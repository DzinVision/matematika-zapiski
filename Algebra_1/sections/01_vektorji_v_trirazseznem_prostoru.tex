\(\mathcal{P}\) - prostor\\
\(T \in \mathcal{P}\) - to"cka

\(A, B \in \mathcal{P}\)\\
\(\overrightarrow{AB}\) - usmerjena daljica

\underline{Formalno:} \(\overrightarrow{AB} = (A, B) \in \mathcal{P} \times \mathcal{P}\) (urejen par)

\subsubsection*{Ekvivalentnost usmerjenih daljic:}

\(\overrightarrow{CD} \sim \overrightarrow{AB}\), kadar je \(\overrightarrow{AB}\) z vzporednim premikom mogo"ce premakniti v \(\overrightarrow{CD}\).
\begin{itemize}
	\item \(|AB| = |CD|\) (dol"zini daljic sta enaki)
	\item ista smer ("ce potegnemo premico "cez izhodi"sca daljic (\(AC\)), morata biti to"cki \(B\) in \(D\) na istem ``bregu'' te premice)
	\item \(AB \parallel CD\)
\end{itemize}
\[\overrightarrow{CD} \sim \overrightarrow{AB} \Rightarrow \overrightarrow{AB} \sim \overrightarrow{CD}\]

\underline{Def:} Vektor \(\vec{AB}\) je mno"zica \(\vec{AB} = \{\overrightarrow{XY}: \overrightarrow{XY} \sim \overrightarrow{AB}\}\) (usmerjene daljice ekvivalentne daljici \(\overrightarrow{AB}\))

\begin{itemize}
	\item \underline{ni"celni vektor:} \(\vec{AA} = \vec{0}\)
	\item \underline{nasprotni vektor} vektorja \(\vec{AB}\) je \(\vec{BA}\) (\(\vec{BA} = -\vec{AB}\))
\end{itemize}

Dodatna oznaka: \(\vec{a}\), \(-\vec{a}\) nasprotni vektor

\(V = \{\vec{v}: \vec{v} \text{ vektor} \}\) - \underline{vektorski prostor}.

\(O \in \mathcal{P}\); \(O\) fiksiramo

\[f: \mathcal{P} \rightarrow V\]
\[f(T) = \vec{OT}\]
\(f\) je bijekcija (vsaki to"cki priredi natanko en vektor).\\
\(\vec{a} = \vec{OT}\)

\subsection{Operacije z vektorji}
\subsubsection*{Se"stevanje:}
\[\vec{a}, \vec{b} \in V\]
\[\vec{a} = \vec{AB}, \vec{b} = \vec{BC}\]
\[\vec{a} + \vec{b} = \vec{AC}\]
\[\vec{AB} + \vec{BC} = \vec{AC}\]

\underline{Lastnosti:}\footnote{Dokaz lastnosti (1) in (2) s skico.}
\begin{itemize}
	\item[(1)] \((\vec{a} + \vec{b}) + \vec{c} = \vec{a} + (\vec{b} + \vec{c})\) asociativnost
	\item[(2)] \(\vec{a} + \vec{b} = \vec{b} + \vec{a}\) komutativnost
	\item[(3)] \(\vec{a} + \vec{0} = \vec{a}\)
	\item[(4)] \(\vec{a} + (-\vec{a}) = \vec{0}\)
\end{itemize}

Za lastnosti od (1) do (4) = \((V, +)\) \textbf{Abelova grupa}.

\[\vec{a} - \vec{b} := \vec{a} + (-\vec{b})\]

\subsubsection*{Mno"zenje s skalarjem}
Skalar je realno "stevilo.
\[\vec{a}, \alpha \in \mathbb{R}\]
\(\alpha \vec{a}\) je vektor.
\begin{itemize}
	\item ima isto smer kot \(\vec{a}\) za \(\alpha > 0\)
	\item ima nasprotno smer kot \(\vec{a}\) za \(\alpha < 0\)
	\item \(|\alpha \vec{a}| = |\alpha| |\vec{a}|\)
\end{itemize}

\[\vec{a} = \vec{OA} \neq \vec{0}\]
\[\alpha \vec{a} = \vec{OT}, O, A, T \text{ so na isti premici} \]

S tem uvedemo koordinatni sistem na premici \(OA\).

\underline{Lastnosti:}
\begin{itemize}
	\item[(5)] \(\alpha (\beta \vec{a}) = (\alpha \beta) \vec{a}\)
	\item[(6)] \((\alpha + \beta) \vec{a} = \alpha \vec{a} + \beta \vec{a} \)
	\item[(7)] \(\alpha(\vec{a} \vec{b}) = \alpha \vec{a} + \alpha \vec{b}\)
	\item[(8)] \(1\vec{a} = \vec{a}\)
\end{itemize}

\(V, +\) in mno"zenje s skalaji je \textbf{vektorski prostor}: veljajo lastnosti od (1) do (8).

\subsection{Linearna neodvisnost}
\[\vec{a}, \vec{b} \in V\]
\(\vec{a}, \vec{b}\) sta linearno odvisna kadar je:\\
\hspace*{10pt} bodisi \(\vec{b} = \alpha \vec{a}\) za ustrezen \(\alpha \in \mathbb{R}\),\\
\hspace*{10pt} bodisi \(\vec{a}  = \beta \vec{b}\) za ustrezen \(\beta \in \mathbb{R}\).

V nasprotnem primeru sta \(\vec{a}\) in \(\vec{b}\) linearno neodvisna.

\[\vec{a} = \vec{OA}, \vec{b} = \vec{OB}\]

\begin{enumerate}
	\item \(\vec{OA}\) in \(\vec{OB}\) sta linearno odvisna \(\Leftrightarrow O, A, B\) kolinearne (li"zijo na isti premici).
	\item \(\vec{a}, \vec{b}\) sta linearno neodvisna \(\Leftrightarrow (\alpha \vec{a} + \beta \vec{b} = \vec{0} \Rightarrow \alpha = \beta = 0)\)
\end{enumerate}

Privzamemo da sta \(\vec{a}, \vec{b}\) linearno neodvisna:
\[\{T: \vec{OT} = \alpha \vec{a} + \beta \vec{b}, \alpha, \beta \in \mathbb{R}\} = \mathcal{R}\]
\( \alpha \vec{a} + \beta \vec{b}\) - linearna kombinacija\\
\(\mathcal{R}\) - ravnina dolo"cena z \(O, A, B\) (z vektorji \(\vec{a}, \vec{b}\)) in to"cko \(O\).

\[\vec{r} = \vec{OT}, T \in \mathcal{R}\]
\[\exists \alpha, \beta \in \mathbb{R}: \vec{r} = \alpha \vec{a} + \beta \vec{b} \]
Pri tem sta \(\alpha\) in \(\beta\) enoli"cna dolo"cena skalarja.

V \(\mathcal{R}\) smo z vektorjema \(\vec{a}, \vec{b}\) vpeljali koordinatni sistem.

\(\vec{a}, \vec{b}, \vec{c} \in V\) so linearno odvisni, kadar je vsaj eden od njih linearna kombinacija drugih dveh.\\
npr: \(\vec{c} = \alpha \vec{a} + \beta \vec{b}\)

V nasprotnem primeru so \(\vec{a}, \vec{b}, \vec{c}\) linearno neodvisni.

\begin{enumerate}
	\item \(\vec{a} = \vec{OA}, \vec{b} = \vec{OB}, \vec{c} = \vec{OC}\) so linearno odvisni \(\Leftrightarrow O, A, B, C\) koplanarne (le"zijo na isti ravnini)
	\item \(\vec{a}, \vec{b}, \vec{c}\) so linearno neodvisni \(\Leftrightarrow (\alpha \vec{a} + \beta \vec{b} + \gamma \vec{c} = \vec{0} \Rightarrow \alpha = \beta = \gamma = 0)\)
\end{enumerate}

\(\vec{a}, \vec{b}, \vec{c}\) linearno neodvisni

\(\vec{a} = \vec{OA}\)\\
\(\vec{b} = \vec{OB}\)\\
\(\vec{c} = \vec{OC}\)

\[V = \{\alpha \vec{a} + \beta \vec{b} + \gamma \vec{c}: \alpha, \beta, \gamma \in \mathbb{R}\}\]
\(\alpha \vec{a} + \beta \vec{b} + \gamma \vec{c}\) je linearna kombinacija vektorjev \(\vec{a}, \vec{b}, \vec{c}\).

\(V\) - mno"zica vseh vektorjev prostora \(\mathcal{P}\)
\[\mathcal{P} = \{R \in \mathcal{P}: \vec{OR} = \alpha \vec{a} + \beta \vec{b} + \gamma \vec{c}, \alpha, \beta, \gamma \in \mathbb{R}\}\]

\underline{Dodatek:} V zapisu vektorja \(\vec{r} \in V\): \(\vec{r} = \alpha \vec{a} + \beta \vec{b} + \gamma \vec{c}\), so koeficienti \(\alpha, \beta, \gamma\) enoli"cno dolo"ceni.

\underline{Dokaz:}
\begin{align*}
	\vec{r} &= \alpha \vec{a} + \beta \vec{b} + \gamma \vec{c}\\
	\vec{r} &= \alpha_1 \vec{a} + \beta_1 \vec{b} + \gamma_1 \vec{c}\\
	\Rightarrow \alpha \vec{a} + \beta \vec{b} + \gamma \vec{c} &= \alpha_1 \vec{a} + \beta_1 \vec{b} + \gamma_1 \vec{c}\\
	(\alpha - \alpha_1)\vec{a} + (\beta - \beta_1)\vec{b} + (\gamma - \gamma_1)\vec{c} &= \vec{0}\\
	\vec{a}, \vec{b}, \vec{c} \text{ linearno neodvisni } &\Rightarrow \alpha - \alpha_1 = \beta - \beta_1 = \gamma - \gamma_1 = 0\\
	\alpha = \alpha_1, \beta &= \beta_1, \gamma = \gamma_1
\end{align*}

\(\{\vec{a}, \vec{b}, \vec{c}\}\) je \textbf{baza} vektorskega prostora \(V\). \(\vec{a}, \vec{b}, \vec{c}\) so linearno neodvisni.

\(R \in \mathcal{P}\) (\(O\) - fiksirana to"cka)
\(\vec{OR} = \alpha \vec{a} + \beta \vec{b} + \gamma \vec{c}\)
\[R \mapsto (\alpha, \beta, \gamma) \in \mathbb{R}^3 = \{(x, y, z): x, y, x \in \mathbb{R}\}\]

\(\alpha, \beta, \gamma\) je z \(R\) enoli"cno dolo"cena.\\
\(\alpha, \beta, \gamma\) so koordinate to"cke \(R\) glede na koordinaten sistem, ki je dolo"cen z bazo \(\{\vec{a}, \vec{b}, \vec{c}\}\) in to"cko \(O\) (izhodi"s"ce koordinatnega sistema).

Imena koordinat: dfabscisa, ordinata, aplikata

\[\varphi: V \rightarrow \mathbb{R}\]
\[\vec{r} \mapsto (\alpha, \beta, \gamma); \vec{r} = \vbase\]

\(\varphi\) je bijekcija.

S \(\varphi\) prenesemo operaciji se"stevanja vektorjev in mno"zenja vektorjev s skalarji iz \(V\) v \(\mathbb{R}^3\).
\begin{align*}
	\vec{r_1}, \vec{r_2} &\in V\\
	\vec{r_1} &= \alpha_1 \vec{a} + \beta_1 \vec{b} + \gamma_1 \vec{c}\\
	\vec{r_2} &= \alpha_2 \vec{a} + \beta_2 \vec{b} + \gamma_2 \vec{c}\\
	\varphi(\vec{r_1}) &= (\alpha_1, \beta_1, \gamma_1)\\
	\varphi(\vec{r_2}) &= (\alpha_2, \beta_2, \gamma_2)
\end{align*}
\[\vec{r_1} + \vec{r_2} = (\alpha_1 + \alpha_2) \vec{a} + (\beta_1 + \beta_2)\vec{b} + (\gamma_1 + \gamma_2)\vec{c}\]
\[\varphi(\vec{r_1} + \vec{r_2}) = (\alpha_1 + \alpha_2, \beta_1 + \beta_2, \gamma_1 + \gamma_2)\]

Torej velja:
\[(\alpha_1, \beta_1, \gamma_1) + (\alpha_2, \beta_2, \gamma_2) = (\alpha_1 + \alpha_2, \beta_1 + \beta_2, \gamma_1 + \gamma_2)\]
se"stevanje je definirano po komponentah.

Podobno velja za mno"zenje s skalarji:
\[\lambda(\alpha, \beta, \gamma) = (\lambda \alpha, \lambda \beta, \lambda \gamma)\]

\(\mathbb{R}^3\) je za te operaciji \textbf{vektorski prostor} (zado"s"ca A1-A8).
\begin{align*}
	\varphi(\vec{a}) &= (1, 0, 0)\\
	\varphi(\vec{b}) &= (0, 1, 0)\\
	\varphi(\vec{c}) &= (0, 0, 1)
\end{align*}

\[\{(1, 0, 0), (0, 1, 0), (0, 0, 1)\}\] je \textbf{standardna baza} vektorskega prostora \(\mathbb{R}^3\).
\[(\alpha, \beta, \gamma) = \alpha(1, 0, 0) + \beta(0, 1, 0) + \gamma(0, 0, 1)\]

\underline{Oznake:}
\begin{align*}
	\vec{i} &= (1, 0, 0)\\
	\vec{j} &= (0, 1, 0)\\
	\vec{k} &= (0, 0, 1)
\end{align*}

Dodatna zahteva za standardno bazo vektorskega prostora \(\mathbb{R}^3\):
\hspace*{12pt}baza je \textbf{ortonormirana}, torej:
\begin{itemize}
	\item \(|\vec{i}| = |\vec{j}| = |\vec{k}| = 1\)
	\item \(\vec{i}, \vec{j}, \vec{k}\) so paroma pravokotni.
\end{itemize}

\subsection{Skalarni produkt}
\(\vec{a}, \vec{b} \in V\)\\
Kot med njima je \(\varphi, 0 \leq \varphi \leq \pi\)

\underline{Def:} \(\vec{a} \cdot \vec{b} = |\vec{a}| \cdot |\vec{b}| \cos\varphi\)

\(V\) \textbf{identificiramo} z \(\mathbb{R}^3\) (glede na standardno bazo in dano izhodi"s"ce \(O\)).
\begin{align*}
	O &= (0, 0, 0)\\
	\vec{i} &= (1, 0, 0)\\
	\vec{j} &= (0, 1, 0)\\
	\vec{k} &= (0, 0, 1)
\end{align*}

\begin{align*}
	\vec{a} &= (a_1, a_2, a_3) \in \mathbb{R}^3\\
	\vec{b} &= (b_1, b_2, b_3) \in \mathbb{R}^3
\end{align*}

\[\vec{a} \cdot \vec{b} = ?\]
\[\vec{a} = (a_1, a_2, a_3) = \vec{OA}\]
\[|\vec{a}| = |OA| = \sqrt{a_1^2 + a_2^2 + a_3^2}\]
\[d(A, B) = \sqrt{(a_1 - b_1)^2 + (a_2 - b_2)^2 + (a_3 - b_3)^2}\]

Kosinusni izrek:
\[(\vec{a} - \vec{b})^2 = |\vec{a}|^2 + |\vec{b}|^2 - 2|\vec{a}||\vec{b}|\cos\varphi\]
\begin{align*}
	(a_1 - b_1)^2 + (a_2 - b_2)^2 + (a_3 - b_3)^2 &= a_1^2 + a_2^2 + a_3^2 + b_1^2 + b_2^2 + b_3^2 - 2|\vec{a}||\vec{b}|\cos\varphi\\
	\Rightarrow |\vec{a}||\vec{b}|\cos\varphi &= a_1b_1 + a_2b_2 + a_3b_3
\end{align*}
\[\vec{a} \cdot \vec{b} = a_1b_1 + a_2b_2 + a_3b_3\]

\underline{Lastnosti:}
\begin{enumerate}
	\item[(1)] \(\vec{a} \vec{a} = |\vec{a}|^2 \geq 0\) (ena"caj le za \(\vec{a} = \vec{0}\))
	\item[(2)] \((\vec{a} + \vec{b})\vec{c} = \vec{a}\vec{c} + \vec{b}\vec{c}\)
	\item[(3)] \((\alpha\vec{a})\vec{b} = \alpha(\vec{a}\vec{b})\)
	\item[(4)] \(\vec{a}\vec{b} = \vec{b}\vec{a}\)
\end{enumerate}

\begin{align*}
	\vec{a} \bot \vec{b} &\Leftrightarrow \varphi \frac{\pi}{2}, \vec{a} \neq \vec{0}, \vec{b} \neq \vec{0}\\
	\varphi = \frac{\pi}{2} &\Leftrightarrow \cos\varphi = 0 (0 \leq \varphi \leq \pi)\\
	\vec{a} \bot \vec{b} &\Leftrightarrow \vec{a} \cdot \vec{b} = 0
\end{align*}

\underline{Primer:}
\begin{align*}
	\mathbb{R}^3 &\equiv \mathbb{R}^2 \times \{0\}\\
	\vec{a} &= (a_1, a_2, 0)\\
	\vec{a} \text{ v } \mathbb{R}^2: \vec{a} &= (a_1, a_2)\\
	\vec{a} \vec{b} &= a_1b_1 + a_2b_2
\end{align*}
\(p\) - plo"s"cina paralelograma\\
\(p\) izraziti z \(a_1, a_2, b_1, b_2\)
\[p = |\vec{a}||\vec{b}|\sin\varphi\]
\begin{align*}
	\vec{a'} &\bot \vec{a}\\
	|\vec{a'}| &= |\vec{a}|
\end{align*}
\(\vec{a}, \vec{a'}\) pozitivno orientirana\\
\(\vec{a'} = (-a_2, a_1)\)\\
\(\psi = \frac{\pi}{2} - \varphi\) ali \(\varphi - \frac{\pi}{2}\) "ce je orienacija \((\vec{a}, \vec{b})\) pozitivna.

\[|\vec{a}||\vec{b}| \sin\varphi = |\vec{a}||\vec{b}|\cos\theta = \vec{a'}\vec{b} = (-a2, a_1) \cdot (b_1, b_2) = a_1b_2 - a_2b_1\]

\(p = a_1b_2 - a_2b_1\), "ce je orientacija \(\vec{a}, \vec{b}\) pozitivna, "ce pa je negativna velja: \(p = -(a_1b_2 - a_2b_1)\)

\subsection{Vektorski produkt}
$$\vec{a}, \vec{b}$$
$$\vec{a} \times \vec{b}$$
\begin{enumerate}
	\item[(1)] $\vec{a} \times \vec{b}$ je pravokoten na $\vec{a}$ in $\vec{b}$.
	\item[(2)] $|\vec{a} \times \vec{b}|$ je enaka plo"s"cini paralelograma, ki ga dolo"cata $\vec{a}$ in $\vec{b}$. ($=0$, kadar sta $\vec{a}$ in $\vec{b}$ linearno odvisna)
	\item[(3)] Urejena trojica $\vec{a}, \vec{b}, \vec{a} \times \vec{b}$ je pozitivno orientirana.
\end{enumerate}

\begin{align*}
	\vec{a} &= (a_1, a_2, a_3)\\
	\vec{b} &= (b_1, b_2, b_3)\\
	\vec{a} \times \vec{b} &= (x, y, z)	\\\\
	%
	\vec{k} &= (0, 0, 1)\\
	z &= (\vec{a} \times \vec{b}) \cdot \vec{k} =\\
	&=|\vec{a}\times \vec{b}| |\vec{k}| \cos \delta =\\
	&=p \cos \delta	
\end{align*}
$p$ - plo"s"cina paralelograma.\\
$\delta$ - kot med ravninama, ki ju dolo"cata osi (1),(2) in vektorja $\vec{a}, \vec{b}$.

\begin{align*}
	\vec{a'} = (a_1, a_2, 0)\\
	\vec{b'} = (b_1, b_2, 0)\\
	p' = \pm (a_1b_2 - a_2b_1)
\end{align*}
$p'$ ima predznak $+$, kadar sta $\vec{a'}$ in $\vec{b'}$ pozitivno orientirana, ter $-$, kadar sta negativno orientirana.

$$p' = \pm p \cos \delta$$
$+$ kadar: $0 \leq \delta \leq \frac{\pi}{2}$\\
$-$ kadar: $\frac{\pi}{2} \leq \delta \leq \pi$

$$z = \pm p' = a_1b_2 - a_2b_1$$
$\pm$ se izni"ci, ker se predznak, ki nastane zaradi $\cos$ in predznak, ki nastane pri izra"cunu plo"s"cine paralelograma z vektorjema ujemata.

\begin{align*}
	x &= a_2b_3 - a_3b_2\\
	y &= a_3b_1 - a_1b_3
\end{align*}

$$\vec{a} \times \vec{b} = (a_2b_3 - a_3b_2, a_3b_1 - a_1b_3, a_1b_2 - a_2b_1)$$

\[
\begin{vmatrix}
\alpha & \beta\\
\gamma & \delta
\end{vmatrix}
= \alpha \delta - \beta \gamma
\]
determinanta (reda 2)

\[
\vec{a} \times \vec{b} = \left(
\begin{vmatrix}
a_2 & a_3\\
b_2 & b_3
\end{vmatrix}
,
\begin{vmatrix}
a_3 & a_1\\
b_3 & b_1
\end{vmatrix}
,
\begin{vmatrix}
a_1 & a_2 \\
b_1 & b_2
\end{vmatrix}
\right)
\]

\[
\begin{vmatrix}
a_3 & a_1\\
b_3 & b_1
\end{vmatrix}
= -
\begin{vmatrix}
a_1 & a_3 \\
b_1 & b_3
\end{vmatrix}
\]

\begin{align*}
	\vec{a} \times \vec{b} &= 
	\begin{vmatrix}
	a_2 & a_3\\
	b_2 & b_3
	\end{vmatrix}
	\vec{i} +
	\begin{vmatrix}
	a_3 & a_1 \\
	b_3 & b_1
	\end{vmatrix}
	\vec{j} +
	\begin{vmatrix}
	a_1 & a_2 \\
	b_1 & b_2
	\end{vmatrix}
	\vec{k}\\
	%
	\vec{a} \times \vec{b} &=
	\begin{vmatrix}
	\vec{i} & \vec{j} & \vec{k}\\
	a_1 & a_2 & a_3 \\
	b_1 & b_2 & b_3
	\end{vmatrix}
\end{align*}
