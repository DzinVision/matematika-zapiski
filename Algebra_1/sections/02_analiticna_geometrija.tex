\subsection{Premica}
$p$ podana s to"cko $R_0$ na njej in \emph{smernim vektorjem} $\vec{e}$.

$\vec{r_0} = \vec{OR_0} = (x_0, y_0, z_0)$\\
$R \in p$\\
$\vec{r} = \vec{OR} = (x, y, z)$

Koorinatizirali smo premico.

\[\vec{R_0R} = \vec{r} - \vec{r_0}\]
\[\vec{r} = \vec{r_0} + \lambda \vec{e}, \lambda \in \mathbb{R}\]
Ena"cba premice $p$ (vektorska parametri"cna) ($\lambda$ je parameter)
%
\[\vec{e} = (a, b, c)\]
\begin{align*}
	x &= x_0 + \lambda a\\
	y &= y_0 + \lambda b\\
	z &= z_0 + \lambda c
\end{align*}
(Parametri"cna) ena"cba premice.
%
\[
\frac{x - x_0}{a} = \frac{y - y_0}{b} = \frac{z - z_0}{c}
\]
ena"cba premice (brez parametra)

$a = 0 ?$
\[\frac{x - x_0}{0} \equiv (x = x_0 \text{ ali a} x - x_0 = 0)\]
Podobno za $b = 0$ in $c = 0$.

$\vec{R_0R}, \vec{e}$ linearno odvisna $\Leftrightarrow R \in p$

To je kadar: $\vec{R_0R} \times \vec{e} = \vec{0} \Leftrightarrow (\vec{r} - \vec{r_0}) \times \vec{e} = \vec{e} \times (\vec{r} - \vec{r_0}) = \vec{0}$\\
(vektorska ena"cba premice)

"Ce imamo to"cko $R_1$ izven premice, je razdalja med premico $p$ in to to"cko enaka:
\[
\Delta = |\vec{r_1} - \vec{r_0}| \sin \varphi
\]
To ena"cbo lahko preoblikujemo da dobimo:
\[
\Delta = \frac{|\vec{e} \times (\vec{r_1} - \vec{r_0})|}{|\vec{e}|}
\]
To je posebej ugodno, kadar $|\vec{e}| = 1$, saj iz tega sledi $\Delta = |\vec{e} \times (\vec{r_1} - \vec{r_0})|$.

Razdaljo med to"cko in premico lahko zapi"semo tudi kot: $\Delta = d(R_1, p)$.

\subsection{Ravnina}
Da dolo"cimo ravnino $\Sigma$, potrebujemo to"cko $R_0 \in \Sigma$ in vektor normale $\vec{n}$, kjer $\vec{n} \perp \Sigma$ in $\vec{n} \neq \vec{0}$.

Da dolo"cimo kdaj to"cka le"zi na ravnini zapi"semo:
\[
R \in \Sigma \Leftrightarrow \vec{r} - \vec{r_0} \perp \vec{n} \Leftrightarrow \vec{n} \cdot (\vec{r} - \vec{r_0}) = 0
\]

To pomeni da nam ravnino $\Sigma$ dolo"ca ena"cba:
\[
\vec{n} \cdot (\vec{r} - \vec{r_0}) = 0
\]

"Ce zapi"semo vektorje $\vec{r_0}, \vec{r}$ in $\vec{n}$ kot:
\begin{align*}
	\vec{r_0} &= (x_0, y_0, z_0)\\
	\vec{r} &= (x, y, z)\\
	\vec{n} &= (a, b, c)
\end{align*}
lahko zapi"semo ena"cbo ravnine kot:
\[
a(x - x_0) + b(y - y_0) + c(z - z_0) = 0
\]
To ena"cbo lahko naprej pretvorimo v \emph{implicitno obliko}:
\[
ax + by + cz + d = 0
\]
kjer je $d = -ax_0 - by_0 - cz_0$.

"Ce imamo podane to"cke $R_0, R_1$ in $R_2$, lahko izra"cunamo vektor normale kot:
\[
\vec{n} =  (\vec{r_1} - \vec{r_0}) \times (\vec{r_2} - \vec{r_0})
\]
"ce to vstavimo v en"acbo ravnine, dobimo da lahko ravnino $\Sigma$ zapi"semo kot:
\[
((\vec{r_1} - \vec{r_0}) \times (\vec{r_2} - \vec{r_0})) \cdot (\vec{r} - \vec{r_0}) = 0
\]
Opazimo, da nam ta ena"cba predstavlja me"sani produkt kar lahko zapi"semo z determinanto reda 3:
\[
\begin{vmatrix}
x-x_0 & y-y_0 & z-z_0 \\
x_1 - x_0 & y_1 - y_0 & z_1 - z_0\\
x_2 - x_0 & y_2 - y_0 & z_2 - z_0
\end{vmatrix} = 0
\]
kjer je vektor $\vec{r_n}$ zapisan kot: $\vec{r_n} = (x_n, y_n, z_n)$.

"Ce imamo to"cko $R_1$, ki ni na ravnini, lahko zapi"semo razdaljo te to"cke do ravnine kot:
\begin{equation}
	\label{eq:razdalja_tocka_ravnina}
	\Delta = \pm |\vec{r_1} - \vec{r_0}| \cos \varphi
\end{equation}
To ena"cbo lahko s pomo"cjo ena"cbe ravnine preoblikujemo v:
\[
\Delta = \dfrac{|\vec{n} \cdot (\vec{r_1} - \vec{r_0})|}{|\vec{n}|}
\]
v "stevcu lahko uprabimo absolutno vrednost s katero se znebimo predznaka, ki se pojavi v \eqref{eq:razdalja_tocka_ravnina}, ker je razdalja vedno pozitivna.

Razdaljo med ravnino $\Sigma$ in to"cko $R_1$ lahko zapi"semu tudi kot:
\begin{equation*}
\Delta = d(R_1, \Sigma)
\end{equation*}

"Ce si pomagamo z "ze izpeljano implicitno ena"cbo ravnine, se lahko znebimo vektorjev in dobimo naslednjo ena"bo:
\begin{equation*}
d(R_1, \Sigma) = \dfrac{|ax_1 + by_1 + cz_1 + d|}{\sqrt{a^2 + b^2 + c^2}}
\end{equation*}
kjer $\vec{OR_1} = \vec{r_1} = (x_1, y_1, z_1)$.

\subsection{Razdalja med mimobe"znima premicama}
$p_1$: $e_1$ je smerni vektor; $R_1 \in p_1, r_1$\\
$p_2$: $e_2$ je smerni vektor; $R_2 \in p_2, r_2$

Da sta premici mimobe"zni imamo dva pogoja:
\begin{itemize}
	\item $\vec{e_1} \times \vec{e_2} \neq \vec{0} (p_1 \nparallel p_2)$
	\item $p_1 \cap p_2 = \varnothing$ (ne sekata se)
\end{itemize}

\begin{equation*}
d(p_1, p_2) = \min \{d(T_1, T_2): T_1 \in p_1, T_2 \in p_2\}
\end{equation*}

Z pomo"cjo skice in premisleka opazimo, da je najmanj"sa razdalja takrat, ko $S_1S_2 \perp p_1, p_2$. To pomeni:
\begin{align*}
\vec{S_1S_2} & \perp \vec{e_1}, \vec{e_2}\\
\vec{S_1S_2} &= \lambda \vec{e_1} \times \vec{e_2}, \lambda \in \RR
\end{align*}

Tu je spet v veliko pomo"c skica. Ideja je, da z vzporednim premikom premaknemo vektor $\vec{e_2}$ v izhodi"s"ce vektorja $\vec{e_1}$. S tem lahko naredimo ravnino $\Sigma_1$, ki jo tvorita ta dva vektorja. Nato naredimo ravnino $\Sigma_2$ na podoben na"cin -- z vzporednim premikom premaknemo vektor $\vec{e_1}$ v izho"s"ce vektorja $\vec{e_2}$. Velja $\Sigma_1 \parallel \Sigma_2$. Ker sta si ravnini vzporedni lahko premico $p_1$ z vzporednim premikom premaknemo iz $\Sigma_1$ v $\Sigma_2$ in dobimo premico $p_1^*$, ki se seka s premico $p_2$ v to"cko $S_2$. Podobno lahko premaknemo premico $p_2$ v ranino $\Sigma_1$ in dobimo to"cko $S_1$ kjer se sekata $p_1$ in $p_2^*$. Opazimo, da je daljica $S_1S_2$ pravokotna na premici $p_1$ in $p_2$ in je tudi najkraj"sa razdalja med tema premicama. To pomeni, da je dol"zina daljice $S_1S_2$ razdalja med premicama $p_1$ in $p_2$.

Z nadaljnim premislekom in zelo natan"cno narisano skico opazimo, da vektorji $\vec{e_1}, \vec{e_2}$ in $\vec{r_1} - \vec{r_2}$ tvorijo paralelepiped, katerega vi"sina je enaka daljici $S_1S_2$. To pomeni, da lahko uporabimo na"se znanje o me"sanem produktu in naredimo naslednje:
\begin{align*}
V &= |[\vec{r_1} - \vec{r_2}, \vec{e_1}, \vec{e_2}]\\
V &= |\vec{e_1} \times \vec{e_2}| \cdot \Delta\\
\end{align*}
kjer je $\Delta = |S_1S_2|$.

To lahko izena"cimo in dobimo:
\begin{equation*}
\Delta = \dfrac{|[\vec{r_1} - \vec{r_2}, \vec{e_1}, \vec{e_2}]|}{|\vec{e_1} \times \vec{e_2}|}
\end{equation*}