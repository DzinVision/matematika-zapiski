\subsection{Preslikave in relacije}
$A, B$ sta neprazni mno"zici.

Preslikavo, ki slika iz $A$ v $B$ lahko zapi"semo kot $f: A \rightarrow B$ ali $A \stackrel{f}{\rightarrow} B$.

$\forall x \in A$ predpis $f$ dolo"ci natanko en element, ki je iz mno"zice $B$. Mno"zici $A$ re"cemo domena (v"casih tudi definicijsko obmo"cje), mno"zici $B$ pa re"cemo kodomena. $f(x)$ pravimo slika elementa $x$. ($x \mapsto f(x)$)

Zaloga (vrednosti) preslikave $f: A \rightarrow B$ je mno"zica $\{f(x): x \in A\} \subseteq B$.

$f: A \rightarrow B$ je \emph{surjektivna} (surjekcija), kadar je njena zaloga $B$.
\begin{equation*}
\forall y \in B \ \exists x \in A: y = f(x)
\end{equation*}

$f: A \rightarrow B$ je \emph{injektivna} (injekcija), kadar velja sklep:
\begin{equation*}
x_1, x_2 \in A, x_1 \neq x_2 \Rightarrow f(x_1) \neq f(x_2)
\end{equation*}
Za preverjanje uporabimo:
\begin{equation*}
f(x_1) = f(x_2) \Rightarrow x_1 = x_2, x_1, x_2 \in A
\end{equation*}

$f: A \rightarrow B$ je \emph{bijektivna} (bijekcija), kadar je injektivna in hkrati surjektivna. "Ce je $f: A \rightarrow B$ bijekcija, obstaja to"cno dolo"cena preslikava $g: B \rightarrow A$, da velja:
\begin{equation*}
(\forall x \in A: g(f(x)) = x) \land (\forall y \in B: f(g(y)) = y)
\end{equation*}

Preslikavo $g: B \rightarrow A$ imenujemo \emph{inverz} preslikave $f: A \rightarrow B$ in jo ozna"cimo z:
\begin{equation*}
g = f^{-1}
\end{equation*}

\emph{Kompozitum} preslikav $f: A \rightarrow B$ in $g: B \rightarrow C$ je:
\begin{align*}
g \circ f &\text{ ali } gf\\
g \circ f &: A \rightarrow C\\
(g \circ f)(x) &= g(f(x))
\end{align*}
za vsak $x \in A$.

Preslikavo $A \rightarrow A$ imenujemo \emph{identi"cna preslikava} ali \emph{identiteta}:
\begin{align*}
id_A&: A \rightarrow A\\
\forall x \in A&: id_A(x) = x
\end{align*}

\begin{align*}
f&: A \rightarrow B \text{ bijekcija}\\
g&: B \rightarrow A\\
g \circ f &= id_A\\
f \circ g &= id_B
\end{align*}

$f: A\rightarrow B$ je bijekcija in $g: B\rightarrow A$ je inverzana preslikava $f \iff (g \circ f = id_A \land f \circ g = id_B)$

Graf preslikave $f: A \rightarrow B$ je mno"zica:
\begin{align*}
G(f) &= \{(x, f(x)): x \in A\}\\
G(f) &\subseteq A \times B
\end{align*}

\emph{Relacija} med elementi mno"zice $A$ in elementi mno"zice $B$ je podmno"zica mno"zice $A \times B$.

$R \subseteq A\times B$ ($R$ je relacija)\\
$(x, y) \in R \equiv x R y$

\emph{Primeri} kjer $A = B$ (relacija $R \subseteq A \times A$ je \emph{binarjna relacija} na mno"zici $A$).
\begin{enumerate}[(1)]
	\item $A = \RR$
	
	$R$ relacija na $\RR$: $\leq$
	\begin{gather*}
	(x, y) \in R \subseteq \RR \times \RR \iff x \leq y\\
	R = \leq\\
	R = \{(x, y) \in \RR^2: x \leq y\}
	\end{gather*}
	
	\item $A = \{p: \text{$p$ - premica v prostoru}\}$
	
	$R$ relacija vzporednosti
	\begin{equation*}
	p, q \in A \qquad p R q \equiv p \parallel q
	\end{equation*}
	
	\item $M \neq \varnothing, \qquad A = \mathcal{P}M$
	$R$ relacija \emph{inkluzije} $\subseteq$
	\begin{gather*}
		x, y \in A \qquad (x \subseteq A, y \subseteq A)\\
		x R y \equiv x \subseteq y
	\end{gather*}
\end{enumerate}

\emph{Definicije:}
\begin{enumerate}[(1)]
	\item Relacija $R$ nad $A$ je \emph{refleksivna}, kadar velja $x R x$ za vsak $x \in A$.
	\item Relacija $R$ nad $A$ je \emph{tranzitivna}, kadar velja sklep:
	\begin{equation*}
	(x R y \land y R z) \Rightarrow x R z
	\end{equation*}
	
	\item Relacija $R$ nad $A$ je \emph{antisimetri"cna}, kadar velja sklep:
	\begin{equation*}
	(x R y \land y R x) \Rightarrow x = y
	\end{equation*}
	
	\item Relacija $R$ nad $A$ je \emph{simetri"cna}, kadar velja sklep:
	\begin{equation*}
	x R y \Rightarrow y R x
	\end{equation*}
	
	\item $R$ je relacija \emph{delne urejenosti}, kadar je refleksivna, antisimetri"cna in tranzitivna ($R \equiv \leq$).
	
	\item $R$ je relacija \emph{ekvivalence} (ali ekvivalen"cna relacija), kadar je refleksivna, simetri"cna in tranzitivna ($R \equiv \sim$).
\end{enumerate}

Naj bo $A$ neprazna mno"zica, $\sim$ ekvivalen"cna relacija na $A$ in $a \in A$.
\begin{equation*}
[a] = \{x \in A: x \sim a\}
\end{equation*}
$[a]$ je \emph{ekvivalen"cni razred} elementa $a$.
%
\begin{equation*}
a \sim a \Rightarrow a \in [a]
\end{equation*}
%
$a$ je predstavnik tega ekvivalne"cnega razreda.

\begin{equation*}
[a] = [b] ?
\end{equation*}
Predpostavimo $b \sim a$ (zaradi simetri"cnosti sledi $a \sim b$).
\begin{equation*}
x \in [a] \Rightarrow x \sim a \sim b \Rightarrow x \sim b \Rightarrow x \in [b]
\end{equation*}
Torej velja:
\begin{align*}
[a] &\subseteq [b]\\
[b] &\subseteq [a]
\end{align*}
Zato $[a] = [b]$.

\dashuline{Velja tudi $[a] = [b] \Rightarrow a \sim b$}
\begin{equation*}
[a] = [b] \Rightarrow a \in [a] \Rightarrow a \in [b] \Rightarrow a \sim b
\end{equation*}
%
\begin{equation*}
a \sim b \iff [a] = [b]
\end{equation*}

Naj velja $[a] \cap [b] \neq \varnothing$:
\begin{align*}
&\exists c \in [a] \cap [b]\\
&\Rightarrow c \sim a \land c \sim b \Rightarrow a \sim b \Rightarrow [a] = [b]
\end{align*}
%
\begin{gather*}
	[a] \cap [b] \neq \varnothing \Rightarrow [a] = [b]\\
	[a] \neq [b] \Rightarrow [a] \cap [b] = \varnothing
\end{gather*}

$A/_\sim = \{[a]: a \in A\}$ je \emph{kvocientna} ali \emph{faktorska} mno"zica glede na ekvivalen"cno relacijo $\sim$.

$A = \cup [a]$ pravimo \emph{raz"clenitev} $A$-ja.

\emph{Primera:}
\begin{enumerate}[(1)]
	\item $A = \{\overrightarrow{MN}: M,N - \text{to"cki v prostoru} \}$
	
	$\overrightarrow{MN}$ je usmerjena daljica
	
	$\overrightarrow{XY} \sim \overrightarrow{MN} \iff$ obstaja translacija, ki $XY$ prenese v $MN$. $\sim$ je ekvivalen"cna relacija.
	\begin{equation*}
	\left[\overrightarrow{MN}\right] = \left\{\overrightarrow{XY}: \overrightarrow{XY} \sim \overrightarrow{MN}\right\} = \vec{MN}
	\end{equation*}
	
	\item $A = \ZZ \times \NN = \{(m,n); m \in \ZZ, n \in \NN\}$
	\begin{equation*}
	\sim: (m,n) \sim (p,q) \iff mq = np
	\end{equation*}
	$\sim$ je ekvivalen"cna relacija
	
	$A/_\sim = \QQ$
	\begin{equation*}
	\left[(m,n)\right] = \{(p, q): (p, q) \sim (m, n)\}
	\end{equation*}
\end{enumerate}

\subsection{Operacije}
$M \neq \varnothing$

Operacija na $M$ je preslikava $M \times M \rightarrow M, (a, b) \mapsto a \circ b$

$a \circ b$ je \emph{kompozitum} elementov $a$ in $b$.

\textsc{Primeri:}
\begin{enumerate}[1)]
	\item $M = \NN$ ali $\ZZ$ ali $\QQ$ ali $\RR$.
	
	$\circ$ je lahko $+$ ali $\cdot$.
	\item $A \neq \varnothing$
	\begin{equation*}
	M = \{f: A \rightarrow A\} \equiv F(A)
	\end{equation*}
	$\circ$ je kompozitum preslikav
\end{enumerate}

$M$ z dano operacijo $\circ$ je \emph{grupoid} $(M, \circ)$.

Zapis operacije brez znaka $(a, b) \mapsto ab$ je \emph{multiplikativen} zapis operacije.

Imamo grupoid $(M, \sim, \circ)$. Radi bi prenesli $\circ$ v $M/_\sim$.

Operacija $\circ$ je usklajena z ekvivalen"cno relacaijo $\sim$, kadar velja sklep:
\begin{equation*}
(m_1 \sim m \land n_1 \sim n) \Rightarrow m_1 \circ n_1 \sim m \circ n
\end{equation*}
kjer $m, n, m_1, n_1 \in M$.

\textsc{Primer:} $M = \ZZ \times \NN$

$\sim$ iz primera (2)
\begin{gather*}
(p_1, q_1) \sim (p, q) \land (m_1, n_1) \sim (m, n) \Rightarrow (p_1, q_1) + (m_1, n_1) \sim (p, q) + (m, n)\\
(p,q) + (m, n) := (pn + mq, nq)
\end{gather*}v
$+$ iz $\ZZ \times \NN$ lahko prenesemo na $\QQ = (\ZZ \times \NN)/_\sim$.

$(M, \sim, \circ)$, $\sim$ in $\circ$ usklajeni.

V $M/_\sim$ lahko uvedemo operacijo $\scric$ s predpisom:
\begin{equation*}
[a] \scric [b] = [a \circ b]
\end{equation*}
Definicija je dobra zaradi uklajenosti operacije $\circ$ z relacijo $\sim$:
\begin{equation*}
[a_1] = [a] \text{ in } [b_1] = [b] \Rightarrow [a_1 \circ b_1 \sim a \circ b]
\end{equation*}

\subsection{Grupe}

\textsc{Definicije:}
\begin{itemize}
	\item $(M, \circ)$ grupoid
	
	$e \in M$ je \emph{enota} ali \emph{nevtralni element} grupoida $(M, \circ)$ kadar velja:
	\begin{equation*}
	\forall a \in M: a \circ e = e \circ a = a
	\end{equation*}
	
	\dashuline{"Ce enota obstaja je ena sam}
	
	$e_1, e_2 \in M$ sta enoti. Sledi:
	\begin{equation*}
	e_1 \circ e_2 = e_2
	\end{equation*}
	"ce upo"stevamo da je $e_1$ enota,
	\begin{equation*}
	e_1 \circ e_2 = e_1
	\end{equation*}
	"ce upo"stevamo da je $e_2$ enota
	\begin{equation*}
	\Rightarrow e_1 = e_2
	\end{equation*}
	\hfill $\square$
	
	\item Grupoid $(M, \circ)$ je \emph{polgrupa}, kadar je opracije $\circ$ \emph{asociativna}:
	\begin{equation*}
	\forall a, b, c \in M: (a \circ b) \circ c = a \circ (b \circ c)
	\end{equation*}
	V polgrupi oklepaji niso potrebni: $a \circ b \circ c$.
	
	\item Naj bo $(M, \circ)$ polgrupa z enoto $e$.
	
	Element $b \in M$ je \emph{inverz} elementa $a \in M$, kadar velja:
	\begin{equation*}
	a \circ b = b \circ a = e
	\end{equation*}
	Kadar ima element $a \in M$ inverz, pravimo, da je $a$ \emph{invertabilen} ali \emph{obrnljiv}.
	
	\dashuline{"Ce ima $a \in M$ inverz, je ta en sam}
	
	$b_1, b_2$ inverza elementa $a$.
	\begin{align*}
	a \circ b_1 = b_1 \circ a = e\\
	a \circ b_2 = b_2 \circ a = e
	\end{align*}
	\begin{equation*}
	\Rightarrow b_1 = b_1 \circ e = b_1 \circ (a \circ b_2) = (b_1 \circ a) \circ b_2 = e \circ b_2 = b_2
	\end{equation*}
	\hfill $\square$
	
	"Ce je $a \in M$ obrnljiv, njegov inverz zaznamujemo (v splo"snem) z $a^{-1}$.
	\begin{equation*}
	a \circ a^{-1} = a^{-1} \circ a = e
	\end{equation*}
	
	\item Polgrupa z enoto, v kateri je vsak element obrnljiv se imenuje \emph{grupa}.
	
	Z multiplikativnim zapisom: $(G, \circ)$ je grupa, kadar velja:
	\begin{enumerate}[(1)]
		\item $\forall a, b, c \in G: (ab)c = a(bc)$
		\item $\exists e \in G \forall a \in G: ae = ea = a$
		\item $\forall a \in G \exists b \in G: ab = ba = e$
	\end{enumerate}

	\item $(M, \circ)$ grupoid je \emph{komutativen}, kadar velja:
	\begin{equation*}
	\forall a,b \in M: a \circ b = b \circ a
	\end{equation*}
\end{itemize}
%
\textsc{Primeri:}
\begin{enumerate}[(1)]
	\item $(\NN, +)$ polgrupa brez enote ("ce $0 \notin \NN$).
	\item $(\NN, \cdot)$ polgrupa z enoto 1
	\item $(\ZZ, +)$ grupa
	\item $(\ZZ, \cdot)$ polgrupa z enoto 1
	\item $A \neq \varnothing, M = F(A) = \{f: A \rightarrow A\}$
	
	operacija: komponiranje preslikave
	
	$(M, \circ)$ je polgrupa z enoto $e = id$
	
	\item $M = S(A) = \{f: A \mapsto A, \text{$f$ je bijekcija}\}$
	
	$(M, \circ)$ je grupa
\end{enumerate}

Prej"sen primer lahko nekoliko spremenimo in dobimo:
\begin{gather*}
A = \{1, 2, \ldots, n\} \\
S(A) \equiv S_n
\end{gather*}
$S_n$ je \emph{simetri"cna grupa}.
\begin{align*}
\pi &\in S_n \\
\pi &: \{1, 2, \ldots, n\} \rightarrow \{1, 2, \ldots, n\}
\end{align*}
"Ce preslikamo vse elemente s preslikavo $\pi$ dobimo:
\begin{equation*}
\{\pi(1), \pi(2), \ldots, \pi(n)\} = \{1, 2, \ldots, n\}
\end{equation*}
Pravimo, da je $\pi$ \emph{permutacija} in jo zapi"semo kot:
\begin{equation*}
\pi = \begin{pmatrix}
1 & 2 & \ldots & n \\
\pi(1) & \pi(2) & \ldots & \pi(n)
\end{pmatrix}
\end{equation*}
Zapis $\pi(k)$ je ralitvno dolg, zato ga skraj"samo na:
\begin{equation*}
\pi(k) = i_k
\end{equation*}
S tem lahk permutacijo $\pi$ zapi"semo kot:
\begin{equation*}
\pi = \begin{pmatrix}
1 & 2 & \ldots & n \\
i_1 & i_2 & \ldots & i_n
\end{pmatrix}
\end{equation*}
Zelo lahko je izplejati, da $S_n$ ima $n!$ elementov.

Ker so permutacije elementi grupe, ki ima za operacijo komponiranje preslikav (kompozitum), lahko z njimi ra"cunamo. Poglejmo si primer:
\begin{align*}
\rho &= \begin{pmatrix}
1 & 2 & 3 \\
2 & 3 & 1
\end{pmatrix}\\
%
\sigma &= \begin{pmatrix}
1 & 2 & 3 \\
1 & 3 & 2
\end{pmatrix}
\end{align*}
kjer $\rho, \sigma \in S_3$
\begin{align*}
\rho \sigma &= \begin{pmatrix}
1 & 2 & 3 \\
2 & 1 & 3
\end{pmatrix}\\
%
\sigma \rho &= \begin{pmatrix}
1 & 2 & 3 \\
3 & 2 & 1
\end{pmatrix}
\end{align*}
Opazimo, da $\rho \sigma \neq \sigma \rho$.

Poglejmo si, kako lahko v grupi kraj"samo. Naj bo $(G, \cdot)$ grupa.
\begin{align*}
ab &= ac\\
a^{-1} (ab) &= a^{-1} (ac)\\
(a^{-1} a) b &= (a^{-1} a) c\\
eb &= ec \\
b &= c
\end{align*}
Pozorni moramo biti na vrstni red, ker v grupi ni obvezno da velja komutativnost. Pri tem primeru smo na obeh straneh ena"cbe $a$ imeli na levi strani. Analogno bi lahko pravilo kraj"sanja izpeljali, "ce bi bil $a$ na desni strani, vendar ne "ce je na eni strani ena"cbe desni, na drugi pa levi "clen. To pomeni da v grupi vlejajo naslednje trditve:
\begin{gather*}
ab = ac \Rightarrow b = c \\
ab = ca \nRightarrow b = c \\
b \neq c \Rightarrow ab \neq ac
\end{gather*}

\textsc{Grupa s tremi elementi je samo ena}

Naj bo $G$ grupa s tremi elementi.
\begin{equation*}
G = \{e, a, b\}
\end{equation*}
kjer je $e$ enota.

Zapi"simo naslednjo tabelo:
\begin{table}[!htbp]
	\centering
	\begin{tabular}{c|ccc}
		& e & a & b \\ \hline
		e &  &  & \\
		a &  &  &  \\
		b &  &  & 
	\end{tabular}
\end{table}

Prva vrstica in prvi stolpec sta trivialna, saj imamo na eni strani enoto. Tabelo lahko dopolnimo in dobimo:
\begin{table}[!htbp]
	\centering
	\begin{tabular}{c|ccc}
		& e & a & b \\ \hline
		e & e & a & b\\
		a & a &  &  \\
		b & b &  & 
	\end{tabular}
\end{table}

 Potrebujemo premisliti drugo vrstico. Vemo "ze, da $ae = a$, potrebujemo pa se odlo"citi, kaj bomo zapisali pri $aa$ in pri $ab$.

Zgoraj smo zapisali pravilo, ki nam pravi naslednje: $b \neq c \Rightarrow ab \neq ac$. V grupi so trije razli"cni elementi, to pomeni: $e \neq a \neq b \Rightarrow ae \neq aa \neq ab$. Druga"ce povedano, v vsaki vrstici bo vsak element nastopil natanko enkrat in tudi v vsakem stolpcu bo vsak element nastopil natanko enkrat. To si lahko predstavljamo kot nekak"sen sudoku.

"Ce se vrnemo na prej"sen problem - odlo"citev kaj je $aa$ in kaj $ab$. Sedaj vemo da imamo dve mo"znosti:
\begin{enumerate}[1)]
	\item $ab = b \Rightarrow a = e \rightarrow\leftarrow$ ni mo"zno, ker bi potem $a$ bil enota, vemo pa da mora biti razli"cen od enote.
	\item $ab = e$
\end{enumerate}
Torej se odlo"cimo da bo veljalo $ab = e$. Za $aa$ nam torej ostane samo ena mo"znost, to je: $aa = b$. Tabelo lahko "se nekoliko dopolnimo:
\begin{table}[!htbp]
	\centering
	\begin{tabular}{c|ccc}
		& e & a & b \\ \hline
		e & e & a & b\\
		a & a & b & e \\
		b & b &  & 
	\end{tabular}
\end{table}

Za izpolniti nam ostane samo "se $ba$ in $bb$. Zapisali smo "ze, da se mora v vsaki vrstici vsak element nahajat natanko enkrat. Torej lahko samo dopolnimo tabelo do konca in dobimo:
\begin{table}[!htbp]
	\centering
	\begin{tabular}{c|ccc}
		& e & a & b \\ \hline
		e & e & a & b\\
		a & a & b & e \\
		b & b & e & a
	\end{tabular}
\end{table}

Definirajmo potence. To bomo naredili podobno kot pri analizi. Za pozitivne cele eksponente torej velja:
\begin{align*}
aa &= a^2 \\
aaa &= a^3 \\
\underbrace{aa\ldots a}_{n} &= a^n
\end{align*}
Za negativne cele eksponente velja podobno:
\begin{align*}
a^{-1}a^{-1} &= a^{-2} \\
a^{-1}a^{-1}a^{-1} &= a^{-3} \\
\underbrace{a^{-1}a^{-1}\ldots a^{-1}}_{n} = a^{-n}
\end{align*}
Definirati moramo "se $a^0$. To naredimo na slede"c na"cin:
\begin{equation*}
a^0 \equiv e
\end{equation*}

Sedaj lagko zapi"semo $G$ kot $G = \{e, a, a^2\}$. Vemo tudi, da $a^3 = e$.

Primer take je grupe je podmno"zica kompleksnih "stevil kjer je opracija mno"zenje:
\begin{align*}
G &\subseteq \CC\\
G &= \{1, a, a^2\}\\
a &= -\dfrac{1}{2} + \dfrac{\sqrt{3}}{2}
\end{align*}

Za katerikoli $n$ obstaja grupa. Zgornji grupi $G$ pravimo tudi \emph{cikli"cna grupa}.

\textsc{Definicija} transpozicije:

Naj bosta $j, k \in \{1, \ldots, n\}, j \neq k$
\begin{align*}
	\tau &\in S_n \\
	\tau(j) &= k \\
	\tau(k) &= j \\
	\tau(i) &= i \forall i \in \{1, \ldots, n \} \setminus \{j, k\}
\end{align*}
$\tau$ je \emph{transpozicija}.

Vsaka permutacija je kompozitum samih transpozicij.

\textsc{Primer:}
\begin{equation*}
\pi = \begin{pmatrix}
1 & 2 & 3 & 4 & 5 \\
3 & 5 & 1 & 2 & 4
\end{pmatrix}
\end{equation*}
Lahko si naredimo diagram, kjer v vsakem koraku premaknemo en element na pravo mesto. Za"cnemo z 1, nato 2 in tako naprej. Nato samo komponiramo transpozicije, ki smo jih uporabili. Skica takega postopka je v zvezku. "Ce je ni, potem lahko poizkusi"s izumiti toplo vodo, lahko pa vpra"sa"s kak"snega "studenta, ki je bolj priden od tebe in ima to skico v zvezku. Torej lahko permutacijo $\pi$ zapi"semo kot kompozitum transpozicij na nasledenj na"cin:
\begin{equation*}
\pi = (4, 5) (2, 4) (1, 3)
\end{equation*}
Startegija velja v vsaki simetri"cni grupi $S_n$. Zelo lahko je opzaiti, da lahko vsako permutacijo zapi"semo kot kompozitum najve"c $n-1$ transpozicij.

Definirajmo inverzijo. Naj bo
\begin{gather*}
\pi = \begin{pmatrix}
1 & 2 & 3 & \ldots & n \\
i_1 & i_2 & i_3 & \ldots & i_n
\end{pmatrix} \in S_n \\
1 \leq j < k \leq n
\end{gather*}
\textsc{Definicija:} Par $(j, k)$ tvori \emph{inverzijo} v permutaciji $\pi$, kadar v vrstici $i_1, i_2, \ldots, i_n$ $k$ nastopa pred $j$ (z leve proti desni). Druga"ce povedano: indeks mesta elementa $i_k$ je manj"si od indeksa elementa $i_j$.
\begin{equation*}
\inv \pi = \text{ "stevilo inverzij v $\pi$}
\end{equation*}
%
\textsc{Primer:}
\begin{equation*}
\pi = \begin{pmatrix}
1 & 2 & 3 & 4 & 5 \\
3 & 5 & 1 & 2 & 4
\end{pmatrix}
\end{equation*}
\begin{multicols}{2}
	Inverzije v $\pi$ so:
	\columnbreak
	\begin{gather*}
		(1, 3), (1, 5) \\
		(2, 3), (2, 5) \\
		(4, 5)
	\end{gather*}
\end{multicols}
\begin{equation*}
\inv \pi = 5
\end{equation*}
Definirajmo naslednjo funkcijo:
\begin{equation*}
s(\pi) = (-1)^{\inv \pi} = \begin{cases}
1 & \text{$\pi$ ima sodo inverzij} \\
-1 & \text{$\pi$ ima liho inverzij}
\end{cases}
\end{equation*}
Pravimo da:

$\pi$ je soda $\iff s(\pi) = 1$ \\
$\pi$ je liha $\iff s(\pi) = -1$

\textsc{Trditev:} Naj bo $\tau \in S_n$ transpozicija. Potem $\forall \rho \in S_n$ velja:
\begin{equation*}
s(\tau \rho) = -s(\rho)
\end{equation*}
\textsc{Dokaz:}
\begin{equation*}
\rho = \begin{pmatrix}
1 & \ldots & n \\
i_1 & \ldots & i_n
\end{pmatrix}
\end{equation*}
\begin{enumerate}[1)]
	\item $\tau = (i_k, i_{k+1})$
	\begin{gather*}
	\inv (\tau \rho) = \inv (\rho) \pm 1 \\
	\Rightarrow s(\tau \rho) = - s(\rho)
	\end{gather*}
	
	\item $\tau (i_k, i_{k+p}), p > 1$
	
	$\tau$ dose"zemo s produktom transpozicij podobni tisti v primeru (1). To pomeni, da najprej element $i_k$ premikamo v desno proti $i_{k+p}$, vsaki"c za eno mesto, nato pa "se element $i_{k+p}$ premikamo nazaj na prvotno mesto elementa $i_k$. "Ce znamo vsaj malo algoritmov, se lahko spomnimo na bubble sort. Za ostale, ki ne znajo algoritmov pa obstaja skica, ki se "zal ponovno nahaja samo v zvezku in domi"sliji bralca.
	
	Torej potrebujemo $p$ transpozicij, da premakno element $i_k$ na mesto elementa $i_{k+p}$.  V tem trenutnku, je $i_{k+p}$, "ze premaknjen eno mesto proti ciljni poziciji, zato potrebujemo samo "se $p-1$ transpozicij, da ga damo na mesto elementa $i_k$. Torej je skupno "stevilo potrebnih transpozicij:
	\begin{equation*}
	p + p-1 = 2p - 1
	\end{equation*}
	Vemo, da se na vsakem koraku predznak premutacije zamenja, zato velja:
	\begin{equation*}
	s(\tau \rho) = (-1)^{2p - 1} s(\rho) = -s (\rho)
	\end{equation*}
	saj je $2p - 1$ liho "stevilo. \hfill $\square$
\end{enumerate}

\textsc{Izrek:} Naj bo $\pi \in S_n$ in naj velja:
\begin{equation*}
\pi = \tau_1 \tau_2 \ldots \tau_k
\end{equation*}
kjer so $\tau_i$ transpozicije.

Potem je $\pi$ soda (oziroma liha) natanko takrat, kadar je "stevilo $k$ sodo (oziroma liho).

\textsc{Dokaz:} $s(e) = 1$ kjer je $e = id_{\{1, \ldots, n\}}$ enota grupa $S_n$. Z uporabo prej"snje trditve lahko naredimo naslednje:
\begin{multline*}
s(\pi) = s(\underbrace{\tau_1}_{\tau} \underbrace{\tau_2 \ldots \tau_k e}_{\rho}) =\\
(-1)s(\tau_2 \ldots \tau_k e) = (-1)^2 s(\tau_3 \ldots \tau_k e) = \cdots  \\
 (-1)^k s(e) = (-1)^k
\end{multline*}

Naj bo $A_n = \{\pi \in S_n: \text{$\pi$ soda}\}$, $e \in A_n$
\begin{enumerate}[(1)]
	\item \dashuline{$\rho, \sigma \in A_n \Rightarrow \rho \sigma \in A_n$}
	
	$\rho, \sigma$ zapi"semo kot produkt samih transpozicij. Nato uporabimo prej"snji izrek. 
	
	\textbf{Opomba:} to velja samo za sode premutacje. Produkt 2 lihih permutacij je soda permutacija.
	
	\item \dashuline{$\rho \in A_n \Rightarrow \rho^{-1} \in A_n$}
	\begin{align*}
	\rho &= \tau_1 \tau_2 \ldots \tau_{k-1} \tau_k \\
	\rho^{-1} &= \tau_k \tau_{k-1} \ldots \tau_2 \tau_1
	\end{align*}
	kjer $\tau_i$ transpozicija in $k$ je sodo.
	\begin{equation*}
	\rho \rho^{-1} = \tau_k \tau_{k-1} \ldots \tau_2 \tau_1 \tau_1 \tau_2 \ldots \tau_{k-1} \tau_k
	\end{equation*}
	Ker je $S_n$ grupa velja asociatovnost, torej lahko za"cnemo v sredini: $\tau_1 \tau_1 = e$, nato $\tau_2 \tau_2 = e$ in tako naprej.
	
	$A_n \subseteq S_n, e \in A_n$. Torej je $A_n$ zaprta za mno"zenje in zaprta za invertiranje. Zato je $A_n$ grupa. Pravimo ji \emph{alternirajo"ca grupa}.
\end{enumerate}
Naj bo $\tau$ transpozicija, $\rho \in A_n \Rightarrow \tau \rho$ je liha

Naj bosta $\rho_1 \rho_2 \in A_n, \rho_1 \neq \rho_2$. Sledi $\tau \rho_1 \neq \tau \rho_2$.

$n>1$ "stevilo lihih permutacij je enako "stevilu sodih permutacij. Torej ima $A_n$ $\dfrac{n!}{2}$ elementov.

\textsc{Definirajmo} podgrupo:

Naj bo $(G, \cdot)$ grupa in $H \subseteq G, H \neq \varnothing$. $H$ naj izpolnjuje pogoja:
\begin{enumerate}[(1)]
	\item $a, b \in H \Rightarrow ab \in H$
	
	Temu pravimo \emph{zaprtost za mno"zenje}
	
	\item $a \in H \Rightarrow a^{-1} \in H$
	
	Temu pravimo \emph{zaprtost za invertiranje}
\end{enumerate}
\dashuline{Potem je $H$ za operacijo iz $G$ grupa.}

\begin{gather*}
a \in H \stackrel{(2)}{\Rightarrow} a^{-1} \in H \\
a, a^{-1} \in H \stackrel{(1)}{\Rightarrow} e = aa^{-1} \in H
\end{gather*}
$e$ enota grupa $G$ le"zi v $H$ in je enota v $H$. Pravimo, da je $H$ \emph{podgrupa} grupe $G$.

\textsc{Primeri:}
\begin{enumerate}[(1)]
	\item $A_n$ je podgrupa $S_n$
	\item $G$ grupa, $G$ je podgrupa v $G$.
	
	$\{e\}$ je \emph{trivialna podgrupa} $G$
	
	\item $(G, \cdot)$ je grupa
	\begin{align*}
	a &\in G \\
	H &= \{a^m; m \in \ZZ\} \\
	H &= \{\ldots, a^{-2}, a^{-1}, e ,a, a^2, a^3, \ldots \} \\
	\end{align*}
	$H$ je najmanj"sa podgrupa grupe $G$, ki vsebuje $a$.
	\begin{equation*}
	H \equiv \left<a\right>
	\end{equation*}
\end{enumerate}

Recimo, da velja $a^{m_1} = a^{m_2}$ za celi "stevili $m_1 < m_2$.
\begin{gather*}
a^m a^{-m_1} = a^{m_2} a^{-m_1} = a^{m_2 - m_1} \\
k = m_2 - m \in \NN, k \geq 1 \\
\exists k \in \NN: a^k = e
\end{gather*}
Naj bo $k \in \NN$ najmanj"se naravno "stevilo, ki izpolnjuje pogoj $a^k = e$. Ponavljal se bo vzorec:
\begin{equation*}
e, a, a^2, \ldots, a^{k-1}
\end{equation*}
in veja:
\begin{align*}
	a^{k+1} &= a^k a = a \\
	a^{k+2} &= a^k a^2 = a^2
\end{align*}
\begin{equation*}
H = \{e, a, a^2, \ldots, a^{k-1}\}
\end{equation*}
$H$ ima $k$ elementov. Pravimo, da je $H$ \emph{cikli"cna grupa reda $k$}.

\subsection{Abelove grupe}
Pravimo jim tudi \emph{komutativne grupe}.

$(G, +)$ je grupa in je komutativna:
\begin{equation*}
\forall a, b \in G: a + b = b + a
\end{equation*}

\textsc{Primeri:} $(\ZZ, +), (\RR, +), (\CC, +)$

Naj bo $(G, \cdot)$ grupa (ne nujno komutativna).
\begin{equation*}
a \in G, \left< a \right> = \{a^m: m \in \ZZ \}
\end{equation*}
$\left<a\right>$ je abelova grupa:
\begin{equation*}
a^i a^j = a^{i+j} = a^j a^i
\end{equation*}

\subsubsection*{Oznake v abelovi grupi:}
\begin{itemize}
	\item $0$ -enota Abelove grupe
	\item $-a$ \emph{nasprotni element} od $a$
	\item $\underbrace{a + a + \ldots + a}_{n, n \in \NN} \equiv na, n \in \NN$
	\item $(-n)a \equiv -(na) = \underbrace{(-a) + (-a) + \ldots + (-a)}_{n}, n \in \NN$
	\item $0a \equiv 0$ 
	
	\textbf{Opomba:} na levi strani je 0 "stevilo 0, na desni pa je enota grupe
	\item  $a, b \in GG$
	\begin{equation*}
	a - b \equiv a + (-b)
	\end{equation*}
\end{itemize}

Naj bo $(G, +)$ Abelova grupa in $H \subseteq G, H \neq \varnothing$, $H$ podgrupa
\begin{enumerate}[(1)]
	\item $a, b \in H \Rightarrow a + b \in H$
	\item $a \in H \Rightarrow -a \in H$
	\item[(1) \& (2)] $\iff (a, b \in H \Rightarrow a - b \in H)$
\end{enumerate}
\textsc{Primer:} $(G, +) = (\ZZ, +)$, $+$ je obi"cajno se"stevanje.
\begin{gather*}
n \in \NN \\
H = \{kn: k \in \ZZ\} = \{m \in \ZZ: n | m\}
\end{gather*}
$H$ je podgrupa grupe $(\ZZ, +)$ in je mno"zica ve"ckratnikov $n$. Pi"semo:
\begin{equation*}
H \equiv n\ZZ
\end{equation*}

Naj bo $(G, +)$ Abelova grupa, $H \subseteq G$, $H$ podgrupa.
\begin{equation*}
a, b \in G: a \sim b \stackrel{\text{def.}}{\iff} a - b \in H
\end{equation*}
\dashuline{$\sim$ je ekvivalen"cna relacija}
\begin{enumerate}[(1)]
	\item \emph{refleksivnost:} $\forall a \in G: a \sim a$
	\begin{equation*}
	a \sim a \iff \underbrace{a - a}_{\text{enota $H$}} \in H
	\end{equation*}
	
	\item \emph{simetri"cnost} $a \sim b \Rightarrow b \sim a$
	\begin{equation*}
	a \sim b \Rightarrow a - b \in H \Rightarrow b - a = - (a - b) \in H
	\end{equation*}
	Dokazati je potrebno korak $b - a = -(a - b)$:
	\begin{equation*}
	(b-a) + (a - b) = b + (-a) + a + (-b) = 0
	\end{equation*}
	
	\item \emph{tranzitivnost} $a \sim b \land b \sim c \Rightarrow a \sim c$
	\begin{align*}
	a - b &\in H \\
	b - c &\in H
	\end{align*}
	Po definiciji $a, b \in H: a + b \in H$, torej v na"sem primeru:
	\begin{equation*}
	(a - b )+ (b - c)  = b - c \in H \Rightarrow a \sim c
	\end{equation*}
\end{enumerate}
\hfill $\square$

Se"stevanje in ekvivalen"cna relacija $\sim$ sta usklajeni:
\dashuline{$x \sim a, y \sim b \Rightarrow x + y \sim a + b$}
\begin{align*}
x - a &\in H \\
y - b &\in H
\end{align*}
Po definiciji relacije potrebuje veljati: $(x + y) - (a + b) \in H$
\begin{equation*}
(x + y) - (a + b) = \underbrace{x - a}_{\in H} + \underbrace{y - b}_{\in H} \in H
\end{equation*}
\hfill $\square$

Zato lahko operacijo $+$ prenesemo na kvocientno mno"zico:
\begin{gather*}
G/_\sim = \left\{[a]: a \in G\right\} \\
\forall a, b \in G: [a] + [b] = [a + b]
\end{gather*}
$(G/_\sim, +)$ je Abelova grupa

\textbf{Opomba:} $+$ je operacija med ekvivalen"cnimi razredi in je razli"cna od operacije med elementi

\textsc{Oznaka:} $G/_H$ (namesto $G/_\sim$, ker $\sim$ definiramo s pomo"cjo $H$)

Komutativnost in asociativnost se hitro preveri. Za enoto vzamemo $[0]$. Nasprotni element definiramo kot $-[a] = [-a]$