\subsection{Preslikave in relacije}
$A, B$ sta neprazni mno"zici.

Preslikavo, ki slika iz $A$ v $B$ lahko zapi"semo kot $f: A \rightarrow B$ ali $A \stackrel{f}{\rightarrow} B$.

$\forall x \in A$ predpis $f$ dolo"ci natanko en element, ki je iz mno"zice $B$. Mno"zici $A$ re"cemo domena (v"casih tudi definicijsko obmo"cje), mno"zici $B$ pa re"cemo kodomena. $f(x)$ pravimo slika elementa $x$. ($x \mapsto f(x)$)

Zaloga (vrednosti) preslikave $f: A \rightarrow B$ je mno"zica $\{f(x): x \in A\} \subseteq B$.

$f: A \rightarrow B$ je \emph{surjektivna} (surjekcija), kadar je njena zaloga $B$.
\begin{equation*}
\forall y \in B \ \exists x \in A: y = f(x)
\end{equation*}

$f: A \rightarrow B$ je \emph{injektivna} (injekcija), kadar velja sklep:
\begin{equation*}
x_1, x_2 \in A, x_1 \neq x_2 \Rightarrow f(x_1) \neq f(x_2)
\end{equation*}
Za preverjanje uporabimo:
\begin{equation*}
f(x_1) = f(x_2) \Rightarrow x_1 = x_2, x_1, x_2 \in A
\end{equation*}

$f: A \rightarrow B$ je \emph{bijektivna} (bijekcija), kadar je injektivna in hkrati surjektivna. "Ce je $f: A \rightarrow B$ bijekcija, obstaja to"cno dolo"cena preslikava $g: B \rightarrow A$, da velja:
\begin{equation*}
(\forall x \in A: g(f(x)) = x) \land (\forall y \in B: f(g(y)) = y)
\end{equation*}

Preslikavo $g: B \rightarrow A$ imenujemo \emph{inverz} preslikave $f: A \rightarrow B$ in jo ozna"cimo z:
\begin{equation*}
g = f^{-1}
\end{equation*}

\emph{Kompozitum} preslikav $f: A \rightarrow B$ in $g: B \rightarrow C$ je:
\begin{align*}
g \circ f &\text{ ali } gf\\
g \circ f &: A \rightarrow C\\
(g \circ f)(x) &= g(f(x))
\end{align*}
za vsak $x \in A$.

Preslikavo $A \rightarrow A$ imenujemo \emph{identi"cna preslikava} ali \emph{identiteta}:
\begin{align*}
id_A&: A \rightarrow A\\
\forall x \in A&: id_A(x) = x
\end{align*}

\begin{align*}
f&: A \rightarrow B \text{ bijekcija}\\
g&: B \rightarrow A\\
g \circ f &= id_A\\
f \circ g &= id_B
\end{align*}

$f: A\rightarrow B$ je bijekcija in $g: B\rightarrow A$ je inverzana preslikava $f \iff (g \circ f = id_A \land f \circ g = id_B)$

Graf preslikave $f: A \rightarrow B$ je mno"zica:
\begin{align*}
G(f) &= \{(x, f(x)): x \in A\}\\
G(f) &\subseteq A \times B
\end{align*}

\emph{Relacija} med elementi mno"zice $A$ in elementi mno"zice $B$ je podmno"zica mno"zice $A \times B$.

$R \subseteq A\times B$ ($R$ je relacija)\\
$(x, y) \in R \equiv x R y$

\emph{Primeri} kjer $A = B$ (relacija $R \subseteq A \times A$ je \emph{binarjna relacija} na mno"zici $A$).
\begin{enumerate}[(1)]
	\item $A = \RR$
	
	$R$ relacija na $\RR$: $\leq$
	\begin{gather*}
	(x, y) \in R \subseteq \RR \times \RR \iff x \leq y\\
	R = \leq\\
	R = \{(x, y) \in \RR^2: x \leq y\}
	\end{gather*}
	
	\item $A = \{p: \text{$p$ - premica v prostoru}\}$
	
	$R$ relacija vzporednosti
	\begin{equation*}
	p, q \in A \qquad p R q \equiv p \parallel q
	\end{equation*}
	
	\item $M \neq \varnothing, \qquad A = \mathcal{P}M$
	$R$ relacija \emph{inkluzije} $\subseteq$
	\begin{gather*}
		x, y \in A \qquad (x \subseteq A, y \subseteq A)\\
		x R y \equiv x \subseteq y
	\end{gather*}
\end{enumerate}

\emph{Definicije:}
\begin{enumerate}[(1)]
	\item Relacija $R$ nad $A$ je \emph{refleksivna}, kadar velja $x R x$ za vsak $x \in A$.
	\item Relacija $R$ nad $A$ je \emph{tranzitivna}, kadar velja sklep:
	\begin{equation*}
	(x R y \land y R z) \Rightarrow x R z
	\end{equation*}
	
	\item Relacija $R$ nad $A$ je \emph{antisimetri"cna}, kadar velja sklep:
	\begin{equation*}
	(x R y \land y R x) \Rightarrow x = y
	\end{equation*}
	
	\item Relacija $R$ nad $A$ je \emph{simetri"cna}, kadar velja sklep:
	\begin{equation*}
	x R y \Rightarrow y R x
	\end{equation*}
	
	\item $R$ je relacija \emph{delne urejenosti}, kadar je refleksivna, antisimetri"cna in tranzitivna ($R \equiv \leq$).
	
	\item $R$ je relacija \emph{ekvivalence} (ali ekvivalen"cna relacija), kadar je refleksivna, simetri"cna in tranzitivna ($R \equiv \sim$).
\end{enumerate}

Naj bo $A$ neprazna mno"zica, $\sim$ ekvivalen"cna relacija na $A$ in $a \in A$.
\begin{equation*}
[a] = \{x \in A: x \sim a\}
\end{equation*}
$[a]$ je \emph{ekvivalen"cni razred} elementa $a$.
%
\begin{equation*}
a \sim a \Rightarrow a \in [a]
\end{equation*}
%
$a$ je predstavnik tega ekvivalne"cnega razreda.

\begin{equation*}
[a] = [b] ?
\end{equation*}
Predpostavimo $b \sim a$ (zaradi simetri"cnosti sledi $a \sim b$).
\begin{equation*}
x \in [a] \Rightarrow x \sim a \sim b \Rightarrow x \sim b \Rightarrow x \in [b]
\end{equation*}
Torej velja:
\begin{align*}
[a] &\subseteq [b]\\
[b] &\subseteq [a]
\end{align*}
Zato $[a] = [b]$.

\dashuline{Velja tudi $[a] = [b] \Rightarrow a \sim b$}
\begin{equation*}
[a] = [b] \Rightarrow a \in [a] \Rightarrow a \in [b] \Rightarrow a \sim b
\end{equation*}
%
\begin{equation*}
a \sim b \iff [a] = [b]
\end{equation*}

Naj velja $[a] \cap [b] \neq \varnothing$:
\begin{align*}
&\exists c \in [a] \cap [b]\\
&\Rightarrow c \sim a \land c \sim b \Rightarrow a \sim b \Rightarrow [a] = [b]
\end{align*}
%
\begin{gather*}
	[a] \cap [b] \neq \varnothing \Rightarrow [a] = [b]\\
	[a] \neq [b] \Rightarrow [a] \cap [b] = \varnothing
\end{gather*}

$A/_\sim = \{[a]: a \in A\}$ je \emph{kvocientna} ali \emph{faktorska} mno"zica glede na ekvivalen"cno relacijo $\sim$.

$A = \cup [a]$ pravimo \emph{raz"clenitev} $A$-ja.

\emph{Primera:}
\begin{enumerate}[(1)]
	\item $A = \{\overrightarrow{MN}: M,N - \text{to"cki v prostoru} \}$
	
	$\overrightarrow{MN}$ je usmerjena daljica
	
	$\overrightarrow{XY} \sim \overrightarrow{MN} \iff$ obstaja translacija, ki $XY$ prenese v $MN$. $\sim$ je ekvivalen"cna relacija.
	\begin{equation*}
	\left[\overrightarrow{MN}\right] = \left\{\overrightarrow{XY}: \overrightarrow{XY} \sim \overrightarrow{MN}\right\} = \vec{MN}
	\end{equation*}
\end{enumerate}
