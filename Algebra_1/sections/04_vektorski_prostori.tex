\textsc{Definicija:} \emph{Vektorski prostor} na obsegom \OO je Abelova grupa $(V, +)$ skupaj z \emph{zunanjo operacijo}
\begin{align*}
\OO \times V &\to V \\
(\alpha, v) &\mapsto \alpha v
\end{align*}
ki ustreza naslednjim pogojem:
\begin{enumerate}
	\item $(\alpha + \beta) v = \alpha v + \beta v$ \hfill $\forall \alpha, \beta \in \OO, \forall v \in V$
	\item $\alpha(u + v) = \alpha u + \alpha v$ \hfill $\forall \alpha \in \OO, \forall u, v \in V$
	\item $\alpha(\beta v) = (\alpha \beta)v$ \hfill $\forall \alpha, \beta \in \OO, \forall v \in V$
	\item $1v = v$ \hfill $\forall v \in V$
\end{enumerate}
Elemente iz \OO imenujemo \emph{skalarji}, elemente iz $V$ imenujemo \emph{vektorji}, zunanjo operacijo pa imenujemo \emph{mno"zenje z skalarji}.

\textsc{Primer:}
\begin{enumerate}[(1)]
	\item $V = \RR^3, \OO = \RR$ obi"cajen trirazse"zen vektorski prostor
	\item $V = \OO^n$, $\OO$ - obseg
	
	Naj bosta $x$ in $y$ naslednja vektorja:
	\begin{align*}
	x &= (x_1, x_2, \ldots, x_n) \in \OO^n (x_i \in \OO \forall i) \\
	y &= (y_1, y_2, \ldots, y_n) \in \OO^n (y_i \in \OO \forall i)
	\end{align*}
	Operaciji definiramo slede"ce:
	\begin{align*}
	x + y &= (x_1 + y_1, x_2 + y_2, \ldots, x_n + y_n) \in \OO^n \\
	\alpha x &= (\alpha x_1, \alpha x_2, \ldots, \alpha x_n) \in \OO^n
	\end{align*}
	Za ti dve operacijie je $\OO^n$ vektorski prostor na obsegom $\OO$. Ni"celni element je
	\begin{equation*}
	0 = (0, 0, \ldots, 0) \in \OO^n
	\end{equation*}
	Nasprotni element je:
	\begin{equation*}
	-(x_1, x_2, \ldots x_n) = (-x_1, -x_2, \ldots, -x_n) \in \OO^n
	\end{equation*}
	
	\item $M \neq \varnothing$ \qquad $\mathcal{F}(M, \RR) \equiv \RR^M = \{f: M \to \RR\}$
	
	Operaciji definiramo po to"ckah:
	\begin{align*}
	(\alpha f)(t) &= \alpha f(t) && \forall t \in M (\alpha \in \RR) \\
	(f + g)(t) &= f(t) + g(t) && \forall t \in M
	\end{align*}
	$V = \RR^M, \OO = \RR$
	
	$V$ je vekotrski prostor nad $\RR$.
\end{enumerate}
\subsection{Nekaj osnovnih lastnosti vektorskih prostorov}
Naj bo $V$ vektorski prostor nad \OO. Velja:
\begin{enumerate}[(1)]
	\item $0v = 0$ \hfill $\forall v \in V$
	\item $\alpha 0 = 0$ \hfill $\forall \alpha \in \OO$
	\item $\alpha v = 0 \Rightarrow (\alpha = 0 \lor v = 0)$
	\item $(-1)v = -v$ \hfill $\forall v \in V$
\end{enumerate}
\textsc{Dokaz:}
\begin{enumerate}[(1)]
	\item
	\begin{multline*}
		0v = x \in V \Rightarrow \\
		x + x = 0v + 0v = (0 + 0)v = 0v = x  \\
		\Rightarrow x + x = x \Rightarrow x = 0 \\
		\Rightarrow 0v = 0
	\end{multline*}
	\item Podoben dokaz kot za (1).
	\item $\alpha v = 0$. "Ce $\alpha = 0$ optem velja (2). Druga"ce:
	\begin{multline*}
		\alpha \neq 0 \Rightarrow \exists \alpha^{-1} \in \OO \Rightarrow \\
		\Rightarrow \alpha^{-1} (\alpha v) = \alpha^{-1} 0 = 0 \\
		\underbrace{(\alpha^{-1} \alpha)}_1 v = 1v = v \\
		\Rightarrow v = 0
	\end{multline*}
	\item
	\begin{equation*}
	(-1)v + v = (-1)v + 1v = (-1 + 1) v = 0v = 0
	\end{equation*}
\end{enumerate}
%
\subsection{Vektorski podprostor}
\textsc{Definicija:} Naj bo $V$ vektorski prostor nad $\OO$ in $U \subseteq V, U \neq \varnothing$. $U$ je vektorski poprostor vektorskega prostora $V$, kadar velja:
\begin{enumerate}[(1)]
	\item $x, y \in U \Rightarrow x + y \in U$
	\item $x \in U \Rightarrow \alpha x \in U$ \hfill $\forall \alpha \in \OO$
\end{enumerate}
Obe zahtevi lahko zdru"zimo v eno:
\begin{equation*}
(1) \land (2) \iff (x, y \in U \Rightarrow \forall \alpha, \beta \in U: \alpha x + \beta y \in U)
\end{equation*}
$(U, +)$ je podgrupa grupe $(V, +)$.

\textsc{Primeri:}
\begin{enumerate}[(1)]
	\item $V = \RR^3$, $U$ je ravnina skozi $0$ v $\RR^3$
	
	\item 
	\begin{align*}
	V &= \RR^3 \\
	U &= \RR[x]
	\end{align*}
	
	\item 
	\begin{align*}
	V &= \RR[x] \\
	U &= \RR_m[x] = \{p(x) \in \RR[x]: \text{stp}(x) \leq m\}
	\end{align*}
\end{enumerate}
"Ce je $V$ vektorski prostor nad $\OO$ in $U \subseteq V$ podprostor, uporabljamo oznako:
\begin{equation*}
U \leq V
\end{equation*}
Vsak podprostor vsebuje ni"clo:
\begin{equation*}
x \in U \Rightarrow 0x = 0 \in U
\end{equation*}
Nasprotni element je element podprostora:
\begin{equation*}
x, y \in U \Rightarrow x - y = x + (-1) \in U
\end{equation*}
Ker velja $\alpha x \in U$ in $\beta y \in U$, lahko zapi"semo:
\begin{equation*}
\alpha x + \beta y \in U
\end{equation*}
Zapi"semo lahko:
\begin{equation*}
x_1, x_2, \ldots, x_k \in U \Rightarrow \underbrace{\alpha x_1 + \alpha x_2 + \ldots + \alpha x_k}_{\text{\emph{linearna kombinacija vektorjev} $x_1, \ldots, x_k$}} \in U
\end{equation*}
\subsection{Linearna ogrinja"ca}
\textsc{Definicija:} Naj bo $M \in V, M \neq \varnothing$. \emph{Linearno ogrinja"ca mno"zice} $M$ je
\begin{equation*}
\Lin M = \{\alpha_1 x_1 + \ldots +\alpha_k x_k: x_1, \ldots, x_k \in M, \alpha_1, \ldots \alpha_k \in \OO, k \in \NN\}
\end{equation*}
Velja:
\begin{equation*}
M \subseteq U \leq V \Rightarrow \Lin M \subseteq U
\end{equation*}
\dashuline{$\Lin M$ je vektorski podprostor vektorskega prostora $V$ ($\Lin M \leq V$)}
\begin{itemize}
	\item Zaprtost za se"stevanje:
	\begin{gather*}
	\alpha_1 x_1 + \ldots + \alpha_k x_k \in \Lin M \\
	\beta_1 x_1 + \ldots + \beta_n y_n \in \Lin M
	\end{gather*}
	Opazimo, da so po definiciji $\Lin M$ posame"cni "cleni $\alpha_1 x_1, \ldots \alpha_k x_k \in \Lin M$ in $\beta_1 y_1, \ldots, \beta_n y_n \in \Lin M$, torej je tudi vsota vseh "clenov $\in \Lin M$.
	
	\item Zaprtost za mno"zenje s skalarjem:
	\begin{gather*}
		\beta(\alpha_1 x_1 + \ldots + \alpha_k x_k) = (\beta \alpha_1) x_1 + \ldots + (\beta \alpha_k)x_k \in \Lin M \\
		x_1, \ldots, x_k \in M
	\end{gather*}
	\hfill $\square$
\end{itemize}
Iz tega sledi, da je $\Lin M$ najmanj"si vektorski podprostor, ki vsebuje $M$. Simbolno za malo naprednej"se:
\begin{equation*}
M \subseteq U \leq V \Rightarrow \Lin M \subseteq U
\end{equation*}
Za prazno mno"zico velja:
\begin{equation*}
\Lin \varnothing = \{0\}
\end{equation*}
Poglejmo si, kako je s preseki in unijami. Za preseke velja:
\begin{equation*}
V_i \leq V \forall i \in I \Rightarrow \bigcap_{i \in I}V_i \leq V
\end{equation*}
To je o"citno. Zaprtost za se"stevanje velja, ker "ce sta neka dva vektorja $x, y$ v $\bigcap_{i \in I}V_i$, potem se nahajata v vseh $V_i$. Ker so $V_i$ vektorski podprostori, v njih tudi velja zaprtost za se"stevanje. Zato je vsota $x + y$ tudi v vseh $V_i$, torej je tudi v $\bigcap_{i \in I}V_i$. Podobno lahko naredimo za zaprtost za mno"zenje s skalarjem.

Malo ve"c je za videti pri uniji. $V_1, V_2 \leq V \Rightarrow \Lin (V_1 \cup V_2)$ je najmanj"si vektorski podprostor, ki vsebuje $V_1$ in $V_2$. Primer na katerem se lahko predstavljamo, sta dve premici. Unija dveh premic, ki se sekata ni vektorski podrpostor, zato okoli naredimo linearno ogrinja"co. Poglejmo si eno zanimivost:
\begin{gather*}
x \in \Lin (V_1 \cup V_2) \\
x = \underbrace{\alpha_1 x_1 + \ldots \alpha_k x_k}_{\in V_1} + \underbrace{\beta_1 y_1 + \ldots + \beta_n y_n}_{\in V_2} = u + v
\end{gather*}
Torej velja:
\begin{equation*}
x \in \Lin (V_1 \cup V_2) \iff x = u + v, u \in V_1, v \in V_2
\end{equation*}
Zapi"semo:
\begin{equation*}
V_1 + V_2 = \{u + v: u \in V_1, v \in V_2\}
\end{equation*}
Torej velja:
\begin{equation*}
\Lin (V_1 \cup V_2) = V_1 + V_2
\end{equation*}
Analogno naredimo za ve"c sumandov:
\begin{gather*}
	\Lin (V_1 \cup V_2 \cup \ldots \cup V_k) = V_1 + V_2 + \ldots + V_k \\
	V_i \leq V \forall i \\
	V_1 + \ldots + V_k = \{x_1 + \ldots + x_k: x_i \in V_i \forall i\}
\end{gather*}
\textsc{Definicija:} $V_1 + \ldots + V_k$ je \emph{prema} ali \emph{direktna}, kadar za vsak $x \in V_1 + \ldots + V_k$ obstajajo in so z $x$ enoi"cno dolo"ceni taki vektorji $x_i \in V_i (i = 1, \ldots, k)$, da je $x = x_1 + \ldots + x_k$. Ozani"cimo:
\begin{equation*}
V_1 \oplus \ldots \oplus V_k
\end{equation*}
\textsc{Trditev:} Vsota $V_1 + V_2$ vektorskih podprostorov $V_1$ in $V_2$ je direktna natanko takrat, kadar je $V_1 \cup V_2 = \{0\}$.

\textsc{Dokaz:}
\begin{itemize}
	\item[($\Rightarrow$)] Naj bo vsota $V_1 + V_2$ direktna ($V_1 \oplus V_2$). Vzemimo $x \in V_1 \cup V_2$.
	\begin{equation*}
	x = \underbrace{x}_{\in V_1} + \underbrace{0}_{\in V_2} = \underbrace{0}_{\in V_1} + \underbrace{x}_{\in V_2} \Rightarrow x = 0
	\end{equation*}
	$\Rightarrow V_1 \cup V_2 = \{0\}$
	
	\item[($\Leftarrow$)] Naj bo $V_1 \cup V_2 = \{0\}$.
	\begin{align*}
	x &\in V_1 + V_2 \\
	x &= x_1 + x_2, x_1 \in V_1, x_2 \in V_2 \\
	x &= x_1' + x_2', x_1' \in V_1, x_2' \in V_2 \\
	x_1 + x_2 &= x_1' + x_2' \\
	\underbrace{x_1 - x_1'}_{\in V_1} &= \underbrace{x_2 - x_2'}_{\in V_2} = z
	\end{align*}
	\begin{align*}
	&\Rightarrow z \in V_1 \cap V_2 = \{0\} \\
	&\Rightarrow x = 0 \Rightarrow \\
	&\Rightarrow x_1' = x_1 \land x_2' = x_2
	\end{align*}
	\begin{equation*}
	V_1 \oplus V_2
	\end{equation*}
	\hfill $\square$
\end{itemize}

\subsection{Kvocientni vektorski prostor}
Naj bo $U$ vektorski prostor nad $\OO$, $U \leq V$. Definiramo:
\begin{equation*}
v_1 \sim v_2 \iff v_1 - v_2 \in U
\end{equation*}
kjer je $\sim$ ekvivalen"cna relacija. $U$ je Abelova podgrupa Abelove grupe $V$. $V/_U$ je torej Abelova grupa in velja:
\begin{align*}
[x] + [y] &= [x+y] \forall x, y \in V \\
[z] &= z + U \forall z \in V
\end{align*}
V $V/_U$ uvedemo mno"zenje s skalarji:
\begin{equation*}
\alpha [x] := [\alpha x], \alpha \in \OO, x \in V
\end{equation*}
\dashuline{Definicija je dobra} "ce velja:
\begin{align*}
y \sim x &\Rightarrow \alpha x \sim \alpha y \\
y -x \in U &\Rightarrow \underbrace{\alpha y - \alpha x}_{\alpha (y - x) = z} \in U
\end{align*}
Ker je $U$ podprostor zaprt za mno"zenje s skalarjem, vemo:
\begin{equation*}
z \in U \Rightarrow \alpha z \in U \forall \alpha \in \OO
\end{equation*}
\hfill $\square$

Sledi, da je $V/_U$ vektorski prostor nad $\OO$. Elementi so $x + U, x \in V$.

\textsc{Primer:} $U$ premica skozi 0 v $V = \RR^3$. Elementi $V/_U: x + U, x \in \RR^3$ so premice vzporedne premici $U$.
%
\subsection{Linearne preslikave}
So neke vrste homomorfizmi vektorskih prostorov.

\textsc{Definicija:} Naj bosta $V$ in $U$ vektorska prostora nad istim $\OO$. Preslikava $\A : V \to U$ je \emph{linearna} (= homomorfizem vektorskih prostorov), kadar velja:
\begin{enumerate}[(1)]
	\item $\A(x + y) = \A x + \A y$ \hfill $\forall x, y \in V$
	\item $\A(\alpha x) = \alpha \A x$ \hfill $\forall \alpha \in \OO, \forall x \in V$
\end{enumerate}
Pogoju (1) pravimo, da je $\A$ \emph{aditivna}, pogoju (2) pa pravimo, da je $\A$ \emph{homogena}.
\subsubsection*{Nekaj lastnosti:}
\begin{itemize}
	\item $\A0 = 0$ (pride iz Abelove grupe)
	\item $\A (-x) =  -\A x$ (pride iz Abelove grupe) \hfill $\forall x \in V$
	\item $\A(x-y) = \A x - \A y$ \hfill $\forall x, y \in V$
	\item[(3)] $\A(\alpha x + \beta y) = \A(\alpha x) + \A(\beta y) = \alpha \A x + \beta \A y$ \hfill $\forall x, y \in V, \forall \alpha, \beta \in \OO$
	\begin{equation*}
	\A(\alpha x + \beta y) = \alpha \A x + \beta \A y
	\end{equation*}
	Ta lastnost sledi iz pogojev (1) in (2). Iz te lastnosti lahko dobimo nazaj pogoj (1) in (2).
	\begin{equation*}
	((1) \land (2)) \iff (3)
	\end{equation*}
\end{itemize}
\textsc{Splo"sno:}
\begin{equation*}
	\A(\alpha_1 x_1 + \alpha_2 x_2 + \cdots + \alpha_n x_n) = \alpha_1 \A x_1 + \alpha_2 \A x_2 + \cdots + \alpha_n \A x_n
\end{equation*}
%
\textsc{Definicija:} $\A V \to U$ je \emph{izomorfizem} vektorskega prostora, kadar je $\A$ bijektivna in sta $\A$ in $\A^{-1}$ linearni preslikavi. \textbf{Velja:} bijektivna linearna preslikava je izomorfizem vektorskega prostora.

Naj bo $\A: V \to U$ linearna bijekcija. \dashuline{$\A^{-1}: U \to V$ je linearna}

Aditivnost sledi iz dejstva, da je $A$ izomorfizem Abelovih grup $(V, +)$, $(U, +)$.
\begin{equation*}
\A^{-1}(\alpha u) = \A^{-1}(\alpha \A v) = \A^{-1}(\A(\alpha v)) = \alpha v = \alpha \A^{-1} u
\end{equation*}
kjer upo"stevamo, da $\exists v \in V: u = \A v (v = \A^{-1}u)$

$\Rightarrow \A^{-1}$ je homogena \hfill $\square$

\textsc{Primeri:}
\begin{enumerate}[(1)]
	\item $V = U = \RR^3$
	\begin{itemize}
		\item $\A: \RR^3 \to \RR^3$ pravokotna projekcija na ravnino skozi 0.
		\item $\A: \RR^3 \to \RR^3$ zasuk za dolo"cen kot okrog dane osi skozi 0.
	\end{itemize}
	\item $\A: \RR[x] \to \RR[x]$, $\A$ odvajanje.
	\item $\A: \RR[x] \to \RR$, $\A$ je dolo"ceno integriranje.
\end{enumerate}
%
\subsubsection{Slika in jedro linearnih preslikav}
\textsc{Definicija:} Naj bo $\A: V \to U$ linearna preslikava. Definiramo:
\begin{itemize}
	\item $\im \A = \{\A x: x \in V\}$ slika preslikave $\A$
	\item $\ker \A = \{x \in V: \A x = 0\}$ jedro preslikave $\A$
\end{itemize}
\textsc{Velja:} $\im \A \leq U$ in $\ker \A \leq V$

\textsc{Dokaz:} za $\im \A$: \dashuline{$u_1, u_2 \in \im\A \Rightarrow \alpha_1 u_1 + \alpha_2 u_2 \in \im \A$}
\begin{gather*}
\exists x_1, x_2 \in V: u_1 = \A x_1, u_2 = \A x_2 \\
\alpha_1 u_1 + \alpha_2 u_2 = \alpha_1 \A x_1 + \alpha_2 \A x_2 = \A(\alpha_1 x_1 + \alpha_2 x_2) \in \im \A
\end{gather*}
%
\textsc{Definicija:} Naj bo $\A: V \to U$. Velja:
\begin{enumerate}[(1)]
	\item $\A$ je surjektivna $\iff \im \A = U$
	\item $\A$ je injektivna $\iff \ker \A = \{0\}$
\end{enumerate}
\textsc{Dokaz} za (2):
\begin{itemize}
	\item[($\Rightarrow$)] $\A$ je injektivna. Vemo $\A0 = 0$. Zanima nas, za katere $x$ velja $\A x = 0$. Ker je injektivna je $x = 0 \Rightarrow \ker A = \{0\}$.
	\item[($\Leftarrow$)] $\ker \A = \{0\}$ Naj bosta $\A x = \A y$, $x, y \in V$.
	\begin{align*}
	&\Rightarrow \underbrace{\A x - \A y}_{\A(x-y) = 0} = 0 \\
	&\Rightarrow x - y \in \ker \A = \{0\} \\
	&\Rightarrow x - y = 0 \Rightarrow x = y
	\end{align*}
\end{itemize}
\hfill $\square$

\textsc{Izrek:} Naj bo $\A: V \to U$ linearna preslikava. Potem obstaja izomorfizem med vektorskima prostoroma $V/_{\ker \A}$ in $\im \A$. Izomorfizem deluje s predpisom:
\begin{equation*}
\hat{\A}: [x] \mapsto \A x
\end{equation*}

\textsc{Dokaz:}
\begin{itemize}
	\item \dashuline{Predpis je dober} t.j: $[x] = [y] \Rightarrow \A x = \A y$.
	\begin{equation*}
	x \sim y \Rightarrow x - y \in \ker \A \Rightarrow \underbrace{\A(x-y)}_{\A x - \A y = 0} = 0 \Rightarrow \A x = \A y
	\end{equation*}

	\item \dashuline{$\hat{\A}$ je linearna}
	\begin{multline*}
	\hat{\A}(\underbrace{\alpha[x]}_{[\alpha x]} + \underbrace{\beta[y]}_{[\beta y]}) = \\
	=\hat{\A}(\alpha x + \beta y) = \A(\alpha x + \beta y) = \alpha \A x + \beta \A y =\\
	= \alpha \hat{\A}([x]) + \beta \hat{\A}([y])
	\end{multline*}
	
	\item \dashuline{$\hat{\A}$ je surjektivna} -- sledi neposredno iz definicije $\hat{\A}$
	\item \dashuline{$\hat{\A}$ je injektvina}
	\begin{gather*}
	\underbrace{\hat{\A}([x])}_{\A x} = \underbrace{\hat{\A}([y])}_{\A y} \\
	\Rightarrow \A (x-y) = \A x - \A y = 0 \\
	\Rightarrow x - y \in \ker \A \Rightarrow \\
	\Rightarrow x \sim y \Rightarrow [x] = [y]
	\end{gather*}
\end{itemize}
$\Rightarrow \hat{\A}$ je linearne in bijektivna $\Rightarrow \hat{\A}: V/_{\ker \A} \to \im \A$ je izomorfizem vektorskih prostorov. \hfill $\square$

\textsc{Posledici:} Naj bo $\A: V \to U$ linearna preslikava
\begin{enumerate}[(1)]
	\item "Ce je $\A$ surjektivna, je vektorski prostor $V/_{\ker \A}$ izomorfen $U$.
	\item "Ce je $\A$ injektivna, je vektorski prostor $V$ izomorfen vektorskemu prostoru $\im \A$
	\begin{equation*}
	\A \text{ injektivna} \Rightarrow \ker \A = \{0\} \Rightarrow V/_{\{0\}} = V
	\end{equation*}
\end{enumerate}
%
\subsection{Vektorski prostor linearnih preslikav}
$V, U$ naj bosta vektorska prostora nad komutativnim obsegom $\OO$.
\begin{equation*}
\LL(V, U) = \{\A: V \to U; \text{ $\A$ je linearna}\}
\end{equation*}
Ni"celna preslikava 0 je element te mno"zice $0 \in \LL(V, U)$.

V $\LL(V, U)$ uvedemo operavijo $+$ (se"stevanje) po to"ckah:
\begin{align*}
\A, \mathcal{B} &\in \LL(V, U) \\
(\A + \mathcal{B})(x) &= \A x + \mathcal{B} x, \forall x \in V
\end{align*}
Velja $\A + \mathcal{B} \in \LL(V, U)$. Preverimo homogenost (aditivnost za DN):
\begin{equation*}
(\A + \mathcal{B})(\alpha x) = \alpha \A x + \alpha \mathcal{B} x = \alpha (\A x + \mathcal{B} x) = \alpha ((\A + \mathcal{B})x)
\end{equation*}
\textsc{Velja:}
\begin{itemize}
	\item $(\LL(V, U), +)$ je Abelova grupa
	\item $0$ (ni"celna preslikava) je ni"celni element
	\item $\A \in \LL(V, U); -\A = -\A x \forall x \in V$
	\begin{gather*}
		(-A)x = -\A x, \forall x \in V \\
		(\A + (-A))x = \A x + (-\A) x = \A x +(-\A)x = 0 (\in U), \forall x \in V \\
		\Rightarrow \A + (-\A) = 0
	\end{gather*}
\end{itemize}
\textbf{Mno"zenje s skalarji} definiramo po to"ckah:
\begin{gather*}
	(\alpha A)x = \alpha (A x), \forall x \in V, \alpha \in \OO \\
	\A \in \LL(V, U) \Rightarrow \alpha A \in \LL(V, U)
\end{gather*}
$\LL(V, U)$ postane z obema operacijama vektorski prostor nad $\OO$.
%
\textbf{Poseben primer} $U = V$

$\LL(V, V) \equiv \LL(V)$ -- mno"zica vseh endomorfizmov vektorskega prostora $V$. V mno"zico $\LL(V)$ uvedemo "ze mno"zenje (= komponiranje preslikav).
\begin{align*}
\A, \mathcal{B} &\in \LL(V) \\
(\A \mathcal{B})x &= A(Bx), \forall x \in V
\end{align*}
Mno"zenje je operacija na $\LL(V): A, \mathcal{B} \in \LL(V) \Rightarrow \A \mathcal{B} \in \LL(V)$.