\textsc{Definicija:} \emph{Vektorski prostor} na obsegom \OO je Abelova grupa $(V, +)$ skupaj z \emph{zunanjo operacijo}
\begin{align*}
\OO \times V &\to V \\
(\alpha, v) &\mapsto \alpha v
\end{align*}
ki ustreza naslednjim pogojem:
\begin{enumerate}
	\item $(\alpha + \beta) v = \alpha v + \beta v$ \hfill $\forall \alpha, \beta \in \OO, \forall v \in V$
	\item $\alpha(u + v) = \alpha u + \alpha v$ \hfill $\forall \alpha \in \OO, \forall u, v \in V$
	\item $\alpha(\beta v) = (\alpha \beta)v$ \hfill $\forall \alpha, \beta \in \OO, \forall v \in V$
	\item $1v = v$ \hfill $\forall v \in V$
\end{enumerate}
Elemente iz \OO imenujemo \emph{skalarji}, elemente iz $V$ imenujemo \emph{vektorji}, zunanjo operacijo pa imenujemo \emph{mno"zenje z skalarji}.

\textsc{Primer:}
\begin{enumerate}[(1)]
	\item $V = \RR^3, \OO = \RR$ obi"cajen trirazse"zen vektorski prostor
	\item $V = \OO^n$, $\OO$ - obseg
	
	Naj bosta $x$ in $y$ naslednja vektorja:
	\begin{align*}
	x &= (x_1, x_2, \ldots, x_n) \in \OO^n (x_i \in \OO \forall i) \\
	y &= (y_1, y_2, \ldots, y_n) \in \OO^n (y_i \in \OO \forall i)
	\end{align*}
	Operaciji definiramo slede"ce:
	\begin{align*}
	x + y &= (x_1 + y_1, x_2 + y_2, \ldots, x_n + y_n) \in \OO^n \\
	\alpha x &= (\alpha x_1, \alpha x_2, \ldots, \alpha x_n) \in \OO^n
	\end{align*}
	Za ti dve operacijie je $\OO^n$ vektorski prostor na obsegom $\OO$. Ni"celni element je
	\begin{equation*}
	0 = (0, 0, \ldots, 0) \in \OO^n
	\end{equation*}
	Nasprotni element je:
	\begin{equation*}
	-(x_1, x_2, \ldots x_n) = (-x_1, -x_2, \ldots, -x_n) \in \OO^n
	\end{equation*}
	
	\item $M \neq \varnothing$ \qquad $\mathcal{F}(M, \RR) \equiv \RR^M = \{f: M \to \RR\}$
	
	Operaciji definiramo po to"ckah:
	\begin{align*}
	(\alpha f)(t) &= \alpha f(t) && \forall t \in M (\alpha \in \RR) \\
	(f + g)(t) &= f(t) + g(t) && \forall t \in M
	\end{align*}
	$V = \RR^M, \OO = \RR$
	
	$V$ je vekotrski prostor nad $\RR$.
\end{enumerate}
\subsection{Nekaj osnovnih lastnosti vektorskih prostorov}
Naj bo $V$ vektorski prostor nad \OO. Velja:
\begin{enumerate}[(1)]
	\item $0v = 0$ \hfill $\forall v \in V$
	\item $\alpha 0 = 0$ \hfill $\forall \alpha \in \OO$
	\item $\alpha v = 0 \Rightarrow (\alpha = 0 \lor v = 0)$
	\item $(-1)v = -v$ \hfill $\forall v \in V$
\end{enumerate}
\textsc{Dokaz:}
\begin{enumerate}[(1)]
	\item
	\begin{multline*}
		0v = x \in V \Rightarrow \\
		x + x = 0v + 0v = (0 + 0)v = 0v = x  \\
		\Rightarrow x + x = x \Rightarrow x = 0 \\
		\Rightarrow 0v = 0
	\end{multline*}
	\item Podoben dokaz kot za (1).
	\item $\alpha v = 0$. "Ce $\alpha = 0$ optem velja (2). Druga"ce:
	\begin{multline*}
		\alpha \neq 0 \Rightarrow \exists \alpha^{-1} \in \OO \Rightarrow \\
		\Rightarrow \alpha^{-1} (\alpha v) = \alpha^{-1} 0 = 0 \\
		\underbrace{(\alpha^{-1} \alpha)}_1 v = 1v = v \\
		\Rightarrow v = 0
	\end{multline*}
	\item
	\begin{equation*}
	(-1)v + v = (-1)v + 1v = (-1 + 1) v = 0v = 0
	\end{equation*}
\end{enumerate}
%
\subsection{Vektorski podprostor}
\textsc{Definicija:} Naj bo $V$ vektorski prostor nad $\OO$ in $U \subseteq V, U \neq \varnothing$. $U$ je vektorski poprostor vektorskega prostora $V$, kadar velja:
\begin{enumerate}[(1)]
	\item $x, y \in U \Rightarrow x + y \in U$
	\item $x \in U \Rightarrow \alpha x \in U$ \hfill $\forall \alpha \in \OO$
\end{enumerate}
Obe zahtevi lahko zdru"zimo v eno:
\begin{equation*}
(1) \land (2) \iff (x, y \in U \Rightarrow \forall \alpha, \beta \in U: \alpha x + \beta y \in U)
\end{equation*}
$(U, +)$ je podgrupa grupe $(V, +)$.

\textsc{Primeri:}
\begin{enumerate}[(1)]
	\item $V = \RR^3$, $U$ je ravnina skozi $0$ v $\RR^3$
	
	\item 
	\begin{align*}
	V &= \RR^3 \\
	U &= \RR[x]
	\end{align*}
	
	\item 
	\begin{align*}
	V &= \RR[x] \\
	U &= \RR_m[x] = \{p(x) \in \RR[x]: \text{stp}(x) \leq m\}
	\end{align*}
\end{enumerate}
"Ce je $V$ vektorski prostor nad $\OO$ in $U \subseteq V$ podprostor, uporabljamo oznako:
\begin{equation*}
U \leq V
\end{equation*}
Vsak podprostor vsebuje ni"clo:
\begin{equation*}
x \in U \Rightarrow 0x = 0 \in U
\end{equation*}
Nasprotni element je element podprostora:
\begin{equation*}
x, y \in U \Rightarrow x - y = x + (-1) \in U
\end{equation*}
Ker velja $\alpha x \in U$ in $\beta y \in U$, lahko zapi"semo:
\begin{equation*}
\alpha x + \beta y \in U
\end{equation*}
Zapi"semo lahko:
\begin{equation*}
x_1, x_2, \ldots, x_k \in U \Rightarrow \underbrace{\alpha x_1 + \alpha x_2 + \ldots + \alpha x_k}_{\text{\emph{linearna kombinacija vektorjev} $x_1, \ldots, x_k$}} \in U
\end{equation*}
\subsection{Linearna ogrinja"ca}
\textsc{Definicija:} Naj bo $M \in V, M \neq \varnothing$. \emph{Linearno ogrinja"ca mno"zice} $M$ je
\begin{equation*}
\Lin M = \{\alpha_1 x_1 + \ldots +\alpha_k x_k: x_1, \ldots, x_k \in M, \alpha_1, \ldots \alpha_k \in \OO, k \in \NN\}
\end{equation*}
Velja:
\begin{equation*}
M \subseteq U \leq V \Rightarrow \Lin M \subseteq U
\end{equation*}
\dashuline{$\Lin M$ je vektorski podprostor vektorskega prostora $V$ ($\Lin M \leq V$)}
\begin{itemize}
	\item Zaprtost za se"stevanje:
	\begin{gather*}
	\alpha_1 x_1 + \ldots + \alpha_k x_k \in \Lin M \\
	\beta_1 x_1 + \ldots + \beta_n y_n \in \Lin M
	\end{gather*}
	Opazimo, da so po definiciji $\Lin M$ posame"cni "cleni $\alpha_1 x_1, \ldots \alpha_k x_k \in \Lin M$ in $\beta_1 y_1, \ldots, \beta_n y_n \in \Lin M$, torej je tudi vsota vseh "clenov $\in \Lin M$.
	
	\item Zaprtost za mno"zenje s skalarjem:
	\begin{gather*}
		\beta(\alpha_1 x_1 + \ldots + \alpha_k x_k) = (\beta \alpha_1) x_1 + \ldots + (\beta \alpha_k)x_k \in \Lin M \\
		x_1, \ldots, x_k \in M
	\end{gather*}
	\hfill $\square$
\end{itemize}
Iz tega sledi, da je $\Lin M$ najmanj"si vektorski podprostor, ki vsebuje $M$. Simbolno za malo naprednej"se:
\begin{equation*}
M \subseteq U \leq V \Rightarrow \Lin M \subseteq U
\end{equation*}
Za prazno mno"zico velja:
\begin{equation*}
\Lin \varnothing = \{0\}
\end{equation*}
Poglejmo si, kako je s preseki in unijami. Za preseke velja:
\begin{equation*}
V_i \leq V \forall i \in I \Rightarrow \bigcap_{i \in I}V_i \leq V
\end{equation*}
To je o"citno. Zaprtost za se"stevanje velja, ker "ce sta neka dva vektorja $x, y$ v $\bigcap_{i \in I}V_i$, potem se nahajata v vseh $V_i$. Ker so $V_i$ vektorski podprostori, v njih tudi velja zaprtost za se"stevanje. Zato je vsota $x + y$ tudi v vseh $V_i$, torej je tudi v $\bigcap_{i \in I}V_i$. Podobno lahko naredimo za zaprtost za mno"zenje s skalarjem.

Malo ve"c je za videti pri uniji. $V_1, V_2 \leq V \Rightarrow \Lin (V_1 \cup V_2)$ je najmanj"si vektorski podprostor, ki vsebuje $V_1$ in $V_2$. Primer na katerem se lahko predstavljamo, sta dve premici. Unija dveh premic, ki se sekata ni vektorski podrpostor, zato okoli naredimo linearno ogrinja"co. Poglejmo si eno zanimivost:
\begin{gather*}
x \in \Lin (V_1 \cup V_2) \\
x = \underbrace{\alpha_1 x_1 + \ldots \alpha_k x_k}_{\in V_1} + \underbrace{\beta_1 y_1 + \ldots + \beta_n y_n}_{\in V_2} = u + v
\end{gather*}
Torej velja:
\begin{equation*}
x \in \Lin (V_1 \cup V_2) \iff x = u + v, u \in V_1, v \in V_2
\end{equation*}
Zapi"semo:
\begin{equation*}
V_1 + V_2 = \{u + v: u \in V_1, v \in V_2\}
\end{equation*}
Torej velja:
\begin{equation*}
\Lin (V_1 \cup V_2) = V_1 + V_2
\end{equation*}
Analogno naredimo za ve"c sumandov:
\begin{gather*}
	\Lin (V_1 \cup V_2 \cup \ldots \cup V_k) = V_1 + V_2 + \ldots + V_k \\
	V_i \leq V \forall i \\
	V_1 + \ldots + V_k = \{x_1 + \ldots + x_k: x_i \in V_i \forall i\}
\end{gather*}
\textsc{Definicija:} $V_1 + \ldots + V_k$ je \emph{prema} ali \emph{direktna}, kadar za vsak $x \in V_1 + \ldots + V_k$ obstajajo in so z $x$ enoi"cno dolo"ceni taki vektorji $x_i \in V_i (i = 1, \ldots, k)$, da je $x = x_1 + \ldots + x_k$. Ozani"cimo:
\begin{equation*}
V_1 \oplus \ldots \oplus V_k
\end{equation*}