\textsc{Definicija:} \emph{Vektorski prostor} na obsegom \OO je Abelova grupa $(V, +)$ skupaj z \emph{zunanjo operacijo}
\begin{align*}
\OO \times V &\to V \\
(\alpha, v) &\mapsto \alpha v
\end{align*}
ki ustreza naslednjim pogojem:
\begin{enumerate}
	\item $(\alpha + \beta) v = \alpha v + \beta v$ \hfill $\forall \alpha, \beta \in \OO, \forall v \in V$
	\item $\alpha(u + v) = \alpha u + \alpha v$ \hfill $\forall \alpha \in \OO, \forall u, v \in V$
	\item $\alpha(\beta v) = (\alpha \beta)v$ \hfill $\forall \alpha, \beta \in \OO, \forall v \in V$
	\item $1v = v$ \hfill $\forall v \in V$
\end{enumerate}
Elemente iz \OO imenujemo \emph{skalarji}, elemente iz $V$ imenujemo \emph{vektorji}, zunanjo operacijo pa imenujemo \emph{mno"zenje z skalarji}.

\textsc{Primer:}
\begin{enumerate}[(1)]
	\item $V = \RR^3, \OO = \RR$ obi"cajen trirazse"zen vektorski prostor
	\item $V = \OO^n$, $\OO$ - obseg
	
	Naj bosta $x$ in $y$ naslednja vektorja:
	\begin{align*}
	x &= (x_1, x_2, \ldots, x_n) \in \OO^n (x_i \in \OO \forall i) \\
	y &= (y_1, y_2, \ldots, y_n) \in \OO^n (y_i \in \OO \forall i)
	\end{align*}
	Operaciji definiramo slede"ce:
	\begin{align*}
	x + y &= (x_1 + y_1, x_2 + y_2, \ldots, x_n + y_n) \in \OO^n \\
	\alpha x &= (\alpha x_1, \alpha x_2, \ldots, \alpha x_n) \in \OO^n
	\end{align*}
	Za ti dve operacijie je $\OO^n$ vektorski prostor na obsegom $\OO$. Ni"celni element je
	\begin{equation*}
	0 = (0, 0, \ldots, 0) \in \OO^n
	\end{equation*}
	Nasprotni element je:
	\begin{equation*}
	-(x_1, x_2, \ldots x_n) = (-x_1, -x_2, \ldots, -x_n) \in \OO^n
	\end{equation*}
	
	\item $M \neq \varnothing$ \qquad $\mathcal{F}(M, \RR) \equiv \RR^M = \{f: M \to \RR\}$
	
	Operaciji definiramo po to"ckah:
	\begin{align*}
	(\alpha f)(t) &= \alpha f(t) && \forall t \in M (\alpha \in \RR) \\
	(f + g)(t) &= f(t) + g(t) && \forall t \in M
	\end{align*}
	$V = \RR^M, \OO = \RR$
	
	$V$ je vekotrski prostor nad $\RR$.
\end{enumerate}
\subsection{Nekaj osnovnih lastnosti vektorskih prostorov}
Naj bo $V$ vektorski prostor nad \OO. Velja:
\begin{enumerate}[(1)]
	\item $0v = 0$ \hfill $\forall v \in V$
	\item $\alpha 0 = 0$ \hfill $\forall \alpha \in \OO$
	\item $\alpha v = 0 \Rightarrow (\alpha = 0 \lor v = 0)$
	\item $(-1)v = -v$ \hfill $\forall v \in V$
\end{enumerate}
\textsc{Dokaz:}
\begin{enumerate}[(1)]
	\item
	\begin{multline*}
		0v = x \in V \Rightarrow \\
		x + x = 0v + 0v = (0 + 0)v = 0v = x  \\
		\Rightarrow x + x = x \Rightarrow x = 0 \\
		\Rightarrow 0v = 0
	\end{multline*}
	\item Podoben dokaz kot za (1).
	\item $\alpha v = 0$. "Ce $\alpha = 0$ optem velja (2). Druga"ce:
	\begin{multline*}
		\alpha \neq 0 \Rightarrow \exists \alpha^{-1} \in \OO \Rightarrow \\
		\Rightarrow \alpha^{-1} (\alpha v) = \alpha^{-1} 0 = 0 \\
		\underbrace{(\alpha^{-1} \alpha)}_1 v = 1v = v \\
		\Rightarrow v = 0
	\end{multline*}
	\item
	\begin{equation*}
	(-1)v + v = (-1)v + 1v = (-1 + 1) v = 0v = 0
	\end{equation*}
\end{enumerate}
%
\subsection{Vektorski podprostor}
\textsc{Definicija:} Naj bo $V$ vektorski prostor nad $\OO$ in $U \subseteq V, U \neq \varnothing$. $U$ je vektorski poprostor vektorskega prostora $V$, kadar velja:
\begin{enumerate}[(1)]
	\item $x, y \in U \Rightarrow x + y \in U$
	\item $x \in U \Rightarrow \alpha x \in U$ \hfill $\forall \alpha \in \OO$
\end{enumerate}
Obe zahtevi lahko zdru"zimo v eno:
\begin{equation*}
(1) \land (2) \iff (x, y \in U \Rightarrow \forall \alpha, \beta \in U: \alpha x + \beta y \in U)
\end{equation*}
$(U, +)$ je podgrupa grupe $(V, +)$.

\textsc{Primeri:}
\begin{enumerate}[(1)]
	\item $V = \RR^3$, $U$ je ravnina skozi $0$ v $\RR^3$
	
	\item 
	\begin{align*}
	V &= \RR^3 \\
	U &= \RR[x]
	\end{align*}
	
	\item 
	\begin{align*}
	V &= \RR[x] \\
	U &= \RR_m[x] = \{p(x) \in \RR[x]: \text{stp}(x) \leq m\}
	\end{align*}
\end{enumerate}
"Ce je $V$ vektorski prostor nad $\OO$ in $U \subseteq V$ podprostor, uporabljamo oznako:
\begin{equation*}
U \leq V
\end{equation*}
Vsak podprostor vsebuje ni"clo:
\begin{equation*}
x \in U \Rightarrow 0x = 0 \in U
\end{equation*}
Nasprotni element je element podprostora:
\begin{equation*}
x, y \in U \Rightarrow x - y = x + (-1) \in U
\end{equation*}
Ker velja $\alpha x \in U$ in $\beta y \in U$, lahko zapi"semo:
\begin{equation*}
\alpha x + \beta y \in U
\end{equation*}
Zapi"semo lahko:
\begin{equation*}
x_1, x_2, \ldots, x_k \in U \Rightarrow \underbrace{\alpha x_1 + \alpha x_2 + \ldots + \alpha x_k}_{\text{\emph{linearna kombinacija vektorjev} $x_1, \ldots, x_k$}} \in U
\end{equation*}
\subsection{Linearna ogrinja"ca}
\textsc{Definicija:} Naj bo $M \in V, M \neq \varnothing$. \emph{Linearno ogrinja"ca mno"zice} $M$ je
\begin{equation*}
\Lin M = \{\alpha_1 x_1 + \ldots +\alpha_k x_k: x_1, \ldots, x_k \in M, \alpha_1, \ldots \alpha_k \in \OO, k \in \NN\}
\end{equation*}
Velja:
\begin{equation*}
M \subseteq U \leq V \Rightarrow \Lin M \subseteq U
\end{equation*}
\dashuline{$\Lin M$ je vektorski podprostor vektorskega prostora $V$ ($\Lin M \leq V$)}
\begin{itemize}
	\item Zaprtost za se"stevanje:
	\begin{gather*}
	\alpha_1 x_1 + \ldots + \alpha_k x_k \in \Lin M \\
	\beta_1 x_1 + \ldots + \beta_n y_n \in \Lin M
	\end{gather*}
	Opazimo, da so po definiciji $\Lin M$ posame"cni "cleni $\alpha_1 x_1, \ldots \alpha_k x_k \in \Lin M$ in $\beta_1 y_1, \ldots, \beta_n y_n \in \Lin M$, torej je tudi vsota vseh "clenov $\in \Lin M$.
	
	\item Zaprtost za mno"zenje s skalarjem:
	\begin{gather*}
		\beta(\alpha_1 x_1 + \ldots + \alpha_k x_k) = (\beta \alpha_1) x_1 + \ldots + (\beta \alpha_k)x_k \in \Lin M \\
		x_1, \ldots, x_k \in M
	\end{gather*}
	\hfill $\square$
\end{itemize}
Iz tega sledi, da je $\Lin M$ najmanj"si vektorski podprostor, ki vsebuje $M$. Simbolno za malo naprednej"se:
\begin{equation*}
M \subseteq U \leq V \Rightarrow \Lin M \subseteq U
\end{equation*}
Za prazno mno"zico velja:
\begin{equation*}
\Lin \varnothing = \{0\}
\end{equation*}
Poglejmo si, kako je s preseki in unijami. Za preseke velja:
\begin{equation*}
V_i \leq V \forall i \in I \Rightarrow \bigcap_{i \in I}V_i \leq V
\end{equation*}
To je o"citno. Zaprtost za se"stevanje velja, ker "ce sta neka dva vektorja $x, y$ v $\bigcap_{i \in I}V_i$, potem se nahajata v vseh $V_i$. Ker so $V_i$ vektorski podprostori, v njih tudi velja zaprtost za se"stevanje. Zato je vsota $x + y$ tudi v vseh $V_i$, torej je tudi v $\bigcap_{i \in I}V_i$. Podobno lahko naredimo za zaprtost za mno"zenje s skalarjem.

Malo ve"c je za videti pri uniji. $V_1, V_2 \leq V \Rightarrow \Lin (V_1 \cup V_2)$ je najmanj"si vektorski podprostor, ki vsebuje $V_1$ in $V_2$. Primer na katerem se lahko predstavljamo, sta dve premici. Unija dveh premic, ki se sekata ni vektorski podrpostor, zato okoli naredimo linearno ogrinja"co. Poglejmo si eno zanimivost:
\begin{gather*}
x \in \Lin (V_1 \cup V_2) \\
x = \underbrace{\alpha_1 x_1 + \ldots \alpha_k x_k}_{\in V_1} + \underbrace{\beta_1 y_1 + \ldots + \beta_n y_n}_{\in V_2} = u + v
\end{gather*}
Torej velja:
\begin{equation*}
x \in \Lin (V_1 \cup V_2) \iff x = u + v, u \in V_1, v \in V_2
\end{equation*}
Zapi"semo:
\begin{equation*}
V_1 + V_2 = \{u + v: u \in V_1, v \in V_2\}
\end{equation*}
Torej velja:
\begin{equation*}
\Lin (V_1 \cup V_2) = V_1 + V_2
\end{equation*}
Analogno naredimo za ve"c sumandov:
\begin{gather*}
	\Lin (V_1 \cup V_2 \cup \ldots \cup V_k) = V_1 + V_2 + \ldots + V_k \\
	V_i \leq V \forall i \\
	V_1 + \ldots + V_k = \{x_1 + \ldots + x_k: x_i \in V_i \forall i\}
\end{gather*}
\textsc{Definicija:} $V_1 + \ldots + V_k$ je \emph{prema} ali \emph{direktna}, kadar za vsak $x \in V_1 + \ldots + V_k$ obstajajo in so z $x$ enoi"cno dolo"ceni taki vektorji $x_i \in V_i (i = 1, \ldots, k)$, da je $x = x_1 + \ldots + x_k$. Ozani"cimo:
\begin{equation*}
V_1 \oplus \ldots \oplus V_k
\end{equation*}
\textsc{Trditev:} Vsota $V_1 + V_2$ vektorskih podprostorov $V_1$ in $V_2$ je direktna natanko takrat, kadar je $V_1 \cup V_2 = \{0\}$.

\textsc{Dokaz:}
\begin{itemize}
	\item[($\Rightarrow$)] Naj bo vsota $V_1 + V_2$ direktna ($V_1 \oplus V_2$). Vzemimo $x \in V_1 \cup V_2$.
	\begin{equation*}
	x = \underbrace{x}_{\in V_1} + \underbrace{0}_{\in V_2} = \underbrace{0}_{\in V_1} + \underbrace{x}_{\in V_2} \Rightarrow x = 0
	\end{equation*}
	$\Rightarrow V_1 \cup V_2 = \{0\}$
	
	\item[($\Leftarrow$)] Naj bo $V_1 \cup V_2 = \{0\}$.
	\begin{align*}
	x &\in V_1 + V_2 \\
	x &= x_1 + x_2, x_1 \in V_1, x_2 \in V_2 \\
	x &= x_1' + x_2', x_1' \in V_1, x_2' \in V_2 \\
	x_1 + x_2 &= x_1' + x_2' \\
	\underbrace{x_1 - x_1'}_{\in V_1} &= \underbrace{x_2 - x_2'}_{\in V_2} = z
	\end{align*}
	\begin{align*}
	&\Rightarrow z \in V_1 \cap V_2 = \{0\} \\
	&\Rightarrow x = 0 \Rightarrow \\
	&\Rightarrow x_1' = x_1 \land x_2' = x_2
	\end{align*}
	\begin{equation*}
	V_1 \oplus V_2
	\end{equation*}
	\hfill $\square$
\end{itemize}

\subsection{Kvocientni vektorski prostor}
Naj bo $U$ vektorski prostor nad $\OO$, $U \leq V$. Definiramo:
\begin{equation*}
v_1 \sim v_2 \iff v_1 - v_2 \in U
\end{equation*}
kjer je $\sim$ ekvivalen"cna relacija. $U$ je Abelova podgrupa Abelove grupe $V$. $V/_U$ je torej Abelova grupa in velja:
\begin{align*}
[x] + [y] &= [x+y] \forall x, y \in V \\
[z] &= z + U \forall z \in V
\end{align*}
V $V/_U$ uvedemo mno"zenje s skalarji:
\begin{equation*}
\alpha [x] := [\alpha x], \alpha \in \OO, x \in V
\end{equation*}
\dashuline{Definicija je dobra} "ce velja:
\begin{align*}
y \sim x &\Rightarrow \alpha x \sim \alpha y \\
y -x \in U &\Rightarrow \underbrace{\alpha y - \alpha x}_{\alpha (y - x) = z} \in U
\end{align*}
Ker je $U$ podprostor zaprt za mno"zenje s skalarjem, vemo:
\begin{equation*}
z \in U \Rightarrow \alpha z \in U \forall \alpha \in \OO
\end{equation*}
\hfill $\square$

Sledi, da je $V/_U$ vektorski prostor nad $\OO$. Elementi so $x + U, x \in V$.

\textsc{Primer:} $U$ premica skozi 0 v $V = \RR^3$. Elementi $V/_U: x + U, x \in \RR^3$ so premice vzporedne premici $U$.
%
\subsection{Linearne preslikave}
So neke vrste homomorfizmi vektorskih prostorov.

\textsc{Definicija:} Naj bosta $V$ in $U$ vektorska prostora nad istim $\OO$. Preslikava $\A : V \to U$ je \emph{linearna} (= homomorfizem vektorskih prostorov), kadar velja:
\begin{enumerate}[(1)]
	\item $\A(x + y) = \A x + \A y$ \hfill $\forall x, y \in V$
	\item $\A(\alpha x) = \alpha \A x$ \hfill $\forall \alpha \in \OO, \forall x \in V$
\end{enumerate}
Pogoju (1) pravimo, da je $\A$ \emph{aditivna}, pogoju (2) pa pravimo, da je $\A$ \emph{homogena}.
\subsubsection*{Nekaj lastnosti:}
\begin{itemize}
	\item $\A0 = 0$ (pride iz Abelove grupe)
	\item $\A (-x) =  -\A x$ (pride iz Abelove grupe) \hfill $\forall x \in V$
	\item $\A(x-y) = \A x - \A y$ \hfill $\forall x, y \in V$
	\item[(3)] $\A(\alpha x + \beta y) = \A(\alpha x) + \A(\beta y) = \alpha \A x + \beta \A y$ \hfill $\forall x, y \in V, \forall \alpha, \beta \in \OO$
	\begin{equation*}
	\A(\alpha x + \beta y) = \alpha \A x + \beta \A y
	\end{equation*}
	Ta lastnost sledi iz pogojev (1) in (2). Iz te lastnosti lahko dobimo nazaj pogoj (1) in (2).
	\begin{equation*}
	((1) \land (2)) \iff (3)
	\end{equation*}
\end{itemize}
\textsc{Splo"sno:}
\begin{equation*}
	\A(\alpha_1 x_1 + \alpha_2 x_2 + \cdots + \alpha_n x_n) = \alpha_1 \A x_1 + \alpha_2 \A x_2 + \cdots + \alpha_n \A x_n
\end{equation*}
%
\textsc{Definicija:} $\A V \to U$ je \emph{izomorfizem} vektorskega prostora, kadar je $\A$ bijektivna in sta $\A$ in $\A^{-1}$ linearni preslikavi. \textbf{Velja:} bijektivna linearna preslikava je izomorfizem vektorskega prostora.

Naj bo $\A: V \to U$ linearna bijekcija. \dashuline{$\A^{-1}: U \to V$ je linearna}

Aditivnost sledi iz dejstva, da je $A$ izomorfizem Abelovih grup $(V, +)$, $(U, +)$.
\begin{equation*}
\A^{-1}(\alpha u) = \A^{-1}(\alpha \A v) = \A^{-1}(\A(\alpha v)) = \alpha v = \alpha \A^{-1} u
\end{equation*}
kjer upo"stevamo, da $\exists v \in V: u = \A v (v = \A^{-1}u)$

$\Rightarrow \A^{-1}$ je homogena \hfill $\square$

\textsc{Primeri:}
\begin{enumerate}[(1)]
	\item $V = U = \RR^3$
	\begin{itemize}
		\item $\A: \RR^3 \to \RR^3$ pravokotna projekcija na ravnino skozi 0.
		\item $\A: \RR^3 \to \RR^3$ zasuk za dolo"cen kot okrog dane osi skozi 0.
	\end{itemize}
	\item $\A: \RR[x] \to \RR[x]$, $\A$ odvajanje.
	\item $\A: \RR[x] \to \RR$, $\A$ je dolo"ceno integriranje.
\end{enumerate}
%
\subsubsection{Slika in jedro linearnih preslikav}
\textsc{Definicija:} Naj bo $\A: V \to U$ linearna preslikava. Definiramo:
\begin{itemize}
	\item $\im \A = \{\A x: x \in V\}$ slika preslikave $\A$
	\item $\ker \A = \{x \in V: \A x = 0\}$ jedro preslikave $\A$
\end{itemize}
\textsc{Velja:} $\im \A \leq U$ in $\ker \A \leq V$

\textsc{Dokaz:} za $\im \A$: \dashuline{$u_1, u_2 \in \im\A \Rightarrow \alpha_1 u_1 + \alpha_2 u_2 \in \im \A$}
\begin{gather*}
\exists x_1, x_2 \in V: u_1 = \A x_1, u_2 = \A x_2 \\
\alpha_1 u_1 + \alpha_2 u_2 = \alpha_1 \A x_1 + \alpha_2 \A x_2 = \A(\alpha_1 x_1 + \alpha_2 x_2) \in \im \A
\end{gather*}
%
\textsc{Definicija:} Naj bo $\A: V \to U$. Velja:
\begin{enumerate}[(1)]
	\item $\A$ je surjektivna $\iff \im \A = U$
	\item $\A$ je injektivna $\iff \ker \A = \{0\}$
\end{enumerate}
\textsc{Dokaz} za (2):
\begin{itemize}
	\item[($\Rightarrow$)] $\A$ je injektivna. Vemo $\A0 = 0$. Zanima nas, za katere $x$ velja $\A x = 0$. Ker je injektivna je $x = 0 \Rightarrow \ker A = \{0\}$.
	\item[($\Leftarrow$)] $\ker \A = \{0\}$ Naj bosta $\A x = \A y$, $x, y \in V$.
	\begin{align*}
	&\Rightarrow \underbrace{\A x - \A y}_{\A(x-y) = 0} = 0 \\
	&\Rightarrow x - y \in \ker \A = \{0\} \\
	&\Rightarrow x - y = 0 \Rightarrow x = y
	\end{align*}
\end{itemize}
\hfill $\square$

\textsc{Izrek:} Naj bo $\A: V \to U$ linearna preslikava. Potem obstaja izomorfizem med vektorskima prostoroma $V/_{\ker \A}$ in $\im \A$. Izomorfizem deluje s predpisom:
\begin{equation*}
\hat{\A}: [x] \mapsto \A x
\end{equation*}

\textsc{Dokaz:}
\begin{itemize}
	\item \dashuline{Predpis je dober} t.j: $[x] = [y] \Rightarrow \A x = \A y$.
	\begin{equation*}
	x \sim y \Rightarrow x - y \in \ker \A \Rightarrow \underbrace{\A(x-y)}_{\A x - \A y = 0} = 0 \Rightarrow \A x = \A y
	\end{equation*}

	\item \dashuline{$\hat{\A}$ je linearna}
	\begin{multline*}
	\hat{\A}(\underbrace{\alpha[x]}_{[\alpha x]} + \underbrace{\beta[y]}_{[\beta y]}) = \\
	=\hat{\A}(\alpha x + \beta y) = \A(\alpha x + \beta y) = \alpha \A x + \beta \A y =\\
	= \alpha \hat{\A}([x]) + \beta \hat{\A}([y])
	\end{multline*}
	
	\item \dashuline{$\hat{\A}$ je surjektivna} -- sledi neposredno iz definicije $\hat{\A}$
	\item \dashuline{$\hat{\A}$ je injektvina}
	\begin{gather*}
	\underbrace{\hat{\A}([x])}_{\A x} = \underbrace{\hat{\A}([y])}_{\A y} \\
	\Rightarrow \A (x-y) = \A x - \A y = 0 \\
	\Rightarrow x - y \in \ker \A \Rightarrow \\
	\Rightarrow x \sim y \Rightarrow [x] = [y]
	\end{gather*}
\end{itemize}
$\Rightarrow \hat{\A}$ je linearne in bijektivna $\Rightarrow \hat{\A}: V/_{\ker \A} \to \im \A$ je izomorfizem vektorskih prostorov. \hfill $\square$

\textsc{Posledici:} Naj bo $\A: V \to U$ linearna preslikava
\begin{enumerate}[(1)]
	\item "Ce je $\A$ surjektivna, je vektorski prostor $V/_{\ker \A}$ izomorfen $U$.
	\item "Ce je $\A$ injektivna, je vektorski prostor $V$ izomorfen vektorskemu prostoru $\im \A$
	\begin{equation*}
	\A \text{ injektivna} \Rightarrow \ker \A = \{0\} \Rightarrow V/_{\{0\}} = V
	\end{equation*}
\end{enumerate}
%
\subsection{Vektorski prostor linearnih preslikav}
$V, U$ naj bosta vektorska prostora nad komutativnim obsegom $\OO$.
\begin{equation*}
\LL(V, U) = \{\A: V \to U; \text{ $\A$ je linearna}\}
\end{equation*}
Ni"celna preslikava 0 je element te mno"zice $0 \in \LL(V, U)$.

V $\LL(V, U)$ uvedemo operavijo $+$ (se"stevanje) po to"ckah:
\begin{align*}
\A, \mathcal{B} &\in \LL(V, U) \\
(\A + \mathcal{B})(x) &= \A x + \mathcal{B} x, \forall x \in V
\end{align*}
Velja $\A + \mathcal{B} \in \LL(V, U)$. Preverimo homogenost (aditivnost za DN):
\begin{equation*}
(\A + \mathcal{B})(\alpha x) = \alpha \A x + \alpha \mathcal{B} x = \alpha (\A x + \mathcal{B} x) = \alpha ((\A + \mathcal{B})x)
\end{equation*}
\textsc{Velja:}
\begin{itemize}
	\item $(\LL(V, U), +)$ je Abelova grupa
	\item $0$ (ni"celna preslikava) je ni"celni element
	\item $\A \in \LL(V, U); -\A = -\A x \forall x \in V$
	\begin{gather*}
		(-A)x = -\A x, \forall x \in V \\
		(\A + (-A))x = \A x + (-\A) x = \A x +(-\A)x = 0 (\in U), \forall x \in V \\
		\Rightarrow \A + (-\A) = 0
	\end{gather*}
\end{itemize}
\textbf{Mno"zenje s skalarji} definiramo po to"ckah:
\begin{gather*}
	(\alpha A)x = \alpha (A x), \forall x \in V, \alpha \in \OO \\
	\A \in \LL(V, U) \Rightarrow \alpha A \in \LL(V, U)
\end{gather*}
$\LL(V, U)$ postane z obema operacijama vektorski prostor nad $\OO$.
%
\textbf{Poseben primer} $U = V$

$\LL(V, V) \equiv \LL(V)$ -- mno"zica vseh endomorfizmov vektorskega prostora $V$. V mno"zico $\LL(V)$ uvedemo "ze mno"zenje (= komponiranje preslikav).
\begin{align*}
\A, \mathcal{B} &\in \LL(V) \\
(\A \mathcal{B})x &= A(Bx), \forall x \in V
\end{align*}
Mno"zenje je operacija na $\LL(V): A, \mathcal{B} \in \LL(V) \Rightarrow \A \mathcal{B} \in \LL(V)$.

$(\LL(V), \cdot)$ je polgrupa (mno"zenje je asociativno) in velja
\begin{itemize}
	\item $\A(\mathcal{B} + \mathcal{C}) = \A \mathcal{B} + \A \mathcal{C}$
	\item $(\mathcal{B} + \mathcal{C}) \A = \mathcal{B} \A + \mathcal{C} \A$
\end{itemize}
$(\LL(V), +, \cdot)$ je kolobar. Velja "se:
\begin{equation*}
(\alpha \A) (\beta \mathcal{B}) = (\alpha \beta) (\A \mathcal{B})
\end{equation*}
Pravimo, da je $\LL(V)$ \emph{algebra} nad $\OO$.

\textsc{Definicija:} $\A$ je \emph{algebra} nad komutativnim obsegom $\OO$, kadar je $\A$ vektorski prostor nad $\OO$, v katerem je dano mno"zenje
\begin{equation*}
\A \times \A \to \A \quad ((a, b) \mapsto ab)
\end{equation*}
ki ustreza pogojem:
\begin{itemize}
	\item $(\A, +, \cdot)$ je kolobar
	\item $(\alpha a) (\beta b) = (\alpha \beta) (ab)$ \hfill $\forall \alpha, \beta \in \OO, \quad \forall a, b, \in \A$
\end{itemize}
\textsc{Primeri:}
\begin{enumerate}[(1)]
	\item $\LL(V)$ je algebra
	\item $(\RR^M) \equiv \mathcal{F}(M, \RR)$ za operacije definirane po to"ckah je algebra
	\item $\RR[x]$ algebra polinomov z realnimi koeficienti, kjer so operacije definirane po to"ckah
\end{enumerate}
$id_V \in \LL(V)$ je enota algebre $\LL(V)$
\begin{equation*}
id_V (x) = x \forall x \in V
\end{equation*}
%
\subsection{Kon"cno razse"zni vektorski prostori}
\textsc{Definicija:} Naj bo $V$ vektorski prostor nad $\OO$ in $M \subseteq V$. $M$ je \emph{ogrodje} vektorskega prostora $V$, kadar velja $\Lin M = V$

$M \neq \varnothing$ je ogrodje vektorskega prostora $V$, kadar za vsak $x \in V$ velja
\begin{equation*}
\exists v_1, \dots v_m \in M, \alpha_1, \ldots, \alpha_m \in \OO: x = \alpha_1 v_1 + \cdots + \alpha_m v_m
\end{equation*}
\textsc{Definicija:} Vektorski prostor $V$ je \emph{kon"cno razse"zen}, kadar ima kak"sno kon"cno ogrodje.
\begin{gather*}
M = \{v_1, \ldots v_m\} \text{ ogrodje v.p. $V$} \\
x \in V \Rightarrow x = \alpha_1v_1 + \cdots + \alpha_m v_m, \quad \alpha_1,\ldots \alpha_m \in \OO
\end{gather*}
Poglejmo si kako je z enoli"snostjo zapisa. Naj bo
\begin{align*}
0 &= 0v_1 + 0v_2 + \cdots + 0v_m \\
0 &= \alpha_1 v_1 + \alpha_2 v_2 + \cdots + \alpha_m v_m
\end{align*}
"Ce je zapis enoli"cen, velja sklep
\begin{equation*}
\alpha_1v_1 + \cdots + \alpha_m v_m = 0 \Rightarrow \alpha_1 = \cdots = \alpha_m = 0
\end{equation*}
Poglejmo si "se, kako je v obratno smer. Naj velja prej"snji sklep
\begin{gather*}
\begin{aligned}
x &= \alpha_1 v_1 + \cdots + \alpha_m v_m \\
x &= \beta_1 v_1 + \cdots + \beta_m v_m
\end{aligned}\\
\begin{aligned}
&\Rightarrow (\alpha_1 - \beta_1)v_1 + \cdots + (\alpha_m - \beta_m)v_m = 0 \\
&\Rightarrow \alpha_1 -\beta_1 = \cdots = \alpha_m - \beta_m = 0 \\
&\Rightarrow \beta_j = \alpha_j \quad \forall j = 1, \ldots, m
\end{aligned}
\end{gather*}
Torej velja enoli"cnost zapisa.

\textsc{Definicija:} Vektorji $v_1, \ldots v_m$ so \emph{linearno neodvisni}, kadar velja sklep
\begin{equation*}
\alpha_1 v_1 + \cdots + \alpha_m v_m = 0 \Rightarrow \alpha_1 = \cdots \alpha_m = 0
\end{equation*}

"Ce je $M = \{v_1, \ldots, v_m\}$ ogrodje vektorskega prostora $V$, potem vsak $x \in V$ lahko zapi"semo v obliki $x = \alpha_1 v_1 + \cdots + \alpha_m v_m$, pri "cemer so $\alpha_1, \ldots, \alpha_m$ enoli"cno dolo"ceni z $x$ natanko takrat, kadar so $v_1, \ldots v_m$ linearno neodvisni.

"Ce so $v_1, \ldots, v_m$ linearno neodvisni, potem so razli"cni ($i \neq j \Rightarrow v_i \neq v_j$). Naj bo $v_1 = v_2$. Zapi"semo lahko:
\begin{equation*}
\underbrace{1}_{\neq 0}v_1 + \underbrace{(-1)}_{\neq 0}v_2 + 0v_3 + \cdots + 0v_m = 0
\end{equation*}
$\Rightarrow$ vektorji niso neodvisni.

\textsc{Definicija:} Naj bo $M \subseteq V$. $M$ je linearno neodvisna, kadar je vsaka njena kon"cna podmno"zica linearno neodvisna.

\textsc{Definicija:} Naj bo $M \subseteq V$. $M$ je \emph{baza} vektorskega prostora $V$, kadar je linearno neodvisna in hkrati ogrodje vektorskega prostora $V$.

\textsc{Primeri:}
\begin{enumerate}[1)]
	\item Baze v $\RR^3$ so oblike $\{\vec{a}, \vec{b}, \vec{c}\}$, kjer so $\vec{a}, \vec{b}, \vec{c} \in \RR^3$ linearno neodvisni.
	\item $V = \OO^n$
	\begin{equation*}
	e_j  (0, \ldots, 0, \underbrace{1}_{\text{$j$-to mesto}}, 0, \ldots, 0) \in \OO^n
	\end{equation*}
	$\{e_1, e_2, \ldots e_n\}$ je \emph{standardna baza} $\OO^n$.
	\begin{equation*}
	x = (\alpha_1, \ldots, \alpha_n) \in \OO^n \Rightarrow x = \alpha_1 e_1 + \cdots + \alpha_n e_n
	\end{equation*}
	
	\item $V = \RR[x]$\\
	\begin{gather*}
		p(x) \in \RR[x] \\
		p(x) = a_0 + a_1 x + \cdots + a_n x^n
	\end{gather*}
	Baza tega prostora je
	\begin{equation*}
	\{p_j(x) = x^j: j = 0, 1, \ldots\} = \{1, x, x^2, x^3, \ldots\}
	\end{equation*}
\end{enumerate}
%
\textsc{Definicija:} Vektorji $v_1, \ldots v_m$ so \emph{linearno odvisni}, kadar niso linearno neodvisni.

Naj bodo $v_1, \ldots v_m$ linearno odvisni ($m > 1$). Potem obstajajo tudi skalarji $\alpha_1, \ldots \alpha_m \in \OO$, da niso vsi enako 0, vendar pa je
\begin{equation*}
\alpha_1 v_1 + \cdots + \alpha_m v_m = 0
\end{equation*}
Recimo, da $\alpha_1 \neq 0$, Potem je
\begin{equation*}
v_1 = \underbrace{(-\alpha_1^{-1} \alpha_2)}_{\beta_2}v_2 + \cdots + \underbrace{(-\alpha_1^{-1} \alpha_m)}_{\beta_m}v_m
\end{equation*}
$v_1$ je linearna kombinacija elemetnov $v_2, \ldots v_m$. 

Potem obstaja tak $j \in \{j, \ldots, m\}$, da je $v_j$ linearna kombinacija vektorjev
\begin{equation*}
v_1, \ldots v_{j-1}, v_{j+1}, \ldots v_m
\end{equation*}
\textbf{Obratno:} "Ce velja prej"snja trditev, potem so $v_1, \ldots v_m$ linearno odvisni
\begin{gather*}
v_1 = \beta_2 v_2 + \cdots + \beta_m v_m \\
1 v_1 + (-\beta_2)v_2 + \cdots + (-\beta_m)v_m = 0
\end{gather*}
%
\textsc{Trditev:} Naj bodo $v_1, \ldots v_m$ linearno odvisni in $v_1 \neq 0$, $m > 1$. Potem obstaja tak $k > 1$, $k \leq m$, da je $v_k$ linearna kombinacija vektorjev $v_1, \ldots v_{k-1}$.

\textsc{Dokaz:} Naj bo $\alpha_1 v_1 + \cdots + \alpha_m v_m = 0$, pri "cemer niso vsi $\alpha_j = 0$.
\begin{gather*}
\exists \alpha_j \neq 0: j > 1 \\
k = \max \{j: \alpha_j \neq 0\} \quad (k > 1) \\
\Rightarrow v_k = \beta_1 v_1 + \cdots + \beta_1{k-1}v_{k-1}
\end{gather*}
\hfill $\square$

\textsc{Trditev:} Naj vektorji $x_1, \ldots, x_m$ tvorijo ogrodje vektorskega prostora $V$. "Ce obstaja $j \in \{1, \ldots, m\}$, da je $x_j$ linearna kombinacija vektorjev $x_i, i \in \{1, \ldots, m\} \setminus \{j\}$, potem vektorji $\{x_i: i \in \{1, \ldots, m\} \setminus \{j\}\}$ sestavljajo ogrodje vektorskega prostora $V$.

\textsc{Dokaz:} Smemo vzeti $j=1$, ker lahko spremenimo indekse.
\begin{gather*}
x_1 = \alpha_2 x_2 + \cdots + \alpha_m x_m\\
v \in V
\end{gather*}
\begin{multline*}
v = \beta_1 x_1 + \cdots + \beta_m x_m = \\
= \beta_1 (\alpha_2 x_2 + \cdots + \alpha_m x_m) + \beta_2 x_2 + \cdots + \beta_m x_m = \\
= (\beta_1 \alpha_2 + \beta_2)x_2 + \cdots + (\beta_1 \alpha_m + \beta_m)x_m
\end{multline*}
Torej $x_2, \ldots, x_m$ sestavljajo ogrodje vektorskega prostora $V$.

\hfill $\square$

\textsc{Trditev:} Iz vsakega kon"cnega ogrodja vektorskega prostora $V \neq \{0\}$, lahko izberemo bazo.

\textsc{Dokaz:} Iz ogrodja postopoma odstanjujemo vektorje, ki so linearna kombinacija drugih. Na koncu ostane baza. (Predpostavimo lahko, da so vektorji v ogrodju razli"cni).

\textsc{Posledica:} Vsak netrivialen kon"cno razse"zen vektorski prostor ima bazo.

\textsc{Trditev:} Naj vektorji $x_1 \ldots x_m$ sestavljajo ogrodje vektorskega prostora $V$, vektorji $y_1, \ldots, y_n$ pa naj bodo linearno neodvisni. Potem je $m \geq n$.

\textsc{Dokaz:} Predpostavimo, da je $n > m$. Imamo dve vrsti vektorjev:
\begin{equation*}
x_1, \ldots , x_m \quad y_1, \ldots, y_n
\end{equation*}
Premaknemo $y_1$ v bazo in dobimo
\begin{equation*}
y_1, x_1, \ldots, x_m
\end{equation*}
To je ogorodje, vektorji $y_1, x_1, \ldots x_m$ pa so linearno odvisni. Torej obstaja tak vektor, ki je linearna kombinacija predhodnih. To je eden od vektorjev $x_1, \ldots x_m$. Tega odstranimo in ostane ogrodje
\begin{equation*}
y_1, x_1', \ldots, x_{m-1}'
\end{equation*}
Postopem ponovimo "se enkrat in dobimo
\begin{equation*}
y_2, y_1, x_1', \ldots, x_{m-1}'
\end{equation*}
Ti vektorji sestavljajo ogrodji in so linearno odvisni. Odstranimo vektor, ki je linearna kombinacija predhotnih. To je eden od vektorjev $x_1', \ldots, x_{m-1}'$, ker so $y_i$ linearno neodvisni. Dobimo ogrodje
\begin{equation*}
y_2, y_1, x_1'', \ldots, x_{m-2}''
\end{equation*}
Postopoma izpodrinemo vse $x$-e in dobimo ogrodje $y_m, y_{m-1}, \ldots, y_1$. Zato je $y_{m+1}$ linearna kombinacija vektorjev $y_1, \ldots, y_m$. $\rightarrow \leftarrow$ ($y_1, \ldots, y_n$ so linearno neodvisni).

\textbf{Sklep:} $m \geq n$ \hfill $\square$

\textbf{Posledica:} Vse baze netrivialnega kon"cno razse"znege vektorksega prostora imajo enako elementov.

\textsc{Definicija:} "Stevilo elementov v bazi kon"cno razse"znega vektorskega prostora imenujemo \emph{razse"znost} ali \emph{dimenzija} tega vektorskega prostora. \textbf{Oznaka:} $\dim V$

\textsc{Dokaz posledice:} $V \neq \{0\}$. Naj bosta
\begin{align*}
X &= \{x_1, \ldots, x_m\} \\
Y &= \{y_1, \ldots, y_n\}
\end{align*}
bazi vektorskega prostora $V$ in velja $x_i \neq x_j \forall i \neq j$ in $y_i \neq y_j \forall i \neq j$. Potem velja:

$X$ je ogrodje, $Y$ niz linearno neodvisnih vektorjev $\Rightarrow m \geq n$\\
$Y$ je ogrodje, $X$ niz linearno neodvisnih vektorjev $\Rightarrow n \geq m$ 

$\Rightarrow m = n$

\textsc{Izrek:} Naj bo $V$ $n$-razse"zen vektorski prostor nad $\OO$ (komutativen), $N \in \NN$. Potem je vektorski prostor $V$ izomorfen vektorskemu prostoru $\OO^n$.

\textsc{Dokaz:} Naj bo $\mathcal{V} = \{v_1, \ldots, v_n\}$ baza $V$ (\emph{urejena}, t.j., dolo"cen vrstni red).
\begin{gather*}
x \in V, x \mapsto (\alpha_1, \ldots, \alpha_n) \in \OO^n \\
x = \alpha_1 v_1 + \cdots + \alpha_n v_n
\end{gather*}
ker je $\{v_1, \ldots, v_n\}$ baza, so $\alpha_1, \ldots, \alpha_n$ enoli"cno dolo"ceni. Zapi"semo lahko preslikavo
\begin{align*}
\Phi_v&: V \to \OO^n \\
\Phi_v (x) &= (\alpha_1, \ldots, \alpha_n)
\end{align*}
$\Phi_v$ je odvisen od vrstenga reda baze in je izomorfen . Zapi"semo lahko tudi preslikavo
\begin{align*}
\Psi_v&: \OO^n \to V \\
\Psi_v (\alpha_1, \ldots, \alpha_n) &= \alpha_1 v_1 + \cdots + \alpha_n v_n
\end{align*}
$\Psi_v$ je inverz preslikave $\Phi_v \Rightarrow \Psi_v, \Phi_v$ sta bijekciji. Zado"s'ca dokazati, da je $\Psi$ linearna. Torej je potrebno dokazati homogenost in aditivnost. Oboje je o"citno, zato nismo napisali dokaza. Lahko ga napi"se"s za vajo doma (ni te"zek, saj je o"citen).

\textsc{Izrek:} Kon"cno razse"zna vektorska prostora nad istim obsegom sta izomorfna natanko takrat, kadar imata enako dimenzijo.

\textsc{Dokaz:} Smemo privzeti, da sta $V, U$ netrivialna. Kot se je izrazil profesor: ,,"ce sta $V$ in $U$ trivialna, je tudi dokaz trivialen.''
\begin{itemize}
	\item[($\Leftarrow$)] $\dim V = \dim U = n \Rightarrow$ obstajata izomorfizma  $\Phi, \Psi$:
	\begin{align*}
	\Phi &: V \to \OO^n \\
	\Psi &: \OO^n \to U
	\end{align*}
	$\Rightarrow \Psi \Phi: V \to U$ je izomorfizem
	
	\item[($\Rightarrow$)] Naj bo $F: V \to U$ izomorfizem vektorskih prostorov in $\dim V = n, n \in \NN$, ter $\{v_1, \ldots, v_n\}$ baza $V$. Trdimo, da je $\{F(v_1), \ldots, F(v_n)\}$ baza $U$.
	\begin{enumerate}
		\item \dashuline{linearna neodvisnost}
		\begin{gather*}
			\alpha_1 F(v_1) + \cdots + \alpha_n F(v_n) = 0 \\
			F(\alpha_1 v_1 + \cdots + \alpha_n v_n) = F(0) \\
			\Rightarrow \alpha_1 v_1 + \cdots + \alpha_n v_n = 0 \Rightarrow \\
			\Rightarrow \alpha_1 = \cdots = \alpha_n = 0
		\end{gather*}
		
		\item \dashuline{$\{F(v_1) , \ldots, F(v_n)\}$ je ogrodje}
		\begin{gather*}
			u \in U \Rightarrow \exists v \in V: F(v) = u \\
			v = \beta_1 v_1 + \cdots + \beta_n v_n \Rightarrow \\
			\Rightarrow u = f(v) = F(\beta_1 v_1 + \cdots + \beta_n v_n) = \\
			= \beta_1 F(v_1) + \cdots + \beta_n F(v_n)
		\end{gather*}
	\end{enumerate}
	\hfill $\square$
\end{itemize}
%
\textsc{Trditev:} Naj bo $V \neq \{0\}$ kon"cno razse"zen vektorski prostor. "Ce so $v_1, \ldots, v_m \in V$ linearno neodvisni, obstaja baza $V$, ki vsebuje $v_1, \ldots, v_m$.

\textsc{Dokaz:} $u_1, \ldots, u_n$ naj tvorijo ogrodje $V$.

$\Rightarrow \{v_1, \ldots, v_m, u_1, \ldots, u_n\}$ je ogrodje $V$. Postopoma iz tega ogrodja odtranjujemo vektorje, ki so linearna kombinacija vektorjev pred njimi. Vsi vektorji $v_1, \ldots, v_m$ ostanejo, ker so linearno neodvisni. Ostane nam baza, ki vsebuje $\{v_1, \ldots, v_n\}$.

\textsc{Trditev:} Naj bo $V$ kon"cno razse"zen vektorski prostor in $U$ njegov vektorski podprostor. Potem je $\dim U \leq \dim V$, pri "cemer velja ena"caj le v primeru $U = V$.

\textsc{Dokaz:} $V \neq \{0\}, \dim V = n \in \NN$.

$U \subseteq V, U \neq \{0\}$

$u_1, \ldots, u_m \in U$ linearno neodvisni v $U$ ($\Rightarrow$ v $V$)., zato je $m \leq n$. Naj bo $m$ maksimalen. Trdimo, da je potem $\{u_1, \ldots, u_m\}$ baza $U$. Zado"s"ca dokaz, da je $\mathcal{U} = \{u_1, \ldots, u_m\}$ ogrodje $U$.

"Ce $\mathcal{U}$ ni ogrodje vektorskega prostora $U$, obstaja tak $u \in U$ da $u$ ni linearna kombinacija vektorjev $u_1, \ldots, u_m$ ($u \notin \Lin \mathcal{U}$). Potem so vektorji $u_1, \ldots, u_m, u$ linearno neodvisni, to pa je protislovje z maksimalnostjo "stevila $m$. Torej je $\mathcal{U}$ ogrodje vektorskega prostora $U$, zato je baza $U$ in $\dim U = m (\leq n)$ "Ce je $\dim U = n$, je $U$ baza $V$, zato $U = V$.

\hfill $\square$

\textsc{Trditev:} Naj bo $V$ kon"cno razse"zen vektorski prostor in $U$ njegov vektorski podprostor. Potem obstaja tak vektorski podprostor $W \subset V$, da velja $V = U \oplus W$

\textsc{Dokaz:} $U = \{0\}, W = V$. Bolj zanimivo je, "ce $U \neq \{0\}, \{u_1, \ldots, u_m\}$ baza $U$. Dopolnimo jo do baze $V$
\begin{equation*}
\{u_1, \ldots u_m, u_{m+1}, \ldots, u_{m+k}\}
\end{equation*}
Postavimo $W = \Lin \{u_{m+1}, \ldots, u_{m+k}\}$. "Ce dopolnimo tako, da ni"c ne dopolnimo potem:
\begin{align*}
	W &= \Lin\{\} = \{0\} \\
	U &= V
\end{align*}
\dashuline{Trdimo, da je $V = U \oplus W$}
\begin{gather*}
v \in V \Rightarrow v = \underbrace{\alpha_1 u_1 + \cdots + \alpha_m u_m}_{x \in U} +\underbrace{\alpha_1{m+1} u_{m+1} + \cdots + \alpha_{m+k} u_{m+k}}_{y \in W} \\
v = x + y, x \in U, y \in W \\
\Rightarrow V = U + W
\end{gather*}
\dashuline{$U \cap W = \{0\}$}
\begin{gather*}
z \in U \cap W \\
z = \beta_1 u_1 + \cdots + \beta_m u_m = \beta_1{m+1} u_{m+1} + \cdots + \beta_{m+k} u_{m+k} \\
\beta_1 u_1 + \cdots + (-\beta_{m+k})u_{m+k} = 0 \\
\Rightarrow \beta_1 = \cdots = \beta_{m+k} = 0 \Rightarrow z = 0
\end{gather*}
$\Rightarrow V = U \oplus W$

\hfill $\square$

Tej trditvi pravimo \emph{trditev o eksistenci direktnega komplementa}.

\textsc{Trditev:} Naj bo $V$ kon"cno razse"zen vektorski prostor in $U, W$ njegova vektorska podprostora. "Ce je $U \cap W = \{0\}$, potem velja $\dim U \oplus W = \dim U + \dim W$.

\textsc{Dokaz:} $U, W$ sta netrivialna, druga"ce je o"citno. Naj bosta
\begin{gather*}
\{u_1, \ldots, u_m\} \text{ baza $U$, $\dim U = m$} \\
\{w_1, \ldots, w_n\} \text{ baza $W$, $\dim W = n$}
\end{gather*}
Trdimo, da je $\{u_1, \ldots, u_m , w_1, \ldots, w_n\}$ baza $U \oplus W$.
\begin{enumerate}
	\item linearna neodvisnost
	\begin{gather*}
	\alpha_1 u_1 + \cdots + \alpha m u_m + \beta_1 w_1 + \cdots + \beta_n w_n = 0 \\
	z = \underbrace{\alpha_1 u_1 + \cdots + \alpha_m u_m}_{\in U} = \underbrace{(-\beta_1)w_1 + \cdots + (-\beta_n)w_n}_{\in W} \\
	z \in U \cap W = \{0\} \Rightarrow z = 0 \\
	\Rightarrow \alpha_1 = \cdots = \alpha_m = 0, \\
	\beta_1 = \cdots = \beta_n = 0
	\end{gather*}
	(1) $\Rightarrow u_1 \cdots u_m, w_1 \cdots w_n$ so razli"cni
	
	\item $\Lin \{u_1, \ldots, u_m, w_1, \ldots, w_n\} = U \oplus W$

	O"citno je, da je $\Lin \{u_1, \ldots, u_m, w_1, \ldots, w_n\} \subseteq U \oplus W$. Dokazati je treba "se obratno smer ($\supseteq$).
	\begin{gather*}
	x \in U \oplus W \Rightarrow \\
	\Rightarrow x = u + 2, u \in U, w \in W \\
	u = \alpha_1 u_1 + \cdots + \alpha_m u_m \\
	w = \beta_1 w_1 + \cdots + \beta_n w_n \\
	\Rightarrow x = \alpha_1 u_1 + \cdots + \alpha_m u_m + \beta_1 w_1 + \cdots + \beta_n w_n
	\end{gather*}
\end{enumerate}
$(1), (2) \Rightarrow \dim U \oplus W = m + n = \dim U + \dim W$

\textsc{Trditev:} Naj bo $V$ kon"cno razse"zen vektorski prostor in $U \leq V, W \leq V$. Ptem velja enakost (= \emph{dimeznisjska formula}):
\begin{equation*}
\dim (U + W) = \dim U + \dim W - \dim (U \cap W)
\end{equation*}
\textsc{Osnovna ideja dokaza:} Vzamemo bazo vektorskega prostora $U \cap W$. V $W$ najdemo vektorje s katerimi raz"sirimo $U \cap W$. Linearno ogrinja"co teh vektorjev ozna"cimo z $Z$. Velja $U + W = U \oplus Z$.

$\Rightarrow \dim (U + W) = \dim U + \dim Z$ in \\
$W = (U \cap W) \oplus  \Rightarrow \dim W = \dim (U \cap W) + \dim Z$

Iz tega sledi zgornja formula.

\textsc{Trditev:} Naj bo $V = U \oplus W, \dim V < \infty$. Potem je vektorski prostor $V/_U$ izomorfen $W$, vektorski prsotor $V/_W$, pa je izomorfen $U$.

\textsc{Dokaz:}
\begin{gather*}
	f: W \to V/_U \\
	f(w) = [w] = w + U
\end{gather*}
\dashuline{$f$ je linearna preslikava}
\begin{equation*}
f (w_1 + w_2) = [w_1 + w_2] = [w_1] + [w_2] = f(w_1) + f(w_2)
\end{equation*}
$\Rightarrow f$ je aditivna. Podobno doka"zemo homogenost.

\dashuline{$f$ je injektvina}
\begin{gather*}
f(w_1) = f(w_2) \Rightarrow \\
\Rightarrow [w_1] = [w_2] \Rightarrow \\
\Rightarrow w_1 \sim w_2 \Rightarrow \\
\Rightarrow w_1 - w_2 \in U \\
w_1 - w_2 \in W \\
\Rightarrow w_1 - w_2 \in U \cap W = \{0\} \Rightarrow w_1 - w_2 = 0 \\
\Rightarrow w_1 = w_2
\end{gather*}
\dashuline{$f$ je surjektivna:}
\begin{gather*}
[v] \in V/_U, v \in V \\
v = u + w, u \in U, w \in W
\end{gather*}
\dashuline{$f(w) = [v]$}
\begin{gather*}
f(w) = w \\
v = u + w \Rightarrow u = v - w \in U \Rightarrow w \sim v
\end{gather*}
\hfill $\square$

Naj bo $V= V_1 \oplus V_2 \Rightarrow V/_{V_1} \cong V_2 \land V/_{V_2} \cong V_1$

\textsc{Trditev:} Naj bo $V$ kon"cno razse"zen vektorski prostor in $U$ njegov vektorski podprostor. Potem je
\begin{equation*}
\dim V/_U = \dim V - \dim U
\end{equation*}
\textsc{Dokaz:} Poi"s"cimo $W \leq V$, da je $V = U \oplus W$. Ker je $V/_U \cong W$, velja $\dim V/_U = \dim W$. Vemo:
\begin{equation*}
\dim V = \dim U + \dim W
\end{equation*}
Zato je
\begin{equation*}
\dim V/_U = \dim V - \dim U
\end{equation*}
%
\subsection{Linearne preslikave na kon"cno razse"znih V.\,P.}
Naj bosta $V, U$ kon"cno razse"zna vektorska prostora nad $\OO$ in naj bo $\mathcal{A} \in \LL (V, U)$ linearna.

Naj bo $\mathcal{V} = \{v_1, \ldots v_n\}$ baza $V$. "Ce poznamo slike $\A v_1, \ldots, \A v_n$, poznamo $\A$:
\begin{gather*}
x \in V, \quad \exists \alpha_1, \ldots, \alpha_n \in \OO: \\
x = \alpha_1 v_1 + \cdots + \alpha_n v_n \\
\Rightarrow \A x = \A(\alpha_1 v_1 + \cdots + \alpha_n v_n) = \alpha_1 \A v_1 + \cdots +  \alpha_n \A v_n
\end{gather*}
\subsubsection{Poseben primer}
\begin{gather*}
V = \OO^n ,\quad U = \OO^m \\
A = \LL(\OO^n, \OO^m) \\
\{e_1, \ldots, e_n\} \text{ standardna baza $\OO^n$} \\
e_j = \begin{bmatrix}
0 \\ \vdots \\ 0 \\ 1 \\ 0 \\ \vdots
\end{bmatrix} \\
x \in \OO^n, \quad x = \begin{bmatrix}
x_1 \\ x_2 \\ \vdots \\ x_n
\end{bmatrix}
\end{gather*}
Poznamo $Ae_1, \ldots Ae_n \Rightarrow$ poznamo $A$.
\begin{gather*}
Ae_j \in \OO^m \\
Ae_j = \begin{bmatrix}
a_{1j} \\ a_{2j} \\ \vdots \\ a_{mj}
\end{bmatrix} \in \OO^m \\
A = \begin{bmatrix}
a_{11} & a_{12} & \cdots & a_{1n} \\
a_{21} & a_{22} & \cdots & a_{2n} \\
a_{31} & a_{32} & \cdots & a_{3n} \\
\vdots & \vdots & \ddots & \vdots \\
a_{m1} & a_{m2} & \cdots & a_{mn}
\end{bmatrix} = \text{$m \times n$ matrika, ki predstavlja linearno preslikavo $A$}
\end{gather*}
$a_{ij} (1 \leq i \leq m, 1 \leq j \leq n)$ je "clen matrike $A$. $a_{ij}$ le"zi v $i$-ti vrstici in $j$-tem stolpcu.
\begin{gather*}
A^{(j)} = \begin{bmatrix}
a_{1j} \\ a_{2j} \\ \vdots \\ a^{mj}
\end{bmatrix} = \text{$j$-ti stolpec matrike} \\
A_{(i)} = \begin{bmatrix}
a_{i1} & a_{i2} & \cdots & a_{in}
\end{bmatrix} \\
A = [a_{ij}]\\\\
A : \OO^n \to \OO^m \\
x = \begin{bmatrix}
x_1 \\ x_2 \\ \vdots \\ x_n
\end{bmatrix} \in \OO^n \\
A = [a_{ij}] \\
Ax = y \\
y = \begin{bmatrix}
y_1 \\ y_2 \\ \cdots \\ y_m
\end{bmatrix}
\end{gather*} 
$y$ izra"cunamo kot:
\begin{multline*}
y = Ax = A(x_1 e_1 + x_2 e_2 + \cdots + x_n e_n) = \\
= x_1 A e_1 + x_2 A e_2 + \cdots + x_n A e_n = \\
= x_1 A^{(1)} + x_2 A^{(2)} + \cdots + x_n A^{(n)} \\
\Rightarrow y_i = x_1 a_{i1} + x_2 a_{i2} + \cdots + x_n a_{in} = \\
y_i = \sum_{j=1}^{n} a_{ij} x_j, \quad i = 1, \ldots, m
\end{multline*}
\textsc{Primer:} $A: \RR^3 \to \RR^3$ zasuk za kot $\varphi$ okrog $z$-osi. Kaj je matrika $A$ in kam $A$ preslika to"cko $(1, 2, 3)$?
\begin{align*}
A e_1 &= A\vec{i} = A^{(1)} \\
A e_2 &= A\vec{j} = A^{(2)} \\
A e_3 &= A\vec{k} = A^{(3)}
\end{align*}
\begin{gather*}
A^{(3)} = A\vec{k} = \begin{bmatrix}0 \\ 0 \\ 1\end{bmatrix} \\
A^{(1)} = A\vec{i} = \begin{bmatrix}\cos \varphi '' \sin \varphi \\ 0 \end{bmatrix} \\
A^{(2)} = A\vec{j} = \begin{bmatrix}- \sin \varphi \\ \cos \varphi \\ 0\end{bmatrix} \\
A = \begin{bmatrix}
\cos \varphi & -\sin \varphi & 0 \\
\sin \varphi & \cos \varphi & 0 \\
0 & 0 & 1
\end{bmatrix}
\end{gather*}
Izra"cunajmo sliko to"cke $(1, 2, 3)$:
\begin{equation*}
A\begin{bmatrix}1 \\ 2 \\ 3\end{bmatrix} = \begin{bmatrix}
\cos \varphi - 2 \sin \varphi \\
\sin \varphi + 2 \cos \varphi \\
3
\end{bmatrix}
\end{equation*}
$\LL(\OO^n, \OO^m) \equiv \OO^{m \times n}$ je mno"zica vseh $m \times n$ matrik s "cleni $\OO$. Preslikave smo \emph{identificirali} (poistovetili) z matrikami.

$\LL(\OO^n, \OO^m)$ je vektorski prostor nad $\OO$. $\OO^{m \times n}$ postane vektorski prostor nad $\OO$.

Naj bosta $A, B \in \OO^{m \times n}$.
\begin{gather*}
(A + B) x = Ax + Bx \quad \forall x \in \OO^n \\
(\underbrace{A + B}_{C \in \OO^{m\times n}})e_j = Ae_j + Be_j = A^{(j)} + B^{(j)}
\end{gather*}
\begin{gather*}
\Rightarrow C^{(j)} = Ce_j = A^{(j)} + B^{(j)} \\
\Rightarrow c_{ij} = a_{ij} + b_{ij} \quad \forall i,j \\
\end{gather*}
$\Rightarrow$ v $\OO^{m \times n}$ se"stevamo po "clenih. Podobno je z mno"zenjem s skalarji.
%
\subsubsection{Splo"sna situacija}
Naj bosta $V, U$ vektorska prostora nad $\OO$. 
\begin{gather*}
\begin{aligned}
\dim V &= n\\
\dim U &= m
\end{aligned} \\
\begin{aligned}
\mathcal{V} &= \{v_1, \ldots, v_n\} \text{ urejena baza $V$} \\
\mathcal{U} &= \{u_1, \ldots, u_n\} \text{ urejena baza $U$}
\end{aligned}
\end{gather*}
$\A = \LL(V, U)$, $\A$ poznamo, "ce poznamo slike $\A v_j, \quad j = 1, \ldots, n$.
\begin{gather*}
\Phi_\mathcal{V}: V \to \OO^n \text{ izomorfizem} \\
\begin{aligned}
v &\in V \\
v &= \alpha_1 v_1 + \cdots + \alpha_n v_n \\
v &\to (\alpha_1, \ldots, \alpha_n) = \Phi_\mathcal{V}(v) = \begin{bmatrix}\alpha_1 \\ \vdots \\ \alpha_n\end{bmatrix}
\end{aligned} \\
\Psi_{\mathcal{V}}(v_j) = \begin{bmatrix}0 \\ \vdots \\ 0 \\ 1 \\ 0 \\ \vdots \\ 0\end{bmatrix} = e_j
\end{gather*}
Podobno velja za izomorfizem $\Phi_\mathcal{U}: U \to \OO^m$.
\begin{figure}[!htbp]
	\centering
	\begin{tikzpicture}
	\node (V) at (0, 0) {V};
	\node (U) at (3, 0) {U};
	\node (On) at (0, -3) {$\OO^n$};
	\node (Om) at (3, -3) {$\OO^m$};
	
	%\draw (Point1) -- (Point2) node [<position>, fill=white] {Label Text};
	
	\draw[->] (V) edge node [midway, above] {$\A$} (U);
	\draw[->] (V) edge node [midway, left] {$\Phi_\mathcal{V}$} (On);
	\draw[->] (U) edge node [midway, right] {$\Phi_\mathcal{U}$} (Om);
	\draw[->] (On) edge node [midway, below] {$A$} (Om);
	\end{tikzpicture}
	\caption{Diagram preslikave}
\end{figure}

Diagram \emph{komutira} $\Phi_\mathcal{U}\A = A\Phi_\mathcal{V}$
\begin{gather*}
(\Phi_{\mathcal{U}}\A)v_j = (A \Phi_{\mathcal{V}})v_j = Ae_j = A^{(j)} \\
(\Phi_{\mathcal{U}}\A)v_j = \Phi_{\mathcal{U}}(\A v_j) = \Phi_{\mathcal{U}} (\alpha_1 u_1 + \cdots + \alpha_m u_m) = \alpha_1 e_1 + \cdots + \alpha_m e_m = \begin{bmatrix}\alpha_1 \\ \alpha_2 \\ \vdots \\ \alpha_m \end{bmatrix} \\
\A v_j = \alpha_1 u_1 + \alpha_2 u_2 + \cdots + \alpha_m u_m \\
\Rightarrow A^{(j)} = \begin{bmatrix}\alpha_1 \\ \alpha_2 \\ \vdots \\ \alpha_m \end{bmatrix} \Rightarrow \alpha_i = a_{ij} \quad \forall i, j
\end{gather*}
\begin{equation*}
\A v_j = a_{1j} u_1 + a_{2j} u_2 + \cdots + a_{mj} u_m
\end{equation*}
Linearni preslikavi $\A \in \LL(V, U)$ priredimo (glede na bazi $\mathcal{U}, \mathcal{V}$) matriko $A \in \OO^{m \times n}$.
\begin{gather*}
\begin{aligned}
F : \LL(V, U) &\to \OO^{m \times n} \\
\A &\mapsto A = F(\A)
\end{aligned} \\
F(\A) = \Phi_{\mathcal{U}}\A\left(\Phi_{\mathcal{V}}\right)^{-1}
\end{gather*}
$F$ je izomorfizem med $\LL(V, U)$ in $\OO^{m \times n}$
\begin{itemize}
	\item  aditivnost in homogenost sta o"citni
	\item iz diagrama hitro dobimo inverz $F^{-1} = \left(\Phi_{\mathcal{U}}\right)^{-1}A\Phi_{\mathcal{V}}$
\end{itemize}
\begin{equation*}
\dim \LL(V, U) = \dim \OO^{m \times n} = ?
\end{equation*}
Standradna baza $\OO^{m \times n}$ je sestavljena iz \emph{elementarnih matrik}. V elementarni matriki se nahaja ena 1, ostali "cleni so 0.
\begin{align*}
E_{pq} &= [e_{ij}] \quad i = 1, \ldots, m \quad j = 1, \ldots, n \\
e_{ij}& = \begin{cases}
1 & i = p \land j = q \\
0 & \text{sicer}
\end{cases}
\end{align*}
%
\subsubsection{Mno"zenje matrik}
Naj bodo $U, V$ in $W$ vektorski prostori, linearna preslikava $\mathcal{B}: W \to V$, $\A: V \to U$ in $\mathcal{C}: W \to U$, t.j.: $\mathcal{C} = \mathcal{A} \mathcal{B}$. 

$\mathcal{U}$ je urejeneba baza v.p. $U$, $\mathcal{V}$ urejena baza $V$, $\mathcal{W}$, pa urejena baza $W$.

Poznamo $\Phi_{\mathcal{V}}: V \to \OO^n$, $\Phi_{\mathcal{U}}: U \to \OO^m$ in $\Phi_\mathcal{W}: W \to \OO^p$. in poznamo preslikavi $A: \OO^n \to \OO^m$, ter $B: \OO^p \to \OO^n$. Zanima nas $C = AB$.
\begin{align*}
A &= [a_{ij}] \in \OO^{m \times n} \\
B &= [b_{ij}] \in \OO^{n \times p} \\
C &= [c_{ij}] \in \OO^{m \times p}
\end{align*}
\begin{gather*}
C = AB \\
C^{(j)} = C e_j = (AB)e_j = A(B e_j) = AB^{(j)} \Rightarrow \\
C^{(j)} = AB^{(j)} \quad \forall j \\
\Rightarrow C_ij = A_{(i)} B^{(j)} \\
C_{ij} = \sum_{k = 1}^{n} a_{ik} b_{kj}
\end{gather*}
%
\subsubsection{Poseben primer}
Naj bo $U = V = W$, in $m = n = p$.
\begin{equation*}
A, B \in \OO^{n \times n} \Rightarrow C = AB \in \OO^{n \times n}
\end{equation*}
$\LL(\OO^n) \equiv \OO^{n \times n}$  je \emph{algebra kvadratnih matrik}.
Naj bodo baze $U, V, W$ enake, to je $\mathcal{U} = \mathcal{V} = \mathbb{V}$. Skonstruiramo preslikavo
\begin{align*}
F: \LL(V) &\to \LL(\OO^n) \equiv \OO^{n \times n} \\
\A &\mapsto A \\
F(\A) &= \Phi_{\mathcal{V}} \A \Phi_{\mathcal{V}}^{-1}
\end{align*}
\textbf{Vemo:} $F$ je izomorfizem vektorskih prostorov $\LL(V)$ in $\OO^{n \times n}$.

\textsc{Trditev:} $F$ je izomorfizem med algebrama $\LL(V)$ in $\OO^{n \times n}$.

\textsc{Dokaz:} Zado"s"ca ugotoviti, da $F$ ohranja mno"zenje
\dashuline{$F(\mathcal{A} \mathcal{B}) = F(\A) F(\mathcal{B})$}

\begin{gather*}
F(\A \mathcal{B}) = \Phi_{\mathcal{V}} (\A \mathcal{B}) \Phi_{\mathcal{V}}^{-1} \\
F(\A) F(\mathbb{B}) = \Phi_{\mathcal{V}} \A \Phi_{\mathcal{V}}^{-1} \Phi_{\mathcal{V}} \mathcal{B} \Phi_{\mathcal{V}}^{-1} = \Phi_{\mathcal{V}} \A \mathcal{B} \Phi_{\mathcal{V}}^-1
\end{gather*}
\hfill $\square$

$id_V$ je enota algebre $\LL(V)$. $F(id_V)$ je enota algebre $\OO^{n \times n}$
\begin{equation*}
F(id_V) = id_{\OO^{n \times n}} = I
\end{equation*}
$I$ je \emph{enotska} (ali identi"cna) matrika. Velja:
\begin{equation*}
I^{(j)} = I e_j = e_j
\end{equation*}
torej 
\begin{equation*}
I = [e_1, e_1, \ldots, e_n]
\end{equation*}
Zapi"semo lahko tudi kot
\begin{equation*}
I = [\delta_{ij}]\quad i,j = 1, \ldots, n
\end{equation*}
kjer je $\delta_{ij}$ \emph{Kroneckerjeva delta}, za katero velja predpis
\begin{equation*}
\delta_{ij} = \begin{cases}
1 & i = j \\
0 & \text{sicer}
\end{cases}
\end{equation*}
%
\textsc{Definicija:} Naj bo $\A \in \LL(V)$ bijekcija ($\exists \mathcal{B} \in \LL(V): \A \mathbb{B} = \mathcal{B} \A = id_V$). Pravimo, da je $F(\A) = A$ \emph{obrnljiva}:
\begin{equation*}
\exists B \in \OO^{n \times n} : A B = BA = I
\end{equation*}
$A$ obrnljiva $\Rightarrow B$ je enoli"cno dolo"cena. Ozna"cimo:
\begin{equation*}
B = A^{-1}
\end{equation*}
\subsubsection{Rang linearne preslikave in matrike}
Naj bosta $V, U$ kon"cno razse"zna vektorksa prostora nad $\OO$ in $\A \in \LL(V, U)$.

\textsc{Izrek:} Za $\A$ velja formula
\begin{equation}
\label{eq:rang-izrek}
\dim (\im \A) + \dim (\ker \A) = \dim V
\end{equation}
\textsc{Dokaz:} Vemo, da sta vekotrska prostora $V_{\ker \A}$ in $\im \A$ izomorfna. Zato je $\dim V/_{\ker \A} = \dim (\im \A)$. Vemo $\dim V/_{\ker \A} = \dim V - \dim (\ker \A)$. $\Rightarrow$~\refeq{eq:rang-izrek}.

\hfill $\square$

\textsc{Definicija:} Rang preslikave $\A$ je $\dim (\im \A)$. Oznaka $\rang \A = \dim (\im \A)$.

\textsc{Trditev:} $\A = \LL(V, U)$
\begin{enumerate}
	\item $\A$ je injektivna $\iff \rang \A = \dim V$
	\item $\A$ je surjektivna $\iff \rang A = \dim U$
	\item $\A$ je bijektivna $\iff \dim V = \dim U = \rang \A$
\end{enumerate}
\textsc{Dokaz:}
\begin{enumerate}
	\item vemo: $\A$ injektivna $\iff \ker \A = \{0\}$
	\begin{equation*}
	\ker \A = \{0\} \iff \dim (\ker \A) = 0
	\end{equation*}
	
	\item vemo: $\A$ surjektivna $\iff \im \A = U$
	\begin{equation*}
	\im \A = U \iff \underbrace{\dim (\im \A)}_{\rang \A} = \dim U
	\end{equation*}
	
	\item kombiniramo 1 in 2.
\end{enumerate}
%
\textsc{Trditev:} Za $\A \in \LL(V)$ so ekvivalentne naslednje izjave:
\begin{enumerate}
	\item $\A$ je bijekcija
	\item $\A$ je surjekcija
	\item $\A$ je injekcija
	\item $\rang \A = \dim V$
\end{enumerate}
\textsc{Dokaz:} $U = V$ v prej"snji trditvi $\Rightarrow$
\begin{equation*}
\Rightarrow [(1) \iff (4), (2) \iff (4), (3) \iff (4)]
\end{equation*}
\textsc{Posledica:} Matrika $A \in \OO^{n \times n}$ je obrnljiva natanko takrat, kadar je $\rang A = n$.

\textsc{Dokaz:} V prej"snji trditvi vzamemo $V = \OO^n$ in $A$ razumemo kot endomorfizem vektorskih prostorov $\OO^n$. 

$A$ obrnljiva $\iff A$ bijekcija, t.j.: $(1) \iff (4)$ po prej"snji trditvi.

\textsc{Trditev:} Naj bo $\A \in \LL(V, U)$ in $A$ matrika, ki pripada $\A$ gelde na dai baz $\mathcal{V} \in V, \mathcal{U} \in U$. Potem velja:
\begin{equation*}
\rang \A = \rang A
\end{equation*}
\textsc{Dokaz:}
Nari"semo si diagram in opazimo, da komutira.
\begin{gather*}
\Phi_{\mathcal{U}} \A =  A \Phi_{\mathcal{V}} \\
(\Phi_{\mathcal{U}} \underbrace{\A) V}_{\im \A} = (A \underbrace{\Phi_{\mathcal{V}}) V}_{\OO^n} \Rightarrow \\
\Rightarrow \Phi_{\mathcal{U}} (\im \A) = \im A
\end{gather*}
$\Phi_{\mathcal{U}}$ je izomorfizem, zato je
\begin{gather*}
\begin{aligned}
\dim (\im \A) &= \dim (\im A) \\
\rang \A &= \rang A
\end{aligned}
\end{gather*}
%
Naj bo $A \in \OO^{m \times n} \equiv \LL(\OO^n, \OO^m)$. Velja
\begin{gather*}
\begin{aligned}
\im A &= \{Ax: x \in \OO^n\} = \\
&= \{A(x_1 e_1 + \cdots + x_n e_n): x \in \OO^n\} = \\
&= \{x_1Ae_1 + \cdots + x_n A e_n\} = \\
&= \{x_1 A^{(1)} + \cdots + x_n A^{(n): x_j \in \OO \forall j} = \\
& = \Lin \{A^{(1)}, \ldots, A^{(n)}\}
\end{aligned}
\end{gather*}
\textsc{Trditev:} Stolpci matrike $A$ tvorijo ogrodje vektorskega prostora $\im A$.

\textsc{Posledica:} Rang matrike $A$ je najve"cje "stevilo linearno neodvisnih stolpcev te matrike.

Operacije na matrkah, ki ohranjujejo rang:
\begin{enumerate}[S1)]
	\item  med sabo zamenjamo dva stolpca
	\item stolpec pomno"zimo z neni"celnim skalarjem
	\item Stolpci pri"stejemo ve"ckratnik kak"snega drugega stolpca
\end{enumerate}
V1, V2, V3 so analogne operacije na vrsticah.

\textsc{Trditev:} S1, S2, S3 in V1, V2, V3 ohranjajo rang.

\textsc{Dokaz:}
\begin{enumerate}[S1)]
	\item o"citno
	\item  Dokazati moramo, da velja
	\begin{equation*}
	L1 = \Lin \{\alpha A^{(1)}, A^{(2)}, \ldots, A^{(n)}\} = \Lin \{A^{(1)}, A^{(2)}, \ldots, A^{(n)}\} = L2
	\end{equation*}
	Torej morajo biti vsi vektorji, ki so linearne kombinacije vektorjev L1 tudi linearne kombinacije vektorjev L2 in obratno. Zado"s"ca
	\begin{equation*}
	A^{(1)} = \alpha^{-1} (\alpha A^{(1)})
	\end{equation*}
	To velja, ker $\alpha \neq 0$. Torej lahko vsak vektor, ki je zapisan z linearno kombinacijo L1 pretvorimo v linearno kombinacijo vektorjev L2 in obratno, zato se ohranja slika preslikave $A$ in posledi"cno tudi rang.
	
	\item Dokazati moramo
	\begin{equation*}
	\Lin \{A^{(1)} + \alpha A^{(2)}, A^{(2)}, \ldots, A^{(n)}\} = \Lin \{A^{(2)}, A^{(2)}, \ldots, A^{(n)} \}
	\end{equation*}
	Velja podoben razmislek kot pri S2, zato zado"s"ca
	\begin{equation*}
	A^{(1)} = \left(A^{(1)} + \alpha A^{(2)}\right) + (- \alpha) A^{(2)}
	\end{equation*}
\end{enumerate}
\dashuline{V1, V2, V3 ohranjajo $\ker A$}
\begin{equation*}
x \in \ker A \iff Ax = 0 \iff A_{i} x = 0, \quad i = 1, \ldots, m
\end{equation*}
\begin{enumerate}[V1)]
	\item o"citno iz zgornje enakosti
	\item \begin{equation*}
	A \to 
	\begin{bmatrix}
	\alpha A_{(1)} \\
	A_{(2)} \\
	\vdots \\
	A_{(m)}
	\end{bmatrix}
	\end{equation*}
	Pokazati moramo, da $\alpha A_{(1)} x = 0 \quad \forall x \in \OO^n$. Ker $\alpha \neq 0$, velja
	\begin{equation*}
	\alpha A_{(1)} x = 0 \iff A_{(1)} x = 0
	\end{equation*}
	Torej operacija ohranja $\ker A$.
	
	\item
	\begin{equation*}
	A \to 
	\begin{bmatrix}
	A_{(1)} +  \alpha A_{(2)} \\
	A_{(2)} \\
	\vdots \\
	A_{(m)}
	\end{bmatrix}
	\end{equation*}
	Pokazati moramo
	\begin{equation*}
	\begin{Bmatrix}
	(A_{(1)} + \alpha A_{(2)}) x = 0 \\
	A_{(2)} x = 0 \\
	\cdots \\
	A_{(n)} x = 0
	\end{Bmatrix}
	\iff
	\begin{Bmatrix}
	A_{(1)}x = 0 \\
	A_{(2)} x = 0 \\
	\cdots \\
	A_{(n)} x = 0
	\end{Bmatrix}
	\end{equation*}
	\begin{itemize}
		\item[($\Leftarrow$)] o"citno
		\item [($\Rightarrow$)] vemo $(A_{(1)} + \alpha A_{(2)}) x = 0$. Dokazati moramo, da je $A_{(1)} x = 0$.
		\begin{gather*}
			(A_{(1)} + \alpha A_{(2)}) x = 0 \\
			A_{(1)} x + \alpha \underbrace{A_{(2)} x}_{0} = 0 \\
			\Rightarrow A_{(1)} x = 0
		\end{gather*}
	\end{itemize}
\end{enumerate}
Ker se ohranja jedro, se ohranja rang ($= n - \dim (\ker A)$).

\textsc{Trditev:} Naj bo $A \in \OO^{m \times n}$. Z uporabo operacij S1 - S3, V1 - V3 lahko postopoma pridemo iz matrike $A$, do matrike $A_0$ oblike
\begin{equation*}
A_0 = \begin{bmatrix}
1 &  &  &  \\
 & \ddots &  & \\
 &  & 1 & 
\end{bmatrix} \in \OO^{m \times n}
\end{equation*}
Kjer je $\rang A$ "stevilo stolpcev z eno enico. Natan"cna shema postopka je v zvezku.

\textsc{Primer}
\begin{equation*}
A = \begin{bmatrix}
1 & 2 & 3 & 4 \\
3 & 2 & 1 & 0 \\
2 & 3 & 4 & 5
\end{bmatrix} \sim
\begin{bmatrix}
1 & 2 & 3 & 4 \\
0 & -4 & -8 & -12 \\
0 & -1 & -2 & -3
\end{bmatrix} \sim
\begin{bmatrix}
1 & 0 & 0 & 0 \\
0 & 1 & 2 & 3 \\
0 & 0 & 0 & 0
\end{bmatrix} \sim
\begin{bmatrix}
1 & 0 & 0 & 0 \\
0 & 1 & 0 & 0 \\
0 & 0 & 0 & 0
\end{bmatrix} = A_0
\end{equation*}
Torej je $\rang A = \rang A_0 = 2$.

\textsc{Posledica} Rang matrike je enak rangu njene \emph{transponiranke}.
\begin{equation*}
\rang A = \rang A^\intercal
\end{equation*}
"Ce je $A = [a_{ij}] \in \OO^{m \times n}$, transponiranko $B = A^\intercal$, t.j.: $B = [b_{ij}] \in \OO^{n \times m}$ tvorimo na nasleden na"cin:
\begin{equation*}
b_{ij} = a_{ji} \quad \forall i, j
\end{equation*}
\textsc{Dokaz:} O"citno je, da "ce na matriki $A$ izvedemo operaijo S$i$, se bo ta pretvorila v V$i$ na matriki $A^\intercal$. Analogno za operacije V$i$.

\textsc{Posledica:} Najve"cje "stevilo linearno neodvisnih stolpcev matrike je enako najve"cjemu "stevilu njenih linearno neodvisnih vrstic.

\subsubsection{Sistemi linearnih ena"cb}
Naj bo
\begin{gather*}
a_{11} x_1 + a_{12} x_2 + \cdots + a_{1n} x_n = b_1 \\
a_{21} x_1 + a_{22} x_2 + \cdots + a_{2n} x_n = b_2 \\
\vdots \\
a_{m1} x_1 + a_{m2} x_2 + \cdots + a_{mn} x_n = b_m \\
\end{gather*}
sistem linearnih ena"cb. Zapi"semo lahko $A \in \OO^{m \times n}, \quad A = [a_{ij}]$. Za $b$ lahko zapi"semo
\begin{equation*}
b \in \OO^m, \quad b = \begin{bmatrix}
b_1 \\
\vdots \\
b_m
\end{bmatrix}
\end{equation*}
Podobno lahko $x$ zapi"semo kot
\begin{equation*}
x = \begin{bmatrix}
x_1 \\
\vdots \\
x_n
\end{bmatrix} \in \OO^n
\end{equation*}
I"s"cemo $x \in \OO^n$, da bo veljalo
\begin{equation*}
Ax = b
\end{equation*}
"Ce gledamo na $A$ kot na preslikavo, velja $A \in \LL(\OO^n, \OO^m)$. Naj bo $\mathcal{R}$ mno"zica vseh re"sitev sistema $Ax = b$, t.j.:
\begin{equation*}
\mathcal{R} = \{x \in \OO^n: Ax = b\}
\end{equation*}
$[A|b] \in \OO^{m \times (n + 1)}$ je \emph{raz"sirjena matrika} sistema $Ax = b$. 

"Ce je $b = 0$, potem imamo \emph{homogen sistem} $Ax = 0$, potem
\begin{equation*}
\mathcal{R} = \ker A
\end{equation*}

"Ce je $b \neq 0$. imamo \emph{nehomogen sistem} $Ax = b$. Sistem je \emph{protisloven}, kadar je $\mathcal{R} = \varnothing$, sicer pa je \emph{neprotisloven}. Naj bo sistem $Ax = b$ neprotisloven in $w$ ena od re"sitev ($w \in \mathcal{R}$). Pravimo, da je $w$ \emph{partikularna re"sitev}.

Naj bo 
\begin{gather*}
A w = b, \quad x \in \mathcal{R} \Rightarrow Ax = b \\
\Rightarrow A(x - w) = \underbrace{Ax}_{b} - \underbrace{Aw}_{b} = 0 \Rightarrow x - w \in \ker A \\
\Rightarrow x \in w + \ker A
\end{gather*}
Torej velja $\mathcal{R} \subseteq w + \ker A$.

Naj bo $x \in w + \ker A$. Potem je $x = w + y, y \in \ker A$.
\begin{equation*}
\Rightarrow Ax = A(w + y) = \underbrace{Aw}_{b} + \underbrace{Ay}_{0} = b \\
\Rightarrow x \in \mathcal{R}
\end{equation*}
Torej velja $w + \ker A \subset \mathcal{R}$

Iz $(1) \& (2)$ sledi
\begin{equation*}
\mathcal{R} = w + \ker A
\end{equation*}
\textsc{Trditev:} "Ce je $w$ partikularna re"sitev sistema $Ax = b$, je 
\begin{equation*}
\mathcal{R} = w + \ker A
\end{equation*}
\textsc{Izrek} (Kronecker, Capelli): $\mathcal{R} \neq \varnothing$ natanko takrat, kadar je
\begin{equation*}
\rang [A | b] = \rang[A]
\end{equation*}
\textsc{Dokaz:} $\mathcal{R} \neq \varnothing \iff b \in \im A$ (ker $Ax = b$ pomeni, da je $b \in \im A$)
\begin{gather*}
\begin{aligned}
\im [A | b] &= \Lin \{A^{(1)}, \ldots, A^{(n)}, b\} \\
\im A = &= \Lin \{A^{(1)}, \ldots, A^{(n)}\}
\end{aligned}\\
\dim (\im [A | b]) = \dim (\im A) \iff \im A = \im [A | b] \iff b \in \im A
\end{gather*}
ker $\im A \subseteq \im [A | b]$, preveriti je treba "se $\im [A | b] \subseteq \im A \iff b \im A$.
%
\subsubsection{Gaussov algoritem za re"sevanje sistema}
Dovoljene operacije so V1 - V3 in S1. S1 je dovoljena operacija samo za prvih $n$ stolpcev in paziti je treba, da med seboj ustrezno zamenjamo spremenljivke. S temi operacijami bo mno"zica re"sitev ostala ista.

Skica poteka je v zvezku. Na za"cetku leta sem opozoril, da tu ne bo skic. "Ce si pri"cakoval spremembo toplo priporo"cam da zniza"s pri"cakovanja. Ko pridemo do kon"cne matrike, katere skica nje je v prej omenjenem zvezku, velja
\begin{gather*}
Ax = b \iff\\
1 x_{i1} + 0 x_{i2} + \cdots + 0 x_{ir} + \ast x_{ir+1} + \cdots + \ast x_{in} = \ast \\
0 x_{i1} + 1 x_{i2} + \cdots + 0 x_{ir} + \ast x_{ir + 1} + \cdots + \ast x_{in} = \ast \\
\vdots \\
0 x_{i1} + 0 x_{i2} + \cdots + 1 x_{ir} + \ast x_{ir + 1} + \cdots + \ast x_{in} = \ast \\
0 x_{i1} + 0 x_{i2} + \cdots + 0 x_{ir} + 0 x_{ir + 1} + \cdots + 0 x_{in} = \delta
\end{gather*}
"Ce je $\delta = 1$, je sistem protisloven, "ce pa je $\delta = 0$, ima sistem $n - r$ parametri"cno dru"zino re"sitev. Prametri so
\begin{gather*}
x_{ir+1} = \alpha _1 \\
\vdots \\
x_{in} = \alpha_{n-r}
\end{gather*}
re"sitve ena"cbe pa so
\begin{gather*}
x_{i1} = \ast + \ast \alpha_1 + \cdots + \ast \alpha_{n-r} \\
x_{i2} = \ast + \ast \alpha_1 + \cdots + \ast \alpha_{n-r} \\
\vdots \\
x_{ir} = \ast + \ast \alpha_1 + \cdots + \ast \alpha_{n-r}
\end{gather*}
%
\textsc{Primer:} Obravnavaj sistem linearnih ena"cb:
\begin{align*}
x_1 + 2x_2 + 3x_3 + 4x_4 &= 1\\
5x_1 + 4x_2 + 3x_3 + 2x_4 &= -1 \\
3x_1 + 4x_2 + 5x_3 + 6x_4 &= p
\end{align*}
glede na realen parameter $p$.

\begin{gather*}
\begin{bmatrix}
1 & 2 & 3 & 4 & | & 1 \\
5 & 4 & 3 & 2 & | & -1 \\
3 & 4 & 5 & 6 & | & p
\end{bmatrix} \to
\begin{bmatrix}
1 & 2 & 3 & 4 & | & 1 \\
0 & -6 & -12 & -18 & | & -6 \\
0 & -2 & -4 & -6 & | & p-3
\end{bmatrix} \to
\begin{bmatrix}
1 & 2 & 3 & 4 & | & 1 \\
0 & 1 & 2 & 3 & | & 1 \\
0 & -2 & -4 & -6 & | & p-3
\end{bmatrix} \to \\
\begin{bmatrix}
1 & 2 & 3 & 4 & | & 1 \\
0 & 1 & 2 & 3 & | & 1 \\
0 & 0 & 0 & 0 & | & p - 1
\end{bmatrix} \to
\begin{bmatrix}
1 & 0 & -1 & -2 & | & -1 \\
0 & 1 & 2 & 3 & | & 1 \\
0 & 0 & 0 & 0 & | & p-1
\end{bmatrix}
\end{gather*}
Sistem je neprotisloven natanko takrat, kadar je $p-1 = 0$, torej $p=1$.
\begin{align*}
x_1 - x_3 - 2x_4 &= -1 \\
x_2 + 2x_3 + 3x_4 &= 1 \\
x_3 &= \alpha_1 \\
x_4 &= \alpha_2
\end{align*}
Za $p=1$ torej velja:
\begin{align*}
x_1 &= -1 + \alpha_1 + 2\alpha_2 \\
x_2 &= 1 - 2\alpha_1 - 3\alpha_2 \\
x_3 &= \alpha_1 \\
x_4 &= \alpha_2
\end{align*}
kjer sta $\alpha_1$ in $\alpha_2$ realna parametra.

V resnici re"sujemo sistem $Ax = b$, kjer je $x = \begin{bmatrix}x_1 \\ x_2 \\ x_3 \\ x_4\end{bmatrix}$

Spomnimo se, da za mno"zico re"sitev $R$ velja $R = w + \ker A$, kjer je $w$ partikularna re"sitev. Torej velja
\begin{gather*}
x \in R \iff x = \begin{bmatrix}-1 \\ 1 \\ 0 \\ 0\end{bmatrix}
+ \alpha_1 \begin{bmatrix}1 \\ -2 \\ 1 \\ 0\end{bmatrix}
+ \alpha_2 \begin{bmatrix}2 \\ -3 \\ 0 \\ 1\end{bmatrix}
\\
R = \begin{bmatrix}-1 \\ 1 \\ 0 \\ 0\end{bmatrix}
+ \Lin \left\{
\begin{bmatrix}1 \\ -2 \\ 1 \\ 0\end{bmatrix},
\begin{bmatrix}2 \\ -3 \\ 0 \\ 1\end{bmatrix}
 \right\}
\end{gather*}
$\Rightarrow \begin{bmatrix}1 \\ -2 \\ 1 \\ 0\end{bmatrix}$ in $\begin{bmatrix}2 \\ -3 \\ 0 \\ 1\end{bmatrix}$ tvorita bazo jedra $\ker A$.
%
\subsubsection{Simultano re"sevanje sistemov z isto matriko koeficienotv}
\begin{gather*}
\begin{aligned}
AX^{(1)} &= B^{(1)}\\
AX^{(2)} &= B^{(2)} \\
&\cdots \\
AX^{(p)} &= B^{(p)}
\end{aligned}
\end{gather*}
Zapi"semo lahko
\begin{align*}
X &= \begin{bmatrix}X^{(1)} & \ldots & X^{(p)}\end{bmatrix} \in \OO^{n \times p} \\
B &= \begin{bmatrix}B^{(1)} & \ldots & B^{(p)}\end{bmatrix} \in \OO^{m \times p}
\end{align*}
Torej re"sujemo sistem $AX = B$.

Z Gaussom dobimo $\begin{bmatrix}A & | & B\end{bmatrix} \to \begin{bmatrix}A' & | & B'\end{bmatrix}$.

\textsc{Poseben primer:}

$A \in \OO^{n \times n}$, $A$ obrnljiva ($\Rightarrow \rang A = n$).

$A' = I$.
\begin{align*}
AX = B \iff IX = B' \Rightarrow B' &= X = A^{-1}B \\
B' &= A^{-1}B
\end{align*}
Vzamemo $B = I$, dobimo $B' = A^{-1}$.
\begin{equation*}
\begin{bmatrix}A & | & I\end{bmatrix} \to
\begin{bmatrix}I & | & A^{-1}\end{bmatrix}
\end{equation*}
%
\subsubsection{Sprememba baze}
$V$ v.p. nad $\OO$.

$\mathcal{V} = \{v_1, \ldots, v_n\}$ urejena baza

$\mathcal{V'} = \{v_1', \ldots, v_n'\}$ urejena baza
\begin{gather*}
v_j' = p_{1j} v_1 + p_{2j}v_2 + \cdots + p_{nj}v_n \\
P = \begin{bmatrix}p_{ij}\end{bmatrix} i,j = 1, \ldots, n \quad \in \OO^{n \times n}
\end{gather*}
$P$ je \emph{prehodna matrika} med $\V$ in $\V'$.

Skica, ki naj bi bila v zvezku zelo pomaga pri naslednjem sklepu
\begin{gather*}
P = \Phi_\V (\Phi_V')^{-1} \\
id_V v_j' = v_j' = p_{1j}v_1 + \cdots + p_{nj}v_n
\end{gather*}
%
\textsc{Poseben primer:}

$V = \OO^n$

$\V = \{e_1, e_2, \ldots, e_n\}$ standardna baza

$P$ - prehodna matrika
\begin{gather*}
\Rightarrow v_j' = p_{1j}e_1 + p_{2j}e_2 + \cdots + p_{nj}e_n = \begin{bmatrix}p_{1j} \\ p_{2j} \\ \vdots \\ p_{nj}\end{bmatrix} = P^{(j)} \\
\Rightarrow \V' = \{P^{(1)}, P^{(2)}, \ldots, P^{(n)}\}
\end{gather*}

Naj bo $\A \in \LL (V, U)$ in naj bosta $\V , \V '$ bazi $V$ in $\U, \U'$  bazi $U$. Naj $A \in \OO^{m \times n}$ pripada $\A$ glede na $\V, \U$ in naj $\A' \in \OO^{m \times n}$ pripada $\A$ glede na $\V', \U'$.

$P$ naj bo prehodna matrika med $\V$ in $\V'$, $Q$ pa naj bo prehodna matrika med $\U$ in $\U'$.

$P \in \OO^{n \times n}, \quad Q \in \OO^{m \times m}$

Zanima nas zveza med $A'$ in $A$.

V zvezku na tem mestu stoji (ali pa le"zi, odvisno v kak"sni poziciji bere"s zvezek) en velik diagram, ki komutira. Iz tega diagrama razberemo
\begin{equation*}
A' = Q^{-1}AP
\end{equation*}
%
\textsc{Poseben primer:}

$\A = A$, $\V, \U$ standardni bazi v $V = \OO^n$ in $U = \OO^m$

$\Rightarrow A' = Q^{-1}AP$ je matrika, ki pripada $A$ glede na urejeni bazi $\V' = \{P^{(1)}, \ldots, P^{(n)}\}$ in $\U = \{Q^{(1)}, \ldots, Q^{(m)}\}$
%
\subsubsection{Ekivalentnost matrik}
Naj bosta $A, B \in \OO^{m \times n}$

\textsc{Definicija:} $B$ je \emph{ekvivalentna} $A$ ($B \sim A$), kadar obstajata taki obrnljivi matriki $P, Q$, da velja
\begin{equation*}
B = Q^{-1}AP
\end{equation*}
%
$\sim$ je ekvivalen"cna relacija:
\begin{itemize}
	\item $A \sim A$ (refleksivnost) (za $Q, P$ vzamemo $I$)
	\item $A \sim B \Rightarrow B \sim A$ (simetri"cnost) ($B = Q^{-1}AP \Rightarrow A = \underbrace{QBP^{-1}}_{(Q^{-1})^{-1}B(P^{-1})}$)
	\item $A \sim B \land B \sim C \Rightarrow A \sim C$ (tranzitivnost) (dokaz za DN)
\end{itemize}
%
\textsc{Trditev:} Naj bo $\A \in \LL(V, U)$. Potem obstajata v $V$ in $U$ taki urejeni bazi, da ima matrika, ki pripada $\A$ v teh dveh bazah obliko
\begin{equation*}
A_0 = \begin{bmatrix}
1 & & & &\\
& \ddots && & \\
&& 1 & &\\
&&&&&
\end{bmatrix}
\end{equation*}
kjer je $r = \rang \A$.

\textsc{Dokaz:} I"s"cemo bazi $\{v_1, \ldots, v_n\}$ v $V$ in $\{u_1, \ldots, u_n\}$ v $U$, tako da bo veljalo:
\begin{align*}
\A v_1 &= u_1 \\
\A v_2 &= u_2 \\
&\cdots \\
\A v_r &= u_r \\
\A v_{r+1} &= 0 \\
&\cdots \\
\A v_n &= 0
\end{align*}
V $\im \A$ izberemo bazo $\{u_1, \ldots, u_r\}$. Izberemo "se praslike teh elementov $\{v_1, \ldots, v_r\} \in V$. Velja $\A v_j = u_j$ za $j = 1, \ldots, r$.

Raz"sirimo $\{u_1, \ldots, u_r\}$ do baze $\{u_1, \ldots, u_r, \ldots, u_m\}$ v. p. $U$.

Spomnimo se: $\dim (\ker \A) = n - \dim (\im \A) = n - r$.

Izberemo bazo $\ker \A: \{\underbrace{v_{r+1}, \ldots, v_n}_{n-r \text{ vektorjev}}\}$.

Trdimo, da je $\{v1, \ldots, v_n\}$ baza $V$. Zado"s"ca ugotoviti, da so ti vektorji linearno neodvisni (ker je $\dim V = n$).
\begin{gather*}
\alpha_1 v_1 + \cdots + \alpha_r v_r + \alpha_{r+1} v_{r+1} + \cdots + \alpha_n v_n = 0 \\
\alpha_1 \underbrace{\A v_1}_{u_1} + \cdots + \alpha_r \underbrace{\A v_r}_{u_r} + \alpha_{r+1} \underbrace{\A v_{r+1}}_0 + \cdots + \alpha_n \underbrace{\A v_n}_0 = 0 \tag{*} \\
\alpha_1 u_1 + \cdots + \alpha_r u_r = 0 \Rightarrow \alpha_1 = \cdots = \alpha_r = 0
\end{gather*}
"Ce to vstavimo v (*) dobimo:
\begin{equation*}
\alpha_{r+1} v_{r+1} + \cdots + \alpha_n v_n = 0 \Rightarrow \alpha_{r+1}  = \cdots = \alpha_n = 0
\end{equation*}
Torej so $v_1, \ldots, v_n$ res linearno neodvisni.

\hfill $\square$

\textsc{Posledica:} Vsaka matrika je ekvivalentna matriki oblike $A_0$.

\textsc{Dokaz:} $A$ razumemo kot prelikavo. Matrika, ki pripada $A$ glede na bazi $\{P^{(1)}, \ldots, P^{(n)}\}$ in $\{Q^{(1)}, \ldots, Q^{(m)}\}$, naj bo $A_0$. Ker je $A_0 = Q^{-1} A P$, sta matriki $A$ in $A_0$ ekvivalentni.

\textbf{Opomba:} 
\begin{gather*}
A_0 = Q^{-1}AP \iff AP = Q A_0 \\
AP = [Q^{(1)}, \ldots, Q^{(r)}, 0, \ldots, 0] \\
[AP^{(1)}, \ldots, AP^{(r)}, \ldots AP^{(n)}] = [Q^{(1)}, \ldots, Q^{(r)}, 0, \ldots, 0] \\
\iff AP^{(j)} = Q^{(j)} \text{ za } j = 1, \ldots r \\
AP^{(j)} = 0 \text{ za } j = r+1, \ldots, n
\end{gather*}
%
\textsc{Trditev:} Matriki $A, B \in \OO^{m \times n}$ sta ekvivalentni natanko takrat, kadar velja
\begin{equation*}
\rang A = \rang B
\end{equation*}
\textsc{Dokaz:}
\begin{itemize}
	\item[$(\Rightarrow)$] Naj bosta $A, B$ ekvivalentni. Vemo, da velja
	\begin{equation*}
	B = Q^{-1}AP
	\end{equation*}
	kjer sta $P$ in $Q$ obrnljivi matriki. Torej $B$ pripada preslikavi $A: \OO^n \to \OO^m$ glede na bazi $\{P^{(1)}, \ldots, P^{(n)}\}$ in $\{Q^{(1)}, \ldots, Q^{(m)}\}$. Po neki trditvi velja $\rang B = \rang A$.
	
	\item[$(\Leftarrow)$] Naj velja $\rang A = \rang B = r$. Vemo, da je $A$ ekvivalentna $A_0$. Prav tako je $B$ ekvivalentna $B_0$. Ker $A, B \in \OO^{m \times n}$ imata isti rang, zato je $A_0 = B_0$. Torej velja $A \sim A_0 = B_0 \sim B$. Iz tranzitivnosti sledi $A \sim B$.
\end{itemize}
\hfill $\square$
%
\subsubsection{Podobnost matrik}
Naj bo $\A \in \LL(V), \dim V = n$. In naj bosta $\V, \V'$ urejeni bazi $V$, ter $P$ prehodna matrika. Matrika $A \in \OO^{n \times n}$ naj pripada preslikavi $\A$ glede na bazo $\V$, matrika $A' \in \OO^{m \times n}$ pa naj pripada preslikavi $\A$ glede na bazo $\V'$. Vemo, da je zveza med $A'$ in $A$
\begin{equation*}
A' = P^{-1}AP
\end{equation*}
%
\textsc{Definicija:} Matrika $B \in \OO^{n \times n}$ je \emph{podobna} matriki $A \in \OO^{n \times n}$, kadar obstaja taka obrnljiva matrika $P \in \OO^{n \times n}$, da velja
\begin{equation*}
B = P^{-1}AP
\end{equation*}
Ozna"cimo z $B \mpod A$.

Relacija podobnosti je ekvivalen"cna relacija
\begin{itemize}
	\item refleksivnost: $A \mpod A$, za $P$ vzamemo $I$.
	\item simetri"cnost $B = P^{-1}AP \Rightarrow A = PBP^{-1} = (P^{-1})^{-1}BP^{-1}$
	\item tranzitivnost: $B \mpod B, B \mpod C \Rightarrow A = P^{-1} BP, B = Q^{-1} CQ \Rightarrow A = P^{-1} Q^{-1} CQP = (QP)^{-1} C (QP) \Rightarrow A \mpod C$
\end{itemize}
Vse matrike, ki pripadajo danemu endomorfizmu so med seboj podobne.

\subsubsection{Diagonalne matrike in diagonalizacija}
Naj bo $A \in \OO^{n \times n}, A = [a_{ij}] \quad i, j = 1, \ldots, n$. Pravimo, da je $A$ \emph{diagonalna}, kadar je $a_{ij} = 0$ za vsak $i \neq j$. Oblika $A$:
\begin{equation*}
A = \begin{bmatrix}
a_{11}&&&&& \\
&a_{22}&&&& \\
&&a_{33}&&& \\
&&& a_{44}&& \\
&&&& \ddots & \\
&&&&& a_{nn}
\end{bmatrix}
\end{equation*}
\textbf{Zapis:} $A = \diag (a_{11}, a_{22}, \ldots, a_{nn}) = \diag (b_1, b_2, \ldots, b_n)$.

Velja:
\begin{gather*}
\diag (a_1, \ldots, a_n) + \diag (b_1, \ldots, b_n) = \diag (a_1 + b_1, \ldots, a_n + b_n) \\
\diag (a_1, \ldots, a_n)  \diag (b_1, \ldots, b_n) = \diag(a_1 b_1, \ldots, a_n b_n)
\end{gather*}
%
\textsc{Definicija:} Endomorfizem $\A \in \LL(V)$ se da \emph{diagonalizirati}, kadar obstaja taka baza $\V \in V$, da je matrika $A$, ki pripada $\A$ v tej bazi, diagonalna.

Naj preslikavi $\A$ v bazi $\V$ pripada matrika $A = \diag (a_1, \ldots, a_n)$ in naj bo $\V = \{v_1, \ldots, v_n\}$. Takrat velja:
\begin{equation*}
\A v_j = 0 v_1 + 0 v_2 + \cdots + a_j v_j + \cdots + 0 v_n = a_j v_j \qquad j = 1, \ldots, n
\end{equation*}
Torej velja:
\begin{equation*}
\A v_j = a_j v_j \quad \forall j
\end{equation*}
%
\textsc{Definicija:} Vektor $x \in V \setminus \{0\}$ imenujemo \emph{lastni vektor} endomorfizma $\A \in \LL(V)$, kadar velja
\begin{equation*}
\A x = \lambda x
\end{equation*}
za kak"sen $\lambda \in \OO$. Velja, da je $\lambda$ enoli"cno dolo"cen z $\A$ in lastnim vektorjem $x$. $\lambda$ imenujemo \emph{lastna vrednost} endomorfizma $\A$, ki pripada danemu vektorju $x$.

\textsc{Dokaz} enoli"cnosti $\lambda$:

Naj velja $\A x = \lambda x$ in $\A x = \mu x$. Potem velja
\begin{equation*}
\lambda x = \mu x \Rightarrow (\lambda - \mu) \underbrace{x}_{\neq 0} = 0 \Rightarrow \lambda - \mu = 0 \Rightarrow \lambda = \mu
\end{equation*}

Naj bo $\A \in \LL(V), \lambda \in \OO$. $\lambda$ je lastna vrednost end. $\A$, kadar obstaja kak"sen lasten vektor $x$, da je 
\begin{equation*}
\A x = \lambda x
\end{equation*}
%
Poglejmo si mno"zico $\{x \in V: \A x = \lambda x\}$ za fiksiran $\lambda \in \OO$.
\begin{equation*}
\{x \in V: \A x = \lambda x\} = \{x \in V: (\A - \lambda \I) x = 0\} = \ker (\A - \lambda \I)
\end{equation*}
kjer je $\I = id_V$.

$\ker (\A - \lambda \I)$ se imenuje \emph{lastni podprostor} end. $\A$, ki pripada l. vrednosti $\lambda$.

"Ce je $x$ lastni vektor (za $\A$ in $\lambda$), potem je $\alpha x$ lastni vektor, "ce je $\alpha \neq 0$.

V $\ker (\A - \lambda \I)$ so vsi lastni vektorji end. $\A$, ki pripadajo l. vrednosti $\lambda$, poleg njih pa "se vektor 0.

\textsc{Trditev:} Endomorfizem $\A$ se da diagonalizirati natanko takrat, kadar obstaja baza v. p. $V$, sestavljena iz lastnih vektorjev end. $\A$. Pripadajo"ca diagonalna matrika ima na diagonali lastne vrednosti $\A$.

\textsc{Dokaz:} Naj se da $\A$ diagonaliziragi v $\{v_1, \ldots, v_n\} \Rightarrow \A v_j = a_j v_j \quad \forall j$. 

$A = \diag (a_1, \ldots, a_n)$

$v_j$ je l. vektor $\forall j (v_j \neq 0)$. $a_j$ je lastna vrednost za $\A$ in $v_j$.

Obratno: $\{v_1, \ldots, v_n\}$ je baza iz l. vektorjev $Av_j = \lambda_j v_j$. $A = \diag (\lambda_1, \ldots, \lambda_n)$

\textsc{Trditev:} Naj bodo $\lambda_1, \ldots, \lambda_h$ razli"cne lastne vrednosti endomorfizma $\A \in \LL(V)$ in $x_1, \ldots, x_h$ pripadajo"ci vektorji. Potem so $x_1, \ldots x_h$ linearno neodvisni.

\textsc{Dokaz:} Recimo, da so $x_1, \ldots, x_h$ linearno odvisni. Izberemo najmanj"si $j > 1$, tako da je $x_j = \alpha_1 x_1 + \cdots + a_{j-1} x_{j-1}$. Preslikamo z $\A$:
\begin{gather*}
\A x_j = \alpha_1 \A x_1 + \cdots + \alpha_{j-1} \A x_{j-1} \\
\Rightarrow \lambda_j x_j = \alpha_1 \lambda_1 x_1 + \cdots + a_{j-1} \lambda_{j-1} x_{j-1} \\
\end{gather*}
Velja tudi:
\begin{equation*}
\lambda_j x_j = \lambda_j \alpha_1 x_1 + \cdots + \lambda_j \alpha_{j-1}x_{j-1}
\end{equation*}
"Ce te ena"cbi od"stejemo dobimo:
\begin{gather*}
(\alpha_1 \lambda_1 - \lambda_j \alpha_1) x_1 + \cdots + (\alpha_{j-1} \lambda_{j-1} - \lambda_j \alpha_{j-1})x_{j-1} = 0 \\
\alpha_1 (\lambda_1 - \lambda_j) x_1 + \alpha_2 (\lambda_2 - \lambda_j) x_2 + \cdots + \alpha_{j-1} (\lambda_{j-1} - \lambda_j) x_{j-1} = 0
\end{gather*}
Zaradi izbire $j$ (minimalnost) so $x_1, \ldots, x_{j-1}$ linearno neodvisni.
\begin{gather*}
\Rightarrow \alpha_1 \underbrace{(\lambda_1 - \lambda_j)}_{\neq 0} = \cdots = \alpha_{j-1} \underbrace{(\lambda_{j-1} - \lambda_j)}_{\neq 0} = 0 \\
\Rightarrow \alpha_1 = \cdots = \alpha_{j-1} = 0
\end{gather*}
Torej je $x_j = 0$ $\rightarrow \leftarrow$.

Zato so $x_1, \ldots, x_h$ linearno neodvisni.

\textsc{Posledica:} "Ce ima endomorfizem $\A \in \LL(V)$ $n$ razli"cnih lastnih vrednosti, kjer je $n = \dim (V)$, potem se da $\A$ diagonalizirati.

\textsc{Dokaz:} $\lambda_1, \ldots, \lambda_n$ razli"cne lastne vrednosti.
\begin{gather*}
\A x_j = \lambda_j x_j, \qquad j = 1, \ldots, n \\
x_j \neq 0 \quad \forall j
\end{gather*}
$\Rightarrow \{x_1, \ldots, x_n\}$ je baza $V$ sestavljena iz lastnih vektorjev. Zato se da $\A$ diagonalizirati.

Naj bo $A \in \OO^{n \times n} = \LL(\OO^{n})$. $A$ se da diagonalizirati, kadar je $A$ podobna diagonalni matriki:
\begin{equation*}
\exists P \in \OO^{n \times n} \text{ obrnljiva}: P^{-1}AP = D = \diag(d_1, \ldots, d_n)
\end{equation*}
Velja:
\begin{gather*}
AP = PD \\
\Rightarrow AP^{(j)} = PD^{(j)} \quad j = 1, \ldots, n \\
PD^{(j)} = P(d_j e_j) = d_j Pe_j \\
\Rightarrow AP^{(j)} = d_jP^{(j)} \quad \forall j
\end{gather*}
$\Rightarrow P^{(1)}, \ldots, P^{(n)}$ so lastni vektorji matrike $A$. $d_1, \ldots, d_n$ so pripadajo"ce lastne vrednosti.
%
\subsubsection{Iskanje lastnih vrednosti in lastnih vektorjev}
Naj bo $\A \in \LL(V), \dim V = n$

$Ax = \lambda x \iff x \in \ker (\A - \lambda \I)$

$\lambda$ je lastna vrednost endomorfizma $\A \iff \ker (\A - \lambda \I) \neq \{0\} \iff \A - \lambda \I$ ni bijekcija (nima inverza) $\iff \rang(\A - \lambda \I) < n$.