\deff Naj bo $a_n$ realno zaporedje. Nekson"cno formalno vrsto
\begin{equation*}
a_1 + a_2 + a_2 + \ldots + a_n + \ldots
\end{equation*}
imenujemo \emph{"stevilska vrsta}. "Clen $a_n$ imenujemo splo"sni "clen "stevilske vrste. Oznai"cimo:
\begin{equation*}
a_1 + a_2 + a_3 + \ldots + a_n + \ldots = \sumalln{1} a_n
\end{equation*}
\textsc{Primeri:}
\begin{enumerate}[1)]
	\item geometrijska vrsta
	\begin{equation*}
	1 + \dfrac{1}{2} + \dfrac{1}{4} + \ldots + \dfrac{1}{2^{n-1}} = \sumalln{0} \dfrac{1}{2^n}
	\end{equation*}
	\textbf{Osnovna lastnost:} kovocient zaporednih "clenov je konstanten
	
	\item 
	\begin{equation*}
	1 - 1 + 1 - 1 \ldots = \sumalln{1} (-1)^n
	\end{equation*}
\end{enumerate}
\deff Dana je "stevilska vrsta $\sumalln{1} a_n$. Zaporedje
\begin{align*}
s_1 &= a_1 \\
s_2 &= a_1 + a_2 \\
s_n &= \sum_{j = 1}^{n} a_j
\end{align*}
imenujemo \emph{zaporedje delnih vsot}. "Ce zaporedje delnih vsot konvergira, pravimo, da vsrta konvergira. V tem primeru limito zaporedja delsnih vsto $s$ imenujemo \emph{vsota vrste} in pi"semo
\begin{equation*}
\sumalln{1} a_n = s
\end{equation*}
"Ce zaporedje delnih vsto divergira, re"cemo, da vrsta divergira.

\textbf{Opomba:} Zapis $\sumalln{1} a_n = s$ pomeni:
\begin{enumerate}
	\item $\sumalln{1} a_n$ konvergira
	\item njena vsota je $s$
\end{enumerate}
\textsc{Primeri:}
\begin{enumerate}[1)]
	\item Obrovnavaj konvergenco geometrijske vrste
	\begin{gather*}
	\sumalln{0} a q^n = a + aq + aq^2 + \ldots + aq^n + \ldots \\
	s_n = a + aq + aq^n + \ldots + aq^{n-1} = a (1 + q + q^2 + \ldots + q^{n-1}) = a \dfrac{1-q^n}{1-q}
	\end{gather*}
	Velja za $q \neq 1$. Oklepaj smo razre"sili po formuli, ki smo jo na"sli v spominu:
	\begin{equation*}
	a^n - b^n = (a-b)(a^{n-1} + a^{n-1}b + \ldots + b^{n-1})
	\end{equation*}
	\begin{equation*}
	\limninf s_n = \limninf a\dfrac{a-q^n}{1-q} = \dfrac{a}{a-1}
	\end{equation*}
	"Ce $|q| < 1 \Rightarrow q^n \to 0$
	
	Za $q > 1: a\dfrac{1-q^n}{1-q} \stackrel{n \to \infty}{\longrightarrow} \infty$ vrsta ne konvergira.
	
	Za $q < -1: a\dfrac{1 - q^n}{1-q}$ ne konvergira
	
	Za $q = -1: a \dfrac{1 - (-1)^n}{2}$ ima dve stekali"s"ci, zato divergira.
	
	Za $q = 1: s_n = na$ divergira (ker konvergira proti $\pm \infty$).
	
	Geiom vrsta $\sumalln{0} aq^n$ konvergira natanko takrat, kadar $|q| < 1$. V tem primeru velja:
	\begin{equation*}
	\sumalln{0}aq^n = \dfrac{a}{1-q}
	\end{equation*}
	
	\item Obravnavaj konvergenco $\sumalln{1} \dfrac{1}{n(n+1)}$
	\begin{equation*}
	s_n = a_1 + a_2 + \ldots + a_n
	\end{equation*}
	$\sumalln{1}a_n$ konvergira po definiciji natanko takrat, kadar zaporedje $s_n$ konvergira.
	\begin{equation*}
	s_n = \dfrac{1}{1 \cdot 2} + \dfrac{1}{2 \cdot 3} + \ldots + \dfrac{1}{n(n+1)}
	\end{equation*}
	Razceptimo na delne ulomke:
	\begin{equation*}
	\dfrac{1}{k(k+1)} = \dfrac{a}{k} + \dfrac{b}{k+1} = \dfrac{1}{k} + \dfrac{-1}{k+1}
	\end{equation*}
	Razpi"semo $s_n$ in dobimo:
	\begin{equation*}
	s_n = \dfrac{1}{1 \cdot 2} + \dfrac{1}{2 \cdot 3} + \ldots + \dfrac{1}{n(n+1)} = \dfrac{1}{1} - \dfrac{1}{2} + \dfrac{1}{2} - \dfrac{1}{3} + \dfrac{1}{3} - \ldots + \dfrac{1}{n} - \ldots \dfrac{1}{n+2} = \dfrac{1}{1} - \dfrac{1}{n+1}
	\end{equation*}
	Torej velja:
	\begin{equation*}
	\limninf s_n = \limninf\left(1 - \dfrac{1}{n+1}\right) = 1
	\end{equation*}
	Vrsta je konvergentna in $\sumalln{1} \dfrac{1}{n(n+1)} = 1$
	
	\item Obravnavaj konvergenco $\sumalln{1} 1$
	
	Vrsta divergira, ker $s_n = 1 + 1 + \ldots + 1 = n$ divergira.
\end{enumerate}
\textsc{Opomba:} Naj bo vrsta $\sumalln{1}a_n$ konvergentna. Potem je zaporedje $a_n$ konvergentno in $\limninf a_n = 0$.

\textsc{Dokaz:} "Ce je vrsta $\sumalln{1} a_n$ konvergenta, je zaporedje delnih vsot $s_n$ konvergentno.
\begin{align*}
s_n &= a_1 + a_2 + \ldots + a_n \\
s_{n+1} &= a_1 + a_2 + \ldots + a_n + a_{n+1} = s_n + a_{n+1}
\end{align*}
Torej:
\begin{equation*}
a_{n+1}  = s_{n+1} - s_n
\end{equation*}
$\Rightarrow a_{n+1}$ je konvergentno in velja:
\begin{equation*}
\limninf a_{n+1} = \limninf s_{n+1} - \limninf s_n = 0
\end{equation*}
\textsc{Trditev:} (Cauchyjev pogoj za vrste) Vrsta $\sumalln{1}a_n$ konvergira natanko tedaj, kadar velja:
\begin{equation*}
\forall \varepsilon > 0 \exists n_0 \in \NN \forall n \in \NN, n \geq n_0 \forall k \in \NN: |a_{n+1} + a_{n+2} + \ldots + a_{n+k}| < \varepsilon
\end{equation*}
\textsc{Dokaz:} $\sumalln{1}a_n$ konvergira po definiciji natanko takrat, kadar zaporedje delnih vsot $s_n$ konvergira. Po definiciji je zporedje $s_n$ Cauchyjevo:
\begin{equation*}
\forall \varepsilon > 0 \exists n_0 \in \NN \forall n, m \geq n_0, n, m \in \NN: |s_n - s_m| < \varepsilon
\end{equation*}
Naj bo $m \geq n$:
\begin{equation*}
|a_{n+1} + a_{n+2} + \ldots + a_m| < \varepsilon
\end{equation*}
Lahko si izberemo $k = m-n$ in trditev velja.

\textsc{Trditev:} Dana je "stevilska vrsta $\sumalln{1} a_n$
\begin{enumerate}[(1)]
	\item "Ce vrsta $\sumalln{1} a_n$ konvergira, potem $\forall m \in \NN$ konvergira vrsta $\sum_{n=m}^{\infty}a_n$.
	\item "Ce za nek $m \in \NN$ vrsta $\sum_{n=m}^{\infty} a_n$ konvergira, potem konvergira vrsta $\sumalln{1}a_n$
\end{enumerate}
\textsc{Ideja dokaza:} 
\begin{align*}
\sum a_n \text{ konvergira } &\Rightarrow s_n = a_1 + a_2 + \ldots + a_n  \text{ konvergira.} \\
\sum_{n=m}a_n &\rightsquigarrow t_n = a_m + a_{m+1} + \ldots + a_{m+n}
\end{align*}
\begin{align*}
s_{m+k} &= \underbrace{a_1 + a_2 + a_{m-1}}_{A} + \underbrace{a_m + \ldots + a_{m+k}}_{t_{k + 1}} \\
s_{m+k} &= A + t_{k+1}
\end{align*}
"Ce konvergira eno zaporedje, konvergira tudi drugo zaporedje.

\textsc{Trditev:} Denima, da sta $\sumalln{1}a_n$ in $\sumalln{1}b_n$ konvergentni vrsti, $c \in \RR$. Potem konvergirajo tudi:
$\sumalln{1} (ca_n)$, $\sumalln{1}(a_n + b_n)$ in $\sumalln{1}(a_n - b_n)$ ter velja:
\begin{align*}
\sumalln{1}ca_n &= c \sumalln{1}a_n \\
\sumalln{1}(a_n + b_n) &= \sumalln{1}a_n + \sumalln{1}b_n \\
\sumalln{1}(a_n - b_n) &= \sumalln{1}a_n -\sumalln{1} b_n
\end{align*}
\textbf{Opomba:} KOnvergentne vrste sestavljajo vektorski prostor (nad obsegom).

\textsc{Dokaz:} "Ce $\sumalln{1} a_n$ konvergira, konvergira tudi njeno zaporedje delnih vsot $s_n = a_1 + a_2 + \ldots + a_n$, zato konvergira $c s_n = ca_1 + ca_2 + \ldots + ca_n$ in velja ustrezna zveza za limito.

\textsc{Primer:} Obravnavaj konvergenco vrsto $\sumalln{1}\dfrac{1}{n}$ (\emph{harmoni"cna vrsta}).

$\sumalln{1}\dfrac{1}{n}$ divergira.

$s_n = 1 + \dfrac{1}{2} + \dfrac{1}{3} + \ldots + \dfrac{1}{n}$ je \dashuline{neomejeno}
\begin{align*}
s_4 &= 1 + \dfrac{1}{2} + \underbrace{\dfrac{1}{3} + \dfrac{1}{4}}_{\geq 2 \frac{1}{4} = \frac{1}{2}} \geq \dfrac{3}{2} + \dfrac{1}{2}\\
s_8 &= s_4 + \underbrace{\dfrac{1}{5} + \dfrac{1}{6} + \dfrac{1}{7} + \dfrac{1}{8}}_{\geq 4 \frac{1}{8}} \geq \dfrac{3}{2} + \dfrac{1}{2} + \dfrac{1}{2}
\end{align*}
Opazimo vzorec:
\begin{equation*}
s_{2^k} \geq \dfrac{3}{2} + \dfrac{1}{2}(k-1)
\end{equation*}
To formulo lahko doka"zemo z indukcijo. $s_k$ je neomejeno in vrsta divergira.

\subsection{Vrte z nenegativnimi "cleni}
\deff Denimo, da je $\sumalln{1}a_n$ vrsta z nenegativnimi "cleni. Potem je zaporedje delnih vsot $s_n$ nara"s"cajao"ce ($s_{n+1} = s_n + \underbrace{a_{n+1}}_{\geq 0} \geq s_n$)

\textsc{Trditev:} Vrsta $\sumalln{1}a_n$ z nenegativnimi "cleni konvergira natanko takrat, kadar je njeno zaporedje delnih vsot $s_n$ navzgor omejeno.

\textsc{Izrek:} (primerjalni kriterij za konvergenco vrst): Naj bosta $\sumalln{1}a_n$ in $\sumalln{1}b_n$ vrsti z nenegativnimi "cleni in naj velja:
\begin{equation*}
\forall n \in \NN: a_n \leq b_n
\end{equation*}
\begin{enumerate}[(1)]
	\item "Ce vsota $\sumalln{1}b_n$ konvergira, potem $\sumalln{a_n}$ konvergira.
	\item "Ce vsota $\sumalln{1}a_n$ divergira, potem $\sumalln{b_n}$ divergira.
\end{enumerate}
\textbf{Opomba:} Pravimo, da je $\sumalln{1}b_n$ \emph{majoranta} za $\sumalln{1} a_n$.

\textsc{Primer:} $\sumalln{1}\dfrac{1}{n^2}$ ali konvergira?
\begin{equation*}
\dfrac{1}{n(n+1)} = \dfrac{1}{n^2 + n} \geq \dfrac{1}{n^2 + 2n + 1}  = \dfrac{1}{(n+1)^2}
\end{equation*}
$\sumalln{1}\dfrac{1}{(n+1)^2}$ konvergira po primerjalnem kriteriju. Ker je to rep zaporedja $\sumalln{1}\dfrac{1}{n^2}$ konvergira.

\textsc{Dokaz} trditve:
\begin{align*}
s_n &= a_1 + \ldots + a_n \\
t_n &= b_1 + \ldots + b_n
\end{align*}
Po predpostavki $s_n \leq t_n$ za vsak $n \in \NN$.
\begin{enumerate}[(1)]
	\item "Ce vrsta $\sumalln{1}b_n$ konvergira, zaporedje $t_n$ konvergir, t.j.: $t_n$ je navzgor omejeno:
	\begin{equation*}
	\exists M \in \RR \forall n \in \NN: t_n \leq M
	\end{equation*}
	Sledi: $s_n \leq M$ za vsak $n \in \NN$, torej $s_n$ in s tem $\sumalln{1}a_n$ konvergira.
	
	\item "Ce $\sumalln{1}a_n$ divergira, potem je zaporedje $s_n$ navzgor neomejeno. Ker $s_n \leq t_n$ je tudi $t_n$ navzgor neomejeno, zato $\sumalln{1} b_n$ divergira.
\end{enumerate}
\textsc{Primer:} Obravnavaj konvergenco $\sumalln{1}\dfrac{1}{n^p}$ v odvisnosti od $p \in \RR$.
\begin{itemize}
	\item[$p = 1$] Vemo, da je harmoni"cna vrsta in divergira.
	\item[$p \leq 1$] $\dfrac{1}{n^p} \geq \dfrac{1}{n}$ odtod po primerjalnem kriteriju sledi, da $\sumalln{1}\dfrac{1}{n^p}$ divergira.
	\item[$p > 1$] \dashuline{$\sumalln{1}\frac{1}{n^p}$ konvergira}
	
	Podoben dokaz kot pri harmoni"cni vrsti (samo druge ocene)
	\begin{align*}
	s_4 &= \dfrac{1}{1^p} + \dfrac{1}{2^p} + \underbrace{\dfrac{1}{3^p} + \dfrac{1}{4^p}}_{\leq 2\frac{1}{2^p} = \frac{1}{2^{p-1}}} \leq 1 + \dfrac{1}{2^p} + \dfrac{1}{2^{p-1}} \\
	s_8 &= s_4 + \underbrace{\dfrac{1}{5^p} + \dfrac{1}{6^p} + \dfrac{1}{7^p} + \dfrac{1}{8^p}}_{\leq 4 \frac{1}{4^p} = \frac{1}{4^{p-1}}} \leq 1 + \dfrac{1}{2^p} + \dfrac{1}{2^{p-1}} + \dfrac{1}{4^{p-1}} \\
	\end{align*}
	\begin{multline*}
		s_{2^k} = 1 + \dfrac{1}{2^p} + \dfrac{1}{2^{p-1}} + \dfrac{1}{4^{p-1}} + \ldots + \dfrac{1}{(2^{-1})^{p-1}} \leq \\
		\leq \underbrace{1 + \dfrac{1}{2^p} + \dfrac{1}{2^{p-1}} + \dfrac{1}{4^{p-1}} + \ldots + \dfrac{1}{(2^{-1})^{p-1}} + \ldots}_{\text{zgornja meja}}
	\end{multline*}
	Opazimo, da je od nekega "clena naprej to geometrijska vrsta z $q = \frac{1}{2^{p-1}} < 1$, kar pomeni, da ima vsoto. Torej je $s_{2^k}$ navzgor omejeno, ker je nara"s"cajo"ce je tudi $s_n$ navzgor omejeno in je zato $s_n$ konvergentno.
\end{itemize}
%
\textsc{Izrek:} (\emph{kvocientni} kriterij ali \emph{d'Alembertov} kriterij) Naj bo $\sumalln{1}a_n$ vrsta s pozitivnimi "cleni. Sestavimo zaporedje:
\begin{equation*}
d_n = \dfrac{a_{n+1}}{a_n}
\end{equation*}
\begin{enumerate}[1)]
	\item "Ce obstaja $q \in (0, 1)$, da velja $d_n \leq q \forall n \in \NN$, potem vrsta $\sumalln{1}a_n$ konvergira.
	\item "Ce velja $d_n \geq 1 \forall n \in \NN$, potem vrsta $\sumalln{1}a_n$ divergira.
\end{enumerate}
"Ce obstaja $\limninf d_n = d$, potem velja:
\begin{enumerate}[1')]
	\item "Ce je $d < 1$, potem $\sumalln{1} a_n$ konvergira.
	\item "Ce je $d > 1$, potem $\sumalln{1} a_n$ divergira.
\end{enumerate}
\textbf{Opombi:}
\begin{itemize}
	\item "Ce je $d = 1$, kriterij ne da odgovora.
	\item "Ce je $\limsupninf d_n < 1$, potem $\sumalln{1} a_n$ konvergira.
\end{itemize}
\textsc{Primer:}
\begin{itemize}
	\item $\sumalln{1}\dfrac{2^n}{n!}$
	\begin{gather*}
	d_n = \dfrac{a_{n+1}}{a_n} = \dfrac{\frac{2^{n+1}}{(n+1)!}}{\frac{2^n}{n!}} = \dfrac{2^{n+1}n!}{(n+1)!2^n} = \dfrac{2}{n+1} \\
	d = \limninf d_n = \limninf\dfrac{2}{n+1} = 0
	\end{gather*}
	$\Rightarrow$ vrsta $\sumalln{1}\dfrac{2^n}{n!}$ konvergira.
	
	\item Obravnavaj konvergenco vrste $\sumalln{1} nx^n$, $x > 0$
	\begin{gather*}
		d_n = \dfrac{a_{n+1}}{a_n} = \dfrac{(n+1)x^{n+1}}{nx^n} = \dfrac{(n+1)x}{n} \\
		d = \limninf d_n = \limninf \dfrac{(n+1)x}{n} = x \limninf \dfrac{n+1}{n} = x
	\end{gather*}
	\begin{itemize}
		\item "Ce $x < 1$ vrsta konvergira.
		\item "Ce $x > 1$ vrsta divergira.
		\item "Ce $x = 1: \sumalln{1}n$ divergira (ker "cleni ne konvergirajo proti 0).
	\end{itemize}
\end{itemize}
\textsc{Dokaz:} 
\begin{enumerate}[1)]
	\item Denimo, da je $d_n \leq q < 1$ za vse $n \in \NN$
	\begin{gather*}
	\dfrac{a_{n+1}}{a_n} \leq q \forall n \\
	a_{n+1} \leq q a_n \leq q \cdot q a_{n-1} \leq q^3 a_{n-2} \leq \ldots \leq q^n a_1 \\
	\sumalln{1}a_{n+1} \leq \sumalln{1}q^n a_1
	\end{gather*}
	Po primerjalnem kriteriju vrsta $\sumalln{1}a_{n+1}$ konvergira $\Rightarrow \sumalln{1}a_n$ konvergira.
	
	\item $d_n \geq q \forall n \in \NN$
	\begin{align*}
	\dfrac{a_{n+1}}{a_n} &\geq 1 \quad \forall n \in \NN \\
	a_{n+1} &\geq a_n \quad \forall n \in \NN
	\end{align*}
	Zaporedje $d_n$ je nara"s"cajo"ce zaporedje pozitivnih "stevil. Zato "cleni $a_n$ ne konvergirajo proti $0$. Torej vrsta divergira.
\end{enumerate}
Iz 1 sledi 1', iz 2 sledi 2'. Dokaz je bil za DN. Si naredil/a?

\textsc{Izrek:}(\emph{korenski} kriter ali \emph{Cauchyjev} kriterij) Naj bo $\sumalln{1}a_n$ vrsta z nenegativnimi "cleni. Sestavimo zaporedje
\begin{equation*}
c_n = \sqrt[n]{a_n}
\end{equation*}
Tedaj velja:
\begin{enumerate}[1)]
	\item "Ce obstaja $q \in (0, 1)$, da velja $c_n \leq q$ za vse $n \in \NN$, potem vrsta $\sumalln{1} a_n$ konvergira.
	\item "Ce velja $c_n \geq 1 \forall n$, potem vrsta $\sumalln{1}a_n$ divergira.
\end{enumerate}
"Ce obstaja $\limninf c_n = c$, potem velja:
\begin{enumerate}[1')]
	\item "Ce je $c < 1$, potem $\sumalln{1}a_n$ konvergira.
	\item "Ce je $c > 1$, potem $\sumalln{1}a_n$ divergira.
\end{enumerate}
\textbf{Opombi:}
\begin{itemize}
	\item "Ce je $c = 1$, kriterij ne da odgovora.
	\item "Ce je $\limsupninf c_n < 1$, potem $\sumalln{1}a_n$ konvergira.
\end{itemize}
\textsc{Dokaz:}
\begin{enumerate}[1)]
	\item Naj bo $q \in (0, 1), c_n \leq q \forall n \in \NN$.
	\begin{align*}
	c_n &\leq q \\
	\sqrt[n]{c_n} &\leq q \\
	a_n &\leq q^n
	\end{align*}
	"Cleni v vrsti $\sumalln{1}a_n$ so majonirani s "cleni konvergentne geometrijske vrste $q^n$, zato $\sumalln{1}a_n$ konvergira po primerjalnem kriteriju.
	
	\item Naj bo $c_n \geq 1, \forall n \in \NN$
	\begin{align*}
	c_n &\geq 1 \\
	\sqrt[n]{a_n} &\geq 1
	a_n &\geq 1^n
	\end{align*}
	Vrsta divergira, ker "cleni ne konvergirajo proti 0.
\end{enumerate}
Tako kot pri prej"snjem primeru iz 1 sledi 1' in iz 2 sledi 2'. Ponovno je bil dokaz za DN.

\textsc{Primer:}
\begin{itemize}
	\item Obravnavaj konvergenco vrste $\sumalln{1}\left(\dfrac{x}{n}\right)^n, x > 0$
	\begin{gather*}
		c_n = \sqrt[n]{\left(\dfrac{x}{n}\right)^n} = \dfrac{x}{n} \\
		c = \limninf c_n = \limninf \dfrac{x}{n} = x \limninf \dfrac{1}{n} = 0
	\end{gather*}
	Po korenskem kriteriju vrsta $\sumalln{1}\left(\dfrac{x}{n}\right)^n$ konvergira za vse $x > 0$.
	
	\item $\sumalln{1}\dfrac{1}{n^2}$. "Ze od prej vemo, da konvergira, ampak si vseeno poglejmo kvocientni in korenski kriterij.
	\begin{itemize}
		\item \textbf{Kvocientni kriterij:}
		\begin{equation*}
		d_n = \dfrac{a_{n+1}}{a_n} = \dfrac{n^2}{(n+1)^2} \stackrel{\toinf{n}}{\longrightarrow} 1
		\end{equation*}
		\item \textbf{Korenski kriterij:}
		\begin{equation*}
		c_n = \sqrt[n]{\dfrac{1}{n^2}} = \dfrac{1}{\left(\sqrt[n]{n}\right)^2} \stackrel{\toinf{n}}{\longrightarrow} 1
		\end{equation*}
	\end{itemize}
\end{itemize}
%
\textsc{Izrek:} (\emph{Raabejev} kriterij) Naj bo $\sumalln{1}a_n$ vrsta s pozitivnimi "cleni. Sestavimo zaporedje:
\begin{equation*}
r_n = n\left(\dfrac{a_n}{a_{n+1}} - 1\right)
\end{equation*}
\begin{enumerate}[1)]
	\item "Ce obstaja $r > 1$, da velja $r_n > r \quad \forall n \in \NN$, potem vrsta $\sumalln{1}a_n$ konvergira.
	\item "Ce velja $r_n \leq 1 \forall n$, potem $\sumalln{1}a_n$ divergira.
\end{enumerate}
"Ce obstaja $\limninf r_n = R$, potem velja:
\begin{enumerate}[1')]
	\item "Ce je $R > 1$, potem $\sumalln{1}a_n$ konvergira.
	\item "Ce je $R < 1$, potem $\sumalln{1}a_n$ divergira.
\end{enumerate}
\textbf{Opombi:}
\begin{itemize}
	\item "Ce je $R=1$ kriterij ne da odgovora.
	\item "Ce je $\limsupninf r_n > 1$, potem $\sumalln{1}a_n$ konvergira.
\end{itemize}
\textsc{Dokaz:}
\begin{enumerate}[1)]
	\item Naj bo $r_n \geq r > 1 \quad \forall n$
	\begin{align*}
	n\left(\dfrac{a_n}{a_{n+1}} - 1\right) &\geq r \\
	\dfrac{a_n}{a_{n+1}} -1 &\geq \dfrac{r}{n} \\
	\dfrac{a_n}{a_{n+1}} &\geq \dfrac{r}{n} + 1
	\end{align*}
	\textbf{Trditev:} "Ce je $s \in \QQ \quad 1 < s < r$, potem velja $1 + \dfrac{r}{n} > \left(1 + \dfrac{1}{n}\right)^s$ \\
	\textbf{Dokaz trditve:} $s = \dfrac{p}{q}, p, q \in \NN$
	\begin{gather*}
	\begin{aligned}
	1 + \dfrac{r}{n} &> \left(1+\dfrac{1}{n}\right)^{\frac{p}{q}} \\
	\left(1 + \dfrac{r}{n}\right)^q &> \left(1 + \dfrac{1}{n}\right)^p
	\end{aligned} \\
	\dashuline{\left(1 + \dfrac{r}{n}\right)^q - \left(1 + \dfrac{1}{n}\right)^p > 0}
	\end{gather*}
	\begin{multline*}
		\left(1 + \dfrac{r}{n}\right)^q - \left(1 + \dfrac{1}{n}\right)^p =\\=
		 1 + q\dfrac{r}{n} + \dfrac{1}{n^2}(\cdots) - (1 + \dfrac{p}{n} + \dfrac{1}{n^2}(\cdots)) = \dfrac{1}{n}(qr - p) + \dfrac{1}{n^2}(\cdots) = \\
		 = \underbrace{\dfrac{1}{n}}_{>0}(\underbrace{(qr - p)}_{>0} + \dfrac{1}{n}(\cdots)) > 0
	\end{multline*}
	$qr - p > 0$, ker $r>\dfrac{p}{q} \Rightarrow qr > p$. \quad $\dfrac{1}{n}(\cdots) > 0$ za dovolj velik $n$, ker je limita tega "clena 0.
	%
	\begin{align*}
	\dfrac{a_n}{a_{n+1}} &\geq 1 + \dfrac{r}{n} > \left(1+\dfrac{1}{n}\right)^s = \left(\dfrac{n+1}{n}\right)^s \\
	a_{n+1} &\leq a_n \left(\dfrac{n}{n+1}\right)^s \text{ za dovolj velik $n$} \\
	a_{n_0 +1} &\leq a_{n_0} \left(\dfrac{n_0}{n_0 + 1}\right)^s \\
	a_{n_0 +2} &\leq a_{n_0+1} \left(\dfrac{n_0+1}{n_0 + 2}\right)^s < a_{n_0} \left(\dfrac{n_0}{n_0 + 1}\right)^s \cdot \left(\dfrac{n_0}{n_0 + 1}\right)^s = a_{n_0} \dfrac{n_0^s}{(n_0 + 2)^s} \\
	a_{n_0 + 3} &\leq a_{n_0 + 2}(\cdots) < a_{n_0} \dfrac{n_0^s}{(n_0 + 3)^s} \\
	a_{n_0 + k} &\leq a_{n_0} \dfrac{n_0^s}{(n_0 + k)^s} \quad \forall k \in \NN
	\end{align*}
	\begin{equation*}
	\sum_{k=1}^{\infty}a_{n_0}\dfrac{n_0^s}{(n_0 + k)^s} = a_{n_0}n_0^s \sum_{k = 1}^{\infty}\dfrac{1}{(n_0+k)^s}
	\end{equation*}
	Ta vrsta konvergira, ker je $s > 1$ in je ostanek vrste $\sum_{k=1}^{\infty}\dfrac{1}{k^s}$. Torej po primerjalnem kriteriju konvergira tudi vrsta $\sumalln{1}a_n$.
	
	\item Naj bo $r_n \leq 1$ za vse $n \in \NN$.
	\begin{align*}
	n\left(\dfrac{a_n}{a_{n+1}} - 1\right) &\leq 1 \\
	\dfrac{a_n}{a_{n+1}} - 1 &\leq \dfrac{1}{n} \\
	\dfrac{a_n}{a_{n+1}} &\leq 1 + \dfrac{1}{n} = \dfrac{n+1}{n} \\
	a_{n+1} &\geq \dfrac{n}{n+1}a_n \\	
	a_{n+2} &\geq \dfrac{n+1}{n+2}a_{n+1} \geq \dfrac{n+1}{n+2} \cdot \dfrac{n}{n+1}a_n = \dfrac{n}{n+2}a_n \\
	a_{n+k} &\geq \dfrac{n}{n+k}a_n \\
	a_{k+1} &\geq \dfrac{1}{k+1}a_1
	\end{align*}
	\begin{equation*}
	\sum_{k = 1}^{\infty}\dfrac{1}{k+1}a_1 = a_1 \sum_{k=1}^{\infty}\dfrac{1}{k+1}
	\end{equation*}
	Je harmoni"cna vrsta, ki divergira. Po primerjalnem kriteriju divergira tudi $\sumalln{1}a_n$.
\end{enumerate}
\textsc{Primer:} Obravnavaj konvergenco $\sumalln{1}\dfrac{n!}{(x+1)(x+2)\ldots(x+n)}, x > 0$.
\begin{itemize}
	\item \textbf{Kvocientni:}
	\begin{equation*}
	d_n = \dfrac{a_{n+1}}{a_n} = \dfrac{(x+1)\ldots(x+n)(n+1)!}{(x+1)(x+2)\ldots(x+n+1)n!} = \dfrac{n+1}{x+n+1}
	\end{equation*}
	$\limninf d_n = 1$ in kriterij ne da odgovora
	\item \textbf{Raabejev:}
	\begin{gather*}
	r_n = n\left(\dfrac{a_n}{a_{n+1}}-1\right) = n\left(\dfrac{x+n+1}{n+1} - 1\right) = \dfrac{nx}{n+1} \\
	\limninf r_n = x
	\end{gather*}
	\begin{itemize}
		\item "Ce je $x > 1$, potem vrsta konvergira.
		\item "Ce je $x < 1$, potem vrsta divergira.
		\item "Ce je $x = 1: \sumalln{1}\dfrac{n!}{2\cdot3\cdot n+1} = \sumalln{1}\dfrac{1}{n+1}$ divergira.
	\end{itemize}
\end{itemize}
%
\subsection{Absolutna konvergenca}
\deff Naj bo $\sumalln{1}a_n$ "stevilska vrsta. Pravimo, da vrsta $\sumalln{1}a_n$ \emph{absolutno konvergira}, "ce konvergira vrsta $\sumalln{1}|a_n|$. \\
\textbf{Opomba:} Za vrste z nenegativnimi "cleni, je absolutna konvergenca isto kot konvergenca.

\textsc{Izrek:} Naj bo $\sumalln{1}a_n$ "stevilska vrsta. "Ce $\sumalln{1}a_n$ absolutno konvergira, potem $\sumalln{1}a_n$ konvergira.

\textsc{Dokaz:} Vemo: $\sumalln{1}a_n$ konvergira natanko takrat, ko $\sumalln{1}a_n$ izpolnjuje Cauchyjev pogoj. Spomnimo se Cauchyjevega pogoja za vrste:
\begin{equation*}
\forall \varepsilon > 0 \exists n_0 \in \NN \forall n \geq n_0 \forall k \in \NN: |a_{n+1} + \cdots + a_{n+k}| < \varepsilon
\end{equation*}
Naj vsota $\sumalln{1}|a_n|$ konvergira, zato izpolnjuje Cauchyjev pogoj. Doka"zimo, da \dashuline{$\sumalln{1}a_n$ izpolnjuje Cauchyjev pogoj.}\\
Izberemo $\varepsilon$.
\begin{equation*}
|a_{n+1} + \cdots + a_{n+k}| \leq \underbrace{|a_{n+1}| + \cdots + |a_{n+k}|}_{||a_{n+1}| + \cdots + |a_{n+k}|| < \varepsilon} < \varepsilon
\end{equation*}
\hfill $\square$

\textsc{Primer:} Ugotovi ali vrsta konvergira
\begin{equation*}
\sumalln{1}\dfrac{\sin n}{n^2}
\end{equation*}
\dashuline{$\sumalln{1}\left|\dfrac{\sin n}{n^2}\right|$ konvergira}. Vemo $|\sin n| \leq 1$:
\begin{equation*}
\left|\dfrac{\sin n}{n^2}\right| \leq \dfrac{1}{n^2} \quad \sumalln{1}\dfrac{1}{n^2} \text{ konvergira}
\end{equation*}
Po primerjalnem kriteriju $\sumalln{1}\left|\dfrac{\sin n}{n^2}\right|$ konvergira.

\textsc{Izrek:}(\emph{Leibnizov kriterij} za alternirajo"ce vrste) Naj bo $a_n$ padajo"ce zaporedje pozitivnih "stevil z limito 0. Tedaj vrsta $\sumalln{1}(-1)^na_n$konvergira. Velja ocena:
\begin{equation*}
\left|\sumalln{0}(-1)^n a_n - \sum_{n=0}^{m-1}(-1)^n a_n\right| \leq a_m
\end{equation*}
\textsc{Primer:} $\sumalln{1}(-1)^n\dfrac{1}{n}$ alternirajo"ca harmoni"cna vrsta. $a_n = \dfrac{1}{n}$ ustreza pogojem, zato vrsta konvergira. Ne konvergira absolutno (ker harmoni"cna vrsta divergira).

\deff naj bo $\sumalln{1}a_n$ "stevilska vrsta. "Ce je $\sumalln{1}a_n$ konvergentna, ni pa absolutna konvergentna, potem re"cemo, da je \emph{pogojno konvergentna}.

\textsc{Dokaz:}
\begin{gather*}
\begin{aligned}
	s_n &= a_0 - a_1 + a_2 - a_3 + \cdots + (-1)^{n-1}a_{n-1} \\
	s_{2n+1} &=  (\underbrace{a_0 - a_1 + a_2 - a_3 + \cdots}_{s_{2n-1}}) - \underbrace{a_{2n-1} + a_{2n}}_{> 0} \leq s_{2n-1} \intertext{$\Rightarrow s_{2n+1}$  je padajo"ce}
	s_{2n+2} &= (\underbrace{a_0 - a_1 + a_2 - a_3 + \cdots}_{s_{2n}}) + \underbrace{a_{2n} - a_{2n+1}}_{>0} \geq s_{2n}
	\intertext{$\Rightarrow s_{2n}$  je nara"s"cajo"ca}
	s_{2_n+1} &= (a_0-a_1)+(a_2-a_3)+\cdots+(a_{2n-2}-a_{2n-1}) + a_{2n} \geq 0 \ \\
	s_{2n+2} &= a_0 +(-a_1 + a_2) +(-a_3 + a_4) + \cdots + (-a_{2n-1} + a_{2n}) - a_{2n+1} \leq a_0
\end{aligned}
\end{gather*}
$\Rightarrow s_{2n+1}$ je navzdol omejeno \\
$\Rightarrow s_{2n+2}$ je navzgor omejeno

$\Rightarrow s_{2n+1}, s_{2n+2}$ sta konvergentni podzaporedji v $s_n$.
\begin{equation*}
s_{2n+1} = s_{2n} + \underbrace{a_{2n}}_{\to 0}
\end{equation*}
Zato velja $\limninf s_{2n+1} = \limninf s_{2n}$. Zato je $s_n$ konvergentno $\Rightarrow \sumalln{1}(-1)^na_n$ konvergira.
\begin{equation*}
|s_{m+k} - s_m| = |(-1)^m||a_m+(-a_{m+1}+a_{m+2})+(-\cdots) + (-1^{k-1})a_{m+k-1}| \leq a_m
\end{equation*}
$k \to \infty: |s - s_m| \leq a_m \hfill \square$
%
\subsection{O preureditvah vrst}
Zamenjava vrstnega reda se"stevanja.
\begin{align*}
\sumalln{1}a_n &= a_1 + a_2  \cdots + a_n + \cdots \\
\pi&: \NN\to \NN \text{ bijekcija} \\
\sumalln{1}a_{\pi(n)} &= a_{\pi(1)} + a_{\pi(2)} + \cdots + a_{\pi(n)} + \cdots
\end{align*}
\textsc{Izrek:} Naj bo $\sumalln{1}a_n$ absolutno konvergentna vrsta in $\pi: \NN \to \NN$ bijektivna preslikava. Potem je $\sumalln{1}a_{\pi(n)}$ konvergentna in velja:
\begin{equation*}
\sumalln{1}a_n = \sumalln{1}a_{\pi(n)}
\end{equation*}
\textsc{Dokaz:}
\begin{equation*}
s_n = a_1 + a_2 + \cdots + a_n, \quad s = \limninf s_n
\end{equation*}
Dokazujemo \dashuline{$\sumalln{1}a_{\pi(n)} = s$}
\begin{equation*}
\left|s-\sum_{k=1}^{m}a_{\pi(k)}\right| = \left|s - s_n - \underbrace{\sum_{k=1}^{m}a_{\pi(k)}}_{\text{razen }\pi(k)\in\{1\ldots n\}}\right| \leq |s - s_n| + \sum_{k=1}^{m}|a_{\pi(k)}| \leq |s-s_n| + \sum_{k = m+1}^{\infty}|a_k| < \varepsilon
\end{equation*}
\begin{itemize}
	\item $|s-s_n < \frac{\varepsilon}{2}$, ker $\exists n_0 \forall n \geq n_0$ to velja.
	\item $\exists m_0 \forall n \geq m_0: \sum_{k=n+1}^{\infty}|a_n| < \frac{\varepsilon}{2}$
	\item $\exists m \in \NN: \{\pi(1), \ldots, \pi(m)\} \supset \{1, 2, \ldots, n\}$
\end{itemize}
%
\textsc{Izrek:}(Reimann) Naj bo $\sumalln{1}a_n$ pogojno konvergenta. Potem $\forall A \in \RR$ obstaja bijekcija $\pi: \NN \to \NN$, da je $\sumalln{1}a_{\pi(n)} = A$. Obstaja "se dve bijekciji $\pi_1, \pi_2: \NN \to \NN$
\begin{align*}
\lim_{m\to \infty} \sum_{n=1}^{m} a_{\pi_1(n)} &= \infty \\
\lim_{m\to \infty} \sum_{n=1}^{m} a_{\pi_2(n)} &= -\infty
\end{align*}
\textsc{Dokaz:} Predpostaviti smemo, da $a_n \neq 0 \forall n$.
\begin{equation*}
s_n = a_1 + \cdots + a_n = P_{k(n)} - Q_{m(n)}
\end{equation*}
Pri "cemer so v $P_{k(n)}$ se"steti vsi pozitivni "cleni, med $a_1$ in $a_n$, v $Q_{m(n)}$ pa nasprotne vrednosti vseh negativnih "clenov med $a_1$ in $a_n$.
\begin{align*}
P_{k(n)} &= p_1 + p_2 + \cdots + p_{k(n)} \\
Q_{m(n)} &= q_1 + q_2 + \cdots + q_{m()}
\end{align*}
Velja tudi $k(n) + m(n) = n$.
\begin{equation*}
P_{k(n)} + Q_{m(n)} = |a_1| + |a_2| + \cdots + |a_n|
\end{equation*}
\dashuline{$\sumalln{1}p_n$ in $\sumalln{1}q_n$ divergirata}
\begin{itemize}
	\item "Ce bi obe konvergirali, bi $\sumalln{1}|a_n$ konvergirala $\rightarrow \leftarrow$
	\item "Ce bi ena konvergirala, druga pa ne: iz $s_n = P_{k(n)} - Q_{m(n)}$ sledi, da "ce bi ena od vrst konvergirala bi tudi druga, ker $s_n$ konvergira.
\end{itemize}
\dashuline{$\limninf p_n = 0, \limninf q_n = 0$}

Vemo: ker $\sumalln{1}a_n$ konvergira, je $\limninf a_n = 0$. $p_n$ je podaporedje v $a_n \Rightarrow \limninf p_n = 0$. Podobno velja za $q_n$. Torej
\begin{equation*}
\limninf a_n = \limninf p_n = \limninf q_n = 0
\end{equation*}

Sedaj lahko sestavimo, tako zaporedje, ki konvergira proti $A \in \RR$. Obstaja najmanj"si $k_1 \in \NN$, da velja
\begin{gather*}
p_1 + p_2 + \cdots + p_{k_1} > A \\
p_1 + p_2 + \cdots + p_{k_1 - 1} \leq A
\end{gather*}
Obstaja najmanj"si $k_2 \in \NN$, da velja:
\begin{gather*}
p_1 + \cdots + p_{k_1} - q_1 - q_2 - \cdots - q_{k_2} < A \\
p_1 + \cdots + p_{k_1} - q_1 - \cdots - q_{k_2-1} \geq A
\end{gather*}
Postopek nadaljujemo. Ker je $\limninf p_n = \limninf q_n = 0$, bo preurejena vrsta konvergentna z vsoto $A$. Dobili smo bijektivno preslikavo po konstrukciji.

"Ce "zelimo skonstruirati vrsto, ki konvergira proti neskon"cno $A = \infty$ naredimo slede"ce
\begin{align*}
p_1 + p_2 + \cdots + p_{k_1} &> 1 \\
p_1 + \cdots + p_{k_1} - q_1 + p_{k_1+1} + p_{k_1 + 2} + \cdots + p_{k_2} &> 2
\end{align*}
\hfill $\square$

\subsection{Mno"zenje vrst}
Naj bosta $\sumalln{1}a_n$ in $\sumalln{1} b_n$ "stevilski vrsti.
\begin{align*}
&(a_1 + a_2 + \cdots + a_n)(b_1 + b_2 + \cdots + b_n) = \\
&= a_1 b_1 + a_1 b_2 + \cdots + a_1 b_n + \\
&+ a_2 b_1 + a_2 b_2 + \cdots + a_2 b_n + \cdots
\end{align*}
To lahko zapi"semo v tabelo, podobno kot urejenost racionalnih "stevil. Vsota iz vseh produktov je vrsta oblke
\begin{equation*}
\sum_{s=1}^{\infty} a_{i_s} b_{k_s}
\end{equation*}
kjer je $\NN \to \NN \times \NN$ bijektivna preslikava s predpisom $s \mapsto (i_s, k_s)$. \emph{Cauchyjev produkt}
\begin{equation*}
c_n = a_1b_{n-1} + a_2 b_{n-2} + \cdots + a_{n-1} b_1
\end{equation*}
je produkt kontra diagonal v tabeli. Vsota produktov je
\begin{equation*}
\sumalln{1}c_n
\end{equation*}
\textsc{Izrek:} Naj bosta $\sumalln{1}a_n$ in $\sumalln{1}b_n$ absolutno konvergentni "stevilski vrsti. Potem je vrsta, ki je sestavljena iz vseh produktov konvergentna in njena vsota je
\begin{equation*}
\left(\sumalln{1}a_n\right)\left(\sumalln{1}b_n\right)
\end{equation*}
Natan"cneje, za vsako bijektivno preslikavo $\varphi: \NN\to \NN \times \NN$, je vrsta
\begin{equation*}
\sum_{s=1}^{\infty}a_{\varphi_1(s)} b_{\varphi_2(s)}
\end{equation*}
konvergentna in velja:
\begin{equation*}
\sum_{s=1}^{\infty}a_{\varphi_1(s)} b_{\varphi_2(s)} = \left(\sumalln{1}a_n\right)\left(\sumalln{1}b_n\right)
\end{equation*}
\textsc{Dokaz:} dokazujemo absolutno konvergenco vrste iz vseh produktov. \dashuline{$\sum_{s=1}^{\infty}|a_{\varphi_1(s)} b_{\varphi_2(s)}|$ je konvergenta}

Zaporedje delnih vsot:
\begin{equation*}
|a_{\varphi_1(1)} b_{\varphi_2(1)}| + |a_{\varphi_1(2)} b_{\varphi_2(2)}| + \cdots + |a_{\varphi_1(m)} b_{\varphi_2(m)}|
\end{equation*}
Obstaja $n_0 = \max \{\varphi_1(1), \varphi_1(2), \ldots, \varphi_1(m)\}$ in $k_0 = \max \{\varphi_2(1), \varphi_2(2), \dots, \varphi_2(m)\}$, tako da lahko zgornjo vsoto omejimo navzgor in dobimo
\begin{multline*}
|a_{\varphi_1(1)} b_{\varphi_2(1)}| + |a_{\varphi_1(2)} b_{\varphi_2(2)}| + \cdots + |a_{\varphi_1(m)} b_{\varphi_2(m)}| \leq  \\
\leq \left(|a_1| + |a_2| + \cdots + |a_{n_0}|\right)\left(|b_1| + |b_2| + \cdots + |b_{k_0}|\right) \leq \\ 
\leq \underbrace{\left(\sumalln{1}|a_n|\right) \left(\sumalln{1}|b_n|\right)}_{\text{"stevilo}}
\end{multline*}
Ker $\sumalln{1}a_n$ absolutno konvergira, $\sumalln{1}|a_n|$ konvergira. Podobno velja za $|b_n|$.

Zaporedje delnih vsot $\sum_{s=1}^{\infty}|a_{\varphi_1(s)} b_{\varphi_2(s)}|$ je navzgor omejeno, zato je vrsta z nenegativnimi "cleni konvergentna.

Vosta vrste je neodvisna od izbire vrstnega reda "clenov, zato izberemo vrstni red ,,po kvadratih''.
\begin{equation*}
s_{m^2} = \left(\sum_{n=1}^{m}a_n\right) \left(\sum_{n=1}^{m}b_n\right)
\end{equation*}
\begin{equation*}
\lim_{m\to \infty} s_{m^2} = \left(\sumalln{1}a_n\right) \left(\sumalln{1}b_n\right)
\end{equation*}
Vsako podzaporedje konvergentnega zaporedja konvergira proti limiti zaporedja, zato tudi $s_m$ konvergira k temu "stevilu.

\hfill $\square$

\textsc{Primer:} $a_n = b_n = (-1)^n \dfrac{1}{\sqrt{n+1}}$.

$\sumalln{1}a_n = \sumalln{1}(-1)^n \dfrac{1}{\sqrt{n+1}}$ konvergira po Leibnizevem kriteriju, vendar ne konvergira absolutno. Izra"cunajmo produkt vrst s Cauchyjevo ureditvijo
\begin{multline*}
c_n = a_1 b_{n-1} + a_2 b_{n-2} + \cdots + a_{n-1} b_1 = \\
= (-1) \dfrac{1}{\sqrt{2}} \cdot (-1)^{n-1} \dfrac{1}{\sqrt{n}} + (-1)^2 \dfrac{1}{\sqrt{3}} \cdot (-1)^{n-2} \dfrac{1}{\sqrt{n-1}} + \cdots + (-1)^{n-1}\dfrac{1}{\sqrt{n}}\cdot (-1)\dfrac{1}{\sqrt 2} = \\
=(-1)^n \sum_{k=1}^{n-1} \dfrac{1}{\sqrt{(k+1)(n+1-k)}}
\end{multline*}
\begin{equation*}
|c_n| = \sum_{k=1}^{n-1} \dfrac{1}{\sqrt{\underbrace{(k+1)}_{\leq n} \underbrace{(n+1-k)}_{\leq n}}} \geq \sum_{k=1}^{n-1}\dfrac{1}{n} \geq \dfrac{n-1}{n}
\end{equation*}
$\sumalln{1}c_n$ divergira, ker "cleni ne gredo proti 0.

\subsection{Dvakratne vrste}
Imamo neskon"cno matriko $[a_{ij}]$
\begin{equation*}
\begin{matrix}
a_{11} & a_{12} & a_{13} & a_{14} & \cdots \\
a_{21} & a_{22} & a_{23} & a_{24} & \cdots \\
a_{31} & a_{32} & a_{33} & a_{34} & \cdots \\
\vdots & \vdots & \vdots & \vdots & \ddots
\end{matrix}
\end{equation*}
Formalni vrsti
\begin{equation*}
\sum_{i=1}^{\infty} \left(\sum_{j=1}^{\infty}a_{ij}\right)
\end{equation*}
in
\begin{equation*}
\sum_{j=1}^{\infty} \left(\sum_{i=1}^{\infty}a_{ij}\right)
\end{equation*}
imenujemo \emph{dvakratni vrsti}, "ce $\sum_{j=1}^{\infty}a_{ij}$ konvergira za vsak $i$, oziroma "ce $\sum_{i = 1}^{\infty} a_{ij}$ konvergira za vsak $j$.

Naj bo $\varphi: \NN \to \NN \times \NN$ bijektivna preslikava. Elemente $[a_{ij}]$ s $\varphi$ uredimo v zaporedje in sestavimo "stevilsko vrsto.
\begin{equation}
\label{eq:dvojna_vrsta}
\sum_{s=1}^{\infty}a_{\varphi(s)}
\end{equation}
\textsc{Velja:}
\begin{enumerate}[1)]
	\item "Ce~\ref{eq:dvojna_vrsta} absolutno konvergira, potem dvakratni vrsti konvergirata in vse tri imajo enako vsoto.
	\item "Ce dvakratna vrsta konvergira, ko njene "clene nadomestimo z absoutnimi vrednostmi, potem~\ref{eq:dvojna_vrsta} konvergira in vsote so enake.
\end{enumerate}