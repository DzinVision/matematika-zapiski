\begin{itemize}
	\item Z njimi "stejemo: 1, 2, 3
	\item Mno"zico naravnih "stevil ozna"cimo z \(\mathbb{N} \)\\
		\[\mathbb{N} = \{1, 2, 3, ...\} \]
	\item Vsako naravno "stevilo \(n\) ima naslednika \(n^+\) (\(n^+ = n + 1\))
\end{itemize}

\subsubsection*{Peanovi aksiomi:}
\(\mathbb{N} \) je mno"zica skupaj s pravilom, ki vsakemu naravnemu "stevilu \(n\) dodeli njegovega naslednika \(n^+ \in \mathbb{N} \) in velja:
\begin{enumerate}
	\item za vse \(m, n \in \mathbb{N} \) "ce \(m^+ = n^+\), potem \(m = n\)
	\item obstaja \(1 \in \mathbb{N} \), ki ni naslednik nobenega naravnega "stevila
	\item  "Ce je \(A \subset \mathbb{N} \) in "ce je \(1 \in A \) \footnote{indukcijska baza} in "ce velja: "ce \(n \in A \), potem \(n^+ \in A \) \footnote{indukcijski korak}, potem \(A = \mathbb{N}\)
\end{enumerate}
Aksiom (3) se imenuje \textbf{aksiom popolne indukcije}.

\begin{itemize}
	\item Naravna "stevila lahko \textbf{se"stevamo, mno"zimo}.
	\item \(\mathbb{N} \) so urejena po velikosti \(1, 2, 3, 4, 5, ...\)
		\[\{3, 5, 6, 10\} \subset \mathbb{N}\]
		\[\{3, 5, 7, 16, 23, ... \} \subset \mathbb{N} \]
	\item Vsaka neprazna podmno"zica \(\mathbb{N} \) ima najmanj"si element.
	\item V splo"snem ne velja\footnote{ne velja za vse (mno"zice)}, da ima vsaka neprazna podmno"zica \(\mathbb{N} \) najve"cji element.
\end{itemize}
