Ozna"cimo jih z \(\mathbb{Z} \)
\[\mathbb{Z} = \{0, 1, -1, 2, -2, ...\} \]

\begin{itemize}
	\item Se"stevanje in mno"zenje se iz \(\mathbb{N} \) raz"sirita na \(\mathbb{Z} \). 
	\item Poleg tega je definirano \textbf{od"stevanje}.
	\item Mno"zico celih "stevil uredimo na obi"cajen na"cin.
	\item Ni res, da bi imela vsaka neprazna podmno"zica \(\mathbb{Z} \) najmanj"si element.
	\item V splo"snem deljenje ni definirano (\(\frac{3}{2} \))
\end{itemize}

