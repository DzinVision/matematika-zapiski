Racionalna "stevila so kvocienti celih "stevil. Bolj natan"cno: kvoceinti celih in naravnih "stevil.

Dva ulomka \(\frac{m}{n}, \frac{k}{l} \) predstavljata isto racionalno "stevilo "ce: \(ml = nk \)\\
lahko naredimo:
\[\mathbb{Z} \times \mathbb{N} = \{(m, n), m \in \mathbb{Z}, n \in \mathbb{N} \}\]

Mno"zico \(\mathbb{Z} \times \mathbb{N} \) razdelimo na razrede: urejena para \((m, n)\) in \((k, l)\) sta v istem razredu, "ce velja \(ml = nk\).

Racionalno "stevilo je razred urejenih parov in ga ozna"cimo z \(\frac{m}{n}\).
\[\mathbb{Q} = \{\frac{m}{n}, m \in \mathbb{Z}, n \in \mathbb{N}\} \]

\subsubsection*{Se"stevanje v \(\mathbb{Q}\):}
\[\frac{m}{n} + \frac{k}{l} = \frac{ml + kn}{nl}, m, k \in \mathbb{Z}, n, l \in \mathbb{N} \]

Se"stevanje ulomkov je \emph{dobro definirano:}\\
"ce je: \(\frac{m'}{n'} = \frac{m}{n}, \frac{k'}{l'} = \frac{k}{l}\)\\
potem je: \(\frac{m'}{n'} + \frac{k'}{l'} = \frac{m}{n} + \frac{k}{l}\)

vemo: \(m'n = mn'\) in \(k'l = kl'\)\\
\emph{Dokaz:}
\[\frac{m'}{n'} + \frac{k'}{l'} =^{(def)} \frac{m'l' + n'k'}{n'l'} \frac{\cdot mk}{\cdot mk}=\]
\[= \frac{m'l'mk + n'k'mk}{n'ml'k} = \]
\[= \frac{m'mk'l + m'nk'k}{m'nk'l} = \frac{ml + nk}{nl} =^{(def)} = \frac{m}{n} + \frac{k}{l}\]

\subsubsection*{Mno"zenje v \(\mathbb{Q}\):}
\[\frac{m}{n} \cdot \frac{k}{l} = \frac{mk}{nl}, m, k \in \mathbb{Z}, n, l, \in, \mathbb{N}\]
Mno"zenje je dobro definirano (izpeljava doma).

\subsubsection*{Deljenje v \(\mathbb{Q}\):}
\[\frac{m}{n} : \frac{k}{l} = \frac{ml}{nk}, m, k \in \mathbb{Z}, n, l \in \mathbb{N}, k \neq 0 \]

\subsubsection*{Lastnosti se"stevanja}

Naj bo \(A\) ("stevilska) mno"zica z operacijama \(+\) in \(\cdot\).

Osnovne lastnosti ra"cunskih operacij bomo imenovali \textbf{aksiomi}. Druge lastnosti izpeljemo iz aksiomov.
\begin{itemize}
	\item[\textbf{A1}] \textbf{asociativnost se"stevanja}

	Za vse \(a, b, c \in A\) velja \((a + b) + c = a + (b + c)\)
	\item[\textbf{A2}] \textbf{komutativnost se"stevanja}
	
	Za vse \(a, b \in A\) velja da \(a + b = b + a\)
	
	\item[\textbf{A3}] \textbf{obstoj enote za se"stevanje}
	
	Obstaja za element \(0 \in A\) za katerega velja da: \(0 + a = a\) za vse \(a \in A\)
	
	\item[\textbf{A4}] \textbf{obstoj nasprotnega "stevila (elementa)}
	
	Za vsak \(a \in A\) obstaja nasprotno "stevilo \(-a \in A\) za katerega velja: \((-a) + a = 0\)
\end{itemize}

\emph{Opomba:} Mno"zica \(A\) za operacijo \(+\), ki ustreza aksiomom od A1 do A4 je \textbf{Abelova grupa} za \(+\).

\emph{Trditev:} Naj \((A, +)\) ustreza aksiomom od A1 do A4.
\begin{itemize}
	\item[(1)] \(\forall a \in A \) ima eno samo nasprotno "stevilo
	\item[(2)] \textbf{Pravilo kraj"sanja:} za vse \(a, x, y \in A\) velja: \(a + x = a + y \Rightarrow x = y\)
	\item[(3)] \(-0 = 0\)
\end{itemize}

\emph{Dokaz:}
\begin{itemize}
	\item[(1)] izberemo poljubno "stevilo \(a \in A\). Dokazujemo da ima \(a\) natanko 1 nasprotni element.
	
	Po A4 nasprotno "stevilo obstaja. Denimo, da sta \(b, c \in A\) nasprotni "stevili od a.
	\[b + a = 0\text{ in }c + a = 0 \]
	\[(a + b) + c \stackrel{\text{A2}}{=} (b + a) + c \stackrel{\text{predp.}}{=} 0 + c \stackrel{\text{A2}}{=} c\]
	\[(a + b) + c\stackrel{\text{A2}}{=} (b + a) + c \stackrel{\text{A1}}{=} b + (a + c) \stackrel{\text{A2}}{=} b + (c + a) \stackrel{\text{predp.}}{=} b + 0 \stackrel{\text{A2}}{=} 0 + b \stackrel{\text{A3}}{=} b \]
	\[c = b\]
	
	\item[(2)]
	\[a + x = a + y \stackrel{\text{A4}}{\Rightarrow}\]
	\[\Rightarrow (-a) + (a + x) = (-a) + (a + y) \stackrel{\text{A1}}{\Rightarrow}\]
	\[\Rightarrow ((-a) + a) + x = ((-a) + a) + y \stackrel{\text{A4}}{\Rightarrow}\]
	\[\Rightarrow 0 + x = 0 + y \stackrel{\text{A3}}{\Rightarrow}\]
	\[\Rightarrow x = y\]
	
	\item[(3)] \(-0 = 0\)
	\[0 \stackrel{\text{A4}}{=} (-0) + 0 \stackrel{\text{A2}}{=} 0 + (-0) \stackrel{\text{A3}}{=} -0\]
\end{itemize}

\emph{Od"stevanje v \(A\):} razlika "stevil \(a\) in \(b\) je vsota \(a\) in nasprotnega elementa od \(b\).
\[a - b := a + (-b)\]
\(b-a\) je re"sitev ena"cbe \(a + x = b\)

\emph{Pozor:} od"stevanje ne ustreza aksiomom od A1 do A4.

\subsubsection*{Lastnosti mno"zenja}
\begin{itemize}
	\item[\textbf{A5}] \textbf{asociativnost mno"zenja}
	
	Za vse \(a, b, c, \in A\) velja: \((ab)c = a(bc)\)
	\item[\textbf{A6}] \textbf{komutativnost mno"zenja}
	
	Za vse \(a, b, c \in A\) velja: \(ab = ba\)
	\item[\textbf{A7}] \textbf{obstoj enote za mno"zenje}
	
	\(\exists 1 \in A: 1 \cdot a = a\), za \(\forall a \in A\)
	\item[\textbf{A8}] \textbf{obstoj obratnega "stevila (elementa)}
	
	Vsak \(a \in A, a \neq 0\), ima obratni element, tj.: \(a^{-1} \in A: a^{-1} \cdot a = 1\)
\end{itemize}

Mno"zici \(A\) z operacijo \(+\), ki ustreza A1-A4, re"cemo \textbf{grupa za se"stevanje} (Abelova grupa).

Mno"zica \(A \setminus \{0\}\) z operacijo \(\cdot\), ki ustreza A5-A8 je \textbf{grupa za mno"zenje.}

Podobno kot za se"stevanje lahko izpeljemo:\\
\emph{Trditev:} veljajo:
\begin{itemize}
	\item[(1)] Vsak \(a \in A \setminus \{0\}\) ima eno samo obratno "stevilo
	\item[(2)] (pravilo kraj"sanja za mno"zenje)

	Za vsak \(a, x, y \in A\) velja: \(ax = ay \Rightarrow x = y\)
	\item[(3)] \(1^{-1} = 1\)
\end{itemize}

\begin{itemize}
	\item[\textbf{A9}] \textbf{"Stevili \(0\) in \(1\) sta razli"cni \(0 \neq 1\)}
	\item[\textbf{A10}] \textbf{Distributivnost}
	
	Za vsake \(a, b, c \in A\) velja:
	\[(a + b)c = ac + bc\]
\end{itemize}

\emph{Def:} Mno"zico \(A\) z operacijama \(+\) in \(\cdot\), ki ustreza aksiomom A1-A10, imenujemo \textbf{komutativen\footnote{komutativnost se nana"sa na komutativnost mno"zenja} obseg} ali \textbf{polje}.\\
\emph{Primer:} \((\mathbb{Q}, +, \cdot)\) so polje.

V \(A\) vpeljemo urejenost z dvema aksiomoma:
\begin{itemize}
	\item[\textbf{A11:}] Za  vsak \(a \in A \setminus \{0\}\) velja, da je natanko eno od "stevil \(a\), \(-a\) pozitivno. "Stevilo \(0\) ni niti pozitivno niti negativno. ("Stevilo \(a\) je negativno, "ce je "stevilo \(-a\) pozitivno).
	\item[\textbf{A12:}] Za vsaka \(a, b \in A\) velja: "ce sta \(a\) in \(b\) pozitivni "stevili, potem sta tudi \(a + b\) in \(a \cdot b\) pozitivni "stevili.
\end{itemize}

\emph{Def:} "Ce ima obseg \((A, +, \cdot)\) urejenost, ki izpolnjuje A11 in A12, \(A\) imenujemo \textbf{urejen obseg} (urejeno polje).\\
\emph{Primer:} \((\mathbb{Q}, +, \cdot)\) z obi"cajno urejenostjo je urejen obseg.

\(\frac{m}{n}\) je pozitiven, "ce \(m \cdot n > 0\)

\emph{Def:} Naj bo \(A\) urejen obseg. Za poljubna \(a, b \in A\) definiramo:

\hspace*{12pt}Pi"semo \(a > b\) natanko tedaj, kadar je \(a - b\) pozitivno "stevilo.

\hspace*{12pt}V tem primeru pi"semo tudi \(b < a\)

\hspace*{12pt}V posebnem primeru pi"semo \(a > 0\), kadar je \(a\) pozitivno "stevilo.

\emph{Def:} Naj bo \(A\) urejen obseg. Za poljubna \(a, b \in A\)

\hspace*{12pt}\(a \leq b\) natanko takrat, kadar \(a < b\) ali \(a = b\)

\emph{Trditev:} V urejenem obsegu \(A\) velja:
\begin{enumerate}
	\item[(1)] Za poljubni "stevili \(a, b \in A\) velja natanko ena od mo"znosti:
	\[a < b \text{, } a = b \text{, } a > b\]
	Sledi iz A11 uporabljen za \(a - b\).
	
	\item[(2)] Za poljubne \(a, b, c \in A\) velja:
	
	"ce je \(a > b \land b > c\), potem \(a > c\) (tranzitivnost)
	\item[(3)] Za poljubne \(a, b, c \in A\) velja:
	
	"ce je \(a > b\), potem \(a + c > b + c\)
	\item[(4)] Za poljubne \(a, b, c, \in A, c > 0\):
	
	"ce je \(a > b\), potem je \(ac > bc\)
	\item[5] Za poljubne \(a, b, c, d, \in A\):
	
	"ce je \(a > b > 0\) in \(c > d > 0\), potem je \(ac > bd\)
\end{enumerate}

\emph{Dokaz:}
\begin{enumerate}
	\item[(2)] \(a > b \land b > c \Rightarrow a > c\)
	
	Po definiciji: \(a - b > 0\) in \(b - c > 0\)
	
	Zato po A12:
	\begin{align*}
		(a -b) + (b -c) &> 0\\
		a + (-b) + b + (-c) &> 0\\
		a + 0 + (-c) &> 0 \\
		a - c &> 0
	\end{align*}
	zato \(a > c\)
	
	\item[(3)] denimo, da je \(a > b\)
	
	dokazujemo, da je \(a + b > b + c\), tj: \((a + c) - (b + c) > 0\)
	\begin{align*}
		(a + c) - (b + c) = &a + c + (-(b + c)) =\\
		&a + c + (-b) + (-c) = \\
		&(a + (-b)) + (c + (-c)) =\\
		&a + (-b) = a - b
	\end{align*}
	\begin{adjustwidth}{24pt}{0pt}
		Dokaz da \(-(b + c) = (-b) + (-c)\):\\
		\hspace*{12pt}\(-(b + c)\) je nasprotni element od \(b + c\), kar pomeni da je njuna vsota enaka \(0\). "Ce velja \(-(b + c) = (-b) + (-c)\), mora biti tudi \(b + c + (-b) + (-c) = 0\):
	\end{adjustwidth}
	\[b + c + (-b) + (-c) \stackrel{A2, A1}{=} (b + (-b)) + (c + (-c)) \stackrel{A4}{=} 0 + 0 \stackrel{A3}{=} 0\]
	
	"Ce \(a > b\) je \(a - b > 0\), zato je \((a + c) - (b + c) > 0\), kar pomni \(a + c > b + c\).
	
	\item[(5)] \(a > b > 0\) in \(c > d > 0\)
	
	Dokazujemo \(ac > bd\):
	
	\(a > b \stackrel{4}{\Rightarrow} ac > bc\) (\(c > 0\))\\
	\(c > d \stackrel{4}{\Rightarrow} bc > bd\) (\(b > 0\))
	
	Z upo"stevanjem tranzitivnosti (2) dobimo: \(ac > bd \square \)
\end{enumerate}

Racionalna "stevila predstavimo na "stevilski premici.

Racionalna "stevila so na "stevilski premici \textbf{povsod gosta} tj: na vsakem nepraznem odprtem intervalu le"zi racionalno "stevilo.

Racionalna "stevila ne pokrijejo "stevilske premice

\emph{Trditev:} re"sitev ena"cbe \(x^2 = 2, x > 0\) ni racionalno "stevilo. (\(\sqrt{2} \notin \mathbb{Q}\))\\
\emph{Dokaz:} Dokazujemo da \(x\) ni ulomek.

Dokazujemo s protislovjem.
\begin{itemize}
	\item privzamemo, da tisto kar dokazujemo ni res. (predpostavimo, da \(x\) je ulomek)
	\item sklepamo
	\item skepi nas privedejo v protislovje s predpostavko
\end{itemize}
\[x = \frac{m}{n}, m, n \in \mathbb{N}\]\\
"Ce je \(x\) ulomek, ga lahko zapi"semo kot okraj"san ulomek, zato sta \(m\) in \(n\) tuji si "stevili.
\begin{align*}
	x^2 = 2 &\Rightarrow (\frac{m}{n})^2 = 2\\
	\frac{m^2}{n^2} &= 2\\
	m^2 &= 2n^2\\
	2 | m^2 &\Rightarrow 2 | m\\
	\exists l \in \mathbb{N}: m &= 2l\\
	4l^2 &= 2n^2\\
	2l^2 &= n^2\\
	2 | n^2 &\Rightarrow 2 | n\\
	\rightarrow&\leftarrow
\end{align*}

Dokaz da \(2 | m^2 \Rightarrow 2 | m\):\\
"Ce je \(m\) liho, potem \(k \in \mathbb{N}_0\)
\begin{align*}
	m &= 2k+ + 1\\
	m^2 &= 4k^2 + 4k + 1\\
	m^2 &\text{ je lih}
\end{align*}
