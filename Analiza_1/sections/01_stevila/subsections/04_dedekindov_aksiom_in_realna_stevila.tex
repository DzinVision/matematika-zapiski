Radi bi skonstruirali "stevilsko mno"zico, ki bo vsaj urejen obseg in, ki zapolni "stevilsko premico.

\subsubsection*{Dedekindov pristop}
\emph{Def:} \textbf{Rez} je podmono"zica \(A \subset \mathbb{Q}\), za katero velja:
\begin{enumerate}
	\item[(i)] \(A \neq \varnothing, A \neq \mathbb{Q}\)
	\item[(ii)] "Ce je \(p \in A\), potem za vsak \(q \in \mathbb{Q}\) in \(q < p\), velja \(q \in A\)
	\item[(iii)] za vsak \(p \in A\) obstaja \(q \in A, q > p\) (\(A\) nima najve"cjega elementa)
\end{enumerate}

\emph{Def:} Mno"zica realnih "stevil je mno"zica vseh rezov, ozna"cimo jo s \(\mathbb{R}\)\\
\emph{Primer:} 16 ustreza rez: \(\{p \in \mathbb{Q}; p < 16\} = B\)

\emph{Trditev:} Preslikava \(\mathbb{Q} \rightarrow \mathbb{R}\) in je definirana s predpisom:
\[q \mapsto \{p \in \mathbb{Q}; p < q\} = p^*\]
\hspace*{48pt}vlo"zi mno"zico racionalnih "stevil v mno"zico realnih "stevil.

Vpeljimo ra"cunski operaciji v \(\mathbb{R}\).

\emph{Def:} Naj bosta \(A\) in \(B\) reza. Vsota rezov \(A\) in \(B\) je
\[A + B = \{a + b, a \in A, b \in B\}\]
\hspace*{24pt}\emph{Opomba:} \((p + q)^* = p* + q*\)

\emph{Trditev:} "Ce sta \(A\) in \(B\) reza, potem je tudi \(A + B\) rez.\\
\emph{Dokaz(trditev)}: Denimo da sta \(A\) in \(B\) reza.

Dokazujemo da je \(A + B\) rez:
\begin{enumerate}
	\item[(i)] \dashuline{\(A + B \neq \varnothing\)}

	Ker sta \(A\) in \(B\) reza, po lastnosti (i) obstaja \(a \in A\) in \(b \in B\). Potem je \(a + b \in A + B\). Sledi \(A + B \neq \varnothing \ \ \square\)

	\dashuline{\(A + B \neq \mathbb{Q}\)}

	Obstaja \(c \in \mathbb{Q}, c \notin A\) in obstaja \(d \in \mathbb{Q}, d \notin B\).
	\[c + d \notin A + B\]
	Denimo, da je \(c + d\ \in A + B\).

	Potem velja, da je \(c + d = a + b\) za \(a \in A, b \in B\).

	Iz (ii) sledi: \(c > a\) in \(d > b \Rightarrow a + b < c + d \ \ \rightarrow\leftarrow\)

	\item[(ii)] Denimo: \(p \in A + B\), dokazujemo da za \(q \in \mathbb{Q}, q < p\) velja \dashuline{\(q \in A + B\)}

	Obstajata \(a \in A\) in \(b \in B\), da \(p = a + b\)
	\[q = a + q - a\]
	"Ce je \(q - a < b\), potem \(q - a \in B\)
	\begin{align*}
		q &< a + b\\
		q &< p\\
		&\square
	\end{align*}

	\item[(iii)] \dashuline{\(A + B\) nima najve"cjega elementa.}

	izberimo \(p \in A + B\)\\
	i"s"cemo \(q \in A + B, q > p\)

	Obstajata \(a \in A, b \in B\), da je \(p = a _ b\)\\
	Obstaja \(a' \in A, a' > a\)
	\begin{align*}
		q &:= a' + b \in A + B\\
		q &> p\\
		&\square
	\end{align*}
\end{enumerate}

Ni te"zko preveriti, da za \((\mathbb{R}, +)\) veljajo A1-A4.\\
\hspace*{12pt}Asociativnost in komutatiovnost se doka"ze z operacijami na elementih reza.\\
\hspace*{12pt}\(0^*\) je enota za se"stevanje.

Nasprotni rez od \(A\):
\[-A = \{r \in \mathbb{Q}, \text{obstaja } r' \in \mathbb{Q}, r' > r \text{ in } r' + p < 0 \text { za vse } p \in A\}\]
\begin{enumerate}
	\item[(i)] \(-A \neq \varnothing\)

	\dashuline{\(-A \neq \mathbb{Q}\)}
	\begin{align*}
		q + a &< 0 \text{ za vse } q \in \mathbb{Q}\\
		q &= -a
	\end{align*}

	\item[(ii)] \(q \in -A: r < q\)

	\(q + a < 0\) za vse \(a \in A\)

	\(r + a < q + a\) za vse \(a \in A\) po tranzitivnosti: \(r + a < 0\)

	\item[(iii)]
	
	Izberemo poljuben $r \in A$. I"s"cemo $q \in A, q > r$.
	\[q := \frac{r + r'}{2} (r' > r, r' \in \mathbb{Q}: r' + p < 0 \text{ za vse } p \in \mathbb{Q})\]
	\[q \in \mathbb{Q}, r' > q\]
\end{enumerate}

\emph{Def:} Pravimo da je \(A\) pozitiven, "ce je \(0^* \subset A, A \neq 0^*\). Denimo da sta \(A\) in \(B\) pozitivna reza:
\[A \cdot B = \{q \in \mathbb{Q}, \text{ obstajata } a \in A, a > 0 \text{ in } b \in B, b > 0 \text{, da je } q < ab\}\]
"Zelimo $(pq)^* = p^* \cdot q^*, p, q \in \mathbb{Q}$

\emph{Def:} Naj bosta $A$ in $B$ poljubna reza.
\[
A \cdot B = \begin{cases}
A \cdot B & A > 0 \land B > 0\\
- A \cdot (-B) & A > 0 \land B < 0\\
- (-A) \cdot B & A < 0 \land B > 0\\
(-A) \cdot (-B) & A < 0 \land B < 0
\end{cases}
\]
"Ce je vsaj eden od rezov enak $0^*$, potem $A \cdot B = 0^*$.

Ni te"zko preveriti, da mno"zenje rezov izpolnjuje A5-A8 in A9, A10.

Enota za mno"zenje je $1^*$.

Urejenost izplonjuje A11 in A12.

$(\mathbb{R}, +, \cdot)$ je \textbf{urejen obseg} in vsebuje $\mathbb{Q}$ kot \textbf{urejen podobseg}.

\emph{Cilj:} Obseg $\mathbb{R}$ izpolnjuje "se dodaten aksion A13 (\textbf{Dedekindov aksiom}), ki pove, da $\mathbb{R}$ zapolnjuje "stevilsko premico.

Aksioma A13 obseg $\mathbb{Q}$ ne izpolnjuje.

\emph{Def:} Naj bo $B$ urejen obseg in $A \subset B$. Pravimo, da je $A$ navzgor omejena, "ce obstaja $M \in B$, da velja:
\[\forall a \in A: a \leq M\]
$\forall M$ s to lastnostjo pravimo zgornja meja mno"zice $A$. "Ce je $A \subset \mathbb{Q} (\text{ali } \mathbb{R})$ in je mno"zica navzgor omejena, potem ima $A$ neskon"cno zgornjih mej.

\emph{Def:} Naj bo $A$ navzgor omejena mno"zica. "Ce obstaja najmanj"sa od vseh zgornjih mej mno"zice $A$ v $B$, jo imenuje natan"cna zgornja meja mno"zice $A$.

Torej je $\alpha \in B$ natan"cna zgornja meja mno"zice $A$, "ce velja:
\begin{itemize}
	\item[(i)] $\alpha$ je zgornja meja $\forall a \in A: a \leq \alpha$
	\item[(ii)] "Ce $b \in B, b < \alpha$, potem $b$ ni zgornja meja mno"zice $A$, t.j.: $\exists a \in A: a > b$
\end{itemize}
Natan"cno zgornjo mejo mno"zice $A$ imenujemo tudi \textbf{supremum} mno"zice $A$ in jo ozna"cimo z $\sup A$

\emph{Def:} "Ce obstaja najve"cji element mno"zice $A$, ga imenujemo \textbf{maksimum} mno"zice $A$ in ozna"cimo z $\max A$.

"Ce ima $A$ maksimum, potem velja:
\[\max A \in A \land \forall a \in A: a \leq \max A\]

"Ce ima $A$ maksumum, potem $\max A = \sup A$. (Za $a$ v (2) lastnosti definicije supremuma vzamemo $\max A$)
\emph{Dokaz:}
\begin{itemize}
	\item[(1)] $\max A$ je zgornja meja $A$ (po definiciji maksimuma).
	\item[(2)] "ce $b < \max A$, potem $b$ ni zgornja meja.
	\[\exists \max A, \text{ ki je ve"cji}\]
\end{itemize}

\emph{Primeri:}
\begin{enumerate}
	\item $A = \{x \in \mathbb{Q}, x < 0\} \subset \mathbb{Q}$
	
	4 je zgornja meja mno"zice $A$, ker $x \in A, x < 0, 0 < 4 \Rightarrow x < 4$
	
	$\Rightarrow A$ je navzgor omejena.
	
	0 je natan"cna zgornja meja mno"zice $A$
	\begin{enumerate}
		\item $x \in A, x < 0$
		\item  Izberemo poljuben $b \in \mathbb{Q}, b < 0$ in dokazujemo, da $b$ ni zgornja meja.
		\[b < \frac{b}{2} < 0; \frac{b}{2} \in A, \frac{b}{2} > b \]
	\end{enumerate}
	Mno"zica $A$ nima maksimuma: $0 \notin A$.
	
	\item $C = \{x \in \mathbb{Q}, x^2 < 2\} \subset \mathbb{Q}$
	
	$C$ je navzgor omejena z 2.
	\[x \in C: x^2 < 2 \land 2 < 4 \Rightarrow x^2 < 4 \Rightarrow x < 2 \]
	\begin{enumerate}
		\item Vsako "stevilo $p \in \mathbb{Q}, p^2 > 2$ je zgornja meja mno"zice $C$.
		\item Nobeno racionalno "stevilo $q \in \mathbb{Q}, q^2 < 2$ ni zgornja meja mno"zice $C$.
	\end{enumerate}

	Vemo: Re"sitev ena"cbe $x^2 = 2, x > 0, x \notin \mathbb{Q}$.\\
	Sledi: $C \subset \mathbb{Q}$ nima natan"cne zgornje meje.
	
	\emph{Dokaz:}
	\begin{enumerate}
		\item \[x^2 < 2 < p^2\]
		sledi: $x^2 < p^2 \Rightarrow x < p$ za vse $x \in C\ \square$
		
		\item I"s"cemo $c \in C$, da je $c > q, c^2 < 2$
		\[c := \frac{2q + 2}{q + 2} = q + \frac{2q + 2 - q^2 - 2q}{q+2} = q + \frac{2 - q^2 > 0}{q + 2} > 0\]
		$(q^2 < 2)$
		\begin{align*}
			c^2 &= (\frac{2q + 2}{q + 2}^2 = \frac{4(q^2 + 2q + 1)}{q^2 + 4q + 4}\\
			c^2 - 2 &=\frac{4q^2 + 8q + 4 - 2q^2 - 8q - 8}{(q + 2)^2} = \frac{2q^2 - 4}{(q+2)^2} = \frac{2(q^2 - 2 > 0)}{(q+2)^2} > 0
		\end{align*}
		\[q^2 < 0\]
	\end{enumerate}
\end{enumerate}