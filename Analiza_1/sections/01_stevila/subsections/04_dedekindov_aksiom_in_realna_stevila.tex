Radi bi skonstruirali "stevilsko mno"zico, ki bo vsaj urejen obseg in, ki zapolni "stevilsko premico.

\subsubsection*{Dedekindov pristop}
\underline{Def:} \textbf{Rez} je podmono"zica \(A \subset \mathbb{Q}\), za katero velja:
\begin{enumerate}
	\item[(i)] \(A \neq \varnothing, A \neq \mathbb{Q}\)
	\item[(ii)] "Ce je \(p \in A\), potem za vsak \(q \in \mathbb{Q}\) in \(q < p\), velja \(q \in A\)
	\item[(iii)] za vsak \(p \in A\) obstaja \(q \in A, q > p\) (\(A\) nima najve"cjega elementa)
\end{enumerate}

\underline{Def:} Mno"zica realnih "stevil je mno"zica vseh rezov, ozna"cimo jo s \(\mathbb{R}\)\\
\underline{Primer:} 16 ustreza rez: \(\{p \in \mathbb{Q}; p < 16\} = B\)

\underline{Trditev:} Preslikava \(\mathbb{Q} \rightarrow \mathbb{R}\) in je definirana s predpisom:
\[q \mapsto \{p \in \mathbb{Q}; p < q\} = p^*\]
\hspace*{48pt}vlo"zi mno"zico racionalnih "stevil v mno"zico realnih "stevil.

Vpeljimo ra"cunski operaciji v \(\mathbb{R}\).

\underline{Def:} Naj bosta \(A\) in \(B\) reza. Vsota rezov \(A\) in \(B\) je
\[A + B = \{a + b, a \in A, b \in B\}\]
\hspace*{24pt}\underline{Opomba:} \((p + q)^* = p* + q*\)

\underline{Trditev:} "Ce sta \(A\) in \(B\) reza, potem je tudi \(A + B\) rez.\\
\underline{Dokaz(trditev)}: Denimo da sta \(A\) in \(B\) reza.

Dokazujemo da je \(A + B\) rez:
\begin{enumerate}
	\item[(i)] \dashuline{\(A + B \neq \varnothing\)}
	
	Ker sta \(A\) in \(B\) reza, po lastnosti (i) obstaja \(a \in A\) in \(b \in B\). Potem je \(a + b \in A + B\). Sledi \(A + B \neq \varnothing \ \ \square\)
	
	\dashuline{\(A + B \neq \mathbb{Q}\)}
	
	Obstaja \(c \in \mathbb{Q}, c \notin A\) in obstaja \(d \in \mathbb{Q}, d \notin B\).
	\[c + d \notin A + B\]
	Denimo, da je \(c + d\ \in A + B\).
	
	Potem velja, da je \(c + d = a + b\) za \(a \in A, b \in B\).
	
	Iz (ii) sledi: \(c > a\) in \(d > b \Rightarrow a + b < c + d \ \ \rightarrow\leftarrow\)
	
	\item[(ii)] Denimo: \(p \in A + B\), dokazujemo da za \(q \in \mathbb{Q}, q < p\) velja \dashuline{\(q \in A + B\)}
	
	Obstajata \(a \in A\) in \(b \in B\), da \(p = a + b\)
	\[q = a + q - a\]
	"Ce je \(q - a < b\), potem \(q - a \in B\)
	\begin{align*}
		q &< a + b\\
		q &< p\\
		&\square
	\end{align*}

	\item[(iii)] \dashuline{\(A + B\) nima najve"cjega elementa.}
	
	izberimo \(p \in A + B\)\\
	i"s"cemo \(q \in A + B, q > p\)
	
	Obstajata \(a \in A, b \in B\), da je \(p = a _ b\)\\
	Obstaja \(a' \in A, a' > a\)
	\begin{align*}
		q &:= a' + b \in A + B\\
		q &> p\\
		&\square
	\end{align*}
\end{enumerate}

Ni te"zko preveriti, da za \((\mathbb{R}, +)\) veljajo A1-A4.\\
\hspace*{12pt}Asociativnost in komutatiovnost se doka"ze z operacijami na elementih reza.\\
\hspace*{12pt}\(0^*\) je enota za se"stevanje.

Nasprotni rez od \(A\):
\[-A = \{\text{treba bo dopolniti kej je prfoksa pozabila}\}\]
\begin{enumerate}
	\item[(i)] \(-A \neq \varnothing\)
	
	\dashuline{\(-A \neq \mathbb{Q}\)}
	\begin{align*}
		q + a &< 0 \text{ za vse } q \in \mathbb{Q}\\
		q &= -a
	\end{align*}
	
	\item[(ii)] \(q \in -A: r < q\)
	
	\(q + a < 0\) za vse \(a \in A\)
	
	\(r + a < q + a\) za vse \(a \in A\) po tranzitivnosti: \(r + a < 0\)
	
	\item[(iii)] \(q \in -A\): i"s"cemo \(q' \in -A, q' > q\)
	
	\(q + a < 0\) za vse \(a \in A\).
\end{enumerate}

\underline{Def:} Pravimo da je \(A\) pozitiven, "ce je \(0^* \subset A, A \neq 0^*\). Denimo da sta \(A\) in \(B\) pozitivna reza:
\[A \cdot B = \{q \in \mathbb{Q}, \text{ obstajata } a \in A, a > 0 \text{ in } b \in B, b > 0 \text{, da je } q < ab\}\]
