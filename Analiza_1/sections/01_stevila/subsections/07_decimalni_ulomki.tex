Vsako $\RR$ "stevilo lahko zapi"semo kot \emph{decimalni ulomek}.

Naj bo $x \in \RR^+$ in naj bo $n \in \NN_0$ najve"cje "stevliko, ki ne presega $x$:
\begin{equation*}
n \leq x < n + 1
\end{equation*}

Interval $[n, n+1]$ razdelimo na 10 enakih delov. Nato poi"s"cemo $n_1 \in \{0, 1, 2, \ldots, 9\}$, da velja:
\begin{equation*}
n + \frac{n_1}{10} \leq x <  n + \frac{n_1 + 1}{10}
\end{equation*}

Postopek nadaljujemo in na ta na"cin sestavimo zaporedje decimalnih pribli"zkov za $x$.
\begin{equation*}
\mathcal{A} = \{n, n + \frac{n_1}{10}, n + \frac{n_1}{10} + \frac{n_2}{100}, \ldots\}
\end{equation*}

\emph{Trditev:} $x = \sup \mathcal{A}$\\
\emph{Dokaz:}
\begin{enumerate}
	\item[(i)] \dashuline{$x$ je zgornja meja mno"zice $\mathcal{A}$}
	
	Velja po konstrukciji: $\forall a \in \mathcal{A}: a \leq x$.
	
	\item[(ii)] \dashuline{$x$ je najmanj"sa zgornja meja}
	
	Denimo da to ni res:
	\begin{align*}
	y &:= \sup \mathcal{A} < x\\
	\exists n \in \NN&: \frac{1}{n} \leq x - y\\
	\exists p \in \NN&: \frac{1}{10^p} < \dfrac{1}{n} < x - y
	\end{align*}
	\begin{equation*}
	y + \frac{1}{10^p} < x
	\end{equation*}
	\begin{align*}
		n + \dfrac{n_1}{10} + \dfrac{n_2}{100} + \ldots + \dfrac{n_p}{10^p} &\leq y\\
		n + \dfrac{n_1}{10} + \dfrac{n_2}{100} + \ldots + \dfrac{n_p+1}{10^p} &\leq y + \dfrac{1}{10^p} < x \rightarrow \leftarrow
	\end{align*}
\end{enumerate}
Trditev utemelji, da $x$ lahko zapi"semo, kot neskon"cni decimalni ulomek.
\begin{equation*}
x = n_0 + \dfrac{n_1}{10} + \ldots + \dfrac{n_p}{10^p} + \ldots = n_0,n_1n_2\ldots n_p\ldots
\end{equation*}

\emph{Trditev:} Naj bosta $x, y \in \RR^+$
\begin{enumerate}
	\item[(1)] Denimo, da obstaja $k \in \NN_0$, za katerega velja:
	\begin{align*}
		x &= n_0,n_1n_2\ldots n_{k-1}n_k99\ldots\\
		y &= n_0,n_1n_2\ldots n_{k-1}(n_k + 1)00\ldots
	\end{align*}
	in $n_k \neq 9$, potem $x = y$.
	
	\item[(2)] Za dva razli"cna decimalna zapisa $x \in \RR^+$ velja (1).
\end{enumerate}

\emph{Dokaz:} Naj bo $\mathcal{A}$ mno"zica decimalnih preslikav za $x$.
\begin{itemize}
	\item[(1)] 
	\begin{equation*}
	\forall a \in \mathcal{A}: y \geq a \text{ ($y$ je zgornja meja)}
	\end{equation*}
	Zato $y \geq x$ ($x$ je $\sup \mathcal{A}$)
	
	\dashuline{Dokzujemo $y$ je natan"cna zgornja meja od $\mathcal{A}$}
	
	Naj bo $l > k$ in $a_l$ $l$-ti decimalni pribli"zek za $x$.
	\begin{align*}
	y - a_l &= \dfrac{1}{10^l}\\
	y - \dfrac{1}{10^l} &= a_l\\
	y - \dfrac{1}{10^l} &\leq a_l < a_{l+1}	
	\end{align*}
	$\Rightarrow y - \dfrac{1}{10^l}$ ni zgornja meja za noben $l$, torej je $y$ natan"cna zgornja meja.
	
	\item[(2)]
	$x$ naj ima dva decimalna zapisa:
	\begin{align*}
		x &= n_0,n_1n_2\ldots\\
		x &= m_0,m_1m_2\ldots
	\end{align*}
	Obstaja najmanj"si indeks $k \in \NN: n_k \neq m_k$.
	
	Predpostavimo, da je $m_k > n_k$
	
	$m_0,m_1m_2\ldots m_k$ je zgornja meja mno"zice decimalnih pribli"zkov za $x$.
	\begin{equation*}
		m_0,m_1m_2\ldots m_k > m_0,m_1m_2\ldots m_{k-1}n_kn_{k+1}n_{k+2}\ldots
	\end{equation*}
	
	"Ce bi veljajo $n_k < m_k - 1$ (dokazujemo, da je razlika lahko najve"c 1, t.j: morajo se ponavljati 9-ke)
	\begin{equation*}
		x \leq n_0,n_1n_2\ldots n_{k-1}(n_k+1) < m_0,m_1\ldots m_{k-1}m_k \leq x \rightarrow \leftarrow
	\end{equation*}
	($x < x$ ni mo"zno)
	
	"Ce bi bil $m_{k+1} \neq 0$ (ali za indeks $l > k$):
	\begin{equation*}
		x \leq m_0,m_1m_2\ldots m_k < m_0,m_1 \ldots m_km_{k+1} \leq x \rightarrow \leftarrow
	\end{equation*}
	
	Na podoben na"cin doka"zemo, da velja:
	\begin{equation*}
	\forall l \in \NN: n_{k+l} = 9
	\end{equation*}
\end{itemize}

Podobno velja tudi za druge osnove. Primer v dvoji"skem bi bil:
\begin{equation*}
1101,011 = 1101,010\overline{11}
\end{equation*}

\emph{Trditev:} Naj bo $x \in \RR$. $x$ ima periodi"cen decimalni zapis natanko tedaj, kadar $x \in \mathbb{Q}$ (tudi kon"cen decimalni zapis je periodi"cen).

\emph{Dokaz:} denimo, da je $x \in \QQ^+$.
\begin{equation*}
x = \dfrac{m}{n}, m, n \in \NN
\end{equation*}
$m$ delimo z $n$ pisno. To pomeni da podpisujemo ostanke. Ker imamo na voljo $n$ razli"cnih ostankov: $o_j \in \{0, 1, \ldots, n-1\}$, se bo med $n+1$ zaporednimi ostanki vsaj eden zagotovo ponovil. Ko se ostanek ponovi, se ponovi tudi cel zapis, kar je perioda.

Denimo, da ima $x$ periodi"cen decimalni zapis:
\begin{align*}
x &= d,d_1d_2\ldots d_n \overline{d_{n+1} d_{n+2}\ldots d_{n+k}}\\
10^k x &= d \cdot 10^k + d_1d_2\ldots d_k, d_{k+1}\ldots d_n \ldots d_{n+k} \overline{d_{n+1} \ldots d_{n+k}}\\
10^kx-x &\text{ ima kon"cen decimalni zapis} = p \in \QQ\\
x &= \dfrac{p}{10^k-1} \in \QQ \qed
\end{align*}