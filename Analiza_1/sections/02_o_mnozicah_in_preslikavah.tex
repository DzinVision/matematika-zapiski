\emph{Definicija:} Naj bosta $A$ in $B$ mno"zici.\\
Preslikava $f$ iz mno"zice $A$ v mno"zico $B$ je predpis, ki vsakemu elementu iz mno"zice $A$ priredi natanko en element iz mno"zice $B$.

Pi"semo $f: A \rightarrow B$\\
\begin{equation*}
a \in A: a \mapsto f(a) \in B
\end{equation*}

$A$ imenujemo \emph{domena} ali \emph{definicijsko obmo"cje}, $B$ pa imenujemo \emph{kodomena}.

Zaloga vrednosti je mno"zica $\{f(a); a \in A\} = Z_f$.

\emph{Definicija:} Naj bosta $A$ in $B$ mno"zici in $f: A \rightarrow B$ preslikava.

Pravimo, da je $f$ \emph{surjektivna}, "ce je $B = Z_f$.

Pravimo, da je $f$ \emph{injektivna}, "ce velja:
\begin{equation*}
\forall x, y \in A: x \neq y \Rightarrow f(x) \neq f(y)
\end{equation*}
Ro lahko zapi"semo tudi kot:
\begin{equation*}
\forall x, y \in A: f(x) = f(y) \Rightarrow x = y
\end{equation*}

Pravimo, da je $f$ \emph{bijektivna (povratno enoli"cna)}, kadar je surjektivna in injektivna.

\emph{Definicija:} Naj bosta $A$ in $B$ mno"zici, $f: A \rightarrow B$ bijektivna preslikava.

\emph{Inverzna} preslikava $f^{-1}: B \rightarrow A$ vsakemu elementu $b \in B$ priredi tisti element $a \in A$, za katerega velja $f(a) = b$.

\dashuline{$f^{-1}$ je dobro definirana}

$\forall b \in B \exists a \in A: f(a) = b$, ker je $f$ surjektivna. Ta $a$ je enoli"cno dolo"cen, ker je $f$ injektivna; $f(a_1) = f(a) \Rightarrow a_1 = a$.

\emph{Definicija:} Naj bosta $A$ in $B$ mno"zici.

Pravim, da sta mno"zici \emph{ekvipolentni} ali enako mo"cni, kadar obstaja bijektivna preslikava $f: A \rightarrow B$.

\emph{Opomba:} "ce je $f: A \rightarrow B$ bijektivna, potem je $f^{-1}: B \rightarrow A$ bijekcija.

\emph{Opomba:} kon"cni mno"zici imata enako mo"c, kadar imata enako "stevilo elementov.

\emph{Definicija:} "Ce ima mno"zica $A$ enako mo"c kot $\NN$, pravimo, da je $A$ \emph{"stevno neskon"cna}.

"Ce je $A$ "stevno neskon"cna, potem obstaja $\NN \rightarrow A; n \mapsto f(n) = a_n$.
\begin{equation*}
A = \{a_1, a_2, \ldots a_n\} = \{a_n; n \in \NN\}
\end{equation*}
pri "cemer velja: $j \neq k: a_j \neq a_k$.

\emph{Trditev:} $\NN, \ZZ, \QQ$ so "stevno neskon"cna.

Dokaz za $\ZZ$: $\{0, 1, -1, 2, -2, \ldots\}$

Pri $\QQ$ je dovolj, da doka"zemo za nenegativna "stevila, nato sledi podobno kot za $\ZZ$. Za pozitivna "stevilo naredimo tabelo~\ref{tab:rac-dokaz}.

\begin{table}[htp]
	\centering
	\begin{tabular}{c|ccccc}
		 & \textbf{1} & \textbf{2} & \textbf{3} & \textbf{4} & \textbf{5} \\ \hline
		\textbf{1} & $\frac{1}{1}$ & $\frac{2}{1}$ & $\frac{3}{1}$ & $\frac{4}{1}$ & $\frac{5}{1}$\\
		\textbf{2} & $\frac{1}{2}$ & $\frac{2}{2}$ & $\frac{3}{2}$ & $\frac{4}{2}$ & $\frac{5}{2}$
	\end{tabular}
	\caption{Dokaz da so $\QQ$ "stevno neskno"cna}
	\label{tab:rac-dokaz}
\end{table}

Nato "stevila pove"zemo po diagonalah ($\frac{1}{1}$, $\frac{2}{1}$, $\frac{1}{2}$, $\frac{3}{1}$, $\frac{2}{2}$) in izlo"cimo "stevila ko so se "ze ponovila (npr: $\frac{2}{2} = \frac{1}{1}$).

\emph{Izrek:} $\RR$ ni "stevno neskon"cna

\emph{Dokaz:} Denimo da je. Potem $\{a_1, a_2, a_3, \ldots\} = \RR$.

$a_j$ zapi"semo kot decimalni ulomek:
\begin{align*}
a_1 &= d_1'd_{11}d_{12}d{13}\ldots\\
a_2 &= d_2'd_{21}d_{22}d{23}\ldots\\
a_3 &= d_3'd_{31}d_{32}d{33}\ldots
\end{align*}
%
\begin{align*}
x &:= 0'x_1x_2x_3\ldots \in \RR\\
x_1 &= 1, \text{"ce je $d_{11} = 0$, sicer $x_1 = 0$}
x_2 &= 1, \text{"ce je $d_{22} = 0$, sicer $x_2 = 0$}
\end{align*}
Skonstruirali smo $x \notin \{a_1, a_2, a_3, \ldots\} \rightarrow \leftarrow$.