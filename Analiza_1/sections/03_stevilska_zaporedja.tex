\deff Zaporedje realnih "stevil je preslikava $\NN \rightarrow \RR$. Zapis je:
\begin{align*}
f: &\NN \rightarrow \RR\\
&f(n) \text{ozna"cimo z $a_n$}
\end{align*}
Preslikava $f$ je podana z $a_1, a_2, a_3, \ldots$.

Zaporedje realnih "stevil podamo \emph{s "cleni} $a_1, a_2, a_3, \ldots$, kar kraj"se zapi"semo $\left\{a_n\right\}_{n=1}^\infty$, $\left\{a_n\right\}$,  $\left(a_n\right)_{n=1}^\infty$ ali pa kar zaporedje $a_n$.

\textsc{Opomba}: zaporedje $\left\{a_n\right\}$ ni mno"zica $\left\{a_n; n \in \NN \right\}$.

\textsc{Primer:}
\begin{enumerate}[1)]
	\item $a_n = 1, n \in \NN$ \emph{konstantno zaporedje}
	
	To zaporedje ustreza preslikavi $f(n) = 1, n \in \RR$
	
	\item $b_n = n, n \in \NN$ Takemu zaporedju pravimo, da je podan s splo"snim "clenom. Nari"semo lahko njegov graf $\left\{(n, b_n); n \in \NN \right\}$
	
	\item $d_1=1, d_2=1, d_{n+2} = d_n + d_{n+1}, n \in \NN$ je \emph{rekurzivno podano}. Predstavlja Fibonaccijevo zaporedje.
	
	\item \emph{aritmeti"cno aporedje}
	\begin{gather*}
	a_n = a_1 + (n-1)d, n \in \NN\\
	a_1, a_1 + d, a_1 + 2d, \ldots
	\end{gather*}
	To lahko zapi"semo tudi z rekurzivno zvezo:
	\begin{equation*}
	\begin{cases}
	a_{n+1} = a_n + d, n \in \NN\\
	a_1 = a_1
	\end{cases}
	\end{equation*}
	
	\item \emph{geometrijsko zaporedje}
	\begin{gather*}
		a_n = a_1 q{n-1}, n \in \NN\\
		a_1, a_1 q, a_1 q^2, \ldots
	\end{gather*}
	Zapisano z rekurzivno zvezo:
	\begin{equation*}
	\begin{cases}
	a_{n+1} = a_n q, n \in \NN \\
	a_1 = a_1
	\end{cases}
	\end{equation*}
\end{enumerate}

\deff
\begin{itemize}
	\item Zaporedje $a_n$ je \emph{navzgor omejeno} "ce je zaloga vrednosti preslikave $n \mapsto a_n$ navzgor omejena, t.j.:
	\begin{equation*}
	\exists M \in \RR \forall n \in \NN: a_n \leq M
	\end{equation*}
	
	\item \emph{Natan"cna zgornja meja} zaporedja $a_n$ je natanc"na zgornja meja zaloge vrednosti preslikave $n \mapsto a_n$ in jo ozna"cimo s $\sup a_n$.
	
	\item "Stevilo $M$ imenujemo \emph{zgornja meja} zaporedja $a_n$.
	
	\item Analogno definiramo navzdol omejeno, natan"cno spodnjo mejo $\inf a_n$, $\max$ in $\min$.
\end{itemize}

\textsc{Primer:}

$a_n = \dfrac{1}{n}, n \in \NN$
	
navzgor omejeno z 1: $\forall n \in \NN: \dfrac{1}{n} \leq 1$
	
navzdol omejeno z 0: $\forall n \in \NN: \dfrac{1}{n} \geq 0$
	
$\sup \dfrac{1}{n} = 1$, ker je 1 zgornja meja in $a_1 = 1$.
	
\dashuline{$\inf 1/n = 0$}
0 je spodnja meja.

Izberemo $\varepsilon > 0$. Dokazujemo da $\varepsilon$ ni spodnja meja. Po arhimedski lastnosti:
\begin{equation*}
\exists n \in \NN: \dfrac{1}{n} < \varepsilon
\end{equation*}
	
$\max \dfrac{1}{n} = 1$
	
$\min \dfrac{1}{n}$ ne obstaja.

\deff Zaporedje $a_n$ \emph{konvergira} proti $a \in \RR$, "ce:
\begin{equation*}
\forall \varepsilon > 0 \exists n_0 \in \NN \forall n \in \NN: n \geq n_0 \Rightarrow |a_n - a| < \varepsilon
\end{equation*}
"Stevilo $a$ imenujemo \emph{limita zaporedja} in ozna"cimo z:
\begin{equation*}
a = \lim_{\toinf{n}} a_n
\end{equation*}

"Ce zaporedje $a_n$ konvergira, je $a_n$ \emph{konvergentno zaporedje}. Sicer je \emph{divergentno zaporedje}.
\begin{equation*}
|a_n - a| < \varepsilon \iff a_n \in (a - \varepsilon, a + \varepsilon)
\end{equation*}
Zunaj $\varepsilon$-te okolice je kve"cjemo kon"cnomnogo "celnov.

Zapis $\lim_{\toinf{n}} a_n = a$ pomeni, da zaporedje konvergira in njegova limita je $a$. To ne velja, "ce zaporedje divergira, ali pa njegova konvergira in njegova limita ni $a$.

\textsc{Primeri:}
\begin{enumerate}[1)]
	\item $a_n = 1, n \in \NN$
	
	\dashuline{$\lim_{\toinf{n}} a_n = 1$}
	
	Izberemo poljuben $\varepsilon > 0: |a_n - 1| = |1-1| = 0 < \varepsilon$ za vse $n \in \NN$.
	
	\item $b_n = \dfrac{1}{n}, n \in \NN$
	\begin{equation*}
	\lim_{\toinf{n}} \dfrac{1}{n} = 0
	\end{equation*}
	Po arhimedski lastnosti:
	\begin{align*}
	\forall \varepsilon > 0: \exists m : \dfrac{1}{m} &< \varepsilon\\
	n \geq m: \dfrac{1}{n} &\leq \dfrac{1}{m}\\
	- \varepsilon < \dfrac{1}{n} &\leq \dfrac{1}{m} < \varepsilon
	\end{align*}
	\begin{equation*}
	\forall n \in \NN, n \geq m: \dfrac{1}{n} \in (-\varepsilon, \varepsilon)
	\end{equation*}
	
	\item $c_n = (-1)^n \dfrac{1}{n}$
	\begin{equation*}
	\lim_{\toinf{n}} c_n = 0
	\end{equation*}
	ker:
	\begin{equation*}
	\left|(-1)^n \dfrac{1}{n} - 0\right| = \dfrac{1}{n} < \varepsilon
	\end{equation*}
	po prej"snjem primeru
	
	\item $d_n = (-1)^n$
	
	\dashuline{zaporedje divergira}
	
	Denimo, da je $x$ limita tega zaporedja:
	\begin{itemize}
		\item "ce $x = -1: \varepsilon = 1$, zunaj $(-1 - \varepsilon, -1 + \varepsilon) = (-2, 0)$ le"zijo vsi sodi "cleni zaporedja, ki jih je neskon"cno, zato -1 ni limita.
		
		\item Analogno za $x = 1$.
		
		\item $x \neq 1 \land x \neq -1$
		\begin{equation*}
		d = \min \{|x-1|, |x+1|\}
		\end{equation*}
		velja:
		\begin{align*}
		1 &\notin (x - \dfrac{d}{2}, x + \dfrac{d}{2}) \\
		-1 &\notin (x - \dfrac{d}{2}, x + \dfrac{d}{2})
		\end{align*}
		Vsi "cleni zaporedja le"zijo izven tega intervala, zato $x$ ni limita. Sledi: $d_n$ divergira.
	\end{itemize}
\end{enumerate}

\textsc{Trditev:} Konvergentno zaporedje ima eno samo limito.

\textsc{Dokaz:} Denimo da sta $a$ in $b$ limiti zaporedja $a_n$.
\begin{multicols}{2}
Izberemo poljuben $\varepsilon > 0:$
\columnbreak
\begin{gather*}
	\exists n_a \forall n: n > n_a \Rightarrow |a_n - a| < \varepsilon \\
	\exists n_b \forall n: n > n_b \Rightarrow |a_b - a| < \varepsilon
\end{gather*}
\end{multicols}
\begin{equation*}
|a - b| = |(a-a_n) + (a_n-b)| \leq |a-a_n| + |a_n-b| < 2\varepsilon
\end{equation*}
\begin{equation*}
\forall \varepsilon > 0: |a-b| < 2\varepsilon \Rightarrow |a-b| = 0 \Rightarrow a = b
\end{equation*}

\textsc{Trditev:} Konvergentno zaporedje je omejeno

\textsc{Dokaz:} Denimo, da je $a_n$ konvergentno zaporedje z limito $a$. Izberemo $\varepsilon = 1$ in po definiciji velja:
\begin{equation*}
\exists n_0 \in \NN \forall n \geq n_0: |a_n - a| < \varepsilon = 1
\end{equation*}
Lahko skonstruiramo mno"zico "clenov, ki so izven $\varepsilon$-te okolice $a$. Tej mno"zici dodamo tudi zgornjo mejo okolice $a + \varepsilon = a + 1$. Ker je izven okolice kon"cnomnogo "clenov, ima ta mno"zica maksimum:
\begin{equation*}
\max \left\{a+1, a_1, a_2, \ldots, a_{n_0-1} \right\} = M
\end{equation*}
Za vsak $n$ velja: $a_n \leq M$, ker "ce $n \leq n_0 - 1$, je $a_n$ v zgornji mno"zici, ki smo ji doli"cili maksimum, "ce $n \geq n_0$ je $a_n$ v okolici $a$, to pomeni $a_n < a + 1$, $a+1$ je v zgornji mno"zici, ki smo ji dolo"cili maksimum.

Analogno lahko naredimo za spodnjo mejo. \hfill $\square$

Ni vsako omejeno zaporedje konvergentno. Primer: $a_n = (-1)^n, n \in \NN$.

\deff Naj bo $a_n$ zaporedje. "Stevilo $s$ je \emph{stekali"c"ce zaporedja} $a_n$, "ce v vsaki okolici $s$ le"zi neskon"cno "clenov zaporedja.

\textsc{Primer:} $a_n = (-1)^n, n \in \NN$

Vemo, da ni konvergentno. -1 in 1 sta stekali"s"ci $a_n$, ker vvsi "cleni z lihimi indeksi le"zijo na $(-1 - \varepsilon, -1 + \varepsilon)$ za $\varepsilon > 0$. Analgono za 1 in sode "clene.

\textsc{Opombi:}
\begin{enumerate}[1)]
	\item $s$ je stekali"s"ce $\iff \forall \varepsilon \in \RR, \varepsilon > 0$ je $|a_n - s| < \varepsilon$ izpolnjen za neskon"cno mnogo indeksov $n$.
	
	\item "Ce je zaporedje $a_n$ konvergentno z limito $a$, potem je $a$ edino stekali"s"ce zaporedja $a_n$, ker na $(a-b, a+b)$ le"zijo vsi, razen kon"cno mnogo "clenov.
\end{enumerate}

\textsc{Primeri:}
\begin{enumerate}[1)]
	\item Zaporedje s 3 stekali"s"ci: $a_n = n \mod3$.
	\item Zaporedje z 2 stekali"s"cama, ki ima same razli"cne "clene: $a_n = (-1)^n \left(1+ \dfrac{1}{n}\right)$
	\item Zaporejde z neskon"cno stekali"s"ci:
	\begin{equation*}
	1, 1, 2, 1, 2, 3, 1, 2, 3, 4, 1, 2, 3, 4, 5, \ldots
	\end{equation*}
	Stekali"s"ca zaporedja so vsa naravna "stevila.
	\item Ali ima zaporedje ne"stevno stekali"s"c? Da.
	
	Obstaja bijekcija $\NN \rightarrow \QQ$.
	\begin{equation*}
	\QQ = \{a_1, a_2, a_3, \ldots\}
	\end{equation*}
	Stekali"s"ca zaporedja $a_n: x \in \RR,  \varepsilon > 0$ na $(x-\varepsilon, x+\varepsilon)$ le"zi neskon"cno nogo $\QQ$ "stevil, torej neskon"cno mnogo "clenov zaporedja. Zato je $x$ stekali"s"ce.
	
	\item Divergentno zaporedje brez stekali"s"c: $a_n = n$.
\end{enumerate}

\textsc{Trditev: } "Ce vsaka okolica "stevila $s \in \RR$ vsebuje "clen zaporedja $a_n, a_n \neq s$, potej je $s$ stekali"s"ce $a_n$.

\textsc{Dokaz: } Izberemo poljuben $\varepsilon > 0$. Obstaja $n_1 \in \NN: a_{n_1} \in (s-\varepsilon, s+\varepsilon), a_{n_1} \neq s$. Definiramo razdaljo: $d_1 = |s - a_{n_1}|$. Obstaja $n_2: a_{n_2} \in (s-d_1, s+d_1), a_{n_2} \neq s, n_1 \neq n_2$. Postopek nadaljujemo $\square$.

\textsc{Izrek:} Vsako omejeno zaporedje ima stekali"s"ce.

\textsc{Dokaz:} Ker je zaopredje $a_n$ omejeno, ima spodnjo mejo $m$ in zgornjo mejo $M$.
\begin{gather*}
\mathcal{U} = \{u \in \RR; a_n <u \text{ je izpolnjeno za kon"cno mnogo "clenov zaporedja}\}\\
m \in \mathcal{U}, M + 1 \notin \mathcal{U}
\end{gather*}
$\Rightarrow \mathcal{U}$ je navzgor omejena in neprazna, torej obstaja $\sup \mathcal{U} = s$.

Izberemo $\varepsilon > 0$.:
\begin{itemize}
	\item $s + \varepsilon \notin \mathcal{U}: a_n < s + \varepsilon$ je izpolnjena za neskon"cno mnogo "clenov.
	\item $s - \varepsilon: \exists u \in (s-\varepsilon, s] \cap \mathcal{U}$.	Ker $u \in \mathcal{U}, s - \varepsilon < u$, je $s - \varepsilon \in \mathcal{U}$, zato velja da $a_n < s-\varepsilon$ velja za kon"cno mnogo "clenov zaporedja.
\end{itemize}
Sledi: na $[s-\varepsilon, s + \varepsilon)$ le"zi neskon"cno mnogo "clenov zaporedja. $s$ je stekali"s"ce. 

\hfill $\square$

\subsection{Monotona zaporedja}
\deff
\begin{itemize}
	\item Zaporedje $a_n$ je \emph{nara"s"cajao"ce}, "ce velja: $\forall n \in \NN: a_{n+1} \geq a_n$.
	\item Zaporedje $a_n$ je \emph{padajo"ce}, "ce velja: $\forall n \in \NN: a_{n+1} \leq a_n$.
	\item Zaporedje $a_n$ je \emph{strogo nara"s"cajao"ce}, "ce velja: $\forall n \in \NN: a_{n+1} > a_n$.
	\item Zaporedje $a_n$ je \emph{strogo padajo"ce}, "ce velja: $\forall n \in \NN: a_{n+1} < a_n$.
	\item Zaporedje $a_n$ je \emph{monotono}, "ce je zaporedje bodisi nara"s"cajo"ce, ali padajo"ce.
	\item Zaporedje $a_n$ je \emph{strogo monotono}, "ce je zaporedje bodisi strogo nara"s"cajo"ce, ali strogo padajo"ce.
\end{itemize}

\textsc{Primer:}
\begin{enumerate}[1)]
	\item $a_n = -n$ (strogo) padajo"ce
	\item $a_n = 1$ nara"s"cajo"ce ali padajo"ce
	\item $a_n = \dfrac{1}{n}$ je padajo"ce in navzdol omejeno
	\item $a_n = (-1)^n$ ni ne padajo"ce ne nara"s"cajo"ce
\end{enumerate}

\textsc{Trditev:} Monotono zaporedje je konvergentno natanko tedaj, kadar je omejeno. "Ce je zaporedje $a_n$ nara"s"cajo"ce in navzgor omejeno potem:
\begin{equation*}
\lim_{\toinf{n}}a_n = \sup a_n
\end{equation*}
"Ce je zaporedje $a_n$ padajo"ce in navzdol omejeno potem:
\begin{equation*}
\lim_{\toinf{n}}a_n = \inf a_n
\end{equation*}
Dokaz za v eno stran "ze vemo.

\textsc{Dokaz: } (v drugo stran ekvivalence)

Denimo, da je zaporedje $a_n$ nara"s"cajo"ce in navzgor omejeno. Ker je $a_n$ navzgor omejeno: $\exists a := \sup a_n$.

Dokazujemo \dashuline{$a_n$ konjugira proti $a$}

Izberemo poljubene $\varepsilon > 0$. Ker $a - \varepsilon$ ni zgornja meja zaporedja: $\exists n_0 \in \NN: a_{n_0} > a - \varepsilon$. Ker je zaporedje $a_n$ nara"s"cajo"ce in navzgor omejeno z $a$ velja:
\begin{gather*}
n \geq n_0: a-\varepsilon < a_{n_0} \leq a_n \leq a\\
\forall n \geq n_0 a_n \in (a - \varepsilon, a + \varepsilon)
\end{gather*}
Sledi: $a = \lim_{\toinf{n}}a_n$ \hfill $\square$

Analogno za navzdol omejeno padajo"ce zaporedje.

\textsc{Primer:} $a_n = \dfrac{1}{\sqrt{n}}, n \in \NN$

Najprej doka"zemo, da je $a_n$ padajo"ce zaporedje, t.j:
\begin{align*}
\dfrac{1}{\sqrt{n + 1}} &\leq \dfrac{1}{\sqrt{n}} \\
\sqrt{n} &\leq \sqrt{n+1}
\end{align*}
Vemo, da je navzdol omejeno z 0. Ker je padajo"ce in navzdol omejeno, je konvergentno.

\dashuline{$\lim_{\toinf{n}} a_n = 0$}

Vemo: $\lim_{\toinf{n}} a_n = \inf a_n = a$

Vemo, da $a$ ni negativen, ker so vsi "cleni zaporedji zaporedja pozitivini. "Ce bi $a > 0$:
\begin{align*}
a_n &\geq a\\
\dfrac{1}{\sqrt{n}} &\geq a\\
\dfrac{1}{n} &\geq a^2 > 0 \text{ za vse $n \in \NN$}
\end{align*}
Zaradi arhimedske lastnosti vemo, da to ni res. $\rightarrow \leftarrow$

Torej $a$ ni negativen in $a$ ni pozitivne, torej $a = 0 \Rightarrow \lim_{\toinf{n}} a_n = 0$.

\subsection{Podzaporedja}
Podzaporedje zaporedja $a_n$ je zaporedje, ki vsebuje samo nekatere "clene zaporedja $a_n$ v enakem vrstnem redu.
\begin{equation*}
a_1, \xcancel{a_2}, \xcancel{a_3}, a_4, \ldots
\end{equation*}

\deff Naj bo $a_n$ zaporedje in naj bo $n_j$ strogo nara"s"cajo"ce zaporedje naravnih "stevil. Zaporedje $(a_{n_j})_{j=1}^\infty$ je podzaporedje $(a_n)_{n=1}$.

\textsc{Primeri:}
\begin{enumerate}[1)]
	\item $(a_n)_{n=1}^\infty$ zaporedje
	
	podzaporedje $(a_n)_{n=1}^\infty$
	
	\item $b_n = (-1)^n, n \in \NN$ ni konvergentno, ima dve stekali"s"ci
	
	$b_{2n-1} = -1$ za vsak $n \in \NN$: $b_1, b_3, b_5, \ldots$ je podzaporedje lihih "clenov.
	
	Podobno lahko naredimo podzaporedje sodih "clenov. Podzaporedja lihih in sodih "clenov sta konvergentna.
\end{enumerate}

\textsc{Trditev} Naj bo $a_n$ zaporedje. "Ce je $a_n$ konvergentno, potem je konvergentno tudi vsako njegovo podzaporedje $(a_{n_j})_{j=1}^\infty$ in velja:
\begin{equation*}
\lim_{\toinf{n}}a_n = \lim_{\toinf{j}} a_{n_j}
\end{equation*}
\textsc{Dokaz:} Naj bo $a = \lim_{\toinf{n}}a_n$

Izberemo poljuben $\varepsilon > 0$. Vemo: $\exists n_0 \in \NN \forall n \geq n_0: |a_n - a| < \varepsilon$

Potem velja:
\begin{equation*}
j \geq n_0: |a_{n_j} - a| < \varepsilon
\end{equation*}
ker je $n_j \geq n_{n_0} \geq n_0$

\textsc{Opomba:} "Ce neko podzaporedje danega zaporedja konvergira, ni nujno, da je dano zaporedje konvergentno. Primer: $(-1)^n$.

\deff \emph{Rep zaporedja} je zaporedje, ki ga dobimo iz danega zaporedja tako, da izpustimo prvih kon"cno mnogo "clenov.

Zaporedje je konvergentno natanko tedaj, kadar je konvergenten njegov rep.

\textsc{Primer:} $1, 2, 4, 16, 36, 16, 4, 2, 1, \dfrac{1}{2}, \dfrac{1}{3}, \dfrac{1}{4}, \ldots, \dfrac{1}{n}, \ldots$

\subsection{Ra"cunanje z zaporedji}
\textsc{Trditev:} Naj bosta $a_n$ in $b_n$ konvergentni zaporedji. Tedaj konvergirajo tudi naslednja zaporedja:
\begin{itemize}
	\item $a_1 + b_1, a_2 + b_2, \ldots, a_n + b_n, \ldots$
	\item $a_1 - b_1, a_2 - b_2, \ldots, a_n - b_n, \ldots$
	\item $a_1 b_1, a_2 b_2, \ldots, a_n b_n, \ldots$
\end{itemize}
in velja:
\begin{itemize}
	\item $\lim_{\toinf{n}} (a_n + b_n) = \lim_{\toinf{n}} a_n + \lim_{\toinf{n}} b_n$
	\item $\lim_{\toinf{n}} (a_n - b_n) = \lim_{\toinf{n}} a_n - \lim_{\toinf{n}} b_n$
	\item $\lim_{\toinf{n}} (a_n b_n) = \lim_{\toinf{n}} a_n \cdot \lim_{\toinf{n}} b_n$
\end{itemize}
\textsc{Dokaz}
\begin{itemize}
	\item[Vsota:]
	Izberemo poljuben $\varepsilon > 0: a = \lim_{\toinf{n}} a_n, b = \lim_{\toinf{n}} b_n$.
	\begin{equation*}
	|(a_n + b_n) - (a + b)| = |(a_n - a) + (b_n - b)| \leq |a_n - a| + |b_n - b| \leq 2\varepsilon \text{ za $n \geq \max \{n_a, n_b\}$}
	\end{equation*}
	\hfill $\square$

	\item[Produkt:]
	\begin{equation*}
	|a_n b_n - ab| = |a_n b_n - ab_n + ab_n - ab| =  = |b_n(a_n - a) + a(b_n - b)| \leq |b_n(a_n - a)| + |a(b_n - b)|
	\end{equation*}
	Ker je $b_n$ konvergentno, je omejeno: $\exists M_b \forall n \in \NN: |b_n| < M_b$
	\begin{equation*}
	\leq M_b|a_n - a| + |a||b_n - b| \leq \varepsilon \text{ "ce $n \geq \max \{n_a, n_b\}$}
	\end{equation*}
	ker vemo: $|a_n - a| \leq \dfrac{\varepsilon}{2M_b}$ za $n \geq n_a$ in $|b_n - b| \leq \dfrac{\varepsilon}{2|a|}$ za $n \geq n_b$.
	
	\hfill $\square$
\end{itemize}
\textsc{Posledica:} "Ce je $a_n$ konvergentno zaporedje in $\lambda \in \RR$, potem je zaporedje $\lambda a_1, \lambda a_2, \ldots$ konvergentno in velja:
\begin{equation*}
\lim_{\toinf{n}} \lambda a_n = \lambda \lim_{\toinf{n}} a_n
\end{equation*}
Dokaz je enak kot dokaz za produkt limit, kjer $b_n = \lambda$.

Ker veljajo pravila za dva "clena, veljajo tudi za kon"cno mnogo.

\textsc{Trditev:} Naj bo $a_n$ konvergentno zaporedje in $\forall n \in \NN a_n \neq 0$ in $\lim_{\toinf{n}} a_n \neq 0$. Potem je zaporedje
\begin{equation*}
\dfrac{1}{a_1}, \dfrac{1}{a_2}, \ldots, \dfrac{1}{a_n}, \ldots
\end{equation*}
konvergetno in velja:
\begin{equation*}
\lim_{\toinf{n}} \dfrac{1}{a_n} = \dfrac{1}{\lim_{\toinf{n}} a_n}
\end{equation*}

\textsc{Dokaz:} $a = \lim_{\toinf{n}} a_n$
\begin{equation*}
\left|\dfrac{1}{a_n} - \dfrac{1}{a}\right| = \left|\dfrac{a - a_n}{a_n a}\right| = \dfrac{|a - a_n}{|a||a_n|} \leq \dfrac{|a - a_n|}{|a| \eta} < \varepsilon \text{ za dovolj velike $n$}
\end{equation*}
$|a_n|$ lahko omejino stran od 0, t.j.: $\exists \eta > 0 \forall n \in \NN: |a_n| \geq \eta$

Konstrukcija $\eta$: Zunaj $\left(a - \dfrac{|a|}{2}, a + \dfrac{|a|}{2}\right)$ le"zi kon"cno mnogo "clenov zaporedja.
\begin{equation*}
\eta := \min \left\{\left|a - \dfrac{|a|}{2}\right|, \left|a + \dfrac{|a|}{2}\right|, |a_1|, |a_2|, \ldots, |a_{n_0}  \right\}
\end{equation*}

\textsc{Posledica:} Naj bosta $a_n$ in $b_n$ konvergentni zaporedji, $\forall n \in \NN: b_n \neq 0$, $\lim_{\toinf{n}} b_n \neq 0$. Potem je zaporedje $\frac{a_n}{b_n}$ konvergentno in velja:
\begin{equation*}
\lim_{\toinf{n}}\dfrac{a_n}{b_n} \ \dfrac{\lim_{\toinf{n}} a_n}{\lim_{\toinf{n}}b_n}
\end{equation*}

\textsc{Izrek o sendvi"cu:} Naj bodo $a_n, b_n, c_n$ taka zaporedja, da 
\begin{equation*}
\forall n \in \NN: a_n \leq b_n \leq c_n
\end{equation*}
"Ce sta zaporedji $a_n$ in $c_n$ konvergentni in $\lim_{\toinf{n}} a_n = \lim_{\toinf{n}}c_n$, potem je $b_n$ konvergentno in velja:
\begin{equation*}
\lim_{\toinf{n}} b_n = \lim_{\toinf{n}} a_n = \lim_{\toinf{n}} c_n
\end{equation*}
\textsc{Primer:} $b_n = \sqrt{n+1} - \sqrt{n}$
\begin{gather*}
\sqrt{n+1} - \sqrt{n} = \dfrac{n+1 - n}{\sqrt{n+1} + \sqrt{n}}\\
0 \leq \sqrt{n+1} - \sqrt{n} \leq \dfrac{1}{2\sqrt{n}}
\end{gather*}
ker $\lim_{\toinf{n}} 0 = 0$ in $\lim_{\toinf{n}} \dfrac{1}{2\sqrt{n}} = 0$:
\begin{equation*}
\lim_{\toinf{n}} b_n = 0
\end{equation*}

\textsc{Dokaz izreka:}
\begin{equation*}
L = \lim_{\toinf{n}}a_n = \lim_{\toinf{n}}c_n
\end{equation*}
\begin{multicols}{2}
Vzamemo poljuben $\varepsilon > 0$:
\columnbreak
\begin{gather*}
\exists n_a: n \geq n_a \rightarrow |a_n - L| < \varepsilon \\
\exists n_c: n \geq n_c \rightarrow |c_n - L| < \varepsilon
\end{gather*}
\end{multicols}
"Ce vzamemo $n \geq \max \{n_a, n_c\}: L - \varepsilon < a_n \leq b_n \leq c_n < L + \varepsilon$.

Torej $b_n \in (L - \varepsilon, L + \varepsilon)$ za vse $n \geq \max \{n_a, n_c\}$. Zato $b_n$ konvergira proti $L$.

\textsc{Trditev:} Naj bosta $a_n$ in $b_n$ konvergetni zaporedji. "Ce $a_n \leq b_n$ za vse $n \in \NN$, potem velja $\lim_{\toinf{n}} a_n \leq \lim_{\toinf{n}} b_n$. \textbf{Opomba:} Trditev s strogimi neena"caji v splo"snem ne velja.
\begin{gather*}
a_n = 0, b_n = \dfrac{1}{n}, a_n < b_n\\
\lim_{\toinf{n}}a_n = \lim_{\toinf{n}}b_n
\end{gather*}

\textsc{Primer:} Obravnavaj konvergenco zaporedja (Newtnova formula):
\begin{align*}
x_1 &= 2 \\
x_{n+1} &= x_n - \dfrac{x_n^2 - 2}{2 x_n}
\end{align*}
Zaporedje je s to rekurzivno zvezo dobro definirano, t.j.: \dashuline{$\forall n \in \NN: x_n \neq 0$}
\begin{equation*}
x_{n+1} = x_n - \dfrac{x_n^2 - 2}{2x_n} = \dfrac{2x_n^2 - x_n^2 + 2}{2x_n} = \dfrac{x_n^2 + 2}{2x_n} > 0
\end{equation*}
"Ce $x_n \neq 0$ potem $x_{n+1} \neq 0$. To lahko preverimo tudi z indukcijo.

Doka"zemo lahko, da: \dashuline{$x_n$ je padajo"ce}
\begin{equation*}
x_{n+1} = x_n - \dfrac{x_n^2 - 2}{2x_n}
\end{equation*}
"Ce je $\dfrac{x_n^2 - 2}{2x_n} > 0$, potem je padajo"ce. Dovolj je dokazati, da je $x_n^2 - 2 \geq 0$, ker tedaj iz rekurzivne zveze sledi: $x_{n+1} \leq x_n$.
\begin{equation*}
x_{n+1}^2 = \left(\dfrac{x_n^2 + 2}{2x_n}\right)^2 - 2 = \dfrac{x_n^4 + 4x_n^2 + 4 - 8x_n^2}{4x_n^2} = \dfrac{x_n^4 - 4x_n^2 + 4}{4x_n^2} = \dfrac{(x_n^2 - 2)^2}{4x_n^2} \geq 0
\end{equation*}
$x_n$ je padajo"ce in navzdol omejeno, zato je konvergentno. Torej lahko izra"cunamo limito. Velja:
\begin{align*}
x_{n+1} &= \dfrac{x_n^2 + 2}{2x_n} \\
2x_n x_{n+1} &= x_n^2 + 2
\end{align*}
in vemo: 
\begin{equation*}
\exists \lim_{\toinf{n}} x_n =: x
\end{equation*}
Ker je $x_{n+1}$ rep zaporedja $x_n$, je konvergentno in ima isto limito kot $x_n$. Ker s konvergentnimi zaporedji lahko ra"cunamo:
\begin{align*}
\lim_{\toinf{n}} (2x_n x_{n+1}) & = \lim_{\toinf{n}} (x_n^2 + 2) \\
2x^2 &= x^2 + 2 \\
x^2 &= 2\\
x &= \pm \sqrt{2}
\end{align*}
vemo $x \geq 0$, torej 
\begin{equation*}
\lim_{\toinf{n}} x_n = \sqrt{2}
\end{equation*}

\textsc{Izrek:} Naj bo $I_n = [a_n, b_n], a_n < b_n$ zaporedje vlo"zenih zaprtih intervalov, t.j.:
\begin{equation*}
\forall n \in \NN: [a_{n+1}, b_{n+1}] \subset [a_n, b_n]
\end{equation*}
Denimo, da zaporedje njihovih dol"zin konvergira proti ni"c:
\begin{equation*}
\lim_{\toinf{n}} (b_n, a_n) = 0
\end{equation*}
Tedaj obstaja natanko eno "stevilo $c \in \bigcap_{n=1}^{\infty} I_n$

\textsc{Dokaz:} $a_n$ je nara"s"cajo"ce zaporedje, $b_n$ je padajo"ce zaporedje. Obe zaporedji sta omejeni (navzdol z $a_1$ in navzgor z $b_1$). Zato sta obe konvergentni. Zato $\lim_{\toinf{n}} (b_n - a_n) = \lim_{\toinf{n}} b_n - \lim_{\toinf{n}} a_n$.
\begin{gather*}
\Rightarrow \lim_{\toinf{n}} a_n = \lim_{\toinf{n}} b_n =: c \\
\forall n \in \NN: c \in I_n \\
c = \sup a_n: a_n \leq c\\
c = \inf b_n: c \leq b_n \\
c \in [a_n, b_n]
\end{gather*}
Dokaz, da je "stevilo $c$ eno samo doma. \hfill $\square$

\textsc{Izrek:} Naj bo $a_n$ zaporedje. "Stevilo $b$ je stekali"s"ce zaporedja $a_n$ natanko tedaj, kadar obstaja podzaporedje zaporedja $a_n$, ki konvergira proti $s$.

\textsc{Dokaz:}
\begin{itemize}
	\item[($\Leftarrow$)] Denimo, da podzaporedje $a_{n_j}$ konvergira proti $s$. Dokazujemo, da je $s$ stekali"s"ce $a_n$.
	
	Izberemo poljuben $\varepsilon > 0$. Obstaja $j_0$, da za vsak $j \geq j_0$ velja:
	\begin{equation*}
	a_{n_j} \in (s - \varepsilon, s+ \varepsilon)
	\end{equation*}
	Torej je na $s - \varepsilon, s + \varepsilon$ neskon"cno "clenov zaporedja. Zato je $s$ stekali"s"ce.
	
	\item[($\Rightarrow$)] Denimo, da je $s$ stekali"s"ce zaporedja $a_n$. Za dokaz bomo skonstruirali tako zaporedje, da bo konvergiralo proti $s$. Naj bo
	\begin{equation*}
	U_m = (s - \dfrac{1}{m}, s + \dfrac{1}{m})
	\end{equation*}
	Na $U_1$ obstaja neskon"cno mnogo "clenov zaporedja, zato lahko izberemo $a_{n_1} \in U_1$.
	
	Na $U_2$ obstaja neskon"cno mnogo "clenov zaopredja, zato obstaja:
	\begin{equation*}
	n_2 > n_1 \land a_{n_2} \in U_2
	\end{equation*}
	Ta postopek induktivno nadaljujemo. Recimo, da "ze imamo:
	\begin{equation*}
	a_{n_1}, a_{n_2}, \ldots, a_{n_k}, \text{ da } n_1 < n_2 < \ldots n_k \text{ in vsak } a_{n_j} \in U_j
	\end{equation*}
	Na intervalu $U_{k+1}$ obstaja neskonc"cno "clenov zaporedja, zato obstaja $n_{k+1} > n_k$ in $a_{n_{k+1}} \in U_{k+1}$
	
	Podzaporedje $a_{n_1}$ konvergira proti $s$ po konstrukciji. \hfill $\square$
\end{itemize}

\subsection{Cauchyjev pogoj}
Zanima nas opis konvergence brez sklicevanja na limito. $a_n$ je konvergentno, "ce obstaja $a \in \RR$:
\begin{equation*}
\forall \varepsilon \exists n_0 \forall n \geq n_0: |a_n - a| < \varepsilon
\end{equation*}
\deff Zaporedje $a_n$ izpolnjuje \emph{Cauchyjev pogoj}, "ce velja:
\begin{equation*}
\forall \varepsilon > 0 \exists n_0 \in \NN \forall n, m \geq n_0: |a_n - a_m| < \varepsilon
\end{equation*}
"Ce zaporedje izpolnjuje Cauchyjev pogoj, pravimo, da je zaporedje \emph{Cauchyjevo}.

\textsc{Izrek:} Zaporedje realnih "stevil $a_n$ je konvergentno natanko tedaj, kadar je Cauchyjevo.

\textbf{Opomba:} Za zaporedje racionalnih "stevil ta izrek ne velja. Obstaja zaporedje $q_n, q_n \in \QQ$, $q_n$ je Cauchyjevo ni pa konvergentno v $\QQ$. Npr. zaporedje racionalnih pribli"zkov za $\sqrt{2}$.

\textsc{Dokaz}:
\begin{itemize}
	\item[($\Rightarrow$)] Denimo, da zaporedje $a_n$ konvergira proti $a$. Za poljuben $\varepsilon > 0$ po definiciji konvergence velja:
	\begin{equation*}
	\exists n_0 \forall n \geq n_0: |a_n - a| < \varepsilon
	\end{equation*}
	Za $n, m \geq n_0$ velja:
	\begin{equation*}
	|a_n - a_m| = |a_n - a + a - a_m| \leq |a_n - a| + |a - a_m| < 2 \varepsilon
	\end{equation*}
	Torej je zaporedje Cauchyjevo.
	
	\item[($\Leftarrow$)] Denimo, da je $a_n$ Cauchyjevo.
	
	\dashuline{Potem je $a_n$ omejeno}
	\begin{equation*}
	\exists n_0: \forall n, m \geq n_0: |a_n - a_m| < 1 = \varepsilon
	\end{equation*}
	Zato je $|a_n - a_{n_0}| < 1$ za vse $n \geq n_0$. (Namesto $m$ vzamemo $n_0$, saj vstreza pogoju)
	
	Torej vsi "cleni razen $a_1, a_2, \ldots, a_{n_1}$ le"zijo na $(a_{n_0} - 1, a_{n_0} + 1)$. Zato je $a_n$ omejeno.
	
	Vemo, da ima vsako omejeno zaporedje stekali"s"ce $s$. Trdimo \dashuline{$s = \lim_{\toinf{n}} a_n$}
	
	Izberemo poljuben $\varepsilon > 0$. Denimo, da je zunaj $(s - \varepsilon, s + \varepsilon)$ neskon"cno "clenov zaporedja. Potem bi zunaj $s - \varepsilon, s + \varepsilon$ obstajalo omejeno podzaporejde zaporedja $a_n$, zato ima to podzaporeje stekali"s"ce $t, t \neq s$. $t$ je stekali"s"ce $a_n$.
	
	\dashuline{Cauchyjevo stekali"s"ce nima nikoli dveh razli"cnih stekali"s"c.}
	
	Denimo, da sta $s$ in $t$ stekali"s"ci.
	\begin{equation*}
	d = |s - t|
	\end{equation*}
	Na $(s - \dfrac{d}{3}, s + \dfrac{d}{3})$ le"zi neskon"cno "clenov zaporedja.
	
	Na $(t - \dfrac{d}{3}, t + \dfrac{d}{3})$ le"zi neskon"cno "clenov zaporedja.
	
	"Ce bi bilo zaporedje Cauchyjevo, $\exists n_0 \forall n, m \geq n_0: |a_n - a_m| < \dfrac{d}{3}$. Pridemo v protislovje s prej"snjima dvema izjavama. $\rightarrow \leftarrow$
	
	\hfill $\square$
\end{itemize}

\textsc{Izrek:} Vsako omejeno zaporedje, ki ima eno stekali"s"ce je konvergentno (v dokazu prej"snjega izreka ("ce si naredil doma"co nalogo, tudi na dolgo)).

\textbf{Opomba:} Izrek (zelo o"citno) ne velja za neomejena zaporedja.

\deff Naj bo $a_n$ zaporedje. Pravimo, da zaporedje $a_n$ \emph{konvergira proti neskon"cno}, "ce velja:
\begin{equation*}
\forall M \in \RR \exists n_o \in \NN \forall n \in \NN, n \geq n_0: a_n \geq M
\end{equation*}
V tem primeru pi"semo:
\begin{equation*}
\limninf a_n = \infty
\end{equation*}
%
Zaporedje, ki konvergira proti neskon"cno \textbf{ni} konvergentno. Z drugimi besedami: zaporedij, ki konvergirajo proti $\infty$ \textbf{ne} "stejemo med konvergentna zaporedja.

Podobno definiramo tudi $\limninf a_n = -\infty$:
\begin{equation*}
\forall m \in \RR \exists n_0 \forall n \in \NN, n \geq n_0: a_n \leq m
\end{equation*}
\textsc{Primer:} $a_n$: $\limninf a_n = \infty$

$b_n = n^{(-1)^n}$ ne konvergira proti neskon"cno.

\subsubsection*{Nekaj posebnih zaporedij:}
\textsc{Trditev:} Naj bo $a \in \RR$
\begin{enumerate}[1)]
	\item $|a| < 1 \Rightarrow a^n$ konvergentno z limito 0
	\item $a > 1 \Rightarrow a^n$ konvergira proti $\infty$
\end{enumerate}
\textsc{Dokaz:}
\begin{enumerate}[1)]
	\item $a \in (0, 1)$ zaporedje $a^n$ je padajo"ce: $a^{n+1} < a^n$ in navzdol omejeno z 0, zato je konvergetno.
	
	Naj bo $b_n = a^n$ in $b_{n+1} = a^{n+1} = ab_n$. Vemo:
	\begin{equation*}
	b = \limninf b_n
	\end{equation*}
	Torej:
	\begin{gather*}
	b = \limninf b_{n+1} = \limninf (ab_n) = a \limninf b_n = ab \\
	b(1-a) = 0
	\end{gather*}
	Ker $a \neq 1$ velja $b = 0$.
	
	"Ce $a \in (-1, 0)$ velja:
	\begin{equation*}
	- |a|^n \leq a^n \leq |a|^n 
	\end{equation*}
	Za $|a|^n$ velja zgoren dokaz. Nato lahko uporabimo izrek o sendvi"cu, zaradi katerega velja:
	\begin{equation*}
	\limninf a^n = 0
	\end{equation*}
	
	\item $a > 1: a^n$ je nara"s"cajo"ce
	
	"Ce bi bilo omejeno, bi bilo konvergentno. Po prej"snjem dokazu bi dobili $\limninf a_n = 0$, kar ni mogo"ce, ker je zaporedje nara"s"cajo"ce in so vsi "cleni pozitivni. Torej $a_n$ ni omejeno.
	
	\dashuline{Zato $\limninf a^n = \infty$}
	
	Ker je $a_n$ neomejeno, za $M \in \RR$ obstaja $n_0: a^{n_0} \geq M$. Ker je nara"s"cajo"ce velja:
	\begin{equation*}
	\forall n \geq n_0: a^n \geq M
	\end{equation*}
\end{enumerate}
%
\textsc{Trditev:} Za vsak $a > 0$ in vsak $m \in \NN$ obstaja natanko en $x  0$, ki re"si ena"cbo $x^m = a$. To re"sitev ozna"cimo z $\sqrt[m]{a}$. Veljajo osnovne lastnosti za $\sqrt[m]{~}$:

$a, b > 0, m , n \in \NN$
\begin{align*}
\sqrt[m]{ab} &= \sqrt[m]{a} \sqrt[m]{b} \\
\sqrt[n]{a^m} &= (\sqrt[n]{a})^m \\
\sqrt[nm]{a} &= \sqrt[m]{\sqrt[n]{a}} \\
\sqrt[np]{a^{nq}} &= \sqrt[p]{a^q} \\
\sqrt[p]{a^q} \sqrt[n]{a^m} &= \sqrt[pn]{a^{qn}} \sqrt[np]{a^{mp}} = \sqrt[np]{a^{qno + np}}
\end{align*}
%
\deff Naj bo $a > 0, a \in \RR, m, n \in \NN$. Pi"semo $a^{\frac{n}{m}} = \sqrt[m]{a_n}$
\begin{align*}
a^0 & = 1 \\
a^{-\frac{n}{m}} & = \dfrac{1}{a^{\frac{n}{m}}}
\end{align*}
Naj bo $q \in \QQ$. Obstajata $m \in \ZZ, n \in \NN: q = \frac{m}{n}$.
\begin{equation*}
a^q = a^{\frac{m}{n}}
\end{equation*}
\textbf{Opomba:} Je dobro definirano, ker ni odvisno od izbere ulomka (ena izmed lastnosi nam dovoljuje ``kraj"sanje'')

\textsc{Trditev:} Naj bo $a \in \RR, a > 0, p, q \in \QQ$. Velja:
\begin{align*}
a^{pq} &= (a^p)^q \\
a^pa^1 &= a^{p+q} \\
a^p b^p &= (ab)^p
\end{align*}

\textsc{Trditev:} Za $a \in \RR, a > 0$ velja: $\limninf \sqrt[n]{a} = 1$

\textsc{Dokaz:} $a > 1$:
\begin{align*}
\sqrt[n]{a} &\geq \sqrt[n+1]{a} \\
a^{n+1} &\geq a^n
\end{align*}
Zaporedje je padajo"ce. Ker je navzdol omejeno z 0 (tudi z 1), je konvergetno in njegova limiti $\geq 1$.

Denimo, da je $\limninf \sqrt[n]{a} = L > 1$:
\begin{align*}
\sqrt[n]{a} &\geq L > 1 \\
\forall n \in \NN: a &\geq L^n
\end{align*}
Ker je $L > 1$, $L^n$ konvergira proti $\infty$, kar pomeni, da bo presegel $a$. Torej protislovje $\leftarrow \rightarrow$.

Zato je $L = 1$. 

Za $a<1$ je podoben dokaz (DN). \hfill $\square$

\textsc{Trditev:} $\limninf \sqrt[n]{n} = 1$

\textsc{Dokaz:} 
\begin{multline*}
n = \left(\sqrt[n]{n}\right)^n = \left(\left(\sqrt[n]{n} - 1\right) + 1\right)^n = \\
= 1 + \binom{n}{1} \left(\sqrt[n]{n} - 1\right) + \binom{n}{2} \left(\sqrt[n]{n} - 1\right)^2 + \ldots \geq \\
\geq \binom{n}{2} \left(\sqrt[n]{n} - 1\right)^2 = \dfrac{n(n-1)}{2} \left(\sqrt[n]{n} - 1\right)^2
\end{multline*}
Dobili smo:
\begin{equation*}
0 \leq \dfrac{n(n-1)}{2} \left(\sqrt[n]{n} - 1\right)^2 \leq n
\end{equation*}
torej velja:
\begin{equation*}
0 \leq \dfrac{n - 1}{2}\left(\sqrt[n]{n} - 1\right)^2 \leq 1
\end{equation*}
za $n \geq 2$.

Po izeku o sendvi"cu velja: $\limninf \left(\sqrt[n]{n} - 1\right) = 0$ \hfill $\square$

\textsc{Izrek:} Zaporedje $a_n = \left(1 + \frac{1}{n}\right)^n$ je konvergentno. Njegovo limito ozna"cimo z:
\begin{equation*}
e = \limninf \left(1 + \dfrac{1}{n}\right)^n
\end{equation*}
\textsc{Dokaz:} \dashuline{$a_n$ je nara"s"cajo"ce}
\begin{equation*}
a_n = \left(1 + \dfrac{1}{n}\right)^n = 1 + \binom{n}{1}\dfrac{1}{n} + \binom{n}{2} \dfrac{1}{n^2} + \ldots + \binom{n}{k}\dfrac{1}{n^k} + \ldots + \binom{n}{n} \dfrac{1}{n^n}
\end{equation*}
\begin{equation*}
\binom{n}{k}\dfrac{1}{n^k} = \dfrac{n!}{k!(n-k)!} \cdot \dfrac{1}{n^k} = \dfrac{1}{k!} \cdot \dfrac{n (n-1) (n-2) \ldots (n - k + 1)}{n n n n \ldots n}
\end{equation*}
Ker je v "stevu in imenvolacu tega ulomka enako "stevil, lahko naredimo re"cemo:
\begin{equation*}
\dfrac{1}{k!} \cdot \dfrac{n (n-1) (n-2) \ldots (n - k + 1)}{n n n n \ldots n} = \underbrace{\dfrac{1}{k!} (1 - \dfrac{1}{n}) (1 - \dfrac{2}{n}) \ldots (1 - \dfrac{k-1}{n})}_{k-1}
\end{equation*}
Torej velja:
\begin{align*}
a_n &= 1 + 1 + \dfrac{1}{2!}(1 - \dfrac{1}{n}) + \dfrac{1}{3!} (1 - \dfrac{1}{n}) (1 - \dfrac{2}{n}) + \ldots + \dfrac{1}{n!}(\ldots) \\
a_{n+1} &= 1 + 1 + \dfrac{1}{2!}(1 - \dfrac{1}{n}) + \dfrac{1}{3!} (1 - \dfrac{1}{n}) (1 - \dfrac{2}{n}) + \ldots + \dfrac{1}{n!}(\ldots) + \dfrac{1}{(n+1)!}(\ldots)
\end{align*}
Prva dva "clena sta enaka, nato pa se za"cnejo razlikovati v oklepajih:
\begin{equation*}
1-\dfrac{1}{n} \leq 1 - \dfrac{1}{n+1}
\end{equation*}
Prav tako ima zaporedje $a_{n+1}$ en "clen ve"c kota zaporedje $a_n$. Iz tega sledi: $a_n \leq a_{n+1}$

\dashuline{$a_n$ je navzgor omejeno}
\begin{multline*}
a_n = 1 + 1 + \dfrac{1}{2!} (1 - \dfrac{1}{n}) + \dfrac{1}{3!} (1 - \dfrac{1}{n}) (1 - \dfrac{2}{n}) + \ldots \leq \\
\leq 1 + 1 + \dfrac{1}{2!} + \dfrac{1}{3!} + \dfrac{1}{4!} + \ldots + \dfrac{1}{n!} \leq 1 + 1 + \dfrac{1}{2} + \dfrac{1}{2^2} + \dfrac{1}{2^3} + \ldots + \dfrac{1}{2^{n-1}}  = \\
= 1 + \dfrac{1 - \left(\frac{1}{2}\right)^n}{1 - \frac{1}{2}} \leq 1 + \dfrac{1}{\frac{1}{2}} = 3
\end{multline*}
Sledi, da je $a_n$ konvergentno.

\subsection{Potence z realnimi eksponenti}
\begin{align*}
a > 0, x \in \RR, a^x && r_n \stackrel{n \to \infty}{\longrightarrow} x, r_n \in \QQ
\end{align*}
\begin{itemize}
	\item "Ce $r_n$ konvergira, potem $a^{r_n}$ konvergira.
	\item "Ce $r_n \stackrel{n\to \infty}{\longrightarrow} x, s_n \stackrel{n\to \infty}{\longrightarrow} x, r_n, s_n \in \QQ$, potem velja: $\limninf a^{r_n} = \limninf a^{s_n}$
\end{itemize}
\begin{equation*}
a^x := \limninf a^{r_n} \text{ , kjer } r_n \stackrel{n\to \infty}{\longrightarrow}, r_n \in \QQ
\end{equation*}
\textsc{Trditev:} Naj bo $a \in \RR, a > 0$. Potem:
\begin{equation*}
\forall \varepsilon > 0 \exists \delta > 0 \forall h \in \QQ, |h| < \delta: |a^h - 1| < \varepsilon
\end{equation*}
\textsc{Dokaz:} $a > 0, \varepsilon > 0$,

Vemo: $\limninf \sqrt[n]{a} = 1$. "Ce $\sqrt[n]{a}$ zapi"semo kot $a^{\frac{1}{n}}$, potem po definiciji limite velja:

Za $\varepsilon > 0 \exists n_0 : |a^{\frac{1}{n}} -1 | < 0$ za vse $n \geq n_0$.

Naj bo $h \in \QQ, 0 < h \leq \frac{1}{n}$. Za $a > 1$ velja:
\begin{equation*}
0 < a^h - 1 < a^{\frac{1}{n}} - 1 < \varepsilon
\end{equation*}
in za $a < 1$ velja:
\begin{equation*}
0 < 1-a^h < 1 - a ^{\frac{1}{n}} < \varepsilon
\end{equation*}
Spomnimo se: "ce $a < 1$ in $n, m \in \QQ, n < m$ potem velja: $a^m < a^n$. V na"sem primeru to pomeni:
\begin{equation*}
h < \dfrac{1}{n}: a^{\frac{1}{n}} < a^h
\end{equation*}
Zato pri $1-a^{\frac{1}{n}}$ od 1 od"stejemo ve"c, kot pri $1-a^h$, torej neenakost velja.

Sklep zgornjih dveh dokazov je:
\begin{equation*}
\forall h \in \left(0, \frac{1}{n}\right): |a^h - 1| < \varepsilon
\end{equation*}
Podobno lahko doka"zemo za $h < 0$. \hfill $\square$

\textsc{Trditev:} Naj bo $a > 0$ in naj bo $r_n$ konvergentno zaporednje racionalnih "stevil z limito $r \in \RR$. Potem velja naslednje:
\begin{enumerate}[(1)]
	\item Zaporedje $a^{r_1}, a^{r_2}, a^{r_3}, \ldots$	konvergira.
	\item "Ce je $r \in \QQ$, potem je $\limninf a^{r_n} = a^r$.
\end{enumerate}

\textsc{Dokaz:}
\begin{enumerate}[(1)]
	\item za primer $a > 1$:
	
	\dashuline{$a^{r_n}$ je omejeno zaporedje} za primer $a > 1$.
	
	$r_n$ je omejeno zaporedje, ker je konvergentno: obstaja $M \in \QQ$, da velja:
	\begin{equation*}
	\forall n \in \NN: r_n \leq M
	\end{equation*}
	iz tega sledi:
	\begin{equation*}
	\forall n \in \NN: a^{r_n} \leq a^M
	\end{equation*}
	
	\dashuline{$a^{r_n}$ je Cauchyjevo zaporedje} (od tod sledi, da je konvergentno)
	
	Naj bo $\varepsilon > 0$
	\begin{multline*}
		|a^{r_n} - a^{r_m}| = \\
		 = |a^{r_m}(a^{r_n - r_m} - 1)| = |a^{r_m}| \cdot |a^{r_n - r_m} - 1| \leq\\
		  \leq a^M |a^{r_n - r_m} - 1| < \varepsilon
	\end{multline*}
	Velja, ker po prej"snji trditvi obstaja $\delta > 0$, tako da velja:
	\begin{equation*}
		|\underbrace{r_n - r_m}_{h}| < \delta \Rightarrow |a^{\overbrace{r_n - r_m}^h} - 1| < \dfrac{\varepsilon}{a^M}
	\end{equation*}
	Ker je $r_n$ konvergentno, je Cauchyjevo, zato obstaja $n_o \in \NN$, tako da velja
	\begin{equation*}
	 \forall n, m \in \NN, m, n \geq n_0: |r_n - r_m|  < \delta
	\end{equation*}
	Podobno naredimo za $a < 1$.
	
	\item "Ce je $r \in \QQ$:
	\begin{equation*}
	|a^r - a^{r_n}| = |a^r(1 - a^{r_n - r})| = |a^r| |1 - a^{r_n - r} < \varepsilon
	\end{equation*}
	za $\varepsilon > 0$. Torej mora veljati:
	\begin{equation*}
	|1-a^{r_n - r}| < \dfrac{\varepsilon}{a^r}
	\end{equation*}
	Po trditvi od prej:
	\begin{equation*}
	\exists \delta > 0 \forall h \in \QQ, |h| < \delta: |a^h - 1| < \dfrac{\varepsilon}{a^r}
	\end{equation*}
	Ker $r_n$ konvergira proti $r$, obstaja $n_0 \in \NN \forall n \in \NN, n \geq n_0: |r_n - r| < \delta$
	
	\hfill $\square$
\end{enumerate}
\textsc{Trditev:} Naj bo $a > 0$ in naj bosta $r_n$ in $s_n$ zaporedji racionalnih "stevil z isto limito
\begin{equation*}
\limninf r_n = \limninf s_n
\end{equation*}
Potem velja:
\begin{equation*}
\limninf a^{r_n} = \limninf a^{s_n}
\end{equation*}
\textsc{Dokaz:} ideja dokaza je:
\begin{equation*}
1 = a^0 = a^{\limninf r_n - \limninf s_n} = \dfrac{a^{\limninf r_n}}{a^{\limninf s_n}}
\end{equation*}
To "se ni dokaz, ker ne vemo kaj je $a^{\limninf r_n}$, saj se lahko zgodi, da je limita zaporedja $r_n$ iracionalno "stevilo, za kar "se nismo definirali potenciranja.

\textbf{Vemo}: $a^{r_n - s_n} = \dfrac{a^{r_n}}{a^{s_n}}$ za vse $n \in \NN$

$\limninf (r_n - s_n) = 0$, zato $1 = a^0 = \limninf a^{r_n - s_n}$. Ta neenakost velja zaradi prej"snje trditve. Ker je $\limninf (r_n - s_n)$ racionalno "stevilo, velja $\limninf a^{r_n - s_n} = a^{\limninf (r_n - s_n)} = a^0$
\begin{equation*}
1 = a^0 = \limninf a^{r_n - s_n} = \limninf \dfrac{a^{r_n}}{a^{s_n}} = \dfrac{\limninf a^{r_n}}{\limninf a^{s_n}}
\end{equation*}
\textbf{Sklep}:
\begin{align*}
1 &= \dfrac{\limninf a^{r_n}}{\limninf a^{s_n}} \\
\limninf a^{s_n} &= \limninf a^{r_n}
\end{align*}
To seveda velja, samo "ce $\limninf a^{s_n} \neq 0$.

Ker je $s_n$ konvergentno, je omejeno:
\begin{equation*}
\exists M \in \NN: |s_n| < M \Rightarrow a^{-M} \leq a^{s_n} \leq a^M
\end{equation*}
zato za $a > 1$ velja:
\begin{equation*}
a^{-M} \leq \limninf a^{s_n} \leq a^M
\end{equation*}
za $0 < a < 1$ velja podobno:
\begin{equation*}
a^{M} \leq \limninf a^{s_n} \leq a^{-M}
\end{equation*}
torej je $\limninf a^{s_n}$ pozitivno "stevilo. 

\hfill $\square$

\deff Naj bo $a > 0$ in $r \in \RR$. Denimo, da zaporedje $r_n$ racionalnih "stevil konvergira proti $r$. Potem definiramo:
\begin{equation*}
a^r = \limninf a^{r_n}
\end{equation*}

\textsc{Opombe:}
\begin{enumerate}[1)]
	\item Definicija je dobra, ko za razli"cni konvergentni zaporedji racionalnih "stevil z isto limito, zaporedji $a^{r_n}$ in $a^{s_n}$ konvergirati k istemu "stevilu.
	
	\item "Ce je $r \in \QQ$, se nova definicija ujema z definicijo za racionalne potence.
	\item Ra"cunska pravila za ra"cunanje z racionalnimi eksponenti se prenesejo na ra"unanje z realnimi eksponenti.
\end{enumerate}
%
\textsc{Trditev:} Naj bo $\alpha > 0$. Potem je $\limninf \dfrac{1}{n^\alpha} = 0$.

\textsc{Dokaz:} za doma"co nalogo. "Ce si priden si ga naredil, "ce si pridna se u"cis iz svojih zapiskov, ker znate "zenske brati za sabo (naredimo se fizike in zanemarimo, da zna Nik brati za sabo kljub temu, da ni "zenskega spola\footnote{lahko se sam odlo"cis ali zanemarimo njegov obstoj, ali pa samo njegov spol})).

\textsc{Trditev:} Naj bo $q \in \QQ, q > 1$ in $\alpha \in \RR$. Potem je
\begin{equation*}
\limninf \dfrac{n^\alpha}{q^n} = 0
\end{equation*}
\textsc{Dokaz:} 
\begin{align*}
a_n &= \dfrac{n^\alpha}{q^n} \\
a_{n+1} &= \dfrac{(n+1)^\alpha}{q^{n+1}} = \dfrac{(n+1)^\alpha}{q n^\alpha} \dfrac{n^\alpha}{q^n} = \left(\dfrac{n+1}{n}\right)^\alpha \cdot \dfrac{1}{q} a_n
\end{align*}
Vemo, da $\frac{1}{q} < 1$

Sledi:
\begin{equation*}
\exists n_0 \in \NN: \forall n \geq n_0: \left(\dfrac{n+1}{n}\right)^\alpha \dfrac{1}{q} < 1
\end{equation*}
Zato bo od nekod naprej $a_{n+1} < a_n$.

Ker je $\forall n \in \NN: 0 a_n > 0$, je $a_n$ konvergentno. Torej obstaja
\begin{equation*}
L = \limninf a_n
\end{equation*}
in velja:
\begin{equation*}
L = \limninf a_{n+1} = \limninf \left( \left(\dfrac{n+1}{n} \right)^\alpha \dfrac{1}{q} a_n \right) = \dfrac{1}{q} L
\end{equation*}
Torej lahko pora"cunamo $L$:
\begin{align*}
L &= \dfrac{1}{q} L \\
L - \dfrac{1}{q} L &= 0 \\
L(1-\dfrac{1}{q}) &= 0
\end{align*}
Ker $1- \dfrac{1}{q} \neq 1$, velja $L = 0$

\hfill $\square$

\subsection{Zgornja in spodnja limita}
\deff Naj bo $a_n$ omejeno zaporedje in $\mathcal{S}$ mno"zica njegovih stekali"s"c. Natan"cno zgornjo mejo mno"zice $\mathcal{S}$ imenujemo \emph{zgornja limita} ali \emph{limes superior} in pi"semo:
\begin{equation*}
\limsup_{\toinf{n}} a_n = \overline{\limninf}a_n = \sup \mathcal{S}
\end{equation*}
\emph{Spodnja limita} ali \emph{limes inferior} je natan"cna sopdnja meja od $\mathcal{S}$ in pi"semo
\begin{equation*}
\liminf_{\toinf{n}} a_n = \underline{\lim}_{\toinf{n}} a_n = \inf \mathcal{S}
\end{equation*}
\textsc{Opombi:} 
\begin{itemize}
	\item Ker je zaporedje $a_n$ omejeno, $\mathcal{S} \neq \varnothing$. 
	\item Mno"zica $\mathcal{S}$ je omejena, ker je zaporedje omejeno (z zgornjo in spodnjo mejo zaporedja).
\end{itemize}
%
\deff
\begin{enumerate}[(1)]
	\item "Ce je zaporedje $a_n$ navzgor neomjeno, potem definiramo:
	\begin{equation*}
		\limsupninf a_n = \infty
	\end{equation*}
	
	\item "Ce je zaporedje $a_n$ navzdol neomejeno, potem definiramo:
	\begin{equation*}
		\liminfninf a_n = - \infty
	\end{equation*}
	
	\item Denimo, da je $a_n$ navzgor neomejeno in navzdol omejeno. "Ce je mno"zica njegovih stekali"s"c $\mathcal{S}$ neprazna, potem definiramo:
	\begin{equation*}
	\liminfninf a_n = \inf \mathcal{S}
	\end{equation*}
	sicer:
	\begin{equation*}
	\liminfninf a_n = \infty
	\end{equation*}
	
	\item Denimo, da je $a_n$ navzdol neomejeno in navzgor omejeno. "Ce je mno"zica njegovih stekali"s"c $\mathcal{S}$ neprazna, potem definiramo:
	\begin{equation*}
	\limsupninf a_n = \sup \mathcal{S}
	\end{equation*}
	sicer:
	\begin{equation*}
	\limsupninf a_n = - \infty
	\end{equation*}
\end{enumerate}
\textsc{Primeri:}
\begin{enumerate}[1)]
	\item $a_n = (-1)^n$
	\begin{align*}
	\limsupninf a_n &= 1 \\
	\liminfninf a_n &= -1 \\
	\end{align*}
	
	\item $b_n$ konvergentno z limito $b$
	\begin{equation*}
	b = \limsupninf b_n = \liminfninf b_n
	\end{equation*}
	
	\item $c_n = n^{(-1)^n}: 1, 2, \dfrac{1}{3}, 4, \dfrac{1}{5}, \ldots$
	\begin{align*}
	\limsupninf c_n &= \infty \\
	\liminfninf c_n &= 0
	\end{align*}
	
	\item $d_n = 1, 2, 3, \ldots$
	\begin{align*}
	\limsupninf d_n &= \infty \\
	\liminfninf d_n &=\infty
	\end{align*}
\end{enumerate}
