\deff Zaporedje realnih "stevil je preslikava $\NN \rightarrow \RR$. Zapis je:
\begin{align*}
f: &\NN \rightarrow \RR\\
&f(n) \text{ozna"cimo z $a_n$}
\end{align*}
Preslikava $f$ je podana z $a_1, a_2, a_3, \ldots$.

Zaporedje realnih "stevil podamo \emph{s "cleni} $a_1, a_2, a_3, \ldots$, kar kraj"se zapi"semo $\left\{a_n\right\}_{n=1}^\infty$, $\left\{a_n\right\}$,  $\left(a_n\right)_{n=1}^\infty$ ali pa kar zaporedje $a_n$.

\textsc{Opomba}: zaporedje $\left\{a_n\right\}$ ni mno"zica $\left\{a_n; n \in \NN \right\}$.

\textsc{Primer:}
\begin{enumerate}[1)]
	\item $a_n = 1, n \in \NN$ \emph{konstantno zaporedje}
	
	To zaporedje ustreza preslikavi $f(n) = 1, n \in \RR$
	
	\item $b_n = n, n \in \NN$ Takemu zaporedju pravimo, da je podan s splo"snim "clenom. Nari"semo lahko njegov graf $\left\{(n, b_n); n \in \NN \right\}$
	
	\item $d_1=1, d_2=1, d_{n+2} = d_n + d_{n+1}, n \in \NN$ je \emph{rekurzivno podano}. Predstavlja Fibonaccijevo zaporedje.
	
	\item \emph{aritmeti"cno aporedje}
	\begin{gather*}
	a_n = a_1 + (n-1)d, n \in \NN\\
	a_1, a_1 + d, a_1 + 2d, \ldots
	\end{gather*}
	To lahko zapi"semo tudi z rekurzivno zvezo:
	\begin{equation*}
	\begin{cases}
	a_{n+1} = a_n + d, n \in \NN\\
	a_1 = a_1
	\end{cases}
	\end{equation*}
	
	\item \emph{geometrijsko zaporedje}
	\begin{gather*}
		a_n = a_1 q{n-1}, n \in \NN\\
		a_1, a_1 q, a_1 q^2, \ldots
	\end{gather*}
	Zapisano z rekurzivno zvezo:
	\begin{equation*}
	\begin{cases}
	a_{n+1} = a_n q, n \in \NN \\
	a_1 = a_1
	\end{cases}
	\end{equation*}
\end{enumerate}

\deff
\begin{itemize}
	\item Zaporedje $a_n$ je \emph{navzgor omejeno} "ce je zaloga vrednosti preslikave $n \mapsto a_n$ navzgor omejena, t.j.:
	\begin{equation*}
	\exists M \in \RR \forall n \in \NN: a_n \leq M
	\end{equation*}
	
	\item \emph{Natan"cna zgornja meja} zaporedja $a_n$ je natanc"na zgornja meja zaloge vrednosti preslikave $n \mapsto a_n$ in jo ozna"cimo s $\sup a_n$.
	
	\item "Stevilo $M$ imenujemo \emph{zgornja meja} zaporedja $a_n$.
	
	\item Analogno definiramo navzdol omejeno, natan"cno spodnjo mejo $\inf a_n$, $\max$ in $\min$.
\end{itemize}

\textsc{Primer:}

$a_n = \dfrac{1}{n}, n \in \NN$
	
navzgor omejeno z 1: $\forall n \in \NN: \dfrac{1}{n} \leq 1$
	
navzdol omejeno z 0: $\forall n \in \NN: \dfrac{1}{n} \geq 0$
	
$\sup \dfrac{1}{n} = 1$, ker je 1 zgornja meja in $a_1 = 1$.
	
\dashuline{$\inf 1/n = 0$}
0 je spodnja meja.

Izberemo $\varepsilon > 0$. Dokazujemo da $\varepsilon$ ni spodnja meja. Po arhimedski lastnosti:
\begin{equation*}
\exists n \in \NN: \dfrac{1}{n} < \varepsilon
\end{equation*}
	
$\max \dfrac{1}{n} = 1$
	
$\min \dfrac{1}{n}$ ne obstaja.

\deff Zaporedje $a_n$ \emph{konvergira} proti $a \in \RR$, "ce:
\begin{equation*}
\forall \varepsilon > 0 \exists n_0 \in \NN \forall n \in \NN: n \geq n_0 \Rightarrow |a_n - a| < \varepsilon
\end{equation*}
"Stevilo $a$ imenujemo \emph{limita zaporedja} in ozna"cimo z:
\begin{equation*}
a = \lim_{\toinf{n}} a_n
\end{equation*}

"Ce zaporedje $a_n$ konvergira, je $a_n$ \emph{konvergentno zaporedje}. Sicer je \emph{divergentno zaporedje}.
\begin{equation*}
|a_n - a| < \varepsilon \iff a_n \in (a - \varepsilon, a + \varepsilon)
\end{equation*}
Zunaj $\varepsilon$-te okolice je kve"cjemo kon"cnomnogo "celnov.

Zapis $\lim_{\toinf{n}} a_n = a$ pomeni, da zaporedje konvergira in njegova limita je $a$. To ne velja, "ce zaporedje divergira, ali pa njegova konvergira in njegova limita ni $a$.

\textsc{Primeri:}
\begin{enumerate}[1)]
	\item $a_n = 1, n \in \NN$
	
	\dashuline{$\lim_{\toinf{n}} a_n = 1$}
	
	Izberemo poljuben $\varepsilon > 0: |a_n - 1| = |1-1| = 0 < \varepsilon$ za vse $n \in \NN$.
	
	\item $b_n = \dfrac{1}{n}, n \in \NN$
	\begin{equation*}
	\lim_{\toinf{n}} \dfrac{1}{n} = 0
	\end{equation*}
	Po arhimedski lastnosti:
	\begin{align*}
	\forall \varepsilon > 0: \exists m : \dfrac{1}{m} &< \varepsilon\\
	n \geq m: \dfrac{1}{n} &\leq \dfrac{1}{m}\\
	- \varepsilon < \dfrac{1}{n} &\leq \dfrac{1}{m} < \varepsilon
	\end{align*}
	\begin{equation*}
	\forall n \in \NN, n \geq m: \dfrac{1}{n} \in (-\varepsilon, \varepsilon)
	\end{equation*}
	
	\item $c_n = (-1)^n \dfrac{1}{n}$
	\begin{equation*}
	\lim_{\toinf{n}} c_n = 0
	\end{equation*}
	ker:
	\begin{equation*}
	\left|(-1)^n \dfrac{1}{n} - 0\right| = \dfrac{1}{n} < \varepsilon
	\end{equation*}
	po prej"snjem primeru
	
	\item $d_n = (-1)^n$
	
	\dashuline{zaporedje divergira}
	
	Denimo, da je $x$ limita tega zaporedja:
	\begin{itemize}
		\item "ce $x = -1: \varepsilon = 1$, zunaj $(-1 - \varepsilon, -1 + \varepsilon) = (-2, 0)$ le"zijo vsi sodi "cleni zaporedja, ki jih je neskon"cno, zato -1 ni limita.
		
		\item Analogno za $x = 1$.
		
		\item $x \neq 1 \land x \neq -1$
		\begin{equation*}
		d = \min \{|x-1|, |x+1|\}
		\end{equation*}
		velja:
		\begin{align*}
		1 &\notin (x - \dfrac{d}{2}, x + \dfrac{d}{2}) \\
		-1 &\notin (x - \dfrac{d}{2}, x + \dfrac{d}{2})
		\end{align*}
		Vsi "cleni zaporedja le"zijo izven tega intervala, zato $x$ ni limita. Sledi: $d_n$ divergira.
	\end{itemize}
\end{enumerate}

\textsc{Trditev:} Konvergentno zaporedje ima eno samo limito.

\textsc{Dokaz:} Denimo da sta $a$ in $b$ limiti zaporedja $a_n$.
\begin{multicols}{2}
Izberemo poljuben $\varepsilon > 0:$
\columnbreak
\begin{gather*}
	\exists n_a \forall n: n > n_a \Rightarrow |a_n - a| < \varepsilon \\
	\exists n_b \forall n: n > n_b \Rightarrow |a_b - a| < \varepsilon
\end{gather*}
\end{multicols}
\begin{equation*}
|a - b| = |(a-a_n) + (a_n-b)| \leq |a-a_n| + |a_n-b| < 2\varepsilon
\end{equation*}
\begin{equation*}
\forall \varepsilon > 0: |a-b| < 2\varepsilon \Rightarrow |a-b| = 0 \Rightarrow a = b
\end{equation*}

\textsc{Trditev:} Konvergentno zaporedje je omejeno

\textsc{Dokaz:} Denimo, da je $a_n$ konvergentno zaporedje z limito $a$. Izberemo $\varepsilon = 1$ in po definiciji velja:
\begin{equation*}
\exists n_0 \in \NN \forall n \geq n_0: |a_n - a| < \varepsilon = 1
\end{equation*}
Lahko skonstruiramo mno"zico "clenov, ki so izven $\varepsilon$-te okolice $a$. Tej mno"zici dodamo tudi zgornjo mejo okolice $a + \varepsilon = a + 1$. Ker je izven okolice kon"cnomnogo "clenov, ima ta mno"zica maksimum:
\begin{equation*}
\max \left\{a+1, a_1, a_2, \ldots, a_{n_0-1} \right\} = M
\end{equation*}
Za vsak $n$ velja: $a_n \leq M$, ker "ce $n \leq n_0 - 1$, je $a_n$ v zgornji mno"zici, ki smo ji doli"cili maksimum, "ce $n \geq n_0$ je $a_n$ v okolici $a$, to pomeni $a_n < a + 1$, $a+1$ je v zgornji mno"zici, ki smo ji dolo"cili maksimum.

Analogno lahko naredimo za spodnjo mejo. \hfill $\square$

Ni vsako omejeno zaporedje konvergentno. Primer: $a_n = (-1)^n, n \in \NN$.

\deff Naj bo $a_n$ zaporedje. "Stevilo $s$ je \emph{stekali"c"ce zaporedja} $a_n$, "ce v vsaki okolici $s$ le"zi neskon"cno "clenov zaporedja.

\textsc{Primer:} $a_n = (-1)^n, n \in \NN$

Vemo, da ni konvergentno. -1 in 1 sta stekali"s"ci $a_n$, ker vvsi "cleni z lihimi indeksi le"zijo na $(-1 - \varepsilon, -1 + \varepsilon)$ za $\varepsilon > 0$. Analgono za 1 in sode "clene.

\textsc{Opombi:}
\begin{enumerate}[1)]
	\item $s$ je stekali"s"ce $\iff \forall \varepsilon \in \RR, \varepsilon > 0$ je $|a_n - s| < \varepsilon$ izpolnjen za neskon"cno mnogo indeksov $n$.
	
	\item "Ce je zaporedje $a_n$ konvergentno z limito $a$, potem je $a$ edino stekali"s"ce zaporedja $a_n$, ker na $(a-b, a+b)$ le"zijo vsi, razen kon"cno mnogo "clenov.
\end{enumerate}

\textsc{Primeri:}
\begin{enumerate}[1)]
	\item Zaporedje s 3 stekali"s"ci: $a_n = n \mod3$.
	\item Zaporedje z 2 stekali"s"cama, ki ima same razli"cne "clene: $a_n = (-1)^n \left(1+ \dfrac{1}{n}\right)$
	\item Zaporejde z neskon"cno stekali"s"ci:
	\begin{equation*}
	1, 1, 2, 1, 2, 3, 1, 2, 3, 4, 1, 2, 3, 4, 5, \ldots
	\end{equation*}
	Stekali"s"ca zaporedja so vsa naravna "stevila.
	\item Ali ima zaporedje ne"stevno stekali"s"c? Da.
	
	Obstaja bijekcija $\NN \rightarrow \QQ$.
	\begin{equation*}
	\QQ = \{a_1, a_2, a_3, \ldots\}
	\end{equation*}
	Stekali"s"ca zaporedja $a_n: x \in \RR,  \varepsilon > 0$ na $(x-\varepsilon, x+\varepsilon)$ le"zi neskon"cno nogo $\QQ$ "stevil, torej neskon"cno mnogo "clenov zaporedja. Zato je $x$ stekali"s"ce.
	
	\item Divergentno zaporedje brez stekali"s"c: $a_n = n$.
\end{enumerate}

\textsc{Trditev: } "Ce vsaka okolica "stevila $s \in \RR$ vsebuje "clen zaporedja $a_n, a_n \neq s$, potej je $s$ stekali"s"ce $a_n$.

\textsc{Dokaz: } Izberemo poljuben $\varepsilon > 0$. Obstaja $n_1 \in \NN: a_{n_1} \in (s-\varepsilon, s+\varepsilon), a_{n_1} \neq s$. Definiramo razdaljo: $d_1 = |s - a_{n_1}|$. Obstaja $n_2: a_{n_2} \in (s-d_1, s+d_1), a_{n_2} \neq s, n_1 \neq n_2$. Postopek nadaljujemo $\square$.

\textsc{Izrek:} Vsako omejeno zaporedje ima stekali"s"ce.

\textsc{Dokaz:} Ker je zaopredje $a_n$ omejeno, ima spodnjo mejo $m$ in zgornjo mejo $M$.
\begin{gather*}
\mathcal{U} = \{u \in \RR; a_n <u \text{ je izpolnjeno za kon"cno mnogo "clenov zaporedja}\}\\
m \in \mathcal{U}, M + 1 \notin \mathcal{U}
\end{gather*}
$\Rightarrow \mathcal{U}$ je navzgor omejena in neprazna, torej obstaja $\sup \mathcal{U} = s$.

Izberemo $\varepsilon > 0$.:
\begin{itemize}
	\item $s + \varepsilon \notin \mathcal{U}: a_n < s + \varepsilon$ je izpolnjena za neskon"cno mnogo "clenov.
	\item $s - \varepsilon: \exists u \in (s-\varepsilon, s] \cap \mathcal{U}$.	Ker $u \in \mathcal{U}, s - \varepsilon < u$, je $s - \varepsilon \in \mathcal{U}$, zato velja da $a_n < s-\varepsilon$ velja za kon"cno mnogo "clenov zaporedja.
\end{itemize}
Sledi: na $[s-\varepsilon, s + \varepsilon)$ le"zi neskon"cno mnogo "clenov zaporedja. $s$ je stekali"s"ce. 

\hfill $\square$

\subsection{Monotona zaporedja}
\deff
\begin{itemize}
	\item Zaporedje $a_n$ je \emph{nara"s"cajao"ce}, "ce velja: $\forall n \in \NN: a_{n+1} \geq a_n$.
	\item Zaporedje $a_n$ je \emph{padajo"ce}, "ce velja: $\forall n \in \NN: a_{n+1} \leq a_n$.
	\item Zaporedje $a_n$ je \emph{strogo nara"s"cajao"ce}, "ce velja: $\forall n \in \NN: a_{n+1} > a_n$.
	\item Zaporedje $a_n$ je \emph{strogo padajo"ce}, "ce velja: $\forall n \in \NN: a_{n+1} < a_n$.
	\item Zaporedje $a_n$ je \emph{monotono}, "ce je zaporedje bodisi nara"s"cajo"ce, ali padajo"ce.
	\item Zaporedje $a_n$ je \emph{strogo monotono}, "ce je zaporedje bodisi strogo nara"s"cajo"ce, ali strogo padajo"ce.
\end{itemize}

\textsc{Primer:}
\begin{enumerate}[1)]
	\item $a_n = -n$ (strogo) padajo"ce
	\item $a_n = 1$ nara"s"cajo"ce ali padajo"ce
	\item $a_n = \dfrac{1}{n}$ je padajo"ce in navzdol omejeno
	\item $a_n = (-1)^n$ ni ne padajo"ce ne nara"s"cajo"ce
\end{enumerate}

\textsc{Trditev:} Monotono zaporedje je konvergentno natanko tedaj, kadar je omejeno. "Ce je zaporedje $a_n$ nara"s"cajo"ce in navzgor omejeno potem:
\begin{equation*}
\lim_{\toinf{n}}a_n = \sup a_n
\end{equation*}
"Ce je zaporedje $a_n$ padajo"ce in navzdol omejeno potem:
\begin{equation*}
\lim_{\toinf{n}}a_n = \inf a_n
\end{equation*}
Dokaz za v eno stran "ze vemo.

\textsc{Dokaz: } (v drugo stran ekvivalence)

Denimo, da je zaporedje $a_n$ nara"s"cajo"ce in navzgor omejeno. Ker je $a_n$ navzgor omejeno: $\exists a := \sup a_n$.

Dokazujemo \dashuline{$a_n$ konjugira proti $a$}

Izberemo poljubene $\varepsilon > 0$. Ker $a - \varepsilon$ ni zgornja meja zaporedja: $\exists n_0 \in \NN: a_{n_0} > a - \varepsilon$. Ker je zaporedje $a_n$ nara"s"cajo"ce in navzgor omejeno z $a$ velja:
\begin{gather*}
n \geq n_0: a-\varepsilon < a_{n_0} \leq a_n \leq a\\
\forall n \geq n_0 a_n \in (a - \varepsilon, a + \varepsilon)
\end{gather*}
Sledi: $a = \lim_{\toinf{n}}a_n$ \hfill $\square$

Analogno za navzdol omejeno padajo"ce zaporedje.