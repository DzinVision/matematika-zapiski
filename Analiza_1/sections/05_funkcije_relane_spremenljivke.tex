\deff Naj bo $D \subseteq \RR$. \emph{(Realna) funkcija (realne spremenljivke)} je preslikava
\begin{equation*}
f: D \to \RR
\end{equation*}
Spomnimo se, da je preslikava predpis, za katerega velja
\begin{equation*}
\forall x \in D \exists! y \in \RR: y = f(x)
\end{equation*}
\textsc{Opombi:}
\begin{enumerate}
	\item V pojmu funkcija sta zdru"zena dve predpostavki:
	\begin{itemize}
		\item definicijsko obmo"cje
		\item predpis
	\end{itemize}

	\item Funkciji $f$ in $g$ sta enki, "ce imata enako definicijsko obmo"cje in predpis.
\end{enumerate}
\textsc{Dogovor:} Funkcijo lahko podamo samo s spredpisom. V tem primeru vzamemo za definicjsko obmo"cje mno"zico vseh $x$, za katero ima predpis smisel.

\textsc{Primer:}
\begin{enumerate}[1)]
	\item $f(x) = \sqrt{x} \quad D_f = [0, \infty)$
	\item Ali sta funkciji, ki sta dani s predpisoma
	\begin{align*}
	f(x) &= \ln (x-1) + \ln (x+1) \\
	g(x) &= \ln (x^2 - 1)
	\end{align*}
	enaki?
	\begin{align*}
	D_f &= (1, \infty) \\
	D_g &= (-\infty, -1) \cup (1, \infty)
	\end{align*}
	Nista enaki, vedar data enake vrednosti na preseku definicijskih obmo"cji.
	\begin{equation*}
	\forall x \in D_f \cap D_g: f(x) = g(x)
	\end{equation*}
\end{enumerate}
\deff Naj bosta $f$ in $g$ funkciji. "Ce velja $D_f \subset D_g$ in "ce velja $f(x) = g(x)$ za vse $x \in D_f$, potem re"cemo, da je funkcija $g$ \emph{raz"siritev} funkcije $f$ in da je $f$ \emph{zo"zitev} funkcije $g$. Zo"zitev zapi"semo kot
\begin{equation*}
g|_{D_f} = f
\end{equation*}
\deff Naj bo $f$ funkcija. Mno"zico
\begin{equation*}
\Gamma_f = \{(x, f(x)): x \in D_f\} \subset \RR^2
\end{equation*}
imenujemo \emph{graf funkcije}.

\textsc{Opomba:}
\begin{enumerate}
	\item Funkcija $f$ je s svojim grafom natan"cno dolo"cena.
	\item Katere podmno"zice $S \subseteq \RR^2$ so grafi funkcij? Presek vsake navpi"cne premice z mno"zico $S$ je to"cka ali prazna mno"zica.
\end{enumerate}
%
\subsection{Operacije s funkcijami}
\deff Naj bosta $f: D_f \to \RR$ in $g: D_g \to \RR$ funckiji. "Ce je $Z_f \subset D_g$, potem lahko definiramo \emph{kompozitum funkcij}
\begin{equation*}
g \circ f: D_f \to \RR
\end{equation*}
s predpisom:
\begin{equation*}
(g \circ f)(x) = g(f(x))
\end{equation*}
Ni nujno, da bi bila $g\circ f$ in $f \circ g$ v kak"sni zvezi.

\textsc{Primer:}
\begin{enumerate}[1)]
	\item $f(x) = x^2 + 1$, $g(x) = \ln x^2$. $D_f = \RR$, $D_g =\RR \setminus \{0\}$
	\begin{align*}
	(f \circ g)(x) &= f(g(x)) = f(\ln x^2) = \ln^2 x^2 + 1 = 4\ln^2|x| + 1 \\
	(g \circ f)(x) &= g(f(x)) = g(x^2 + 1) = \ln(x^2 + 1)^2 = 2 \ln(x^2 + 1)
	\end{align*}
	
	\item $f(x) = x^2$ ni injektivna
	\begin{equation*}
	f|_{[0, \infty)}: [0, \infty) \to \RR
	\end{equation*}
	je injektivna. Naredimo lahko inverzno funkcijo $f^{-1} : Z_f \to \RR$ s predpisom $f^{-1}(x) = \sqrt{x}$
\end{enumerate}
\deff Naj bo $f: D \to \RR$ injektivna funkcija z zalogo vrednosti $Z_f$. Inverzno preslikavo $f^{-1}: Z_f \to \RR$ imenujemo \emph{inverzna funkcija}. Njen predpis je znan "za iz poglavja o preslikavah
\begin{equation*}
f^{-1}(b) = a \iff f(a) = b
\end{equation*}
\textsc{Primer:} Funkcija $f$ je dana s predpisom:
\begin{equation*}
f(x) = \dfrac{2x + 3}{3x - 1}
\end{equation*}
Dolo"ci inverzno funkcijo, "ce obstaja.
\begin{gather*}
D_f = \RR \setminus \left\{\dfrac{1}{3}\right\}\\
\begin{aligned}
y &= \dfrac{2x + 3}{3x-1} \\
y(3x - 1) &= 2x + 3 \\
3xy - 2x &= 3 + y \\
x(3y - 2) &= 3 + y \\
x &= \dfrac{3 + y}{3y- 2}
\end{aligned}
\end{gather*}
Iz ra"cuna sledi: "ce je $y \neq \dfrac{2}{3}$, potem obstaja natanko en $x$ za katerega velja $f(x) = y$. Torej je $Z_f = \RR \setminus \left\{\dfrac{2}{3}\right\}$, $f$ je injektivna in $f^{-1}(y) = \dfrac{y+3}{3y - 2}$.
%
\begin{align*}
\Gamma_f &= \{(x, f(x)): x \in D_f\} = \{(f^{-1}(y), y): y \in Z_f\} \\
\Gamma_{f^{-1}} &= \{(y, f^{-1}(y): y \in Z_f)\}
\end{align*}
Zato je $\Gamma_{f^{-1}}$ dobljen iz $\Gamma_f$ z zrcaljenjem preko simetrale lihih kvadrantov.

\deff Naj bo $f$ funkcija. 
\begin{itemize}
	\item Pravimo, da je $f$ navzogr omejena, "ce je $Z_f$ navzgor omejena, t.j.:
	\begin{equation*}
	\exists M \in \RR \forall x \in D_f : f(x) \leq M
	\end{equation*}
	
	\item Pravimo, da je $f$ navzdol omejena, "ce je $Z_f$ navzdol omejena, t.j.:
	\begin{equation*}
	\exists m \in \RR \forall x \in D_f : f(x) \geq m
	\end{equation*}
	
	\item Funkcija $f$ je omejena, kadar je navzgor in navzdol omejena.
	
	\item "Ce je $f$ navzgor omejena, potem je supremum funkcije $f$ natan"cna zgornja meja zaloge vrednosti
	\begin{equation*}
	\sup f := \sup Z_f
	\end{equation*}
	
	\item "Ce je $f$ navzdol omejena, potem je infimum funkcije $f$ natan"cna spodnja meja zaloge vrednosti
	\begin{equation*}
	\inf f := \inf Z_f
	\end{equation*}
	
	\item "Ce obstaja maksimum zaloge vrednosti, velja
	\begin{equation*}
	\max f := \max Z_f
	\end{equation*}
	
	\item "Ce obstaja minimum zaloge vrednosti, velja
	\begin{equation*}
	\min f := \min Z_f
	\end{equation*}
\end{itemize}
\deff Naj bo $f$ funkcija. Pravimo, da je $x \in D_f$ \emph{ni"cla} funkcije $f$, "ce
\begin{equation*}
f(x) = 0
\end{equation*}
\deff Naj bosta $f, g : D \to \RR$ funkciji. Funkcije $f+g, f-d, f\cdot g: D \to \RR$ definiramo s predpisi:
\begin{align*}
(f+g)(x) &= f(x) + g(x) \\
(f-g)(x) &= f(x) - g(x) \\
(f\cdot g)(x) &= f(x) \cdot g(x)
\end{align*}
"Ce $\forall x \in D g(x) \neq 0$, potem definiramo $\dfrac{f}{g}: D \to \RR$ s predpisom
\begin{equation*}
\left(\dfrac{f}{g}\right)(x) = \dfrac{f(x)}{g(x)}
\end{equation*}
\deff Naj bosta $f, g : D \to \RR$ funkciji. Funkciji $\max \{f, g\}, \min \{f, g\}: D \to \RR$ definiramo s predpisoma:
\begin{align*}
\max \{f, g\}(x) &:= \max\{f(x), g(x)\}\\
\min \{f, g\}(x) &:= \min\{f(x), g(x)\}
\end{align*}
\deff Naj bo $\Gamma$ mno"zica in naj bo za vsak $\gamma \in \Gamma$
\begin{equation*}
f_\gamma: D \to \RR
\end{equation*}
funkcija. "Ce je $\{f_\gamma(x): \gamma \in \Gamma\}$ navzgor omejena za vsak $x \in D$, potem lahko definiramo funkcijo
\begin{equation*}
\sup_{\gamma \in \Gamma}f_\gamma: D \to \RR
\end{equation*}
s predpisom
\begin{equation*}
(\sup_{\gamma \in \Gamma} f_\gamma)(x) = \sup \{f_\gamma(x): \gamma \in \Gamma\}
\end{equation*}
Podobno definiramo 
\begin{equation*}
\inf_{\gamma \in \Gamma}f_\gamma: D \to \RR
\end{equation*}
\textsc{Primer:} $f_\gamma(x) = \dfrac{1}{1 + \gamma x^2}, \gamma \in (0, \infty)$, $D_{f_\gamma} = \RR$
\begin{equation*}
\left\{\dfrac{1}{1 + \gamma x^2}: \gamma \in (0, \infty)\right\} \text{ je navzgor omejena z 1 in navzdol omejena z 0}
\end{equation*}
%
\subsection{Zveznost}
Zvezna funkcija je funkcija, kjer ,,majhna'' spremembna neodvisne spremenljivke povzro"ci ,,majhno'' spremembo odvisne spremenljivke.

\textsc{Primera}
\begin{itemize}
	\item $g(x) = \sin \dfrac{1}{x}$: V 0 ni definirana, zato ne moremo govoriti v zveznosti. Lahko dolo"cimo $g(0) = 0$, vendar ne glede na vrednost, ki jo dolo"cimo v 0, $g$ ne bo zvezna.
	
	\item $h(x) = x \sin \dfrac{1}{x}$. "Ce dolo"cimo $h(0) = 0$, potem je $h$ zvezna.
\end{itemize}
\deff Naj bo $f: D \to \RR$ funkcija in $a \in D$ to"cka. Funkcija $f$ je \emph{zvezna v to"cki $a \in D$}, "ce velja
\begin{equation*}
\forall \varepsilon > 0 \exists \delta > 0 \forall x \in D: |x - a| < \delta \Rightarrow |f(x)-f(a)| < \varepsilon
\end{equation*}
\textbf{Opomba:} $\delta$ je odvisna od $\varepsilon$. Najprej si izberemo $\varepsilon$, nato pa dolo"cimo $\delta$.

\textsc{Primeri:}
\begin{enumerate}[1)]
	\item $C \in \RR, \quad f(x) = C$ \dashuline{$f$ je zvezna v $a \in \RR$}
	
	Izberemo poljuben $\varepsilon > 0$. Vemo $f(a) = C$. "Ce si izberemo $\delta = 1$, velja:
	\begin{equation*}
	|x - a| < 1 \Rightarrow |f(x) - f(a)| = 0 < \varepsilon
	\end{equation*}
	
	\item $g(x) = x$ \dashuline{$g$ je zvezna v $a \in \RR$}
	
	Izberemo $\varepsilon > 0$. Dokazujemo
	\begin{equation*}
		|x - a| < \delta \Rightarrow |f(x) - f(a)| < \varepsilon
	\end{equation*}
	Naj bo $\delta := \varepsilon$. Velja:
	\begin{equation*}
	|f(x)-f(a)| = |x-a| < \varepsilon
	\end{equation*}
	ker $|x-a| < \delta$ in $\varepsilon = \delta$
	
	\item $f(x) = \begin{cases} 0 & x \leq 0 \\ 1 & x > 1\end{cases}$ \dashuline{$f$ ni zvezna v 0}
	
	Negacija definicje je:
	\begin{equation*}
	\exists \varepsilon >0 \forall \delta>0 \exists x \in D: |x-a| < \delta \land |f(x) - f(a)| \geq \varepsilon
	\end{equation*}
	Naj bo $\varepsilon = \dfrac{1}{2}$. Izberemo poljuben $\delta > 0$. Naj bo $x_\delta = \dfrac{\delta}{2}$. $x_\delta$ je v okolici to"cke 0, ker
	\begin{equation*}
	|0 - x_\delta| = |0 - \dfrac{\delta}{2}| = \dfrac{\delta}{2} < \delta
	\end{equation*}
	Funkcijske slike niso v $\varepsilon$-ti okolici to"cke $f(0)$, ker:
	\begin{equation*}
	|f(0) - f(x_\delta)| = |0 - 1| = 1 \geq \varepsilon = \dfrac{1}{2}
	\end{equation*}
\end{enumerate}
\deff Naj bo $\delta \in \RR,\ \delta > 0,\ D \subset \RR,\ a \in D$. Mno"zico $(a - \delta, a + \delta) \cap D$ imenujemo \emph{$\delta$ okolice to"cek $a$ v $D$}. Mno"zica $U \subset \RR$ je \emph{okolica to"cke $a$ v $D$}, "ce vsebuje kak"sno $\delta$-okolico to"cke $a$ v $D$.

\textsc{Ekvivalentni definicija zveznosti}
\begin{itemize}
	\item Naj bo $f: D \to \RR$ funkcija in $a \in D$. Potem je $f$ zvezna v to"cki $a$ natanko tedaj, kadar velja, da za vsak $\varepsilon > 0$ obstaja $\delta > 0$, da $f$ preslika $\delta$-okolico to"cke $a$ v $\varepsilon$-to okolico to"cke $f(a)$.
	
	\item Funkcija $f$ je zvezna v to"cki $a$ natanko tedaj, "ce za vsako okolico $V$ to"cek $f(a)$, obstaja taka okolica $U$ to"cke $a$, da velja $f(U) \subset V$.
\end{itemize}
%
\subsubsection{Opis zveznosti z zaporedji}
Naj bo $f: D \to R$ zvezna v $a \in D$. Izberemo poljubno konvergentno zaopredje $x_n \in D$ z limito $a$. Opazujemo $f(x_n)$. Trdimo \dashuline{$f(x_n)$ konvergira proti $f(a)$}.

Izberemo $\varepsilon > 0$. Ker je $f$ zvezna v $a$, obstaja $\delta > 0$, da za $x \in D$ velja, "ce $|x -a| < \delta$ sledi $|f(x) - f(a)| < \varepsilon$. Ker $x_n$ konvergira proti $a$, obstaja $n_0 \in \NN$, da velja $|x_n - a| < \delta$ za vse $n \geq n_0$. Potem velja $|f(x_n) - f(a)| < \varepsilon$ za vse $n \geq n_0$. \hfill $\square$

\textsc{Izrek:} Naj bo $f: D \to \RR$ funkcija in $a \in D$. $f$ je zvezna v $a \in D$ natanko tedaj, kadar za \textbf{vsako} zaporedje $x_n \in D$, ki konvergira proti $a$, zaporedje $f(x_n)$ konvergira proti $f(a)$.

\textsc{Dokaz:}
\begin{itemize}
	\item[($\Rightarrow$)] "Ze dokazano.
	\item[($\Leftarrow$)] Denimo, da $f$ ni zvezna v to"cki $a$. Potem velja:
	\begin{equation*}
	\exists \varepsilon > 0 \forall \delta > 0 \exists x \in D: |x-a| < \delta \land |f(x) - f(a)| \geq \varepsilon
	\end{equation*}
	Torej za vsak $n$ velja:
	\begin{equation*}
	x_n \in D: |x_n - a| < \dfrac{1}{n} \land |f(x_n) - f(a)| \geq \varepsilon
	\end{equation*}
	kjer je $\dfrac{1}{n} = \delta$.
	
	Zaporedje $x_n$ konvergira proti $a$, ker $|x_n - a| < \dfrac{1}{n}$ za vsak $n$. Zaporedje $f(x_n)$ ne konvergira proti $f(a)$, ker $|f(x_n) - f(a)| \geq \varepsilon$ za vsak $n$. 
	
	\hfill $\square$
\end{itemize}
\textsc{Primeri:}
\begin{enumerate}[1)]
	\item $f(x) = \begin{cases}0 & x \in \RR \setminus \QQ \\
	1 & x \in \QQ
	\end{cases}$ \quad \emph{Dirichletova funkcija}
	
	Ni zvezna v nobeni to"cki. Naj bo $a \in \RR$. Velja
	\begin{align*}
	\exists &q_n \in \QQ, q_n \to a \\
	\exists &a_n \in \RR \setminus \QQ, a_n \to a
	\end{align*}
	Torej je $f(q_n) = 1$ za vsak $n$ iin $\limninf f(q_n) = 1$. Vemo tudi, da $f(a_n) = 0$ za vsak $n$ in $\limninf f(a_n) = 0$. Torej $f$ ni zvezna v $a$.
	
	\item \begin{align*}
	g&: (0, 1) \to \RR \\
	g(x) &= \begin{cases}
	\dfrac{1}{k} & x = \dfrac{m}{k} \text{ okraj"san ulomek}\\
	0 & x \in \RR \setminus \QQ
	\end{cases}
	\end{align*}
	$g$ ni zvezna v $\QQ$ to"ckah.
\end{enumerate}
%
\textsc{Izrek:} Naj bosta $f, g: D \to \RR$ zvezni funkciji v to"cki $a \in D$. Potem so funkcije $f + g$, $f-g$, $f\cdot g$ zvezne v to"cki $a$. "Ce je $g(a) \neq 0$, potem je $\dfrac{f}{g}$ zvezna v to"cki $a$.

\textsc{Dokaz} Vemo, da je $f$ zvezna v $a \in D$ $\iff$ za vsako zaporedje $x_n$ v $D$, ki konvergira proti $a$, zaporedje $f(x_n)$ konvergira proti $f(a)$. Izberemo poljubno zaporedje $x_n \in D$, ki konvergira proti $a$. Dokazujemo $(f+g)(x_n)$ kovergira proti $(f+g)(a)$.

$(f+g)(x_n) = f(x_n) + g(x_n)$ ker sta $f$ in $g$ zvezni, $f(x_n)$ konvergira proti $f(a)$ in $g(x_n)$ konvergira proti $g(a)$. Torej $f(x_n) + g(x_n)$ konvergira proti $f(a) + g(a) = (f+g)(a)$.

"Ce $g(a) \neq 0$, potem je $\dfrac{f}{g}$ definirana na okolici to"cke $a \in D$. Obstaja $\delta > 0$, da za
\begin{equation*}
x \in (a - \delta, a + \delta) \cap D
\end{equation*}
velja $g(x) \neq 0$. Torej lahko izberemo poljubno zaporedje $x_n \in (a - \delta, a + \delta) \cap D$, ki konvergira proti $a$. Tedaj je zaporedje $\left(\frac{f}{g}\right)(x_n)$ dobro definirano in konvergira proti $\left(\frac{f}{g}\right)(a)$ (podoben sklep kot prej).

\textsc{Izrek:} Naj bosta $f$ in $g$ funkciji, za kateri velja $Z_f \subseteq D_g$. "Ce je $f$ zvezna v $a$ in $g$ zvezna v $f(a)$, potem je $g \circ f$ zvezna v $a$.

\textsc{Dokaz:} Izberemo poljubno konvergentno zaporeje $x_n \in D_f$ z limito $a$. Ker je $f$ zvezna v $a$, zaporedje $f(x_n)$ konvergira proti $f(a)$. Ker je $g$ zvezna v $f(a)$, zaporedje $g(f(x_n))$ konvergira proti $g(f(a))$.

\deff Naj bo $f: D \to R$ funkcija. Pravimo, da je $f$ \emph{zvezna funkcija}, "ce je zvezna v vsaki to"cki $a \in D$.

\textsc{Primeri:}
\begin{enumerate}[1)]
	\item Konstante so zvezne funkcije. Polinome in racionalne funkcije lahko dobimo z deljenjem in mno"zenjem linearnih funkcij. Posledica je, da so polinomi in racionalne funkcije zvezne.
\end{enumerate}
%
\deff Naj bo $f: D \to R$ funkcija. Pravimo, da je $f$ \emph{enakomerno zvezna} na $D$, "ce velja:
\begin{equation*}
\forall \varepsilon > 0 \exists \delta > 0 \forall x, x' \in D: |x - x'| < \delta \Rightarrow |f(x) - f(x')| < \varepsilon
\end{equation*}
\textbf{Opomba:} "Ce je $f$ enakomerno zvezna na $D$, potem je $f$ zvezna na $D$. $f$ je zvezna na $D$ ($f$ je zvezna na vsaki to"cki $a \in D$):
\begin{equation*}
\forall a \in D \forall \varepsilon > 0 \exists \delta > 0 \forall x \in D: |x-a| < \delta \Rightarrow |f(x) - f(a)| < \varepsilon
\end{equation*}
\dashuline{$f(x) = \frac{1}{x}$ ni enakomerna zvezna na $(0, \infty)$}

Naj bo $0 < a < \dfrac{1}{2}$. Velja:
\begin{equation*}
|f(a) - f(2a)| = \left|\dfrac{1}{a} - \dfrac{1}{2a}\right| = \dfrac{1}{2a}
\end{equation*}
Za $\varepsilon = 1$ izveremo poljuben $\delta > 0$. Obtaja $a$, da velja: $\dfrac{1}{2a} > 1$ in $2a \in (0, \delta)$.
\begin{equation*}
|a - 2a| = a < \delta \land |f(a) - f(2a)| \geq 1
\end{equation*}
\textbf{Lema o pokritjih:} Dan je nek zaprt interval $[a, b]$ in za vsak $x \in [a, b]$ imamo $\delta(x) > 0$. Ozna"cimo $O_x = (x - \delta(x), x + \delta(x)$ $\delta(x)$ okolica to"cke $x$. Tedaj v dru"zini okolic $\{O_x; x \in [a, b]\}$ obstaja kon"cno "stevilo okolic, ki pokrijejo $[a, b]$, t.j. obstaja $n \in \NN$ in obstaja $x_1, x_2, \ldots, x_n \in [a, b]$, da velja $[a, b] \subset \bigcup_{j=1}^n O_{x_j}$

\textsc{Dokaz:} $O_a$ pokrije vsak interval $[a, c)$, kjer je $c < a + \delta(a)$.
\begin{equation*}
S = \{c \in [a, b], \text{interval $[a, c]$ je mogo"ce pokriti s kon"cno mnogo okolicami iz dru"zine $O_x$}\}
\end{equation*}
Vemo $S \neq \varnothing$ in $S \subset [a, b]$. Torej je $S$ neprazna navzgor omejena mno"zica in ima $\sup S := M$.

\dashuline{Dokazujemo $M \in S$}

Vemo $M \in [a, b]$. Ker je $M - \delta(M) < M$, obstaja $c > M - \delta(M), c \in S$. Velja
\begin{align*}
[a, c] & \subset O_{x_1} \cup O_{x_2} \cup \cdots \cup O_{x_k} \\
[a, M] & \subset O_{x_1} \cup O_{x_2} \cup \cdots \cup O_{x_k} \cup O_M
\end{align*}
$\Rightarrow M \in S$

"Ce $M \neq b$ potem zgornji interval pokrije $[a, M + \delta(M)) \cap [a, b]$, kar je ve"c kot $[a, M]$. To je v protislovju s tem, da $M = \sup S$. $\rightarrow \leftarrow$

$\Rightarrow M = b$

\textsc{Posledica:} Naj bo $K = [a, b]$. Iz vsakega pokritja $K$ z odpritmi intervali je mogo"ce izbrati kon"cno podpokritje. To pomeni: $\{I_\gamma: \gamma \in \Gamma\}$ dru"zina odprtih intervalov, za katero velja:
\begin{equation*}
K \subset \bigcup_{\gamma \in \Gamma} I_\gamma
\end{equation*}
Obstaja $m \in \NN$ in $\gamma_1, \gamma_2, \ldots, \gamma_m \in \Gamma$, da velja:
\begin{equation*}
K \subset \bigcup_{i = 1}^m I_{\gamma_i}
\end{equation*}
\textsc{Dokaz posledice}
Za $x \in [a, b]$ velja $x \in \bigcup_{\gamma \in \Gamma} I_\gamma$.

Obstaja $\gamma \in \Gamma: x \in I_\gamma$.

Obstaja $\delta(x) > 0: (x - \delta(x), x + \delta(x)) \subset I_\gamma$

Na ta na"cin dobimo pokritje $[a, b]$ z $O_x$. Po lemi o pokritjih obstaja $m \in \NN, x_1, \ldots, x_m \in [a, b]$
\begin{equation*}
[a, b] \subset \bigcup_{i=1}^m O_{x_i} \subset \bigcup_{i=1}^m I_{\gamma_i}
\end{equation*}
ker $O_{x_i} \subset I_{\gamma_i}$ za ustrazno izbrane $\gamma$. \hfill $\square$

\textsc{Izrek:} Naj bo $f: [a, b] \to \RR$ zvezna funkcija. Potem je $f$ enakomerna zvezna na $[a, b]$

$c \in [a, b]: f$ je zvezna v to"cki $c$:
\begin{equation*}
\forall \varepsilon > 0 \exists \delta > 0 \forall x \in [a, b]: |x - c| < \delta \Rightarrow |f(x) - f(c)| < \varepsilon
\end{equation*}
$f$ je enakomerno zvezna na $[a, b]$:
\begin{equation*}
\forall \varepsilon > 0 \exists \delta > 0 \forall x, x' \in [a, b]: |x - x'| < \delta \Rightarrow |f(x) - f(x')| < \varepsilon
\end{equation*}
\textsc{Dokaz:} Ker je $f$ zvezna na $[a, b]$ je za $x \in [a, b]$ $f$ zvezna v $x$. Izberemo poljuben $\varepsilon > 0$. Obstaja $\delta(x) > 0$, za katerega velja
\begin{equation}
\label{eq:enakomerna_zveznost}
x' \in [a, b]: |x' - x| < 2\delta(x) \Rightarrow |f(x) - f(x)| < \dfrac{\varepsilon}{2}
\end{equation}
\begin{equation*}
O_x = (x - \delta(x), x + \delta(x)): \{O_x: x \in [a, b]\}
\end{equation*}
je odprto pokritje $[a, b]$ in po lemi o pokritjih obstaja kon"cno podpokritje:
\begin{gather*}
m \in \NN: x_1, \ldots, x_m \in [a, b] \\
[a, b] \subset \bigcup_{i=1}^m O_{x_i} \\
\delta := \min \{\delta(x_1), \ldots, \delta(x_m)\}
\end{gather*}
Naj bosta $x, x' \in [a, b], |x - x'| < \delta$.

Obstaja $i \in {1, \ldots, m}: x \in (x_i - \delta(x_i), x_i + \delta(x_i))$:
\begin{equation*}
|x' - x_i| = |x' -x + x - x_i| \leq |x' - x| + |x - x_i| < 2 \delta(x_i)
\end{equation*}
Po~\eqref{eq:enakomerna_zveznost} sledi:
\begin{gather*}
|x - x_i| < \delta(x_i) < 2\delta(x_i) \Rightarrow |f(x) - f(x_i)| < \dfrac{\varepsilon}{2} \\
|x' - x_| < 2\delta(x_i)i \Rightarrow |f(x') - f(x_i)| < \dfrac{\varepsilon}{2} \\
|f(x) - f(x')| = |f(x) - f(x_i) + f(x_i) - f(x')| \leq |f(x) - f(x_i)| + |f(x_i) - f(x')| < \varepsilon
\end{gather*}
\hfill $\square$
%
\subsubsection{Lastnosti zveznih funkcij}
\textsc{Izrek:} (\emph{bisekcija}) Naj bo $f$ zvezna funkcija na $[a, b]$. "Ce ima $f$ v kraji"s"cih intervala $[a, b]$ nasprotno predzna"ceni vrednosti, potem ima $f$ na $[a, b]$ ni"clo.
\begin{equation*}
f(a)f(b) < 0 \Rightarrow (\exists c \in [a, b]: f(c) = 0)
\end{equation*}
\textsc{Dokaz:} $c_1 = \dfrac{a+b}{2}$. "Ce je $f(c_1) = 0$, potem kon"camo.

Sicer ozna"cimo z $[a_1, b_1]$ tistega od podintervalov $[a, c_1], [c_1, b]$, na katerem ima $f$ v kraji"s"cih nasprotno predzna"ceni vrednosti.

Nadaljujemo podobno $c_1 = \dfrac{a_1 + b_1}{2}, \cdots$. "Ce se postpoek ni ustavil.

Na ta na"cin konstruiramo zaporedje vlo"zenih intervalov:
\begin{gather*}
[a, b] \supset [a_1, b_1] \supset [a_2, b_2] \supset \cdots \\
b_n - a_n = \dfrac{1}{2^n}(b-1) \text{\quad in \quad} f(a_n)f(b_n) < 0
\end{gather*}
Po izreku obstaja natanko ena to"cka $c$, ki je vsebovana v vseh intervalih
\begin{equation*}
c = \limninf a_n = \limninf b_n
\end{equation*}
Ker je $f$ zvezna, velja:
\begin{equation*}
f(c) = \limninf f(a_n) = \limninf f(b_n)
\end{equation*}
Ker velja $f(a_n) f(b_n) < 0$
\begin{equation*}
(f(c))^2 = \limninf f(a_n) f(b_n) \leq 0 \Rightarrow f(c) = 0
\end{equation*}
\hfill $\square$

\textbf{Opomba:} Metoda bisekcije je metoda za numberi"cno iskanje ni"cel

\textsc{Izrek:} Naj bo $f$ zvezna funkcija na $[a, b]$. Potem je $f$ omejena. Ozani"cimo:
\begin{equation*}
m = \inf_{[a, b]} f \quad \text{in} \quad M = \sup_{[a, b]} f
\end{equation*}
Obstajata $x_m, x_M \in [a, b]$, za kateri velja
\begin{equation*}
f(x_m) = m \quad \text{in} \quad f(x_M) = M
\end{equation*}
\textbf{Opomba:} Zvezna funkcija na zaprtem intervalu dose"ze minimum in maksimum.

\textsc{Dokaz:} Denimo, da $f$ ni omejena. Recimo, da ni navzgor omejena. Torej velja:
\begin{equation*}
\forall n \in \NN \exists x_n \in [a, b]: f(x_n) > n
\end{equation*}
Zaporedje $x_n$ je omejeno, zato ima stekali"s"ce $s$. Torej obstaja konvergentno podzaporedje $x_{n_k} \to s$. Ker je $f$ zvezna, je $f(x_{n_k})$ konvergentna z limito $f(s)$. Pridemo do protislovja, ker mora biti $f(x_{n_k}) > n$ neomejeno. $\rightarrow \leftarrow$

Naj bo $M = \sup f$. Vemo $\forall x \in [a, b]: f(x) \leq M$. Recimo, da funkcija $f$ ne zadodose"ze vrednosti $M$.
\begin{equation*}
M - f(x) > 0 \quad \forall x \in [a, b]
\end{equation*}
$M-f$ je zvezna na $[a, b]$, zato $\frac{1}{M-f}$ zvezna na $[a, b]$ Po pravkar dokazanem je $\frac{1}{M-f}$ omejena. Torej
\begin{gather*}
\exists A \in \RR, A > 0: \dfrac{1}{M-f} \leq A \quad \forall x \in [a, b] \\
\dfrac{1}{A} \leq M - f(x) \\
f(x) \leq M - \dfrac{1}{A} \quad \forall x \in [a, b]
\end{gather*}
Pridemo do protislovja, ker $M = \sup f$ $\rightarrow \leftarrow$.

Torej obstaja $x_M \in [a ,b]: f(x_M) = M$ \hfill $\square$

\textsc{Posledica:} Naj bo $A$ zvezna funkcija na $[a, b]$. Ozna"cimo $m = \min f$, $M = \max f$. Potem funkcija $f$ dose"ze vse vrednosti med $m$ in $M$.
\begin{equation*}
\forall c \in [m, M] \exists x_c \in [a, b]: f(x_c) = c
\end{equation*}
\textsc{Dokaz:} "Ce je $m = M$ je $f$ konstantna.

"Ce $m \neq M:$ naj bo $c \in (m, M)$. Definiramo $g(x) = f(x) - c$. $g$ je definirana na $[a, b]$ in je zvezna. Obstajata $x_m$ in $x_M: f(x_m) = m, f(x_M) = M$. $g$ zo"zimo med $x_m$ in $x_M$. $g$ ima na tem intervalu v kraji"s"cih nasprotno predzna"ceni vrednosti: $m-c < 0$ in $M-c > 0$. Zato obstaja $x_c: g(x_c) = 0$. Velja: $f(x_c) = c$.

\hfill $\square$