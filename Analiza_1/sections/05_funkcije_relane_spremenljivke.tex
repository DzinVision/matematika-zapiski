\deff Naj bo $D \subseteq \RR$. \emph{(Realna) funkcija (realne spremenljivke)} je preslikava
\begin{equation*}
f: D \to \RR
\end{equation*}
Spomnimo se, da je preslikava predpis, za katerega velja
\begin{equation*}
\forall x \in D \exists! y \in \RR: y = f(x)
\end{equation*}
\textsc{Opombi:}
\begin{enumerate}
	\item V pojmu funkcija sta zdru"zena dve predpostavki:
	\begin{itemize}
		\item definicijsko obmo"cje
		\item predpis
	\end{itemize}

	\item Funkciji $f$ in $g$ sta enki, "ce imata enako definicijsko obmo"cje in predpis.
\end{enumerate}
\textsc{Dogovor:} Funkcijo lahko podamo samo s spredpisom. V tem primeru vzamemo za definicjsko obmo"cje mno"zico vseh $x$, za katero ima predpis smisel.

\textsc{Primer:}
\begin{enumerate}[1)]
	\item $f(x) = \sqrt{x} \quad D_f = [0, \infty)$
	\item Ali sta funkciji, ki sta dani s predpisoma
	\begin{align*}
	f(x) &= \ln (x-1) + \ln (x+1) \\
	g(x) &= \ln (x^2 - 1)
	\end{align*}
	enaki?
	\begin{align*}
	D_f &= (1, \infty) \\
	D_g &= (-\infty, -1) \cup (1, \infty)
	\end{align*}
	Nista enaki, vedar data enake vrednosti na preseku definicijskih obmo"cji.
	\begin{equation*}
	\forall x \in D_f \cap D_g: f(x) = g(x)
	\end{equation*}
\end{enumerate}
\deff Naj bosta $f$ in $g$ funkciji. "Ce velja $D_f \subset D_g$ in "ce velja $f(x) = g(x)$ za vse $x \in D_f$, potem re"cemo, da je funkcija $g$ \emph{raz"siritev} funkcije $f$ in da je $f$ \emph{zo"zitev} funkcije $g$. Zo"zitev zapi"semo kot
\begin{equation*}
g|_{D_f} = f
\end{equation*}
\deff Naj bo $f$ funkcija. Mno"zico
\begin{equation*}
\Gamma_f = \{(x, f(x)): x \in D_f\} \subset \RR^2
\end{equation*}
imenujemo \emph{graf funkcije}.

\textsc{Opomba:}
\begin{enumerate}
	\item Funkcija $f$ je s svojim grafom natan"cno dolo"cena.
	\item Katere podmno"zice $S \subseteq \RR^2$ so grafi funkcij? Presek vsake navpi"cne premice z mno"zico $S$ je to"cka ali prazna mno"zica.
\end{enumerate}
%
\subsection{Operacije s funkcijami}
\deff Naj bosta $f: D_f \to \RR$ in $g: D_g \to \RR$ funckiji. "Ce je $Z_f \subset D_g$, potem lahko definiramo \emph{kompozitum funkcij}
\begin{equation*}
g \circ f: D_f \to \RR
\end{equation*}
s predpisom:
\begin{equation*}
(g \circ f)(x) = g(f(x))
\end{equation*}
Ni nujno, da bi bila $g\circ f$ in $f \circ g$ v kak"sni zvezi.

\textsc{Primer:}
\begin{enumerate}[1)]
	\item $f(x) = x^2 + 1$, $g(x) = \ln x^2$. $D_f = \RR$, $D_g =\RR \setminus \{0\}$
	\begin{align*}
	(f \circ g)(x) &= f(g(x)) = f(\ln x^2) = \ln^2 x^2 + 1 = 4\ln^2|x| + 1 \\
	(g \circ f)(x) &= g(f(x)) = g(x^2 + 1) = \ln(x^2 + 1)^2 = 2 \ln(x^2 + 1)
	\end{align*}
	
	\item $f(x) = x^2$ ni injektivna
	\begin{equation*}
	f|_{[0, \infty)}: [0, \infty) \to \RR
	\end{equation*}
	je injektivna. Naredimo lahko inverzno funkcijo $f^{-1} : Z_f \to \RR$ s predpisom $f^{-1}(x) = \sqrt{x}$
\end{enumerate}
\deff Naj bo $f: D \to \RR$ injektivna funkcija z zalogo vrednosti $Z_f$. Inverzno preslikavo $f^{-1}: Z_f \to \RR$ imenujemo \emph{inverzna funkcija}. Njen predpis je znan "za iz poglavja o preslikavah
\begin{equation*}
f^{-1}(b) = a \iff f(a) = b
\end{equation*}
\textsc{Primer:} Funkcija $f$ je dana s predpisom:
\begin{equation*}
f(x) = \dfrac{2x + 3}{3x - 1}
\end{equation*}
Dolo"ci inverzno funkcijo, "ce obstaja.
\begin{gather*}
D_f = \RR \setminus \left\{\dfrac{1}{3}\right\}\\
\begin{aligned}
y &= \dfrac{2x + 3}{3x-1} \\
y(3x - 1) &= 2x + 3 \\
3xy - 2x &= 3 + y \\
x(3y - 2) &= 3 + y \\
x &= \dfrac{3 + y}{3y- 2}
\end{aligned}
\end{gather*}
Iz ra"cuna sledi: "ce je $y \neq \dfrac{2}{3}$, potem obstaja natanko en $x$ za katerega velja $f(x) = y$. Torej je $Z_f = \RR \setminus \left\{\dfrac{2}{3}\right\}$, $f$ je injektivna in $f^{-1}(y) = \dfrac{y+3}{3y - 2}$.
%
\begin{align*}
\Gamma_f &= \{(x, f(x)): x \in D_f\} = \{(f^{-1}(y), y): y \in Z_f\} \\
\Gamma_{f^{-1}} &= \{(y, f^{-1}(y): y \in Z_f)\}
\end{align*}
Zato je $\Gamma_{f^{-1}}$ dobljen iz $\Gamma_f$ z zrcaljenjem preko simetrale lihih kvadrantov.

\deff Naj bo $f$ funkcija. 
\begin{itemize}
	\item Pravimo, da je $f$ navzogr omejena, "ce je $Z_f$ navzgor omejena, t.j.:
	\begin{equation*}
	\exists M \in \RR \forall x \in D_f : f(x) \leq M
	\end{equation*}
	
	\item Pravimo, da je $f$ navzdol omejena, "ce je $Z_f$ navzdol omejena, t.j.:
	\begin{equation*}
	\exists m \in \RR \forall x \in D_f : f(x) \geq m
	\end{equation*}
	
	\item Funkcija $f$ je omejena, kadar je navzgor in navzdol omejena.
	
	\item "Ce je $f$ navzgor omejena, potem je supremum funkcije $f$ natan"cna zgornja meja zaloge vrednosti
	\begin{equation*}
	\sup f := \sup Z_f
	\end{equation*}
	
	\item "Ce je $f$ navzdol omejena, potem je infimum funkcije $f$ natan"cna spodnja meja zaloge vrednosti
	\begin{equation*}
	\inf f := \inf Z_f
	\end{equation*}
	
	\item "Ce obstaja maksimum zaloge vrednosti, velja
	\begin{equation*}
	\max f := \max Z_f
	\end{equation*}
	
	\item "Ce obstaja minimum zaloge vrednosti, velja
	\begin{equation*}
	\min f := \min Z_f
	\end{equation*}
\end{itemize}
\deff Naj bo $f$ funkcija. Pravimo, da je $x \in D_f$ \emph{ni"cla} funkcije $f$, "ce
\begin{equation*}
f(x) = 0
\end{equation*}
\deff Naj bosta $f, g : D \to \RR$ funkciji. Funkcije $f+g, f-d, f\cdot g: D \to \RR$ definiramo s predpisi:
\begin{align*}
(f+g)(x) &= f(x) + g(x) \\
(f-g)(x) &= f(x) - g(x) \\
(f\cdot g)(x) &= f(x) \cdot g(x)
\end{align*}
"Ce $\forall x \in D g(x) \neq 0$, potem definiramo $\dfrac{f}{g}: D \to \RR$ s predpisom
\begin{equation*}
\left(\dfrac{f}{g}\right)(x) = \dfrac{f(x)}{g(x)}
\end{equation*}
\deff Naj bosta $f, g : D \to \RR$ funkciji. Funkciji $\max \{f, g\}, \min \{f, g\}: D \to \RR$ definiramo s predpisoma:
\begin{align*}
\max \{f, g\}(x) &:= \max\{f(x), g(x)\}\\
\min \{f, g\}(x) &:= \min\{f(x), g(x)\}
\end{align*}
\deff Naj bo $\Gamma$ mno"zica in naj bo za vsak $\gamma \in \Gamma$
\begin{equation*}
f_\gamma: D \to \RR
\end{equation*}
funkcija. "Ce je $\{f_\gamma(x): \gamma \in \Gamma\}$ navzgor omejena za vsak $x \in D$, potem lahko definiramo funkcijo
\begin{equation*}
\sup_{\gamma \in \Gamma}f_\gamma: D \to \RR
\end{equation*}
s predpisom
\begin{equation*}
(\sup_{\gamma \in \Gamma} f_\gamma)(x) = \sup \{f_\gamma(x): \gamma \in \Gamma\}
\end{equation*}
Podobno definiramo 
\begin{equation*}
\inf_{\gamma \in \Gamma}f_\gamma: D \to \RR
\end{equation*}
\textsc{Primer:} $f_\gamma(x) = \dfrac{1}{1 + \gamma x^2}, \gamma \in (0, \infty)$, $D_{f_\gamma} = \RR$
\begin{equation*}
\left\{\dfrac{1}{1 + \gamma x^2}: \gamma \in (0, \infty)\right\} \text{ je navzgor omejena z 1 in navzdol omejena z 0}
\end{equation*}
%
\subsection{Zveznost}
Zvezna funkcija je funkcija, kjer ,,majhna'' spremembna neodvisne spremenljivke povzro"ci ,,majhno'' spremembo odvisne spremenljivke.

\textsc{Primera}
\begin{itemize}
	\item $g(x) = \sin \dfrac{1}{x}$: V 0 ni definirana, zato ne moremo govoriti v zveznosti. Lahko dolo"cimo $g(0) = 0$, vendar ne glede na vrednost, ki jo dolo"cimo v 0, $g$ ne bo zvezna.
	
	\item $h(x) = x \sin \dfrac{1}{x}$. "Ce dolo"cimo $h(0) = 0$, potem je $h$ zvezna.
\end{itemize}
\deff Naj bo $f: D \to \RR$ funkcija in $a \in D$ to"cka. Funkcija $f$ je \emph{zvezna v to"cki $a \in D$}, "ce velja
\begin{equation*}
\forall \varepsilon > 0 \exists \delta > 0 \forall x \in D: |x - a| < \delta \Rightarrow |f(x)-f(a)| < \varepsilon
\end{equation*}
\textbf{Opomba:} $\delta$ je odvisna od $\varepsilon$. Najprej si izberemo $\varepsilon$, nato pa dolo"cimo $\delta$.

\textsc{Primeri:}
\begin{enumerate}[1)]
	\item $C \in \RR, \quad f(x) = C$ \dashuline{$f$ je zvezna v $a \in \RR$}
	
	Izberemo poljuben $\varepsilon > 0$. Vemo $f(a) = C$. "Ce si izberemo $\delta = 1$, velja:
	\begin{equation*}
	|x - a| < 1 \Rightarrow |f(x) - f(a)| = 0 < \varepsilon
	\end{equation*}
	
	\item $g(x) = x$ \dashuline{$g$ je zvezna v $a \in \RR$}
	
	Izberemo $\varepsilon > 0$. Dokazujemo
	\begin{equation*}
		|x - a| < \delta \Rightarrow |f(x) - f(a)| < \varepsilon
	\end{equation*}
	Naj bo $\delta := \varepsilon$. Velja:
	\begin{equation*}
	|f(x)-f(a)| = |x-a| < \varepsilon
	\end{equation*}
	ker $|x-a| < \delta$ in $\varepsilon = \delta$
	
	\item $f(x) = \begin{cases} 0 & x \leq 0 \\ 1 & x > 1\end{cases}$ \dashuline{$f$ ni zvezna v 0}
	
	Negacija definicje je:
	\begin{equation*}
	\exists \varepsilon >0 \forall \delta>0 \exists x \in D: |x-a| < \delta \land |f(x) - f(a)| \geq \varepsilon
	\end{equation*}
	Naj bo $\varepsilon = \dfrac{1}{2}$. Izberemo poljuben $\delta > 0$. Naj bo $x_\delta = \dfrac{\delta}{2}$. $x_\delta$ je v okolici to"cke 0, ker
	\begin{equation*}
	|0 - x_\delta| = |0 - \dfrac{\delta}{2}| = \dfrac{\delta}{2} < \delta
	\end{equation*}
	Funkcijske slike niso v $\varepsilon$-ti okolici to"cke $f(0)$, ker:
	\begin{equation*}
	|f(0) - f(x_\delta)| = |0 - 1| = 1 \geq \varepsilon = \dfrac{1}{2}
	\end{equation*}
\end{enumerate}
\deff Naj bo $\delta \in \RR,\ \delta > 0,\ D \subset \RR,\ a \in D$. Mno"zico $(a - \delta, a + \delta) \cap D$ imenujemo \emph{$\delta$ okolice to"cek $a$ v $D$}. Mno"zica $U \subset \RR$ je \emph{okolica to"cke $a$ v $D$}, "ce vsebuje kak"sno $\delta$-okolico to"cke $a$ v $D$.

\textsc{Ekvivalentni definicija zveznosti}
\begin{itemize}
	\item Naj bo $f: D \to \RR$ funkcija in $a \in D$. Potem je $f$ zvezna v to"cki $a$ natanko tedaj, kadar velja, da za vsak $\varepsilon > 0$ obstaja $\delta > 0$, da $f$ preslika $\delta$-okolico to"cke $a$ v $\varepsilon$-to okolico to"cke $f(a)$.
	
	\item Funkcija $f$ je zvezna v to"cki $a$ natanko tedaj, "ce za vsako okolico $V$ to"cek $f(a)$, obstaja taka okolica $U$ to"cke $a$, da velja $f(U) \subset V$.
\end{itemize}
%
\subsubsection{Opis zveznosti z zaporedji}
Naj bo $f: D \to R$ zvezna v $a \in D$. Izberemo poljubno konvergentno zaopredje $x_n \in D$ z limito $a$. Opazujemo $f(x_n)$. Trdimo \dashuline{$f(x_n)$ konvergira proti $f(a)$}.

Izberemo $\varepsilon > 0$. Ker je $f$ zvezna v $a$, obstaja $\delta > 0$, da za $x \in D$ velja, "ce $|x -a| < \delta$ sledi $|f(x) - f(a)| < \varepsilon$. Ker $x_n$ konvergira proti $a$, obstaja $n_0 \in \NN$, da velja $|x_n - a| < \delta$ za vse $n \geq n_0$. Potem velja $|f(x_n) - f(a)| < \varepsilon$ za vse $n \geq n_0$. \hfill $\square$

\textsc{Izrek:} Naj bo $f: D \to \RR$ funkcija in $a \in D$. $f$ je zvezna v $a \in D$ natanko tedaj, kadar za \textbf{vsako} zaporedje $x_n \in D$, ki konvergira proti $a$, zaporedje $f(x_n)$ konvergira proti $f(a)$.

\textsc{Dokaz:}
\begin{itemize}
	\item[($\Rightarrow$)] "Ze dokazano.
	\item[($\Leftarrow$)] Denimo, da $f$ ni zvezna v to"cki $a$. Potem velja:
	\begin{equation*}
	\exists \varepsilon > 0 \forall \delta > 0 \exists x \in D: |x-a| < \delta \land |f(x) - f(a)| \geq \varepsilon
	\end{equation*}
	Torej za vsak $n$ velja:
	\begin{equation*}
	x_n \in D: |x_n - a| < \dfrac{1}{n} \land |f(x_n) - f(a)| \geq \varepsilon
	\end{equation*}
	kjer je $\dfrac{1}{n} = \delta$.
	
	Zaporedje $x_n$ konvergira proti $a$, ker $|x_n - a| < \dfrac{1}{n}$ za vsak $n$. Zaporedje $f(x_n)$ ne konvergira proti $f(a)$, ker $|f(x_n) - f(a)| \geq \varepsilon$ za vsak $n$. 
	
	\hfill $\square$
\end{itemize}
\textsc{Primeri:}
\begin{enumerate}[1)]
	\item $f(x) = \begin{cases}0 & x \in \RR \setminus \QQ \\
	1 & x \in \QQ
	\end{cases}$ \quad \emph{Dirichletova funkcija}
	
	Ni zvezna v nobeni to"cki. Naj bo $a \in \RR$. Velja
	\begin{align*}
	\exists &q_n \in \QQ, q_n \to a \\
	\exists &a_n \in \RR \setminus \QQ, a_n \to a
	\end{align*}
	Torej je $f(q_n) = 1$ za vsak $n$ iin $\limninf f(q_n) = 1$. Vemo tudi, da $f(a_n) = 0$ za vsak $n$ in $\limninf f(a_n) = 0$. Torej $f$ ni zvezna v $a$.
	
	\item \begin{align*}
	g&: (0, 1) \to \RR \\
	g(x) &= \begin{cases}
	\dfrac{1}{k} & x = \dfrac{m}{k} \text{ okraj"san ulomek}\\
	0 & x \in \RR \setminus \QQ
	\end{cases}
	\end{align*}
	$g$ ni zvezna v $\QQ$ to"ckah.
\end{enumerate}
%
\textsc{Izrek:} Naj bosta $f, g: D \to \RR$ zvezni funkciji v to"cki $a \in D$. Potem so funkcije $f + g$, $f-g$, $f\cdot g$ zvezne v to"cki $a$. "Ce je $g(a) \neq 0$, potem je $\dfrac{f}{g}$ zvezna v to"cki $a$.

\textsc{Dokaz} Vemo, da je $f$ zvezna v $a \in D$ $\iff$ za vsako zaporedje $x_n$ v $D$, ki konvergira proti $a$, zaporedje $f(x_n)$ konvergira proti $f(a)$. Izberemo poljubno zaporedje $x_n \in D$, ki konvergira proti $a$. Dokazujemo $(f+g)(x_n)$ kovergira proti $(f+g)(a)$.

$(f+g)(x_n) = f(x_n) + g(x_n)$ ker sta $f$ in $g$ zvezni, $f(x_n)$ konvergira proti $f(a)$ in $g(x_n)$ konvergira proti $g(a)$. Torej $f(x_n) + g(x_n)$ konvergira proti $f(a) + g(a) = (f+g)(a)$.

"Ce $g(a) \neq 0$, potem je $\dfrac{f}{g}$ definirana na okolici to"cke $a \in D$. Obstaja $\delta > 0$, da za
\begin{equation*}
x \in (a - \delta, a + \delta) \cap D
\end{equation*}
velja $g(x) \neq 0$. Torej lahko izberemo poljubno zaporedje $x_n \in (a - \delta, a + \delta) \cap D$, ki konvergira proti $a$. Tedaj je zaporedje $\left(\frac{f}{g}\right)(x_n)$ dobro definirano in konvergira proti $\left(\frac{f}{g}\right)(a)$ (podoben sklep kot prej).

\textsc{Izrek:} Naj bosta $f$ in $g$ funkciji, za kateri velja $Z_f \subseteq D_g$. "Ce je $f$ zvezna v $a$ in $g$ zvezna v $f(a)$, potem je $g \circ f$ zvezna v $a$.

\textsc{Dokaz:} Izberemo poljubno konvergentno zaporeje $x_n \in D_f$ z limito $a$. Ker je $f$ zvezna v $a$, zaporedje $f(x_n)$ konvergira proti $f(a)$. Ker je $g$ zvezna v $f(a)$, zaporedje $g(f(x_n))$ konvergira proti $g(f(a))$.

\deff Naj bo $f: D \to R$ funkcija. Pravimo, da je $f$ \emph{zvezna funkcija}, "ce je zvezna v vsaki to"cki $a \in D$.

\textsc{Primeri:}
\begin{enumerate}[1)]
	\item Konstante so zvezne funkcije. Polinome in racionalne funkcije lahko dobimo z deljenjem in mno"zenjem linearnih funkcij. Posledica je, da so polinomi in racionalne funkcije zvezne.
\end{enumerate}
%
\deff Naj bo $f: D \to R$ funkcija. Pravimo, da je $f$ \emph{enakomerno zvezna} na $D$, "ce velja:
\begin{equation*}
\forall \varepsilon > 0 \exists \delta > 0 \forall x, x' \in D: |x - x'| < \delta \Rightarrow |f(x) - f(x')| < \varepsilon
\end{equation*}
\textbf{Opomba:} "Ce je $f$ enakomerno zvezna na $D$, potem je $f$ zvezna na $D$. $f$ je zvezna na $D$ ($f$ je zvezna na vsaki to"cki $a \in D$):
\begin{equation*}
\forall a \in D \forall \varepsilon > 0 \exists \delta > 0 \forall x \in D: |x-a| < \delta \Rightarrow |f(x) - f(a)| < \varepsilon
\end{equation*}
\dashuline{$f(x) = \frac{1}{x}$ ni enakomerna zvezna na $(0, \infty)$}

Naj bo $0 < a < \dfrac{1}{2}$. Velja:
\begin{equation*}
|f(a) - f(2a)| = \left|\dfrac{1}{a} - \dfrac{1}{2a}\right| = \dfrac{1}{2a}
\end{equation*}
Za $\varepsilon = 1$ izveremo poljuben $\delta > 0$. Obtaja $a$, da velja: $\dfrac{1}{2a} > 1$ in $2a \in (0, \delta)$.
\begin{equation*}
|a - 2a| = a < \delta \land |f(a) - f(2a)| \geq 1
\end{equation*}
\textbf{Lema o pokritjih:} Dan je nek zaprt interval $[a, b]$ in za vsak $x \in [a, b]$ imamo $\delta(x) > 0$. Ozna"cimo $O_x = (x - \delta(x), x + \delta(x)$ $\delta(x)$ okolica to"cke $x$. Tedaj v dru"zini okolic $\{O_x; x \in [a, b]\}$ obstaja kon"cno "stevilo okolic, ki pokrijejo $[a, b]$, t.j. obstaja $n \in \NN$ in obstaja $x_1, x_2, \ldots, x_n \in [a, b]$, da velja $[a, b] \subset \bigcup_{j=1}^n O_{x_j}$

\textsc{Dokaz:} $O_a$ pokrije vsak interval $[a, c)$, kjer je $c < a + \delta(a)$.
\begin{equation*}
S = \{c \in [a, b], \text{interval $[a, c]$ je mogo"ce pokriti s kon"cno mnogo okolicami iz dru"zine $O_x$}\}
\end{equation*}
Vemo $S \neq \varnothing$ in $S \subset [a, b]$. Torej je $S$ neprazna navzgor omejena mno"zica in ima $\sup S := M$.

\dashuline{Dokazujemo $M \in S$}

Vemo $M \in [a, b]$. Ker je $M - \delta(M) < M$, obstaja $c > M - \delta(M), c \in S$. Velja
\begin{align*}
[a, c] & \subset O_{x_1} \cup O_{x_2} \cup \cdots \cup O_{x_k} \\
[a, M] & \subset O_{x_1} \cup O_{x_2} \cup \cdots \cup O_{x_k} \cup O_M
\end{align*}
$\Rightarrow M \in S$

"Ce $M \neq b$ potem zgornji interval pokrije $[a, M + \delta(M)) \cap [a, b]$, kar je ve"c kot $[a, M]$. To je v protislovju s tem, da $M = \sup S$. $\rightarrow \leftarrow$

$\Rightarrow M = b$

\textsc{Posledica:} Naj bo $K = [a, b]$. Iz vsakega pokritja $K$ z odpritmi intervali je mogo"ce izbrati kon"cno podpokritje. To pomeni: $\{I_\gamma: \gamma \in \Gamma\}$ dru"zina odprtih intervalov, za katero velja:
\begin{equation*}
K \subset \bigcup_{\gamma \in \Gamma} I_\gamma
\end{equation*}
Obstaja $m \in \NN$ in $\gamma_1, \gamma_2, \ldots, \gamma_m \in \Gamma$, da velja:
\begin{equation*}
K \subset \bigcup_{i = 1}^m I_{\gamma_i}
\end{equation*}
\textsc{Dokaz posledice}
Za $x \in [a, b]$ velja $x \in \bigcup_{\gamma \in \Gamma} I_\gamma$.

Obstaja $\gamma \in \Gamma: x \in I_\gamma$.

Obstaja $\delta(x) > 0: (x - \delta(x), x + \delta(x)) \subset I_\gamma$

Na ta na"cin dobimo pokritje $[a, b]$ z $O_x$. Po lemi o pokritjih obstaja $m \in \NN, x_1, \ldots, x_m \in [a, b]$
\begin{equation*}
[a, b] \subset \bigcup_{i=1}^m O_{x_i} \subset \bigcup_{i=1}^m I_{\gamma_i}
\end{equation*}
ker $O_{x_i} \subset I_{\gamma_i}$ za ustrazno izbrane $\gamma$. \hfill $\square$

\textsc{Izrek:} Naj bo $f: [a, b] \to \RR$ zvezna funkcija. Potem je $f$ enakomerna zvezna na $[a, b]$

$c \in [a, b]: f$ je zvezna v to"cki $c$:
\begin{equation*}
\forall \varepsilon > 0 \exists \delta > 0 \forall x \in [a, b]: |x - c| < \delta \Rightarrow |f(x) - f(c)| < \varepsilon
\end{equation*}
$f$ je enakomerno zvezna na $[a, b]$:
\begin{equation*}
\forall \varepsilon > 0 \exists \delta > 0 \forall x, x' \in [a, b]: |x - x'| < \delta \Rightarrow |f(x) - f(x')| < \varepsilon
\end{equation*}
\textsc{Dokaz:} Ker je $f$ zvezna na $[a, b]$ je za $x \in [a, b]$ $f$ zvezna v $x$. Izberemo poljuben $\varepsilon > 0$. Obstaja $\delta(x) > 0$, za katerega velja
\begin{equation}
\label{eq:enakomerna_zveznost}
x' \in [a, b]: |x' - x| < 2\delta(x) \Rightarrow |f(x) - f(x)| < \dfrac{\varepsilon}{2}
\end{equation}
\begin{equation*}
O_x = (x - \delta(x), x + \delta(x)): \{O_x: x \in [a, b]\}
\end{equation*}
je odprto pokritje $[a, b]$ in po lemi o pokritjih obstaja kon"cno podpokritje:
\begin{gather*}
m \in \NN: x_1, \ldots, x_m \in [a, b] \\
[a, b] \subset \bigcup_{i=1}^m O_{x_i} \\
\delta := \min \{\delta(x_1), \ldots, \delta(x_m)\}
\end{gather*}
Naj bosta $x, x' \in [a, b], |x - x'| < \delta$.

Obstaja $i \in {1, \ldots, m}: x \in (x_i - \delta(x_i), x_i + \delta(x_i))$:
\begin{equation*}
|x' - x_i| = |x' -x + x - x_i| \leq |x' - x| + |x - x_i| < 2 \delta(x_i)
\end{equation*}
Po~\eqref{eq:enakomerna_zveznost} sledi:
\begin{gather*}
|x - x_i| < \delta(x_i) < 2\delta(x_i) \Rightarrow |f(x) - f(x_i)| < \dfrac{\varepsilon}{2} \\
|x' - x_| < 2\delta(x_i)i \Rightarrow |f(x') - f(x_i)| < \dfrac{\varepsilon}{2} \\
|f(x) - f(x')| = |f(x) - f(x_i) + f(x_i) - f(x')| \leq |f(x) - f(x_i)| + |f(x_i) - f(x')| < \varepsilon
\end{gather*}
\hfill $\square$
%
\subsubsection{Lastnosti zveznih funkcij}
\textsc{Izrek:} (\emph{bisekcija}) Naj bo $f$ zvezna funkcija na $[a, b]$. "Ce ima $f$ v kraji"s"cih intervala $[a, b]$ nasprotno predzna"ceni vrednosti, potem ima $f$ na $[a, b]$ ni"clo.
\begin{equation*}
f(a)f(b) < 0 \Rightarrow (\exists c \in [a, b]: f(c) = 0)
\end{equation*}
\textsc{Dokaz:} $c_1 = \dfrac{a+b}{2}$. "Ce je $f(c_1) = 0$, potem kon"camo.

Sicer ozna"cimo z $[a_1, b_1]$ tistega od podintervalov $[a, c_1], [c_1, b]$, na katerem ima $f$ v kraji"s"cih nasprotno predzna"ceni vrednosti.

Nadaljujemo podobno $c_1 = \dfrac{a_1 + b_1}{2}, \cdots$. "Ce se postpoek ni ustavil.

Na ta na"cin konstruiramo zaporedje vlo"zenih intervalov:
\begin{gather*}
[a, b] \supset [a_1, b_1] \supset [a_2, b_2] \supset \cdots \\
b_n - a_n = \dfrac{1}{2^n}(b-1) \text{\quad in \quad} f(a_n)f(b_n) < 0
\end{gather*}
Po izreku obstaja natanko ena to"cka $c$, ki je vsebovana v vseh intervalih
\begin{equation*}
c = \limninf a_n = \limninf b_n
\end{equation*}
Ker je $f$ zvezna, velja:
\begin{equation*}
f(c) = \limninf f(a_n) = \limninf f(b_n)
\end{equation*}
Ker velja $f(a_n) f(b_n) < 0$
\begin{equation*}
(f(c))^2 = \limninf f(a_n) f(b_n) \leq 0 \Rightarrow f(c) = 0
\end{equation*}
\hfill $\square$

\textbf{Opomba:} Metoda bisekcije je metoda za numberi"cno iskanje ni"cel

\textsc{Izrek:} Naj bo $f$ zvezna funkcija na $[a, b]$. Potem je $f$ omejena. Ozani"cimo:
\begin{equation*}
m = \inf_{[a, b]} f \quad \text{in} \quad M = \sup_{[a, b]} f
\end{equation*}
Obstajata $x_m, x_M \in [a, b]$, za kateri velja
\begin{equation*}
f(x_m) = m \quad \text{in} \quad f(x_M) = M
\end{equation*}
\textbf{Opomba:} Zvezna funkcija na zaprtem intervalu dose"ze minimum in maksimum.

\textsc{Dokaz:} Denimo, da $f$ ni omejena. Recimo, da ni navzgor omejena. Torej velja:
\begin{equation*}
\forall n \in \NN \exists x_n \in [a, b]: f(x_n) > n
\end{equation*}
Zaporedje $x_n$ je omejeno, zato ima stekali"s"ce $s$. Torej obstaja konvergentno podzaporedje $x_{n_k} \to s$. Ker je $f$ zvezna, je $f(x_{n_k})$ konvergentna z limito $f(s)$. Pridemo do protislovja, ker mora biti $f(x_{n_k}) > n$ neomejeno. $\rightarrow \leftarrow$

Naj bo $M = \sup f$. Vemo $\forall x \in [a, b]: f(x) \leq M$. Recimo, da funkcija $f$ ne zadodose"ze vrednosti $M$.
\begin{equation*}
M - f(x) > 0 \quad \forall x \in [a, b]
\end{equation*}
$M-f$ je zvezna na $[a, b]$, zato $\frac{1}{M-f}$ zvezna na $[a, b]$ Po pravkar dokazanem je $\frac{1}{M-f}$ omejena. Torej
\begin{gather*}
\exists A \in \RR, A > 0: \dfrac{1}{M-f} \leq A \quad \forall x \in [a, b] \\
\dfrac{1}{A} \leq M - f(x) \\
f(x) \leq M - \dfrac{1}{A} \quad \forall x \in [a, b]
\end{gather*}
Pridemo do protislovja, ker $M = \sup f$ $\rightarrow \leftarrow$.

Torej obstaja $x_M \in [a ,b]: f(x_M) = M$ \hfill $\square$

\textsc{Posledica:} Naj bo $A$ zvezna funkcija na $[a, b]$. Ozna"cimo $m = \min f$, $M = \max f$. Potem funkcija $f$ dose"ze vse vrednosti med $m$ in $M$.
\begin{equation*}
\forall c \in [m, M] \exists x_c \in [a, b]: f(x_c) = c
\end{equation*}
\textsc{Dokaz:} "Ce je $m = M$ je $f$ konstantna.

"Ce $m \neq M:$ naj bo $c \in (m, M)$. Definiramo $g(x) = f(x) - c$. $g$ je definirana na $[a, b]$ in je zvezna. Obstajata $x_m$ in $x_M: f(x_m) = m, f(x_M) = M$. $g$ zo"zimo med $x_m$ in $x_M$. $g$ ima na tem intervalu v kraji"s"cih nasprotno predzna"ceni vrednosti: $m-c < 0$ in $M-c > 0$. Zato obstaja $x_c: g(x_c) = 0$. Velja: $f(x_c) = c$.

\hfill $\square$
%
\subsection{Monotone zvezne funkcije}
\deff Naj bo $I \subset \RR$ interval in $f: I \to \RR$ funkcija. Pravimo, da je $f$ \emph{nara"s"cajo"ca funkcija} (na $I$), "ce velja
\begin{equation*}
\forall x_1, x_2 \in I: x_1 \leq x_2 \Rightarrow f(x_1) \leq f(x_2)
\end{equation*}
Pravimo, da je $f$ \emph{strogo nara"s"cajo"ca} (na $I$), "ce
\begin{equation*}
\forall x_1, x_2 \in I : x_1 < x_2 \Rightarrow f(x_1) < f(x_2)
\end{equation*}
Podobno definiramo padajo"co in strogo padajo"co funkcijo.

Funkcija  je (strogo) \emph{monotona}, "ce je ali (strogo) nara"s"cajo"ca, ali (strogo) padajo"ca.

\textsc{Izrek:} Naj bo $f$ strogo monotona funkcija na $[a, b]$. "Ce je $f$ zvezna, potem je njena inverzna funkcija $f^{-1}$ zvezna.

\textsc{Dokaz:} Denimo, da je $f$ strogo nara"s"cajo"ca. Ker je $f$ strogo nara"s"cajo"ca, je injektivna, zato inverzna funkcija obstaja.

Vemo: Ker je $f$ zvezna, je njena zaloga vrednosti zaprt interval od najmanj"se do najve"cje vrednosti, t.j.: $[f(a), f(b)]$.
\begin{equation*}
f^{-1}: [f(a), f(b)] \to \RR
\end{equation*}
Dokazujemo, da je $f^{-1}$ zvezna $\forall y_0 \in [f(a), f(b)]$.

Izberimo $\varepsilon > 0$. Obstaja natanko en $x_0 \in (a, b): f(x_0) = y_0$ ($f^{-1}(y_0) = x_0$).

Ker je $f$ strogo nara"s"cajo"ca:
\begin{equation*}
f(x_0 - \varepsilon) < \underbrace{y_0}_{f(x_0)} < f(x_0 + \varepsilon)
\end{equation*}
Izberemo
\begin{equation*}
\delta = \min \{f(x_0+\varepsilon) - y_0, y_0 - f(x_0 - \varepsilon)\} > 0
\end{equation*}
"Ce $|y - y_0| < \delta : f(x_0 - \varepsilon) < y < f(x_0 + \varepsilon)$

Ker je $f^{-1}$ strogo nara"s"cajo"ca:
\begin{gather*}
x_0 - \varepsilon < f^{-1}(y) < x_0 + \varepsilon \\
|f^{-1}(y) - \underbrace{x_0}_{f^{-1}(y_0)}| < \varepsilon
\end{gather*}
\hfill $\square$
%
\subsection{Zveznost posebnih funkcij}
Vemo, da so polinomi in racionalne funkcije zvezne.

\textsc{Primer} $f(x) = x^n$ je zvezna funkcija
\begin{itemize}
	\item $n \in \NN$, $n$ liho, potem je $f$ strogo nara"s"cajo"ca na $\RR$. Po izreku $x \mapsto \sqrt[n]{x}$ zvezna funkcija na $\RR$.

	\item $n$ sodo: $x \mapsto \sqrt[n]{x}$ je zezna funkcija na $[0, \infty)$.
\end{itemize}
Zato so \emph{algebrai"cne} funkcije zvezne.

Eksponentna funkcija $a>0, a \neq 1$. Definirali smo "ze $a^x, x \in \RR$
\begin{gather*}
a^{\frac{m}{n}} = \sqrt[n]{a^m} \\
x \in \RR, x = \limninf q_n \\
a^x := \limninf a^{q_n}
\end{gather*}
Iz pravil za ra"cunanje z racionalnimi potencami sledi:
\begin{equation*}
a^{x+y} = a^x a^y, \quad x, y \in \RR
\end{equation*}
\textsc{Izrek:} Eksponentna funkcija $x \mapsto a^x$ je zvezna na $\RR$, njena zaloga vrednosti je $(0, \infty)$.

"Ce je $a> 1$, potem je eksponentna funkcija nara"s"cajo"ca, "ce je $a < 1$, potem je padajo"ca.

\textsc{Dokaz:} Za $a > 1$, za $a < 1$ je podobno. \dashuline{$x \mapsto a^x$ strogo nara"s"cajo"ca}

\dashuline{$x_1 < x_2 \Rightarrow a^{x_1} < a^{x_2}$}
\begin{equation*}
a^{x_2} = a^{x_1 + (x_2 - x_1)} = a^{x_1} \cdot \underbrace{a^{x_2 - x_1}}_{> 1} > a^{x_1}
\end{equation*}
Zveznost: Vemo "ze
\begin{equation*}
\forall \varepsilon > 0 \exists \delta > 0 \forall h \in \QQ, |h| < \delta \Rightarrow |a^h - 1| < \varepsilon
\end{equation*}
Ker je $x \mapsto a^x$ strogo nara"s"cajo"ca, velja zgornja trditev tudi "ce $\QQ$ zamenjamo z $\RR$.
\begin{equation*}
h>0: \exists h_1 \in (h, \delta): 0 < h < h_1 < \delta \Rightarrow a^0 < a^h < a^{h_1}
\end{equation*}
ker je $|a^{h_1} - a^0| < \varepsilon$ je tudi $|a^h - a^0| < \varepsilon$.

Podobno lahko naredimo za $h < 0$.

Torej je $x \mapsto a^x$ zvezna v to"cki 0.

Naj bo $x_0 \in \RR$ in doka"zimo zveznost v $x_0$. Izberemo $\varepsilon > 0$ in i"s"cemo $\delta > 0$:
\begin{gather*}
|a^x - a^{x_0}| = a^{x_0} |a^{x - x_0} - 1| \\
|a^{x - x_0} - 1| < \frac{\varepsilon}{a^{x_0}}
\end{gather*}
Ker je zvezna v to"cki 0, obstaja $\delta > 0$:
\begin{equation*}
|h| = |x - x_0| < \delta \Rightarrow |a^h - 1| < \frac{\varepsilon}{a^{x_0}}
\end{equation*}
%
\begin{align*}
\limninf a^n &= \infty \\
\limninf a^{-1} &= 0
\end{align*}
Na $[-n, n]$ $a^x$ dose"ze vse vrednosti od $[a^{-n}, a^n]$. Za $a^x =\bigcup_{n \in \NN} [a^{-n}, a^n] = (0, \infty)$

\textbf{Posledica:} Naj bo $a > 0, a \neq 1$. Potem velja:
\begin{equation*}
(a^x)^y = a^{xy}, \quad x, y \in \RR
\end{equation*}
\textsc{Dokaz:} (skica)
\begin{align*}
p_n \in \QQ, \quad p_n &\to x \\
q_m \in \QQ, \quad q_m &\to y
\end{align*}
$(a^{p_n})^{q_m} = a^{p_n q_m}$ za $\forall n, m$ (ker so racionalne potence)

$a^{p_n q_m} = a^{x q_m}$, ko gre $n \to \infty$ po definiciji. $a^{x q_m} = a^{xy}$, ko gre $m \to \infty$ po definiciji.

$(a^{p_n})^{q_m} = (a^x)^{q_m}$, ko gre $n \to \infty$, ker je $x \mapsto x^{q_m}$ zvezna. $(a^x)^{q_m} = (a^x)^y$, ko gre $m \to \infty$, zaradi zveznosti.

\deff Naj bo $a > 0, a \neq 1$> inverzno funkcijo eksponentne funkcije $x \mapsto a^x$ imenujemo \emph{logaritemska funkcija} z osnovo $a$ in ozna"cimo z $\log_a$ Logaritem z osnovo $e$ ozna"icmo z $\ln$ in imenujemo \emph{naravni logaritem.}

\textbf{Opomba:}
\begin{itemize}
	\item $\log$ obstaja (po nekem izreku)
	\item $\log_a: (0, \infty) \to \RR$ slika bijektivno
\end{itemize}
Po definiciji inverzne funkcije je
\begin{equation*}
y = \log_a x \iff a^y = x
\end{equation*}
\textsc{Izrek:} Funkcija $\log_a$ je zvezna na $(0, \infty)$. "Ce je $a > 1$, potem je strogor nara"s"cajo"ca, "ce je $a < 1$, potem je strogo padajo"ca. \textbf{Velja}:
\begin{align*}
\log_a(xy) &= \log_a x + \log_a y &&\forall x, y > 0 \\
\log_a x^\lambda &= \lambda \log_a x &&\forall x > 0, \lambda \in \RR
\end{align*}
\textsc{Dokaz:} Vse razen lastnosti so posledica prej"snjih izrekov
\begin{gather*}
c = \log_a x \quad \text{in} \quad d = \log_a y \\
\Rightarrow a^c = x \quad \text{in} \quad a^d = y \\
\Rightarrow \underbrace{a^c \cdot a^d}_{a^{c + d}} = xy \\
\Rightarrow c + d = \log_a (x y) \\
\Rightarrow \log_a x + \log_a y = \log_a (xy)
\end{gather*}
\hfill $\square$
%
\subsection{Trigonometri"cne funkcije}
Iz prosminarja vemo
\begin{align*}
\sin (\varphi + \frac{\pi}{2}) &= \cos \varphi \\
\dfrac{\sin \varphi}{\cos\varphi} &= \tan \varphi
\end{align*}
zato je za "studij osnovnih lastnosti dovolj obravnati le eno funkcijo.

\textsc{Izrek:} Sinus je zvezna funkcija $\RR$.

\textsc{Dokaz:} Nari"semo si enotsko kro"znico in v njo vri"semo kot $\varphi_0$ in $\varphi$, kjer $\varphi_0 < \varphi$. Ozani"cimo trikotnik $ABC$, tako da je $A(\cos \varphi_0, \sin \varphi_0)$, in $B(\cos\varphi, \sin\varphi)$. Vemo, da je $ABC$ pravokotni trikotnik, in je dol"cina loka $AB = l$. Velja $l > AB > AC, BC$.

Vemo: $\overline{BC} = |\sin \varphi - \sin \varphi_0|$ in dol"zina loka $AB = |\varphi - \varphi_0|$. Velja $|\sin \varphi - \sin \varphi_0| \leq |\varphi - \varphi_0|$.

\dashuline{sinus je zvezna v to"cki $\varphi_0$}

Izberemo $\varepsilon > 0$ in i"s"cemo $\delta$. Velja
\begin{equation*}
|\sin\varphi - \sin \varphi_0| \leq |\varphi - \varphi_0|  \varepsilon
\end{equation*}
Izberemo $\delta := \varepsilon$ : "ce je $|\varphi - \varphi_0| < \delta = \varepsilon$, potem je $|\sin\varphi - \sin \varphi_0| \leq |\varphi-\varphi_0| < \delta = \varepsilon$.

\hfill $\square$
%
\subsection{Ciklometri"cne funkcije}.
Ve"cino znamo "ze iz proseminarja. Vemo, da je $\arcsin$ inverzna funkcija $\sin|_{[- \frac{\pi}{2}, \frac{\pi}{2}]}$. Posledica izreka je, da so arkus sinus, arkus kosinus, arkus tangens in arkus kotangens zvezna funkcije.

\emph{Elementarne funkcije} so funkcije, ki jih dobimo iz osnovnih tipov (polinomi, racionalne funkcije, algebrai"cne funkcije, eksponentne, trigonometri"cne) z uporabo aritmeti"cnih opracij, kompozituma in invertiranja. \textbf{Primer:}
\begin{equation*}
\ln (\arcsin ( \sqrt{x^2 + 6x} + e^x))
\end{equation*}
Elementarne funkcije so zvezne.
\begin{equation*}
f(x) = \begin{cases}
x & x > 0 \\
16 & x \leq 0
\end{cases}
\end{equation*}
ni zvezna funkcije (ni elementarna funkcija). Pravimo ji \emph{zlepek}.
%
\subsection{Limita funkcije}
Radi bi opisali lokalno obna"sanje funkcije: imamo funkcijo, ki je definaran v okolici to"cke $a$ in ne nujno v $a$.

\deff Naj bo funkcija $f$ definirana v \emph{prebodeni okolici} to"cke $a$, t.j.: obstaja $r > 0$, da je $f$ definirana na $(a - r, a + r) \setminus \{a\}$. "Stevilo $L \in \RR$ imenujemo \emph{limita} funkcije $f$, ko gre $x$ proti $a$, "ce
\begin{equation*}
\forall \varepsilon > 0 \exists \delta > 0 \forall x \in D_f: 0 < |x - a| < \delta \Rightarrow |f(x) - L| < \varepsilon
\end{equation*}
%
\textsc{Trditev:} Naj bo $f$ definirana v okolici to"cke $a$. Funkcija $f$ je zvezna v to"cki $a$ natanko tedaj, kadar je
\begin{equation*}
\lim_{x \to a} f(x) = f(a)
\end{equation*}
to pomeni: $\lim_{x \to a} f(x)$ obstaja in je enaka $f(a)$.

\textsc{Dokaz:} napi"semo obe definiciji in je o"citno, zato je bil za DN.

\textsc{Posledica:} Naj bo funckija $f$ definirana v prebodeni okolici $D_f$ fo"cke $a$. "Ce obstaja $\lim_{x \to a} f(x) = L$, potem je funkcija
\begin{equation*}
\widetilde{f}(x) = \begin{cases}
f(x) & x \in D_f \setminus \{a\} \\
L & x = a
\end{cases}
\end{equation*}
zvezna funkcija v to"cki $a$.

\textsc{Dokaz:} $\lim_{x \to a} \widetilde{f} (x) = \lim_{x \to a} f(x) = \widetilde{f}(a)$

\textsc{Izrek:} Naj bo funkcija $f$ definirana v prebodeni okolici $\mathcal{U}$ to"cke $a$. Potem obstaja $\lim_{x \to a} f(x) = L$ natanko tedaj, kadar za vsako zaporedje $x_n \in \mathcal{U}$, ki konvergira proti $a$, zaporedje $f(x_n)$ konvergira proti $L$.

\textsc{Dokaz:} Posledica trditve (zvezna $\Rightarrow$ po izreku za zveznost).

\textsc{Izrek:} Naj bosta $f$ in $g$ definirani v prebodeni okolici to"cke $a$ in denimo, da obstaja $\lim_{x \to a} f(x)$ in $\lim_{x \to a} g(x)$. Potem obstajajo $\lim_{x \to a} (f + g)(x), \lim_{x \to a} (f - g)(x), \lim_{x \to a} (f \cdot g) (x)$ in velja:
\begin{align*}
\lim_{x \to a} (f \pm g) (x) &= \lim_{x \to a} f(x) \pm \lim_{x \to a} g(x) \\
\lim_{x \to a} (f \cdot g) (x) &= \left(\lim_{x \to a} f(x)\right) \cdot \left(\lim_{x \to a} g(x)\right)
\end{align*}
"Ce je $\lim_{x \to a} g(x) \neq 0$, potem obstaja $\lim_{x \to a} \frac{f(x)}{g(x)}$ in velja
\begin{equation*}
\lim_{x \to a} \left(\dfrac{f}{g}\right)(x) = \dfrac{\lim_{x \to a} f(x)}{\lim_{x \to a} g(x)}
\end{equation*}
\textsc{Dokaz:} Imamo zveznost po trditvi in delamo z zaporedji.

\deff Naj bo funkcija $f$ definirana na $(a - r, a)$ za nek $r > a$. Pravimo, da je $L \in \RR$ \emph{leva limita} funkcije $f$, ko gre $x$ proti $a$ "ce
\begin{equation*}
\forall \varepsilon > 0 \exists \delta > 0 \forall x \in D_f: x \in (a - \delta, a) \Rightarrow |f(x) - L| < \varepsilon
\end{equation*}
\textbf{Oznaka:}
\begin{equation*}
L = \lim_{x \nearrow a} f(x) = f(a-) = \lim_{x \uparrow a} f(x)
\end{equation*}
Podobno definiramo \emph{desno limito} in ozna"cimo 
\begin{equation*}
D = \lim_{x \searrow a} f(x) = f(a + ) = \lim_{x \downarrow a} f(x)
\end{equation*}
%
\textsc{Trditev:} Naj bo $f$ definirana v prebodeni okolici to"cke $a$. Limita funkcije $f$< ko gre $x$ proti $a$ obstaja natanko tedaj, kadar obstajata leva ind esna limita in sta enaki.

\textsc{Dokaz:} Potrebno je lepo zlo"ziti definicije, zato je bil dokaz za DN.

\textsc{Izrek:} Naj bo $f$ monotona funkcija na $[a, b]$. Potem za vsak $c \in (a, b)$ obstajata $f(c-)$ in $f(c+)$. Funkcija $f$ je zvezna v to"cki $c$ natanko tedaj, kadar $f(c-)= f(c+)$. "Ce je nara"s"cajo"ca, potem je $f(c-) \leq f(c) \leq f(c+)$> "Ce je $f$ padajo"ca, potem je $f(c+) \leq f(c) \leq f(c-)$.

\deff "Ce monotona funckija $f$ na $[a, b]$ ni zvezna v to"cki $c$, potem $f(c+) - f(c-)$ imenujemo \emph{skok} funkcije $f$ v to"cki $c$.

\textsc{Dokaz:} (izreka) Naj bo $f$ nara"s"cajo"ca, $c \in (a, b)$. Vemo
\begin{equation*}
\forall x \in [a, c]: f(x) \leq f(c)
\end{equation*}
Mno"zica $\{f(x): x \in [a, c)\}$ je navzgor omejena, zato ima supremum $S$. Dokazujemo \dashuline{$f(c-) = S$}

Izberemo $\varepsilon > 0$. Obstaja $x_0 \in [a, c)$, da je $f(x_0) > S - \varepsilon$.

Za $x \in [x_0, c)$ velja: ker je $f$ nara"s"cajo"ca: $S - \varepsilon < f(x_0) \leq f(x) \leq S$. Torej $f(x) \in (S - \varepsilon, S + \varepsilon)$. Vzamemo $\delta := c - x_0$. Podobno naredimo za desno limito in padajo"ce zaporedje.

\hfill $\square$

\textsc{Izrek:} Monotona funkcija $f$ na $[a, b]$ ima kve"cjemo "stevno mnogo to"ck nezvenznosti.

\textsc{Dokaz:} Vemo: v to"ckah nezveznosti ima funkcija monotona skok. Naj bo
\begin{equation*}
\mathcal{N} = \text{ mno"zica to"ck nezveznosti}
\end{equation*}
Velja
\begin{equation*}
\forall c \in \mathcal{N} \exists f(c-), f(c+): f(c-) \neq f(c+)
\end{equation*}
"Ce je $f$ nara"s"cajo"ca: obstaja $r_c \in (f(c-), f(c+)) \cap \QQ$. Naj bo
\begin{equation*}
g: \mathcal{N} \to \QQ \qquad g(c) = r_c
\end{equation*}
$g$ je injektivna, zato je $\mathcal{N}$ kve"cjemu "stevna.

\hfill $\square$

\deff Naj bo funkcija $f$ definirana na $(a, \infty)$, $a \in \RR$. "Stevilo $L \in \RR$ je limita funkcije $f$, ko po"sljemo $x$ "cez vse meje, "ce velja:
\begin{equation*}
\forall \varepsilon > 0 \exists M  \in \RR, M > a \forall x \in \RR: x \geq M \Rightarrow |f(x) - L| < \varepsilon
\end{equation*}
V tem primeru pi"semo
\begin{equation*}
\lim_{x \to \infty} f(x) = L
\end{equation*}
Podobno definiramo
\begin{equation*}
\lim_{x \to - \infty} f(x) = L
\end{equation*}

\deff Naj bo funkcija $f$ definirana na prebodeni okolici to"cke $a$. Pravimo, da $f$ izpolnjuje \emph{Cauchyjev pogoj} pri $a$, "ce
\begin{equation*}
\forall \varepsilon > 0 \exists \delta >0 \forall x, x' \in (a - \delta, a + \delta) \setminus \{a\}: |f(x) - f(x')| < \varepsilon
\end{equation*}

\textsc{Trditev:} Naj bo funkcija $f$ definirana na prebodeni okolici to"cke $a$. Funkcija $f$ izplonjuje Cauchyjev pogoj pri $a$ natanko takrat, kadar obstaja limita funkcije $f$, ko gre $x$ proti $a$.

\textsc{Dokaz:} Za DN.

\textbf{Opomba:} Cauchyjev pogoj ima smisel tudi pri $a = \pm \infty$ in trditev velja tudi pri $a = \pm \infty$.

\deff Naj bo funkcija $f$ definirana v prebodeni okolici to"cke $a$. Pravimo, da je limita funkcije $f$ ,ko gre $x$ proti $a$ enaka neskon"cno "ce
\begin{equation*}
\forall M \in \RR \exists \delta > 0, \forall x \in (a - \delta, a + \delta) \setminus \{a\}: f(x) > M
\end{equation*}
Ozna"cimo z $\lim_{x \to a} f(x) = \infty$

Podobno definiramo:
\begin{itemize}
	\item $\lim_{x \to a} f(x) = -\infty$
	\item $\lim_{x \uparrow a} f(x) = \pm \infty$
	\item $\lim_{x \downarrow a} f(x) = \pm \infty$
	\item $\lim_{x \to \pm \infty} f(x) = \pm \infty$
\end{itemize}
%
\textsc{Primer:}
\begin{enumerate}[1)]
	\item $\lim_{x \to 0} \frac{\sin x}{x} = 1$

	"Ce si nari"semo kro"zni izsek enotske kro"znice, lahko z malo geometrije pridemo do ocene $\frac{1}{2} \sin x \leq \frac{x}{2} \leq \frac{1}{2} \tan x$. Nato izraz nekoliko preuredimo in dobimo $\cos x \leq \frac{\sin x}{x} \leq 1$. Uporabimo lahko izrek o sendvi"cu in pridemo do kon"cnega rezultata.
	
	\item $f(x) = (1 + \frac{1}{x}) ^ x$
	
	\dashuline{$\lim_{x \to \infty} f(x) = e$}
	
	Vemo:
	\begin{gather*}
	\lim_{n \to \infty} \left(1 + \frac{1}{n}\right) ^ n = e \\
	\lim_{n \to -\infty} \left(1 - \frac{1}{n}\right) ^{-n} = e
	\end{gather*}
	Za $x \in [n, n+1)$ velja:
	\begin{equation*}
	\left(1 + \dfrac{1}{n+1}\right)^n \leq \left(1 + \dfrac{1}{x}\right)^x \leq \left(1 + \dfrac{1}{n}\right)^{n+1}
	\end{equation*}
	Po izreku o sendvi"cu torej velja $\lim_{x \to \infty} f(x) = e$
	
	\item $\lim_{h \to 0} \dfrac{a^h - 1}{h}$, $a > 0, a \neq 1$
	\begin{gather*}
	a^h = 1 + \dfrac{1}{x} \\
	\begin{aligned}
	h > 0 &: h \to 0, x \to \infty\\
	h < 0 &: h \to 0, x \to -\infty
	\end{aligned} \\
	h = \log_a \left(1 + \dfrac{1}{x}\right) \\
	\dfrac{a^h - 1}{h} = \dfrac{1}{x \log_a \left(1 + \dfrac{1}{x}\right)} = \dfrac{1}{\log_a \left(1 + \dfrac{1}{x}\right)^x} \to \dfrac{1}{\log_a e} = \ln a
	\end{gather*}
\end{enumerate}
%
\subsection{Odvod}
\deff Naj bo funkcija $f$ definirana v okolici to"cke $a$. "Ce obstaja
\begin{equation*}
\lim_{h \to 0} \dfrac{f(a + h) - f(a)}{h}
\end{equation*}
jo imenujemo \emph{odvod} funcije $f$ v to"cki $a$ in jo ozna"cimo s $f'(a)$ in pravimo, da je $f$ \emph{odvadljiva} v to"cki $a$.

\textbf{Opomba:} "Ce je $f$ odvadljiva v to"cki $a$: $f'(a) = \lim_{h \to 0} \dfrac{f(a + h) - f(a)}{h} = \lim_{x \to a} \dfrac{f(x) - f(x)}{x - a}$
\begin{gather*}
\begin{aligned}
&\Delta f = f(x) - f(a) \\
&\Delta x = x - a
\end{aligned}
\end{gather*}
razliki ali \emph{diferenca}. $x = a + h, \quad h = x-a$

Kvoceint $\dfrac{f(x) - f(x)}{x - a}$ se imenuje \emph{diferen"cni kvoecient}.

\textsc{Geometrijski pomen:} Naklonski koeficient sekante skozi $(a, f(a)), (x, f(x))$ je $\frac{f(x) - f(a)}{x-a}$. V limiti $x \to a$ dobimo naklonski koeficient tangente.

\textsc{Primer:} $f(x) = x^2 + 1$. Izra"cunaj $f'(2)$.
\begin{gather*}
f'(2) = \lim_{x \to 2} \dfrac{f(x) - f(2)}{x - 2} = \lim_{x \to 2} \dfrac{x^2 + 1 - 5}{x - 2} = \lim_{x \to 2} \dfrac{(x + 2)(x-2)}{x-2} = 4
\end{gather*}
%
\textsc{Izrek:} Naj bo funkcija $f$ definirana v okolici to"cke $a$. "Ce je $f$ \emph{odvadljiva} v to"cki $a$, potem je $f$ zvezna v to"cki $a$.

\textsc{Dokaz:} Denimo, da je $f$ odvadljiva v to"cki $a$.
\begin{multline*}
\lim_{x \to a} f(x) = \lim_{x \to a} \left(f(a) + (f(x) - f(a)) \dfrac{x-a}{x-a}\right) = \\
= \lim_{x \to a} f(a) + \lim_{x \to a} \dfrac{f(x) - f(a)}{x - 1} \lim_{x \to a} (x - 1) = \\
= f(a) + f'(a) \cdot 0 = f(a)
\end{multline*}
\textsc{Primer:}
\begin{enumerate}[1)]
	\item $f(x) = \sqrt[3]{x}$ je zvezna na $\RR$. Ali je odvadljiva v 0?
	\begin{equation*}
	\lim_{h \to 0} \dfrac{f(h) - f(0)}{h} = \lim_{h \to 0} \dfrac{\sqrt[3]{h} - 0}{h} = \lim_{h \to 0} h^{-\frac{2}{3}} = \infty
	\end{equation*}
	ni odvadljiva.
	
	\item \begin{equation*}
	f(x) = \begin{cases}
	x \sin \frac{1}{x} & x \neq 0 \\
	0 & x = 0
	\end{cases}
	\end{equation*}
	\begin{equation*}
	\lim_{h \to 0} \dfrac{f(h) - f(0)}{h} = \lim_{h \to 0} \dfrac{h \sin \dfrac{1}{h}}{h} = \lim_{h \to 0} \sin \dfrac{1}{h}
	\end{equation*}
	ne obstaja
	
	\item $g(x) = |x|$
	\begin{equation*}
	\lim_{h \to 0} \dfrac{g(h) -  g(0)}{h} = \lim_{h \to 0} \dfrac{|h|}{h}
	\end{equation*}
	ne obstaja, ker $\lim_{h \downarrow 0} \dfrac{|h|}{h} = 1$, $\lim_{h \uparrow 0} \dfrac{|h|}{h} = -1$
\end{enumerate}
%
\deff Naj bo funcija $f$ definirana na $[a, a + r)$ za nek $r > 0$. "Ce obstaja $\lim_{x \downarrow a} \dfrac{f(x) - f(a)}{x - a}$, jo imenujemo \emph{desni odvod} funcije $f$ v to"cki $a$. Podobno definiramo \emph{levi odvod}.

\textsc{Trditev:} Naj bo funkcija $f$ definirana v okolici to"cke $a$. Funcija $f$ je odvavljiva v to"cki $a$ natanko tedaj, kadar obstaja levi in desni odvod funcije $f$ v to"cki $a$ in sta enaka.

\deff Pravimo, da je funcija $f$ odvadljiva na $(a, b)$, "ce je $f$ odvadljiva v vsaki to"cki iz $(a, b)$.

Funkcija $f$ je odvadljiva na $[a, b]$, "ce je $f$ odvadljiva na $(a, b)$, v kraji"s"cu $a$ ima desni odvod in v kraji"s"cu $b$ ima levi odvod.

\deff Naj bo funkcija $f$ definirana na intervalu $I$ in naj bo $f$ odvedljiva vsaj v kak"sni to"cki iz $I$. Ozna"cimo z
\begin{equation*}
I' = \{x \in I: f \text{ je odvedljiva v to"cki $x$}\}
\end{equation*}
Funkcijo $f': I' \to \RR$ in je definirana s predpisom $x \mapsto f'(x)$ imenujemo \emph{odvod} funkcije $f$.

\deff Naj bo funkcija $f$ definirana na intervalu $I$.

Pravimo, da je $f$ \emph{zvezno odvedljiva} na $I$, "ce je $f$ odvedljiva na $I$ in je njen odvod $f'$ zvezna funkcija na $I$.

\deff Naj bo $f: [a, b] \to \RR$ zvezna funkcija. Pravimo, da je $f$ \emph{odsekoma zvezno odvedljiva}, "ce je odvedljiva povsod, razen v kon"cno mnogo to"ckah $c_1, \ldots, c_k$, $a < c_1 < \cdots < c_k < b$ v katerih ima $f$ levi in desni odvod, na vsakem podintervalu $[a, c_1], [c_1, c_2], \ldots, [c_k, b]$, pa je $f$ zvezno odvedljiva.

Primer: $f(x) = |x|$./
%
\subsection{Diferencial}
Naj bo $f$ odvadljiva v to"cki $a$
\begin{equation*}
f'(a) = \lim_{h \to 0} \dfrac{f(a + h) - f(a)}{h}
\end{equation*}
Velja:
\begin{equation*}
\lim_{h \to 0} \dfrac{f(a + h) - f(a) - h f'(a)}{h} = 0
\end{equation*}
Ozna"cimo "stevec $o(h) = f(a+h) - f(a) - f'(a) h$, kjer je $\lim_{h \to 0} \dfrac{o(h)}{h} = 0$.

Velja:
\begin{equation*}
f(a+h) - f(a) = f'(a)h + o(h)
\end{equation*}
Izpeljali smo, da je odvadljivo funkcijo v to"cki $a$ v okolici to"cke $a$ mogo"ce ,,dobro'' aproksimirati z ,,linearno'' funkcijo.

\deff Naj bo funkcija $f$ definirana v okolici to"cke $a$. Pravimo, da je $f$ \emph{diferenciabilna} v to"cki $a$, "ce obstaja linearna preslikava $\LL: \RR \to \RR$, za katero velja
\begin{equation*}
f(a+h) - f(a) = \LL(h) + o(h)
\end{equation*}
kjer je $\lim_{h \to 0} \dfrac{o(h)}{h} = 0$. Linearno preslikavo $\LL$ imenujemo \emph{diferencial} funkcije $f$ v $a$ in jo ozna"cimo $df(a)$.

\textbf{Opombe:}
\begin{enumerate}
	\item "Ce je $f$ odvedljiva v to"cki $a$, potem je $f$ diferenciabilna v to"cki $a$ in $\LL(h) = f'(a)h$.
	\item Vse linearne preslikave $\LL: \RR \to \RR$ so oblike $h \mapsto kh, k \in \RR$.
\end{enumerate}
%
\textsc{Izrek:} Naj bo funkcija $f$ definirana v okolici to"cke $a$. Potem je $f$ diferenciabilna v to"cki $a$ natanko tedaj, kadar je $f$ odvedljiva v to"cki $a$. V tem primeru velja $df(a)h = f'(a)h$.

\textsc{Dokaz:}
\begin{itemize}
	\item[($\Leftarrow$)] "ze vemo
	\item[($\Rightarrow$)] Po predpostavki obstaja $\LL: \RR \to \RR$ linearna preslikava, da je
	\begin{equation*}
	f(a+h) - f(a) = \LL(h) + o(h)
	\end{equation*}
	kjer $\lim_{h \to 0} \dfrac{o(h)}{h} = 0$.
	\begin{equation*}
	0 = \lim_{h \to 0} \dfrac{o(h)}{h} = \lim_{h \to 0} \dfrac{f(a+h) - f(a) - \LL(h)}{h}
	\end{equation*}
	Ker je $\LL$ linearna, obstaja $a \in \RR: \LL(h) = ah, \quad \forall h \in \RR$.
	\begin{multline*}
	\lim_{h \to 0} \dfrac{f(a+h) - f(a) - \LL(h)}{h} = \lim_{h \to 0} \dfrac{f(a+h) - f(a) - ah}{h} =\\
	= \lim_{h \to 0} \dfrac{f(a+h) - f(a)}{h} - a
	\end{multline*}
	Po pravilih za ra"cunanje z limitami torej obstaja $\lim_{h \to 0} \dfrac{f(a +h) - f(a)}{h}$ in je enaka $a$.
\end{itemize}
\textsc{Zapis:} Naj bo $f$ odvedljiva v $x \quad h = \Delta x$.
\begin{equation*}
f(x + \Delta x) - f(x) = f'(x) \Delta x + o(\Delta x)
\end{equation*}
kjer je $\lim_{\Delta x \to 0} \dfrac{o(\Delta x)}{\Delta x} = 0$

Vemo $df(x) \Delta x = f'(x) \Delta x$. Vzamemo $f(x)= x \quad dx = \Delta x$. $df = f' dx \quad f'(x) = \frac{df}{dx} (x)$.
%
\subsection{Pravila za odvajanje}
\begin{enumerate}
\item
$f(x) = c$
\begin{equation*}
f'(x) = \lim_{h \to 0} \dfrac{f(x + h) - f(x)}{h} = \lim_{h \to 0} \dfrac{c - c}{h} = 0
\end{equation*}
\item 
"Ce sta $f$ in $g$ odvedljivi funkciji v to"cki $a$, potem so tudi $f + g$, $f - g$, $f \cdot g$ odvedljive v to"cki $a$ in velja
\begin{gather*}
\begin{aligned}
(f \pm g)' (a) &= f'(a) \pm g'(a) \\
(f g)'(a) &= f'(a) g(a) + f(a)g'(a)
\end{aligned}
\end{gather*}
"Ce je $g(a) \neq 0$, potem je $\frac{f}{g}$ odvedljiva v $a$ in velja
\begin{equation*}
\left(\dfrac{f}{g}\right)'(a) = \dfrac{f'(a) g(a) - f(a)g'(a)}{g^2(a)}
\end{equation*}
\textsc{Dokaz:}
\begin{multline*}
(fg)'(a) = \lim_{h \to 0} \dfrac{(fg)(a + h) - (fg)(a)}{h} = \\
= \lim_{h \to a} \dfrac{f(a + h)g(a+h) - f(a)g(a) - f(a)g(a+h) + f(a)g(a+h)}{h} = \\
= \lim_{h \to 0} \dfrac{f(a+h)g(a+h)-f(a)g(a+h)}{h} + \lim_{h \to a}\dfrac{f(a)g(a+h) - f(a)g(a)}{h} = \\
=\lim_{h \to 0} g(a+h) \lim_{h \to 0} \dfrac{f(a+h)-f(a)}{h} + f(a)\lim_{h \to 0}\dfrac{g(a+h) - g(a)}{h} = \\
= g(a)f'(a) + f(a)g'(a)
\end{multline*}
\textsc{Posledica:} "Ce je $f$ odvedljiva v to"cki $a$ in $\lambda \in \RR$, potem velja
\begin{equation*}
(\lambda f)'(a) = \lambda f'(a)
\end{equation*}
Dokaz z uporabo formule za produkt in $g(x) = \lambda$.

\textsc{Posledica:} "Ce so funkcije $f_1, \ldots, f_n$ odvedljive v to"cki $a$, potem velja
\begin{equation*}
(f_1 f_2 \cdots f_n') (a) = f_1'(a) f_2(a)\ldots f_n(a) + f_1(a)f_2'(a) f_3(a)\ldots f_n(a) + \cdots + f_1(a) \ldots f_n'(a)
\end{equation*}
Dokaz z indukcijo in formulo za produkt.
\item Odvod kompozituma

Naj bo funkcija $f$ odvedljiva v to"cki $a$ in naj bo $g$ odvedljiva v to"cki $f(a)$. Potem je $g \circ f$ odvedljiva v to"cki $a$ in velja
\begin{equation*}
(g \circ f)'(a) = g'(f(a)) \cdot f'(a)
\end{equation*}
\textsc{Dokaz:} Ker je $f$ odvedljiva v to"cki $a$, je $f$ diferenciabilna v to"cki $a$ in velja:
\begin{equation}
f(a+h) = f(a) + hf'(a) + o_f(h) \quad \text{kjer } \lim_{h \to 0} \dfrac{o_f(h)}{h} = 0
\end{equation}
Podobno za $g$:
\begin{equation}
g(f(a) + k) = g(f(a)) + k g'(f(a)) + o_g(k) \quad \text{kjer } \lim_{k \to 0} \dfrac{o_g(k)}{k} = 0
\end{equation}
Po definiciji velja:
\begin{equation*}
(g \circ f)'(a) = \lim_{h \to 0} \dfrac{g(f(a+h)) - g(f(a))}{h}
\end{equation*}
"Ce razpi"semo z zgornjima dvema en"cabama dobimo:
\begin{multline*}
g(f(a+h)) \stackrel{(4)}{=} g(f(a) + \underbrace{hf'(a) + o_f(h)}_{k}) \stackrel{(5)}{=} \\
= g(f(a)) + (hf'(a) + o_f(h)) + g'(f(a)) + o_g(k(h)) = \\
= g(f(a)) + hg'(f(a)) \cdot f'(a) + o_f(h) g'(f(a)) + o_g(k(h))
\end{multline*}
Sledi:
\begin{equation*}
\dfrac{g(f(a+h)) - g(f(a))}{h} = g'(f(a))f'(a) + \underbrace{\dfrac{o_f(h)}{h} g'(f(a))}_{\stackrel{h \to 0}{\longrightarrow} 0} + \dfrac{o_g(k(h))}{h}
\end{equation*}
%
\begin{equation*}
\lim_{h \to 0} \dfrac{o_g(k(h))}{h} = \lim_{k \to 0} \underbrace{\dfrac{o_g(k(h))}{k(h)}}_{\to 0} \underbrace{\dfrac{k(h)}{h}}_{\to f'(a)} = 0 \\
\end{equation*}
ker:
\begin{gather*}
\lim_{h \to 0} k(h) \lim_{h \to 0} \left(hf'(a) + \dfrac{o_f(h) h}{h}\right) = 0 \\
\lim_{h \to 0} \dfrac{k(h)}{h} = \lim_{h \to 0} \dfrac{hf'(a) + o_f(h)}{h} = \lim_{h \to 0} \left(f'(a) + \dfrac{o_f(h)}{h}\right) = f'(a)
\end{gather*}
\begin{equation*}
\Rightarrow \lim_{h \to 0} \dfrac{g(f(a+h)) - g(f(a))}{h} = g'(f(a)) \cdot f'(a) = (g \circ f)'(a)
\end{equation*}
\hfill $\square$
\end{enumerate}
%
\textsc{Izrek:} Naj bo $f$ zvezna, strogo monotona na $[a, b]$, $c \in (a, b)$ in denimo, da je $f$ odvedljiva v to"cki $c$. "Ce je $f'(c) \neq 0$, potem je inverzna funkcija $f^{-1}$ odvedljiva v to"cki $f(c) = d$ in velja:
\begin{equation*}
(f^{-1})'(d) = \dfrac{1}{f'(c)} = \dfrac{1}{f'(f^{-1}(d))}
\end{equation*}
\textsc{Dokaz:} Za strogo nara"s"cajo"ce:

$f^{-1}$ je zvezna na $[f(a), f(b)]$
\begin{multline*}
\lim_{y \to d} \dfrac{\overbrace{f^{-1}(y)}^x - \overbrace{f^{-1}(d)}^c}{y-d} = \text{ker je $f^{-1}$ zvezna, gre $x \to c$, ko gre $y \to d$}\\
= \lim_{x \to c} \dfrac{x - c}{f(x) - f(c)} = \lim_{x \to c} \dfrac{1}{\dfrac{f(x) - f(x)}{x - c}} = \\
= \dfrac{1}{f'(c)}
\end{multline*}
\textbf{Geometrijski pomen:} graf $f^{-1}$ dobimo z zrcaljenjem grafa $f$ preko simetrale lihih kvadrantov. "Ce prezrcalimo premico preko simetrale lihih kvadrantov, je koeficient zrcaljenje premice obratna vrednost koeficienta dane premice.

\textbf{Opomba:} "Ce vemo, da je $f^{-1}$ odveljiva, potem izpeljano formulo dobimo tudi takole:
\begin{gather*}
f^{-1}(f(x)) = x \\
(f^{-1})'(f(x)) f'(x) = 1 \\
(f^{-1})'(\underbrace{f(x)}_y) = \dfrac{1}{x} \\
(f^{-1})'(y) = \dfrac{1}{f'(x)} = \dfrac{1}{f'(f^{-1}(y))}
\end{gather*}
%
\subsection{Odvodi elementarnih funkcij}
\begin{table}[!htbp]
	\centering
	\begin{tabular}{c|c|c}
		$f$ & $f'$ & Dokaz \\ \hline
		$c$ & $0$ & po definiciji \\
		$x$ & $1$ & po definiciji \\
		$x^n$ & $n x^{n-1}$ & \hyperref[proof:1]{(1)}\\
		$\log_a x$ & $\frac{1}{\ln a} \frac{1}{x}$ &  \hyperref[proof:2]{(2)} \\
		$\ln x$ & $\frac{1}{x}$ & $(\log_a x)'$ s prehodom na novo osnovo\\
		$a^x$ & $a^x \ln a$ &  \hyperref[proof:3]{(3)} \\
		$e^x$ & $e^x$ & Uporabimo pravilo za $a^x$ \\
		$\sin x$ & $\cos x$ & \hyperref[proof:4]{(4)} \\
		$\cos x$ & $- \sin x$ & $\cos x = \sin(\frac{\pi}{2} - x)$, odvajamo kompozitum\\
		$\tan x$ & $\frac{1}{\cos^2 x}$ & $\tan x = \frac{\sin x}{\cos x}$, odvajamo ulomek\\
		$\arcsin x$ & $\frac{1}{\sqrt{1 - x^2}}$ & \hyperref[proof:5]{(5)} \\
		$\arccos x$ & $- \frac{1}{\sqrt{1 - x^2}}$ & Uporabimo zvezo $\arcsin x + \arccos x = \frac{\pi}{2}$\\
		$\arctan x$ & $\frac{1}{1 + x^2}$ & \hyperref[proof:6]{(6)} 
	\end{tabular}
	\caption{Tabela odvodov}
\end{table}

\begin{itemize}
\item $f(x) = x^n$
\label{proof:1}
Za $n \in \NN$:
\begin{equation*}
f'(x) = (xx\ldots x)' = 1 x^{n-1} + x 1 x^{n-1} + xx1x^{n-3} + \cdots + x^{n-1}1 = nx^{n-1}
\end{equation*}
Za cela "stevila nastavimo $f(x) = x^{-n}, \quad n \in \NN$
\begin{equation*}
(x^{-n})' = \left(\dfrac{1}{x^n}\right)' = \dfrac{-nx^{n-1} 1}{x^{2n}} = -nx^{-n-1}
\end{equation*}
Za racionaln "stevila si najprej poglejmo $f(x) = x^{\frac{1}{n}}, n \in \NN$. "Ce je $n$ sod, je odvedljiva na $(0, \infty)$, "ce je $n$ lih, je odvedljiva na $\RR \setminus \{0\}$
\begin{gather*}
(f(x))^n = x \\
n (f(x))^{n-1} f'(x) = 1\\
f'(x) = \dfrac{1}{n (x^{\frac{1}{n}})^{n-1}}  = \dfrac{1}{n} x^{- \frac{n - 1}{n}} = \frac{1}{n} x^{\frac{1}{n} - 1}
\end{gather*}
%
$f(x) = x^{\frac{m}{n}}, \quad m \in \ZZ, n \in \NN$ lahko zapi"semo kot:
\begin{multline*}
f'(x) = \left(\left(x^{\frac{1}{n}}\right)^m\right)' = \\
= m \left(x ^ {\frac{1}{n}}\right)^{m-1} \cdot \dfrac{1}{n} x^{\frac{1}{n}-1} = \dfrac{m}{n} x^{\frac{m}{n} - \frac{1}{n} + \frac{1}{n} - 1} = \\
=\dfrac{m}{n} x^{\frac{m}{n}-1}
\end{multline*}
"Ce poznamo odvod poten"cne funkcije, lahko naredimo "se za realne eksponente. Naj bo $h(x) = x^r, \quad r \in \RR, x > 0$:
\begin{gather*}
x^r = e^{\ln x^r} = e^{r \ln x} \\
(x^r)' = (e ^{r \ln x})' = e^{r \ln x} r \dfrac{1}{x} = x^r \cdot r \cdot \dfrac{1}{x} = r \cdot x^{r-1}
\end{gather*}

\item $\log_a x$
\label{proof:2}
\begin{multline*}
(\log_a x)' = \lim_{h \to 0} \dfrac{\log_a (x+h) - \log_a x}{h} = \\
= \lim_{h \to 0} \dfrac{\log \dfrac{x + h}{x}}{h} = \lim_{h \to 0} \log_a \left(\dfrac{x+h}{x}\right)^\frac{1}{h} = \log_a \lim_{h \to 0} \left(\left(1+ \dfrac{h}{x}\right)^{\frac{x}{h}}\right)^{\frac{1}{x}} = \\
= \log_a e^\frac{1}{x} = \frac{1}{x} \log_a e = \dfrac{1}{x} \dfrac{\ln e}{\ln a} = \\
= \dfrac{1}{x \ln a}
\end{multline*}

\item $a^x$ je inverzna funkcija strogo monotone funkcije $\log_a x$, ki ima povsod neni"celen odvod, zato je odvedljiva.
\label{proof:3}
\begin{gather*}
\log_a (a^x) = x \\
\dfrac{1}{\ln a}\dfrac{1}{x^x} (a^x)' = 1 \\
(a^x)' = (\ln a) a^x = a^x \ln a
\end{gather*}

\item $f(x) = \sin x$
\label{proof:4}
\begin{multline*}
f'(x) = \lim_{h \to 0} \dfrac{\sin (x+h) - \sin x}{h} = \\
= \lim_{h \to 0} \dfrac{2\sin \frac{h}{2}\cos(x + \frac{h}{2})}{h} = \lim_{h \to 0} \dfrac{\sin \frac{h}{2}\cos(x + \frac{h}{2})}{\frac{h}{2}} = \\
\cos x
\end{multline*}

\item 
\label{proof:5}
$f(x) = \arcsin x$ je inverzna funkcija $\sin x|_{[- \frac{\pi}{2}, \frac{\pi}{2}]}$, ki je strogo monotona, odvedljiva z neni"celnim odvodom na $(- \frac{\pi}{2}, \frac{\pi}{2})$. Zato je $\arcsin$ odvedljiva na $(-1, 1)$.
\begin{gather*}
\sin (\arcsin x) = x \\
\cos ( \arcsin x) (\arcsin x)' = 1 \\
(\arcsin x)' = \frac{1}{\cos (\arcsin x)} = \dfrac{1}{\sqrt{1 - \sin^2 (\arcsin x)}} = \dfrac{1}{\sqrt{1 - x^2}}
\end{gather*}

\item 
\label{proof:6}
$\arctan x$ je inverz zo"itve strogo nara"s"cajo"ce odveljive funkcije $\tan x$ z neni"celnim odvodm, zato je odvedljiva.
\begin{gather*}
\tan(\arctan x) = x \\
\dfrac{1}{\cos^(\arctan x)} (\arctan x)' = 1 \\
(\arctan x)' = \cos^x (\arctan x) = \dfrac{1}{1 + \tan^2 (\arctan x)} = \dfrac{1}{1 + x^2}
\end{gather*}
\end{itemize}
%
\subsection{Odvodi vi"sjega reda}
Denimo, da je $f$ odvedljiva funkcija na intervalu $I$. Potem za $x \in I$ poznamo $f'(x)$, kar nam definira funkcijo:
\begin{align*}
f': I &\to \RR \\
x &\mapsto f'(x)
\end{align*}
"Ce je funkcija $f'$ odvedljiva na $I$, potem njen odvod ozna"cimo z $f''$ in imenujemo \emph{drugi odvod} funkcije $f$ na $I$.

Vi"sje odvode definiramo induktivno. Torej, $n$-ti odvod funkicje $f$ je odvod $(n-1)$-tega odvoda
\begin{equation*}
f^{(n)} = (f^{(n-1)})', \quad \forall n \in \NN
\end{equation*}
%
\textsc{Oznake:} Naj bo $I$ (zaprti) interval:
\begin{itemize}
	\item $C(I) = C^0(I)$ mno"zica vseh zveznih funkcij na $I$.
	\item $C^1(I)$ mno"zica vseh zvezno odveljivih funkcij na $I$, to so funkcije, ki so odveljive na $I$ in je njihov odvod $f'$ zvezna funkcija na $I$.
	\item $C^r(I)$ mno"zica vseh $r$-krat zvezno odvedljivih funkcij na $I$, to so funkcije, ko so $r$-krat odvedljive in je $f^{(r)}$ zvezna funkcija na $I$.
	\item $C^\infty(I) = \bigcap_{r \in \NN}C^r(I)$ neskon"cnokrat odvedljive funkcije
\end{itemize}
\textbf{Vemo:} $f, g \in C^r(I)$, potem $f+g, \lambda f \in C^r(I)$ za vsak $\lambda \in \RR.$ Torej je $C^r(I)$ \emph{vektorski prostor}.

Oglejmo si preslikavo:
\begin{align*}
D: f &\mapsto f' \\
C^r(I) &\to C^{r-1}(I)
\end{align*}
%
\textsc{Trditev:} $D$ je linearna preslikava

\textsc{Dokaz:} Doma. Ni te"zak, "se posebej "ce si v zadnjem mesecu bil/a prisoten na algebri.

\textsc{Oznaka:}
\begin{align*}
D^2 = \dfrac{d^2}{dx^2} && D^k = \dfrac{d^k}{dx^k}
\end{align*}
%
\textsc{Primeri:}
\begin{enumerate}
\item $f(x) = x \sin \dfrac{1}{x}$ na $\RR \setminus \{0\}$
\begin{gather*}
\lim_{x \to 0} f(x) = \lim_{x \to 0} x \sin \dfrac{1}{x} = 0 \\
%
\widetilde{f}(x) = \begin{cases}
f(x) & x \neq 0 \\
0 & x = 0
\end{cases} \qquad \widetilde{f} \in C(\RR)
\end{gather*}
Za $x \neq 0$:
\begin{equation*}
f'(x) = \sin \dfrac{1}{x} - \dfrac{1}{x} \cos \dfrac{1}{x}
\end{equation*}
Za $x = 0$:
\begin{equation*}
\widetilde{f}'(0) = \lim_{h \to 0} \dfrac{\widetilde{f}(h) - \widetilde{f}(0)}{h} = \lim_{h \to 0} \dfrac{h \sin \dfrac{1}{h}}{h}
\end{equation*}
ne obstaja.

$\Rightarrow \widetilde{f}$ v to"cki 0 ni odvedljiva. torej $\widetilde{f} \in C^1(I)$, "ce $0 \notin I$, sicer $\widetilde{f} \notin C^1(I)$.

\item $g(x) = x^2 \sin\dfrac{1}{x}$.

Za $x \neq 0$:
\begin{equation*}
g'(x) = 2x \sin \dfrac{1}{x} - \cos \dfrac{1}{x}
\end{equation*}
Za $x = 0$:
\begin{equation*}
g'(0) = \lim_{h \to 0} \dfrac{h^2 \sin \dfrac{1}{h}}{h} = 0
\end{equation*}
Torej:
\begin{equation*}
g'(x) = \begin{cases}
2x \sin \dfrac{1}{x} - \cos \dfrac{1}{x} & x \neq 0 \\
0 & x = 0
\end{cases}
\end{equation*}
$g'$ je gotovo zvezna na $\RR \setminus \{0\}$. $g'$ je zvezna v 0 $\iff \lim_{x \to 0} g'(x) = g'(0)$.
\begin{equation*}
\lim_{x \to 0} g'(x) = \lim_{x \to 0} (\underbrace{2x \sin \dfrac{1}{x}}_{\to 0} - \underbrace{\cos \dfrac{1}{x}}_{\text{oscilira}}) \quad \text{ne obstaja}
\end{equation*}
$g'$ ni zvezna v 0. Torej je $g$ odvedljiva na $\RR$, njen odvod $g'$ pa ni zvezna funkcija na $\RR$.
\end{enumerate}
%
\subsection{Rollov in Lagrageev izrek}
\deff Naj bo $S \subset \RR$ in naj bo $f: S \to \RR$ funkcija. Pravimo, da ima $f$ v to"cki $c \in S$ \emph{lokalni maksimum}, "ce:
\begin{equation*}
\exists \delta > 0: \forall x \in S: |x - c| < \delta \Rightarrow f(x) \leq f(c)
\end{equation*}
Funkcija $f$ ima v to"cki $c \in S$ \emph{lokalni minimum}, "ce:
\begin{equation*}
\exists \delta > 0: \forall x \in S: |x - c| < \delta \Rightarrow f(x) \geq f(c)
\end{equation*}
\emph{Lokalni ekstrem} je skupno ime za lokalni minimum in lokalni maksimum.

\textsc{Trditev:} Naj bo $f: [a, b] \to \RR$ odvedljiva v to"cki $c \in (a, b)$. "Ce je $c$ lokalni ekstrem od $f$, potem velja
\begin{equation*}
f'(c) = 0
\end{equation*}
\textsc{Dokaz:} Recimo, da je $c$ lokalni maksimum. Poglejmo si sekante skozi $(c, f(c))$. Vzemimo $\delta$ iz definicije lokalnega maksimuma. Potem velja
\begin{equation*}
|x - c| < \delta \Rightarrow f(x) \leq f(c)
\end{equation*}
Lo"cimo primera:
\begin{itemize}
	\item $x > c$
	\begin{equation*}
	\dfrac{f(x) - f(c)}{x - c} \leq 0 \tag{*}
	\end{equation*}
	
	\item $x < c$
	\begin{equation*}
	\dfrac{f(x) - f(c)}{x - c} \geq 0 \tag{**}
	\end{equation*}
\end{itemize}
Poglejmo si odvod:
\begin{equation*}
f'(c) = \lim_{x \to c} \dfrac{f(x) - f(c)}{x - c}
\end{equation*}
Iz (*) sledi:
\begin{equation*}
\lim_{x \downarrow c} \dfrac{f(x) - f(c)}{x - c} \leq 0
\end{equation*}
Iz (**) sledi:
\begin{equation*}
\lim_{x \uparrow c} \dfrac{f(x) - f(c)}{x - c} \geq 0
\end{equation*}
Torej velja $f'(c) = 0$.

\deff Naj bo funkcija $f$ odvedljiva na intervalu $I$. "Ce je za nek $c \in I$ $f'(c) = 0$, imenujemo $c$ \emph{stacionarna to"cka} funckija $f$.

Trditev pove, da so lokalni ekstremi stacionarne to"cke funkcije $f$.

\textsc{Rollov izrek:} Naj bo $f: [a, b] \to \RR$ zvezna funkcija, odvedljiva na $(a, b)$ in $f(a) = f(b)$. Potem obstaja $c \in (a, b)$, da velja $f'(c) = 0$.

\textsc{Dokaz:} Ker je $f$ zvezna na $[a, b]$ dose"ze minimum in maksimum. Ozna"cimo z $x_m \in [a, b]$ to"cke, v katerih dose"ze minimum in z $x_M \in [a, b]$ to"cke v katerih dose"ze maksimum.
\begin{itemize}
	\item "Ce je $f(x_m) = f(x_M)$, potem je $f$ konstanta, zato je $f' \equiv 0$.
	
	\item "Ce je $f(x_m) \neq f(x_M)$:
	
	Ker je $f(a) = f(b)$, vsaj ena od to"ck $x_m, x_M$ ni kraji"s"ce intervala $[a, b]$. Ozna"cimo jo s $c$. V $c$ je dose"zen lokalni ekstrem, zato je po prej"snji trditvi $f'(c) = 0$.
\end{itemize}
%
\textsc{Lagrangeev izrek:} Naj bo $f: [a, b] \to \RR$ zvezna funkcija, odvedljiva na $(a, b)$. Potem obstaja tak $c \in (a, b)$, da velja
\begin{equation*}
f(b) - f(a) = f'(c) (b-a)
\end{equation*}
\textbf{Opomba:} $f'(c) = \dfrac{f(b) - f(a)}{b - a}$

\textsc{Dokaz:} Definirajmo $F(x) = f(x) - f(a) + d(x - a)$ Funkcija $F$ je zvezna na $[a, b]$, odvedljiva na $(a, b)$.

$d$ dolo"cimo tako, da bo veljalo $F(a) = F(b)$
\begin{gather*}
F(a) = 0, \quad F(b) = f(b) - f(a) + d(b - a) \\
d = - \dfrac{f(b) - f(a)}{b - a}
\end{gather*}
Funkcija $F$ ustreza pogojem Rollovega izreka, zato obstaja $c \in (a, b): F'(c) = 0$.
\begin{gather*}
F'(x) = f'(x) + d \\
f'(c) + d = 0 \\
 f'(c) = -d = \dfrac{f(b) - f(a)}{b - a}
\end{gather*}
\hfill $\square$

\textbf{Opomba:} $a = x, \quad b = x+h$. "Ce $c \in (a, b)$, potem obstaja $\vartheta \in (0, 1)$, tako da $c = x + \vartheta h$. V tem primerju velja
\begin{equation*}
f(x + h) - f(x) = f'(x + \vartheta h) h
\end{equation*}
za nek $\vartheta \in (0, 1)$

\textsc{Posledica:} Naj bo $f$ odvedljiva funkcija na intervalu $I$. Tedaj velja:
\begin{enumerate}[(i)]
	\item $f'(x) \geq 0 \quad \forall x \in I \iff f$ nara"s"cajo"ca na $I$
	\item $f'(x) \leq 0 \quad \forall x \in I \iff f$ padajo"ca na $I$
	\item $f'(x) > 0 \quad \forall x \in I \Rightarrow f$ strogo nara"s"cajo"ca na $I$
	\item $f'(x) < 0 \quad \forall x \in I \Rightarrow f$ strogo padajo"ca na $I$
\end{enumerate}
\textbf{Opomba:} $f(x) = x^3$ strogo nara"scajo"ca na $\RR$, vendar $f'(0) = 0$.

\textsc{Dokaz:}
\begin{enumerate}[(i)]
\item
\begin{itemize}
	\item[$(\Rightarrow)$] denimo, da $f'(x) \geq 0 \quad \forall x \in I$. Dokazujemo \dashuline{$f$ je nara"s"cajo"ca na $I$}
	
	Izberemo $a, b \in I, a < b$ in dokazujemo $f(a) \leq f(b)$. Ker je $f$ zvezna na $[a, b]$ in odvedljiva na $(a, b)$, izpolnjuje pogoje za Lagrangeev izrek. Torej velja
	\begin{equation*}
	\exists c \in (a, b): f(b) - f(a) = \underbrace{f'(c)}_{\geq 0} \underbrace{(b - a)}_{> 0} \Rightarrow f(b) \geq f(a)
	\end{equation*}
	
	\item[$(\Leftarrow)$] Denimo, da je $f$ nara"s"cajo"ca na $I$.
	\begin{equation*}
	x \in I: f'(x) = \lim_{h \to 0} \dfrac{f(x+h) - f(x)}{h}
	\end{equation*}
	"Ce je $x$ notranja to"cka, lo"cimo primera:
	\begin{itemize}
		\item[$h>0$] in dovolj majhen, da $x + h \in I$:
		\begin{equation*}
		f(x+h) - f(x) \geq 0 \Rightarrow \dfrac{f(x+h) - f(x)}{h} \geq 0
		\end{equation*}
		
		\item[$h < 0$] in dovolj majhen, da $x + h \in I$:
		\begin{equation*}
		f(x+h) - f(x) \leq 0 \Rightarrow\dfrac{f(x+h) - f(x)}{h} \geq 0
		\end{equation*}
	\end{itemize}
	$\Rightarrow f'(x) \geq 0$.
	\hfill $\square$
\end{itemize}
\end{enumerate}
Ostalo se doka"ze po istem kopitu, tako da je za DN.

\textsc{Posledica:} "Ce za odvedljivo funkcijo $f$ na intervalu $I$ velja $f'(x) = 0 \quad \forall x \in I$, potem je $f$ konstanta.

\textbf{Opombe:}
\begin{itemize}
	\item Vemo "ze, da je $(c)' = 0$
	\item \begin{gather*}
		f(x) = \begin{cases}
		1 & x \in [0, 1] \\
		2 & x \in [2, 3]
		\end{cases}
	\end{gather*}
	$f'(x) = 0$ za vse $x \in [0, 1] \cup [2, 3]$
\end{itemize}
%
\textsc{Posledica:} Naj bo $f: (a, b) \to \RR$ zvezvna funkcija $c \in (a, b)$ in naj bo $f$ odvedljiva na $(a, b) \cup (c, b)$.
\begin{enumerate}[(i)]
	\item "Ce je $f'(x) \leq 0$ za vse $x \in (a, b)$ in $f'(x) \geq 0$ za vse $x \in (c, b)$, potem ima $f$ v to"cki $c$ minimum.
	\item "Ce je $f'(x) \geq 0$ za vse $x \in (a, b)$ in $f'(x) \leq 0$ za vse $x \in (c, b)$, potem ima $f$ v to"cki $c$ maksimum.
\end{enumerate}
\textsc{Dokaz:}
\begin{enumerate}[(i)]
	\item Vemo, da je $f$ padajo"ca na $(a, c)$ in nara"s"cajo"ca na $(c, b)$. $f$ je zvezna v $c$. Izberimo poljubno nara"s"cajo"ce zaporedje $y_n \uparrow c$.
	
	"Ce $y_n \geq x$ za vse $n: f(x) \geq f(y_n)$. 
	
	Sledi: $f(x) \geq f(c)$ velja $\forall x < c$.	
	
	Simetri"cno na desni: $x > c \Rightarrow f(x) \geq f(c)$
	
	$\Rightarrow f$ ima v $c$ minimum.
	
	\item Podobno
	
	\hfill $\square$
\end{enumerate}
\textsc{Posledica:} Naj bo $f: (a, b) \to \RR$ odvedljiva funkcija in $c \in (a, b)$ stacionarna to"cka funkcije $f$.
\begin{enumerate}[(i)]
	\item "Ce obstaja $\delta > 0$ za katerega velja
	\begin{gather*}
	\begin{aligned}
		f'(x) \leq 0& \quad \forall x \in (c - \delta, c) \quad \land \\
		f'(x) \geq 0& \quad \forall x \in (c, c + \delta)
	\end{aligned}
	\end{gather*}
	potem ima $f$ v $c$ lokalni minimum.
	
	\item "Ce obstaja $\delta > 0$, za katerega velja
	\begin{gather*}
	\begin{aligned}
		f'(x) \geq 0& \quad \forall x \in (c - \delta, c) \quad \land \\
		f'(x) \leq 0& \quad \forall x \in (c, c + \delta)
	\end{aligned}
	\end{gather*}
	potem ime $f$ v $c$ lokalni maksimum.
	
	\item "Ce obstaja $\delta > 0$, da velja bodisi $f'(x) > 0 \quad \forall x \in (c - \delta, c + \delta) \setminus \{c\}$, bodisi $f'(x) < 0 \quad \forall x \in (c - \delta, c + \delta) \setminus \{c\}$, potem $f'$ v to"cki $c$ nima lokalnega ekstrema.
\end{enumerate}
%
\subsection{Iskanje ekstremov odvedljive funkcije na $[a, b]$}
\begin{itemize}
	\item Dolo"cimo stacionarne to"cke $x_1, x_2, \ldots, x_k$
	\item Izra"cunamo vrednosti $f(x_1), f(x_2), \ldots, f(x_k)$ in $f(a), f(b)$
	\item Dolo"cimo najmanj"so in najve"cjo vrednost
\end{itemize}
"Ce funkcija $f$ ni povsod odvedljiva na $[a, b]$, potem med kandidate za ekstrem dodaamo "se to"cke v katerih $f$ ni odvedljiva.

Pri obravnavanju lokalnih ekstremov analiziramo predznak odvoda.

\textsc{Primer:} Dolo"ci lokalne ektreme
\begin{gather*}
f(x) = \dfrac{x^5}{5} - \dfrac{x^4}{4} - \dfrac{x^3}{3} + \dfrac{x^2}{2} + 1 \\
f'(x) = x^4 - x^3 -x^2 + x = x (x-1)^2 (x+1)
\end{gather*}
Stacionarne to"cke so: $-1, 0, 1$. "Ce se nari"semo skico predznaka $f'(x)$ vidimo:
\begin{itemize}
	\item V -1 je lokalni maksimum
	\item V 0 je lokalni minimum
	\item V 1 nima lokalnega ekstrema
\end{itemize}
%
\textsc{Trditev:} Denimo, da je $f$ dvakrat odvedljiva v okolici to"cke $a$ in denimo, da je $a$ stacionarna to"cka od $f$.
\begin{enumerate}[(i)]
	\item "Ce je $f''(x) \leq 0$ za vse $x$ v neki okolici to"cke $a$, potem ima $f$ v $a$ lokalni maksimum.
	\item "Ce je $f''(x) \geq 0$ za vse $x$ v neki okolici to"cke $a$, potem ima $f$ v $a$ lokalni minimum.
\end{enumerate}
\textsc{Dokaz:}
\begin{enumerate}[(i)]
	\item $f''(x) \leq 0$ za $x \in (a - \delta, a + \delta)$, zato je $f'$ padajo"ca na $(a - \delta, a + \delta)$. Vemo tudi, da je $f'(a) = 0$, ker je $a$ stacionarna to"cka. Sledi:
	\begin{align*}
	f'(x) \geq 0& \quad \forall x \in (a- \delta, a) \quad \land \\
	f'(x) \leq 0& \quad \forall x \in (a, a + \delta)
	\end{align*}
	Po posledici ima $f$ v $a$ lokalni maksimum.
	
	\item Podobno
\end{enumerate}
%
\textsc{Posledica:} Naj bo$f$ dvakrat odvedljiva v okolkici to"cke $a$ in $f''$ zvezna v to"cki $a$, ter naj bo $a$ stacionarna to'cka od $f$. Velja:
\begin{enumerate}[(i)]
	\item "Ce je $f''(a) < 0$, potem ima $f$ v $a$ lokalni maksimum.
	\item "Ce je $f''(a) > 0$, potem ima $f$ v $a$ loklani minimum.
\end{enumerate}
\textsc{Dokaz:} "Ce je $f''(a) < 0$, zaradi zveznosti $f''$ v $a$ velja $f''(x) < 0$ za vse $x$ v neki okolici to"cke $a$. Nato uporabimo trditev.

\textsc{Trditev:} Naj bo $f$ odvedljiva na $[a, b]$.
\begin{enumerate}[(i)]
	\item "Ce je $f'(a) > 0$, potem ima $f$ v $a$ lokalni minimum.
	\item "Ce je $f'(a) < 0$, potem ima $f$ v $a$ lokalni maksimum.
	\item "Ce je $f'(b) > 0$, potem ima $f$ v $b$ lokalni maksimum.
	\item "Ce je $f'(b) < 0$, potem ima $f$ v $b$ lokalni minimum.
\end{enumerate}
\textbf{Opomba:} Strogi ena"caj je zato, ker "ce $f'(a) = 0$, se lahko funkcija ,,obrne'' v katerokoli smer.
%
\subsection{Konveksnost in konkavnost}
\deff Naj bo funkcija $f$ definirana na intervalu $I$. Pravimo, da je $f$ \emph{konveksna} na $I$, "ce velja:
\begin{equation*}
\forall a, b \in I, a < b \forall x \in [a, b]: f(x) \leq f(a) + \dfrac{f(b) - f(a)}{b - a} (x - a) \tag{*}
\end{equation*}
\textsc{Geometrijsko:} $\forall a, b \in I, a < b$ graf funkcije $f$ na $[a, b]$ le"zi pod daljico skozi $(a, f(a))$ in $(b, f(b))$.

"Ce pi"semo $t = \dfrac{x - a}{b - a}$: ko $x \in [a, b]$ bo $t\in [0, 1]$.
\begin{align*}
x - a =& t (b - a) \\
x =& (1 - t) a + tb
\end{align*}
(*) se prepi"se v
\begin{equation*}
f((1 - t)a + tb) \leq (1-t) f(a) + t f(b)
\end{equation*}
za vse $t \in [0, 1]$.

\deff Naj bo $f$ definirana na intervalu $I$. Pravimo, da je $f$ \emph{konkavna} na$I$, "ce
\begin{equation*}
\forall a, b \in I, a < b \forall x \in [a, b]: f(x) \geq f(a) + \dfrac{f(b) - f(a)}{b-a}(x-a)
\end{equation*}
\textsc{Geometrijsko:} $\forall a, b \in I, a < b$ graf funkcija $f$ na $[a, b]$ le"zi nad daljico skozi $(a, f(a))$ in $(b, f(b))$.

\textbf{Opomba:} Velja: $f$ konveksna $\iff -f$ konkavna.

\textsc{Izrek:} Naj bo $f$ odvedljiva funkcija na intervalu $I$. Potem je $f$ konveksna natanko tedaj, kadar velja
\begin{equation*}
\forall a, x \in I : f(x) \geq f(a) + f'(a) (x-a) \tag{*}
\end{equation*}
\textbf{Opomba:} Graf funkcije le"zi nad tangento v poljubni to"cki iz $I$.

\textsc{Dokaz:}
\begin{itemize}
	\item[$(\Rightarrow)$] Denimo, da je $f$ konveksna na $I$. Izberemo poljubni $a, x \in I$. "Ce je $a = x$, ocean (*) velja. "Ce $a \neq x$, velja
	\begin{equation*}
	f(y) \leq f(a) + \dfrac{f(x) - f(a)}{x-a} (y-a)
	\end{equation*}
	za vse $y$ med $a$ in $x$.
	\begin{gather*}
	\begin{aligned}
	f(y) - f(a) &\leq (f(x) - f(a)) \dfrac{y - a}{x - a} \\
	\underbrace{\dfrac{f(y) - f(a)}{y - a}}_{f'(a) \text{ ko gre $y$ proti $a$}} (x-a) &\leq f(x) - f(a) \\
	f'(a) (x - a) &\leq f(x) - f(a)
	\end{aligned}
	\end{gather*}
	
	\item[$(\Leftarrow)$] Denimo, da je $f(x) \geq f(a) + f'(a)(x-a)$ za vse $a, x \in I$. Izberemo $a, b \in I, a < b$ in $x \in (a, b)$. Potem to"cki $(a, f(a))$ in $(b, f(b))$ le"zita nad tangento na graf v to"cki $(x, f(x))$.
	
	\begin{align*}
	f(a) &\geq f(x) + f'(x) (a - x)  \quad / \underbrace{(b - x)}_{> 0} \\
	f(b) &\geq f(x) + f'(x) (b - x) \quad / - \underbrace{(a - x)}_{< 0} \\
	f(a) (b-x) - f(b) (a-x) &\geq f(x) (b-x - (a - x)) \\
	f(x) &\leq \dfrac{1}{b-a} (f(a) (b-a) + f(a) (a-x) - f(b) (a-x)) \\
	f(x) & \leq f(a) + \dfrac{f(b) - f(a)}{b-a} (x-a)
	\end{align*}
	za vse $x \in (a, b)$
	
	\hfill $\square$
\end{itemize}
%
\textsc{Izrek:} Naj bo $f$ odvedljiva funkcija na intervalu $I$. Potem je $f$ konveksna natanko tedaj, kadar je $f'$ nara"s"cajo"ca na $I$.

"Ce je $f$ dvakrat odvedljiva na intervalu $I$, potem je $f$ konveksna na $I$ natanko tedaj, kadar velja
\begin{equation*}
\forall x \in I: f''(x) \geq 0
\end{equation*}
\textsc{Dokaz:} Dovolj je dokazati prvo ekvivalenco, saj druga sledi iz prve z uporabo izreka o tem, kdaj je funkcija nara"s"cajo"ca.
\begin{itemize}
	\item[$(\Rightarrow)$] Denimo, da je $f$ konveksna na $I$. Dokazujemo, da je $f'$ nara"s"cajo"ca funkcija na $I$.
	
	\dashuline{$a, b \in I, a < b \Rightarrow f'(a) \leq f'(b)$}
	
	Za kraji"s"ce $a$ velja:
	\begin{align*}
	f(x) & \leq f(a) + \underbrace{\dfrac{f(b) - f(a)}{b-a}}_k (x-a) \quad \forall x \in (a, b) \\
	f(x) &\leq f(a) + k(x-a) \Rightarrow f(x) - f(a) \leq k(x-a) \\
	\end{align*}
	Podobno velja za kraji"s"ce $b$:
	\begin{equation*}
	f(x) \leq f(b) + k(x-b) \Rightarrow f(x) - f(b) \leq k(x - b)
	\end{equation*}
	Iz tega sledi:
	\begin{gather*}
		\dfrac{f(x) - f(a)}{x-a} \leq k \text{ ko po"sljemo $x$ proti $a$ dobimo } f'(a) \leq k\\
		\dfrac{f(x) - f(b)}{x-b} \geq k \text{ ko po"sljemo $x$ proti $b$ dobimo } f'(b) \geq k
	\end{gather*}
	Sledi:
	\begin{equation*}
	f'(a) \leq k \leq f'(b) \Rightarrow f'(a) \leq f'(b)
	\end{equation*}
	
	\item[$(\Leftarrow)$] Denimo, da je $f'$ nara"s"cajo"ca funkcija na $I$. Izberemo $a, x \in I, a < x$. Po Lagrangeevem izreku za $f$ na $[a, x]$ obstaja $c \in (a, x)$, da velja:
	\begin{equation*}
	f'(c) = \dfrac{f(x) - f(a)}{x-a}
	\end{equation*}
	Ker je $f'$ nara"s"cajo"ca, velja $f'(a) \leq f'(c)$. Iz tega sledi:
	\begin{align*}
	f(x) - f(a) &\geq f'(a) (x-a) \\
	f(x) &\geq f(a) + f'(a) (x-a) \quad \text{(ena"cba tangente skozi to"cko $a$)}
	\end{align*}
	Podobno lahko naredimo za $a > x$. Po prej"snjem izreku je $f$ konveksna.
	
	\hfill $\square$
\end{itemize}
%
\deff Naj bo funkcija $f$ definirana na intervalu $I$. "Ce za to"cko $a \in I$ obstaja taka okolica, da je na eni strani to"cke $a$ funkcija konveksna in na drugi strani konkavna, potem ima $f$ v to"cki $a$ \emph{prevoj}.

\textsc{Primer:} Natan"cno nari"si graf funkcija $f(x) = \arctan \frac{1 - x}{1 + x}$.

Dolo"cimo:
\begin{itemize}
	\item $D_f$, ni"cle, simetrija (sodost, lihost), periodi"cnost
	\item prvi odvod (intervali nara"s"canja in padanja, lokalni ekstremi)
	\item obna"sanje na robovih $D_f$ (limite, asimptote)
	\item drugi odvod (intervali konveksnosti in konkavnosti, prevoji)
\end{itemize}
\textbf{Opomba:} Drugi odvod je drugotnega pomena.

Pora"cunati in narisata zna"s verjetno sam/a. "Ce ne, pridi kdaj na vaje.
%
\subsection{L'H\^{o}pitalovi izreki}
\textsc{Lema:} (posplo"seni Lagrangeev izrek)

Naj bosta $f$ in $g$ zvezni funkciji na $[a, b]$, odvedljivi na $(a, b)$. Denimo, da je $g'(x) \neq 0$ za vse $x \in (a, b)$. Potem obstaja $c \in (a, b)$, da velja
\begin{equation*}
\dfrac{f(b) - f(a)}{g(b) - g(a)} = \dfrac{f'(c)}{g'(c)}
\end{equation*}
\textbf{Opomba:} "Ce za $g$ vzamemo funkcijo $g(x) = x$, dobimo Lagrangeev izrek.

\textsc{Dokaz:} Velja $g(b) - g(a) \neq 0$, ker po Lagrangeevem izreku obstaja $d \in (a, b)$, da velja
\begin{equation*}
g(b) - g(a) = \underbrace{g'(d)}_{\neq 0} \underbrace{(b-a)}_{\neq 0}
\end{equation*}
Ozna"cimo $k = \frac{f(b) - f(a)}{g(b) - g(a)}$ in definiramo funkcijo:
\begin{equation*}
F(x) = f(x) - f(a) - k(g(x) - g(a))
\end{equation*}
$F$ je zvezna na $[a, b]$ in odveljiva na $(a, b)$. Velja $F(a) = 0, \quad F(b) = 0$. Po Rollovem izreku za $F$ obstaja $c \in (a, b): F'(c) = 0$.
\begin{gather*}
F'(x) = f'(x) - kg'(x) \\
f'(c) - kg'(c) = 0 \\
k = \dfrac{f'(c)}{g'(c)}
\end{gather*}
\hfill $\square$

\textsc{Izrek} (L'H\^{o}pitalovo pravilo) Naj bosta $f$ in $g$ odvedljivi funkciji na $(a, b)$. Denimo da velja
\begin{enumerate}[(i)]
	\item $g(x) \neq 0, \quad g'(x) \neq 0 \qquad \forall x \in (a, b)$
	\item $\lim_{x \downarrow a} f(x) = 0, \quad \lim_{x \downarrow a} g(x) = 0$
\end{enumerate}
"Ce obstaja limitia $\lim_{x \downarrow a} \frac{f'(x)}{g'(x)}$, potem obstaja $\lim_{x \downarrow a} \frac{f(x)}{g(x)}$ in velja
\begin{equation*}
\lim_{x \downarrow a} \frac{f(x)}{g(x)} = \lim_{x \downarrow a} \frac{f'(x)}{g'(x)}
\end{equation*}
\textbf{Opombe:}
\begin{enumerate}
	\item "Ce sta oba odvoda $f'$ in $g'$ v to"cki $a$ definirana in zvezna, potem velja
	\begin{equation*}
	\lim_{x \downarrow a} \frac{f(x)}{g(x)} = \dfrac{f'(a)}{g'(a)}
	\end{equation*}

	\item "Stevilo $b$ je lahko blizu $a$.
	\item Analogen izrek velja za leve in obojestranske limite.
\end{enumerate}
\textsc{Primera:}
\begin{gather*}
\lim_{x \to 0} \dfrac{\sin x}{x} = \lim_{x \to 0} \dfrac{\cos x}{1} = 1 \\
\lim_{x \to 0} \left(\dfrac{1}{x^2} - \dfrac{\cos 2x}{x^2}\right) = \lim_{x \to 0} \dfrac{1 - \cos 2x}{x^2} = \lim_{x \to 0} \dfrac{2\sin 2x}{2x} = \cdots = 2
\end{gather*}
\textsc{Dokaz:} Definiramo $f(a) = g(a) = 0$: $f$ in $g$ sta na $[a, b)$ zvezni funkciji in odvedljivi na $(a, b)$. 

Izberemo poljuben $x \in (a, b)$. $f$ in $g$ sta zvezni na $[a, x]$ in odvedljivi na $(a, x)$. Po lemi (posplo"sen Lagrangeev izrek) obstaja to"cka $c_x \in (a, x)$, za katero velja
\begin{align*}
\dfrac{f(x) - f(a)}{g(x) - g(a)} &= \dfrac{f'(c_x)}{g'(c_x)} \\
\dfrac{f(x)}{g(x)} &= \dfrac{f'(c_x)}{g'(c_x)}
\end{align*}
Denimo, da je $L = \lim_{x \downarrow a} \frac{f'(x)}{g'(x)}$. Dokazujemo $\lim_{x \downarrow a} \frac{f(x)}{g(x)} = L$.

Izberemo $\varepsilon > 0$. Obstaja $\delta > 0: \frac{f'(x)}{g'(x)} \in (L - \varepsilon, L + \varepsilon)$ za vse $x \in (a, a+ \delta)$.

Za $x \in (a, a+ \delta)$ velja $\frac{f(x)}{g(x)} = \frac{f'(c_x)}{g'(c_x)} = L$ za nek $c_x \in (a, x) \Rightarrow c_x \in (a, a + \delta)$.

\textsc{Izrek:} Naj bosta $f$ in $g$ odvedljivi funkciji na $(a, b)$ in $g'(x) \neq 0$ za vsak $x \in (a, b)$ in naj bo $\lim_{x \to a}g(x) = \infty$. "Ce obstaja $\lim_{x \downarrow a} \frac{f'(x)}{g'(x)}$, potem obstaja $\lim_{x \downarrow a} \frac{f(x)}{g(x)}$ in velja
\begin{equation*}
\lim_{x \downarrow a} \dfrac{f(x)}{g(x)} = \lim_{x \downarrow a} \dfrac{f'(x)}{g'(x)}
\end{equation*}
\textsc{Opombe:}
\begin{itemize}
	\item Izrek velja tudi za leve in obojestranske limite.
	\item Uporaba je smiselna npr.\,v primeru $\lim_{x \downarrow a} f(x) = \pm \infty$.
\end{itemize}
\textsc{Dokaz:} Denimo, da $\lim_{x \downarrow a} \frac{f'(x)}{g'(x)} = L$ obstaja. Za poljuben $\varepsilon > 0$ obstaja $\delta > 0$ da velja
\begin{equation*}
\forall x \in (a, a + \delta): L - \varepsilon < \dfrac{f'(x)}{g'(x)} < L + \varepsilon
\end{equation*}
Izberimo poljuben $x \in (a, a + \delta)$. Uporabimo lemo na intervalu $[x, a + \delta)$ in velja
\begin{equation*}
\dfrac{f(x) - f(a + \delta)}{g(x) - g(a + \delta)} = \dfrac{f'(c)}{g'(c)}
\end{equation*}
za nek $c \in (x, a + \delta)$. Sledi:
\begin{equation*}
L - \varepsilon < \dfrac{f(x) - f(a + \delta)}{g(x) - g(a + \delta)} < L + \varepsilon
\end{equation*}
za vsak $x \in (a, a + \delta)$. Ker je $\lim_{x \downarrow a} g(x) = \infty$, velja
\begin{gather*}
g(x) > 0 \\
g(x) - g(a + \delta) > 0
\end{gather*}
zas vse $x$ dovolj blizu $a$. Neenakost pomno"zimo z $\frac{g(x) - g(a + \delta)}{g(x)}$
\begin{gather*}
(L - \varepsilon) \left( 1 - \dfrac{g(a + \delta)}{g(x)} \right) < \dfrac{f(x) - f(a + \delta)}{g(x)} < (L + \varepsilon) \left( 1 - \dfrac{g(a + \delta)}{g(x)} \right) \\
\underbrace{(L - \varepsilon) \left( 1 - \underbrace{\dfrac{g(a + \delta)}{g(x)}}_{\to 0} \right) + \underbrace{\dfrac{f(a + \delta)}{g(x)}}_{\to 0}}_{\to L - \varepsilon} < \dfrac{f(x)}{g(x)} < \underbrace{(L + \varepsilon) \left( 1 - \underbrace{\dfrac{g(a + \delta)}{g(x)}}_{\to 0} \right) + \underbrace{\dfrac{f(a + \delta)}{g(x)}}_{\to 0}}_{\to L + \varepsilon}
\end{gather*}
Ker je limita leve strani neena"cbe $L - \varepsilon$, se nahaja na $(L - 2\varepsilon, L)$ za vse $x \in (a, a + \delta')$. Podobno lahko re"cemo, da ker je limita desne strani neena"cbe $L + \varepsilon$, se desna stran nahaja na $(L, L + 2\varepsilon)$ za vse $x \in (a, \delta'')$. Zato se $\frac{f(x)}{g(x)}$ nahaja na intervalu $(L - 2\varepsilon, L + 2\varepsilon)$ za $x \in (a, a + \delta') \cap (a, a + \delta'')$.

\textsc{Posledica:} Naj bosta $f$ in $g$ odvedljivi funkciji na $(M, \infty)$ za nek $M \in \RR$. Denimo, da $g(x) \neq 0, g'(x) \neq 0$ za vse $x \in (M, \infty)$. Naj bo $\lim_{x \to \infty} f(x) = 0, \lim_{x \to \infty} g(x) = 0$. "Ce obstaja $\lim_{x \to \infty} \frac{f'(x)}{g'(x)}$, potem obstaja $\lim_{x \to \infty} \frac{f(x)}{g(x)}$ in velja
\begin{equation*}
\lim_{x \to \infty} \dfrac{f(x)}{g(x)} = \lim_{x \to \infty} \dfrac{f'(x)}{g'(x)}
\end{equation*}
\textsc{Dokaz:} Definiramo
\begin{gather*}
F(t) = f\left(\dfrac{1}{t}\right) \\
G(t) = g\left(\dfrac{1}{t}\right)
\end{gather*}
Vzamemo $M > 0$: $F$ in $G$ sta definirani na $\left(0, \frac{1}{M}\right)$. Izrek velja za $F$ in $G$, ko $t \downarrow 0$. Takrat gre $\frac{1}{t} \to \infty$.
\begin{gather*}
\dfrac{F'(t)}{G'(t)} = \dfrac{f'\left(\frac{1}{t}\right) \left(- \frac{1}{t^2}\right)}{g'\left(\frac{1}{t}\right) \left(- \frac{1}{t^2}\right)} \\
\lim_{t \downarrow 0} \dfrac{F'(t)}{G'(t)} = \lim_{t \downarrow 0} \dfrac{f'\left(\frac{1}{t}\right)}{g'\left(\frac{1}{t}\right)} = \lim_{x \to \infty} \dfrac{f'(x)}{g'(x)}
\end{gather*}
Po L.P.\,velja
\begin{gather*}
\lim_{t \downarrow 0} \dfrac{F(t)}{G(t)} = \lim_{t \downarrow 0} \dfrac{F'(t)}{G(t)} \\
\lim_{t \downarrow 0} \dfrac{F(t)}{G(t)} = \lim_{t \downarrow 0} \dfrac{f \left(\frac{1}{t}\right)}{g \left(\frac{1}{t}\right)} = \lim_{x \to \infty} \dfrac{f(x)}{g(x)}
\end{gather*}
\hfill $\square$

Podobna posledica velja za drugi izrek.

Kdaj uporabljamo L'H\^{o}pitalovo pravilo?
\begin{itemize}
	\item $\frac{ 0}{0}, \frac{\infty}{\infty}$ ("ce so izpolnjeni pogoji v izreku)
	\item 
	\begin{gather*}
		0 \infty = \dfrac{0}{\frac{1}{\infty}} = \dfrac{0}{0} \\
		0 \infty = \dfrac{\infty}{\frac{1}{0}} = \dfrac{\infty}{\infty}
	\end{gather*}
\end{itemize}
%
\subsection{Uporaba odvoda v geometriji}
\subsubsection{Podajanje krivulj}
V kartezi"cnem koordinatnem sistemu:
\begin{enumerate}
	\item \textbf{Eksplicitno:} Krivulja $K$ je dana kot graf funkcije $f: [a, b] \to \RR$, torej
	\begin{equation*}
	K = \Gamma_f = \{(x, f(x)): x \in [a, b]\}
	\end{equation*}
	
	\item \textbf{Implicitno:} Krivulja $K$ je dana kot mno"zica re"sitev ena"cbe $g(x, y) = 0$, kjer je $g: \RR^2 \to \RR$ dana funkcija
	\begin{equation*}
	K = \{(x, y) \in \RR^2: g(x, y) = 0\}
	\end{equation*}
	Primer: $g(x, y) = x^2 + y^2 - 1$ (enotska kro"znica)
	
	\item \textbf{Parametri"cno:} Krivulja $K$ je dana kot mno"zica to"ck $(x, y)$, ki so dolo'cene z $x = \alpha(t), y = \beta(t)$, kjer sta $\alpha, \beta: [t_0, t_1] \to \RR$ funkciji.
	\begin{equation*}
	K = \{(\alpha(t), \beta(t)): t \in [t_0, t_1]\}
	\end{equation*}
	$t \mapsto (\alpha(t), \beta(t))$ je \emph{vektorska funkcija}.
\end{enumerate}
Implicitni na"cin podajanja krivulj je bolj splo"sen od eksplicitnega.

\textsc{Primeri:}
\begin{enumerate}
	\item enotska kro"znica $x^2 + y^2 = 1$. Parametri"cno:
	\begin{gather*}
		x(t) = \cos t \\
		y(t) = \sin t \\
		x^2(t) + y^2 (t) = 1
	\end{gather*}
	
	\item Elipsa $\underbrace{\frac{(x - x_0)^2}{a^2}}_{\cos^2 t} + \underbrace{\frac{(y - y_0)^2}{b^2}}_{\sin^2 t} = 1$
	
	Parametri"cno:
	\begin{gather*}
	x = x_0 + a \cos t \\
	y = y_0 + b \sin t
	\end{gather*}
	
	\item "Se dva primera kjer smo veliko risali (na enem smo se nau"cili risati grafe parametri"cnih krivulj). Veliko risanja pomeni, da si poglej zvezek.
\end{enumerate}
%
V polarnem koordinatnem sistemu:

Krivulja $K$ je podana kot mno"zica to"ck s polarnima koordinatama $(r, \varphi)$, kjer je $r = h(\varphi)$, kjer je $h: [\varphi_0, \varphi_1] \to \RR$ realna funkcija.

\textsc{Primera:}
\begin{enumerate}
	\item $h(\varphi) = c$, kjer je $c \in \RR$. "Ce nari"ses pride kro"znica.
	\item $h(\varphi) = k\varphi, k > 0$. "Ce nari"ses pride spirala.
\end{enumerate}
%
\deff \emph{Pot v ravnini} je preslikava $F = (\alpha, \beta): I \to \RR^2$, kjer je $I$ interval, $\alpha$ in $\beta$ pa zvezni funkciji na $I$.

\emph{Tir poti} je mno"zica
\begin{equation*}
C = F(I) = \{F(t): t \in I\}
\end{equation*}
$F$ imenujemo \emph{parametrizacija} krivulje $C$. Isto krivuljo $C$ lahko podamo z razli'cnimi parametrizacijami.

\deff Pot $F = (\alpha, \beta): I \to \RR^2$ je odvedljiva, "ce sta $\alpha$ in $\beta$ odvedljivi na $I$. Pot $F$ je razred $C^1$ (zvezno odvedljiva), "ce sta $\alpha$ in $\beta$ razreda $C^1$.

\textbf{Opomba:} Preslikava $F$ je zvezna, oziroma $\alpha$ in $\beta$ sta zvezni.

"Ce je $F$ odvedljiva pot:
\begin{equation*}
\dot{F}(t) = (\dot{\alpha}(t) \dot{\beta}(t))
\end{equation*}
%
\textsc{Izrek:} Naj bo $F: I \to \RR^2$ zvezno odvedljiva pot, $t_0 \in I$ in denimo, da je $\dot{F}(t_0) = (\dot{\alpha(t_0)}, \dot{\beta(t_0)}) \neq 0$. "Ce je $\dot{\alpha(t_0)} \neq 0$, potem obstaja tak $\delta > 0$, da lahko krivuljo
\begin{equation*}
K = \{F(t): |t - t_0| < \delta\}
\end{equation*}
zapi"semo kot graf neke odvedljive funkcije $y = f(x)$ med intervalom $U$ okrog to"cke $x_0 = \alpha(t_0)$:
\begin{equation*}
K = \{(x, f(x)): x \in U\}
\end{equation*}
in velja
\begin{equation*}
f'(\alpha(t)) = \dfrac{\dot{\beta}(t)}{\dot{\alpha}(t)}
\end{equation*}
za vse $t \in (t_0 - \delta, t_0 + \delta)$

\textsc{Dokaz:} Denimo, da je $\dot{\alpha}(t_0) > 0$.

Ker je $\dot{\alpha}$ zvezna funkcija, je $\dot{\alpha} > 0$ na neki okolici $t_0$, zato je $\alpha$ na neki okolici $t_0$ strogo nara"s"cajo"ca funkcija

Zato obstaja $\delta > 0$, da $\alpha$ preslika interval $(t_0 - \delta, t_0 + \delta)$ bijektivno v $(\alpha(t_0 - \delta), \alpha(t_0 + \delta)) =: U$.

Potem ima $\alpha$ inverzno funkcijo $\alpha^{-1}: U \to (t_0 - \delta, t_0 + \delta)$, ki je odvedljiva. Definiramo
\begin{equation*}
f(x) = \beta(\alpha^{-1}(x)), \quad x \in U
\end{equation*}
Velja:
\begin{equation*}
(x, f(x)) = (\alpha(t), \beta(\alpha^{-1}(\alpha(t))) = (\alpha(t), \beta(t))
\end{equation*}
za $x \in U$. Torej velja
\begin{equation*}
f'(x) = \beta'(\alpha^{-1}(x)) \dfrac{1}{\alpha'(\alpha^{-1}(x))}
\end{equation*}
kjer je $x = \alpha(t)$.

\hfill $\square$

\textsc{Posledica:} Naj bosta $\alpha$ in $\beta$ dvakrat odvedljivi na $(t_0, t_1)$. Denimo, da je $\dot{\alpha}(t) \neq 0$ za vse $t \in (t_0, t_1)$. Potem je funkcija $f$ iz izreka dvakrat odvedljiva in velja:
\begin{equation*}
f''(\alpha(t)) = \dfrac{\dot{\alpha}(t)\ddot{\beta}(t) - \ddot{\alpha}(t)\dot{\beta}(t)}{(\dot{\alpha}(t))^3}
\end{equation*}
\textsc{Dokaz:} Velja $f'(\alpha(t)) = \frac{\dot{\beta}(t)}{\dot{\alpha(t)}}$. $f$ je dvakrat odvedljiva, ker je kompozitum dvakrat odvedljivih funkcij.
\begin{equation*}
f''(\alpha(t)) \dot{\alpha}(t) = \dfrac{\dot{\alpha}(t)\ddot{\beta}(t) - \ddot{\alpha}(t)\dot{\beta}(t)}{(\dot{\alpha}(t))^2}
\end{equation*}
\hfill $\square$

\deff Naj bo $F: I \to \RR^2$ odvedljiva pot
\begin{enumerate}[(i)]
	\item "Ce za nek $t \in I$ velja $\dot{F}(t) = 0$, potem $t$ imenujemo \emph{kriti"cna to"cka} preslikave $F$.
	\item "Ce za nek $t \in I$ velja $\dot{F}(t) \neq 0$, potem $t$ imenujemo \emph{regularna to"cka} preslikave $F$.
	\item "Ce so vse to"cke $t \in I$ regularne, potem $F$ imenujemo \emph{regularna parametrizacija} gladkega tira poti $F$.
\end{enumerate}
%
Naj bo $F: I \to \RR^2$ odvedljiva pot v ravnini in naj bo $t$ regularna to"cka od $F$. Potem je $\dot{F}(t)$ smerni vektor tangente skozi to"cko $F(t)$ in ena"cba tangente je
\begin{equation*}
\vec{r} = \vec{r_F(t)} + s \dot{F}(t)
\end{equation*}
v vektorski obliki. V parametri"cni obliki je ena"cba smernega vektorja
\begin{align*}
x &= \alpha(t) + s \dot{\alpha}(t) \\
y &= \beta(t) + s \dot{\beta}(t)
\end{align*}
V segmenti obliki je to
\begin{equation*}
(x - \alpha(t)) \dot{\beta}(t) - (y - \beta(t)) \dot{\alpha}(t) = 0
\end{equation*}
ali
\begin{equation*}
\dfrac{x - \alpha(t)}{\dot{\alpha}(t)} = \dfrac{y - \beta(t)}{\dot{\beta}(t)}
\end{equation*}
V regularni to"cki je smerni vektor normale $(-\dot{\beta}(t), \dot{\alpha}(t))$. Ena"cba normale je torej
\begin{equation*}
(x - \alpha(t)) \dot{\alpha}(t) + (y - \beta(t)) \dot{\beta}(t) = 0
\end{equation*}
%
\textsc{Trditev:} Tangenta je odvisna samo od tira poti, ni pa odvisna od izbire regularne parametrizacije.

\textsc{Dokaz:} Denimo, da je $t_2$ regularna to"cka parametrizacije $F = (\alpha, \beta)$ in da je $s_2$ regularna to"cka parametrizacije $G = (\alpha_1, \beta_1)$ in da velja $F(t_2) = G(s_2)$. Naj bo $\alpha'(t_2) \neq 0$. Potem obstaja odvedljiva funkcija $\varphi$ v okolici $s_2$ za katero velja
\begin{equation*}
\varphi(s_2) = t_2, \quad \varphi'(s_2) \neq 0
\end{equation*}
Funkcijo definiramo kot
\begin{equation*}
\varphi = \alpha^{-1} \circ \alpha_1
\end{equation*}
Smerna koeficienta tangent sta
\begin{gather*}
(\dot{\alpha}(t_2), \dot{\beta}(t_2)) \\
(\dot{\alpha_1}(s_2), \dot{\beta_1}(s_2))
\end{gather*}
Parametrizacijo $G$ lahko zapi"semo kot $G = F \circ \varphi$. Potem velja
\begin{gather*}
(\alpha_1, \beta_1) = (\alpha \circ \varphi, \beta \circ \varphi) \\
(\dot{\alpha_1}(s_2), \dot{\beta_1}(s_2)) = (\dot{\alpha}(t_2) \varphi'(s_2), \dot{\beta}(t_2) \varphi'(s_2)) = \underbrace{\varphi'(s_2)}_{\neq 0} (\dot{\alpha}(t_2), \dot{\beta}(t_2))
\end{gather*}
Ker sta vektorja kolinearna, dolo"cata isto premico.
%
\subsection{Integral}
\subsubsection{Primitivna funkcija in nedolo"cen integral}
\textsc{Motivacija:} Vsaka odvedljiva funkcija $f$ na intervalu $I$ dolo"ca funkcijo $f'$ na $I$. Denimo, da poznamo $f'$. Kako dobimo $f$? Ali je vsaka funkcija $g$ na $I$ odvod neke funkcije?

\deff Naj bo funckija $f$ definirana na mno"zici $I \subset \RR$. "Ce obstaja odvedljiva funkcija $F: I \to \RR$, za katero velja $F' = f$ na $I$, potem jo imenujemo \emph{primitivna funkcija} od $f$.

\textsc{Primer:} $f(x) = x$ na $\RR$

$F(x) = \frac{1}{x} x^2$ je primitivna funkcija od $f$.

$G(x) = \frac{1}{x} x^2 + 100$ je primitivna funkcija od $f$.

\textsc{Lema:} Naj bo $f$ funkcija definirana na intervalu $I$.
\begin{enumerate}
	\item "Ce je $F$ primitivna funkcija od $f$ na $I$, potem je za vsak $c \in \RR$ tudi funkcija $G(x) = F(x) + c$ primitivna funkcija od $f$.
	\item "Ce sta $F$ in $G$ primitivni funkciji od $f$ na intervalu $I$, potem obstaja $c \in \RR$, da velja
	\begin{equation*}
	\forall x \in I: G(x) = F(x) + c
	\end{equation*}
\end{enumerate}
\textsc{Dokaz:}
\begin{enumerate}
	\item $F' = f$ po definiciji
	
	$G' = F' = f$
	
	\item $h(x) = F(x) - G(x)$ je odvedljiva funkcija na intervalu $I$.
	\begin{equation*}
	h'(x) = F'(x) - G'(x) = f(x)  - f(x) = 0 \quad \forall x \in I
	\end{equation*}
	Zato je $h$ konstanta funkcija.
	
	\hfill $\square$
\end{enumerate}
%
\deff \emph{Nedolo"cen integral} funkcije $f$ je skupek vseh njenih primitivnih funkcij. Ozna"cimo ga s
\begin{equation*}
\int f(x) dx
\end{equation*}
Funkcijo $f$ imenujemo \emph{integrand}.

\textsc{Primer:}
\begin{equation*}
\int \frac{1}{x} dx = \begin{cases}
\ln |x| + c_1 & x > 0 \\
\ln |x| + c_2 & x < 0
\end{cases}
\end{equation*}
%
\textsc{Primer:}
\begin{equation*}
f(x) = \begin{cases}
1 & -1 < x \leq 0 \\
-1 & 0 < x < 1
\end{cases}
\end{equation*}
\dashuline{$f$ nima primitivne funkcije na $(-1, 1)$}

Denimo, da je $F$ primitivna funkcija od $f$ na $(-1, 1)$. Torej je $F'(x) = f(x)$ za vse $x \in (-1, 1)$. Naj bo $a \in (0, 1)$. Potem je $F$ odvedljiva funkcija na $[-a, a]$, zatoje $F$ zvezna na $[-a, a]$ in zato dose"ze maksimum.
\begin{itemize}
	\item $F$ nima stacionarne to"cke.
	\item $F'(-a) = f(-a) = 1$, zato v $-a$ ni dose"zen maksimum
	\item $F'(a) = f(a) = -1$, zato v $a$ ni dose"zen maksimum.
\end{itemize}
$\rightarrow \leftarrow$

\textbf{Opomba:} Funkcija $f$ ni zvezna. Dokazali bomo, da ima vsaka zvezna funkcija primitivno funkcijo.

\textsc{Tabela osnovnih integralov:}
\begin{table}[!htbp]
	\centering
	\begin{tabular}{c | c}
		\textbf{Funkcija} & \textbf{Primitivna funkcija} \\ \hline
		$x^n$ & $\frac{x^n+1}{n + 1}, \quad n \neq 0$ \\
		$\frac{1}{x}$ & $\ln |x|$ \\
		$a^x$ & $\frac{1}{\ln a} a^x$ \\
		$e^x$ & $e^x$ \\
		$\sin x$ & $- \cos x$ \\
		$\cos x$ & $\sin x$ \\
		$\sinh x$ & $\cosh x$ \\
		$\cosh x$ & $\sinh x$ \\
		$\frac{1}{1+x^2}$ & $\arctan x$ \\
		$\frac{1}{\sqrt{1 - x^2}}$ & $\arcsin x$ \\
		$\frac{1}{\sqrt{x^2 \pm 1}}$ & $\ln (x + \sqrt{x^2 \pm 1})$
	\end{tabular}
\end{table}
%
\subsubsection{Pravila za integriranje}
\textsc{Trditev:} Za poljubni funkciji $f$ in $g$, ki imata primitivni funkcija na intervalu $I$ velja
\begin{enumerate}
	\item $\int (f(x) \pm g(x)) dx = \int f(x) dx \pm \int g(x) dx$
	\item $\int \lambda f(x) dx = \lambda \int f(x) dx, \quad \lambda \in \RR$
\end{enumerate}
\textsc{Dokaz:} $F' = f$ in $G' = g$, potem velja $(F + G)' = f + g$

Podobno za mno"zenje s konstantno.

\hfill $\square$

\textsc{Trditev:} "Ce je $F$ odvedljiva funkcija na $I$, potem velja
\begin{equation*}
\int F'(x)dx = F(x) + c
\end{equation*}
\textsc{Dokaz:} Po definiciji primitivne funkcije.

\textbf{Opomba:} $F$ je primitivna funkcija do$f$ na $I$, "ce je $F$ odvedljiva in velja $F' = f$ na $I$. Denimo, da je $F$ primitivna funkcija od $f$.
\begin{gather*}
dF(x) = F'(x)dx = f(x)dx \\
\int f(x)dx = F(x) + c
\end{gather*}
%
\textsc{Trditev:} (uvedba nove spremenljivke)

Naj bo funkcija $g$ odvedljiva na intervalu $I$. Naj ima funkcija $f$ primitivno funkcijo $F$, definirano na $g(I) = \{g(x): x \in I\}$. Potem je $F \circ g$ primitivna funkcija od $(f\circ g) g'$ na intervalu $I$. Torej:
\begin{equation*}
\int (f\circ g)(x) g'(x) dx = F(g(x)) + C, \quad x \in I
\end{equation*}
\textbf{Opomba:} "Ce ozna"cimo $t = g(x)$, dobimo:
\begin{gather*}
dt = g'(x) dx \\
\int f(\underbrace{g(x)}_t) \underbrace{g'(x)dx}_{dt} = \int f(t) dt = F(t) + C = F(g(x)) + C
\end{gather*}
\textsc{Dokaz:} $F \circ g$ je odvedljivak, ker je  kompozitum odvedljivih
\begin{equation*}
(F \circ g)' (x) = F'(g(x)) g'(x) = f(g(x)) g'(x) = (f \circ g) g'
\end{equation*}
\hfill $\square$

\textsc{Trditev:} (integriranje po delih - per partes)

Naj bosta $f$ in $g$ odvedljivi funkciji na $I$. Potem velja
\begin{equation*}
\int f(x) g'(x) dx = f(x) g(x) - \int f'(x) g(x) dx, \quad x \in I
\end{equation*}
"Ce ozna"cimo $u = f(x)$ in $v = g(x)$, potem velja
\begin{equation*}
\int u dv = uv - \int v du
\end{equation*}
\textsc{Dokaz:}
\begin{equation*}
(f(x) g(x))' = f'(x) g(x) + f(x) g'(x)
\end{equation*}
Zato je $fg$ primitivna funkcija od $f'g + fg'$ in zato formula velja.
%
\subsubsection{Dolo"ceni integral}
Naj bo $f: [a, b] \to \RR$ nenegativna funkcija. Potem graf funkcije $f$ omejuje obmo"cje $A$ nad intervalom $[a, b]$: $A$ je navzgor omejeno z $\Gamma_f$, navzdol z abscisno osjo, na levi s premico $x = a$ in na desni s premico $x = b$.

"Ce je $f$ konstanta, znamo izra"cunat plo"s"cino, ki jo dolo"ca.

V splo"snem:
\begin{itemize}
	\item Interval $[a, b]$ razdelimo na podintervali in na vsakem podintervalu funkcijo $f$ zamenjamo s primerno konstanto.
	\item Ne vemo kako dobra je aproksimacija.
	\item "Ce namesto poljubne to"cke na grafu na vsakem podintervalu izberemo to"cki, va keterih je dose"zen infimum $f$ oziroma supremum $f$, dobimo zgornjo mejo in spodnjo mejo za plo"s"cino $A$.
\end{itemize}
\textsc{Primer:} $f(x) = x^2$ izra"cunaj plo"s"cino pod grafom na $[0, 1]$.

Interval $[0, 1]$ razdelimo na $n$ enakih delov (\emph{ekvidistan"cna delitev}).

Zgornja meja plo"s"cine lika
\begin{multline*}
\sup A = \dfrac{1}{n} \left(\dfrac{1}{n}\right)^2 + \dfrac{1}{n} \left(\dfrac{2}{n}\right)^2 + \cdots + \dfrac{1}{n} 1^2 = \\
= \dfrac{1}{n^3} (1 + 2^2 + 3^2 + \cdots + n^2) = \dfrac{1}{n^3} \dfrac{n (n+1) (2n + 1)}{6} = \\
= \dfrac{(n + 1) (2n + 1)}{6n^2} \stackrel{n \to \infty}{\longrightarrow} \dfrac{1}{3}
\end{multline*}
Spodnja meja plo"s"cine lika
\begin{multline*}
\inf A = \dfrac{1}{n} 0 + \dfrac{1}{n} \dfrac{1}{n^2} + \dfrac{1}{n} \left(\dfrac{2}{n}\right)^2+ \cdots + \dfrac{1}{n} \left(\dfrac{n-1}{n}\right)^2 = \\
= \dfrac{1}{n^3} (1 + 2^2 + 3^2 + \cdots + (n-1)^2) = \\
= \dfrac{(n - 1) (2n - 1)}{6n^2} \stackrel{n \to \infty}{\longrightarrow} \dfrac{1}{3}
\end{multline*}
%
\subsubsection*{Riemannov integral}
\deff Naj bo $[a, b]$ dan interval. \emph{Delitev} $D$ intervala $[a, b]$ je mno"zica $\{x_1, x_1, \ldots, x_n\}$ \emph{delilnih to"ck}, za katere velja:
\begin{equation*}
a = x_0 < x_1 < x_2 < \cdots < x_{n-1} < x_n = b
\end{equation*}
in $n \in \NN$. Dol"zino $i$-tega intervala $[x_{i-1}, x_i]$ ozna"cimo z $\delta_i = x_i - x_{i-1}$.

\emph{Velikost} delitve $D$ je dol"zina najdalj"sega podintervala v delitvi $D$
\begin{equation*}
\delta(D) = \max \{\delta_1, \delta_2, \ldots, \delta_n\}
\end{equation*}
Na vsakem podintervalu izberemo \emph{testno to"cko} $t_i \in [x_{i-1}, x_i]$ in z
\begin{equation*}
T_D = \{t_1, t_2, \ldots, t_n\}
\end{equation*}
ozna"cimo nabor testnih to"ck. Denimo, da je nabor testnih to"ck $T_D$ \emph{usklajen} z delitvijo $D$, ker smo na vsakem podintervalu delitve $D$ izbrali eno testno to"cko.

\emph{Reimannova vsota} funkcije $f: [a, b] \to \RR$, pridru"zena delitvi $D$ in usklajeni izbiri testnih to"ck $T_D$ je
\begin{equation*}
R(f, D, T_D) = \sum_{k=1}^{n} \delta_k f(t_k)
\end{equation*}
\textbf{Opomba:} Pri"cakujemo, da Riemannove vsote bolje aproksimirajo prlo"s"cino, "ce je $\delta(D)$ manj"si.

\deff Riemannov itnegral ali dolo"ceni integral funkcije $f: [a, b] \to \RR$ je limita Rimeannove vsote, kjer limito vzamemo po vseh delitvah $D$ zaprtega intervala $[a, b]$ in po vseh usklajenih izbirah testnih to"ck $T_D$, ko gre $\delta_D$ proti 0, "ce ta limita obstaja. Pi"semo:
\begin{equation*}
\int_{a}^{b} f(x) dx = \lim_{\stackrel{D, T_D}{\delta(D) \to 0}} R(f, D, T_D)
\end{equation*}
Torej: $I = \lim_{\stackrel{D, T_D}{\delta(D) \to 0}} R(f, D, T_D)$ pomeni, da za vsak $\varepsilon > 0$ obstaja $\delta > 0$, da za vse delitve $D$, kjer $\delta(D) < \delta$ in vsako usklajeno izbiro testnih to"ck $T_D$ velja
\begin{equation*}
\left| R(f, D, T_D) - I \right| < \varepsilon
\end{equation*}
"Ce Riemannov integral $f$ na $[a, b]$ obstaja, pravimo, da je $f$ na $[a, b]$ \emph{Riemannovo integrabilna}.

\textsc{Trditev:} "Ce je funkcija $f: [a, b] \to \RR$ Riemannovo integrabilna, potem je $f$ omejena.

\textsc{Dokaz:} Denimo, da je $f$ Riemannovo integrabilna in da ni omejena. 

Po definiciji integrabilnosti obstaja $I \in \RR$, da za vsak $\varepsilon > 0$, obstaja $\delta > 0$ da za vsako delitev $D$, kjer je $\delta(D) < \delta$ in za vsako usklajeno izbiro testnih to"ck $T_D$, velja $|R(f, D, T_D) - I| < \varepsilon$. Torej obstaja $I$, da za $\varepsilon = 1$ obstaja $\delta > 0$.

Naj bo delitev $D$ taka, da je $\delta(D) < \delta$. Obstaja podinterval $[x_{i-1}, x]$, na katerem je $f$ neomejena.
\begin{equation*}
R(f, \underbrace{D}_{\text{fiksiramo}}, T_D) = \sum_{j = 1}^{n} \delta_j f(t_j)
\end{equation*}
$T_D = \{t_1, \ldots, t_n\}$ fiksiramo to"cke $t_1, t_2, \ldots, t_{i-1}, t_{i+1}, \ldots, t_n$. Vsota se torej zapi"ce kot
\begin{equation*}
R(f, D, T_D^K) = \sum_{j \neq i} f(t_j) \delta_j + f(t_i^k)\delta_i
\end{equation*}
Ker je $f$ neomjena, lahko privzamemo, da je $f$ navzgor neomejena na $[x_{i-1}, x_i]$. Torej za vsak $k \in \NN$ obstaja $s_k \in [x_{i-1}, x_i]$, da je $f(s_k) > k$. Za $t_i^k$ vzamemo $t_i^k = s_k$. $\rightarrow \leftarrow$

\textbf{Opomba:} Ni vsaka omejena funkcija na $[a, b]$ Riemannovo integrabilna
\begin{equation*}
f(x) = \begin{cases}
1 & x \in \QQ \cap [0, 1] \\
0 & x \in \RR \setminus \QQ \cap [0, 1].
\end{cases}
\end{equation*}
$f$ je omejana. \dashuline{$f$ ni Riemannovo integrabilna}
\begin{equation*}
R(f, D, T_D) = \sum_{i = 1}^n f(t_i) \delta_i = \sum_{i = 1}^{n} \delta_i = 1
\end{equation*}
Za testne to"cke, kjer je $t_i \in \QQ \cap [x_{i-1}, x_i]$
\begin{equation*}
R(f, D, T_D) = \sum_{i = 1}^n f(t_i) \delta_i = 0
\end{equation*}
Za testne to"cke, kjer je $t_i \in \RR \cap [x_{i-1}, x_i]$.

Za poljubno drobne delitve lahko izberemo $T_D^1$ in $T_D^2$, da je $R(f, D, T_D^1) = 1$ in $R(f, D, T_D^2) = 0$, torej limita ne obstaja.

\hfill $\square$
%
\subsubsection{Darbouxove vsote}
Odslej naj bo $f: [a, b] \to \RR$ omejena funkcija. Naj bo $D$ delitev intervala $[a, b]$ za delilnimi to"ckami $x_i$. Ozna"cimo:
\begin{align*}
m_k &= \inf \{f(x): x \in [x_{k-1}, x_k]\} \\
M_k &= \sup \{f(x): x \in [x_{k-1}, x_k]\} \\
m &= \inf \{f(x): x \in [a, b]\} \\
M &= \sup \{f(x): x \in [a, b]\}
\end{align*}
Velja:
\begin{equation*}
m \leq m_k \leq M_k \leq M
\end{equation*}
za vse $k = 1, \ldots, n$

\deff "Stevilo $s(D) = \sum_{i = 1}^n m_i \delta_i$ imenujemo \emph{spodnja Darbouxova vsota} prirejena delitvi $D$.

"Stevilo $S(D) = \sum_{i = 1}^{n} M_i \delta_i$ imenujemo \emph{zgornja Darbouxova vsota} prirejena delitvi $D$.

\textsc{Velja:}
\begin{equation*}
m(b-a) \leq s(D) \leq S(D) \leq M(b-a)
\end{equation*}
"se ve"c
\begin{equation*}
s(D) \leq R(f, D, T_D) \leq S(D)
\end{equation*}
za vsako usklajeno izbiro testnih to"ck.

\deff Pravimo, da je delitev $D'$ \emph{finej"sa} od delitve $D$, kadar je $D \subset D'$.

\textsc{Trditev 1:} Naj bo delitev $D'$ finej"sa od delitve $D$. Tedaj velja:
\begin{equation*}
s(D) \leq s(D') \quad \text{in} \quad S(D') \leq S(D)
\end{equation*}
\textsc{Dokaz:} Od $D$ do $D'$ pridemo z dodajanjem (ve"ckrat) ene to"cke naenkrat. Zato je dovolj dokazati v primeru $D' = D \cup \{y\}, y \in (x_{i-1}, x_i)$. Za $m_i$ velja
\begin{align*}
m_i' &= \inf \{f(x): x \in [x_{k-1}, y]\} \geq m_i \\
m_i'' &= \inf \{f(x): x \in [y, x_k]\} \geq m_i
\end{align*}
Torej velja
\begin{equation*}
s(D) = \sum_{j \neq i} m_j \delta_j + m_i \delta_i \leq \sum_{j \neq i} m_i \delta_i + m_i' (y- x_{k-1}) + m_i'' (x_k - y) = s(D'')
\end{equation*}
Podobno naredimo za zgornjo vsoto.

\hfill $\square$

\textsc{Trditev 2:} Naj bosta $D_1$ in $D_2$ poljubni delitvi intervala $[a, b]$. Potem velja
\begin{equation*}
s(D_1) \leq S(D_2)
\end{equation*}
\textbf{Opomba:} Poljubna spodnja Darbouxova vsota je pod poljubno zgornjo Darbouxovo vsoto.

\textsc{Dokaz:} Naj bo $D = D_1 \cup D_2$. $D$ je finej"sa od $D_1$ in od $D_2$. Po trditvi 1 velja
\begin{equation*}
s(D_1) \leq s(D) \leq S(D) \leq S(D_2)
\end{equation*}
\hfill $\square$

Mno"zica spodnjih Darbouxovih vsot je navzgor omejena (z $M(b-a)$), mno"zica zgornjih Darbouxovih vsot pa je navzdol omejena (z $m(b-a)$). Zato obstajata
\begin{align*}
s &= \sup \{s(D): D \text{ delitev } [a, b]\} \\
S &= \inf \{S(D): D \text{ delitev } [a, b]\}
\end{align*}
Po trditvi 2 velja $s \leq S$.
\begin{align*}
s(D_1) &\leq S(D_2) \quad \forall D_1, D_2 \\
\sup D_1 = s &\leq S(D_2) \quad \forall D_2 \\
s &\leq S = \inf D_2
\end{align*}
%
\deff Naj bo $f: [a, b] \to \RR$ omejena funkcija. Pravimo, da je $f$ \emph{Darbouxovo integrabilna} na $[a, b]$, "ce velja
\begin{equation*}
s = S
\end{equation*}
V tem primeru "stevilo $s$ imenujemo \emph{Darbouxov integral} funkcije $f$ na $[a, b]$.

\textsc{Trditev 3:} Omejena funkcija $f$ na $[a, b]$ je Darbouxovo integrabilna natanko tedaj, kadar za vsak $\varepsilon > 0$ obstaja delitev $D$, za katero velja
\begin{equation*}
S(D) - s(D) < \varepsilon
\end{equation*}
\textsc{Dokaz:}
\begin{itemize}
	\item[($\Leftarrow$)] Izberemo poljuben $\varepsilon$. Po predpostavki obstaja delitev $D$, da velja
	\begin{equation*}
	S(D) - s(D) < \varepsilon
	\end{equation*}
	Po definiciji velja
	\begin{equation*}
	s(D) \leq s \leq S \leq S(D)
	\end{equation*}
	Iz tega sledi, da je $\underbrace{S - s}_{\geq 0} < \varepsilon$ za poljuben $\varepsilon > 0$. Torej je $s = S$.
	
	\item[($\Rightarrow$)] Denimo, da je $f$ Darbouxovo integrabilna, torej $s = S$. Izberemo poljuben $\varepsilon > 0$
	
	Obstaja delitev $D_1: s(D_1) \in (s-\varepsilon, s]$
	
	Obstaja delitev $D_2: s(D_2) \in [S, S+ \varepsilon]$
	
	Naj bo $D = D_1 \cup D_2$. 
	\begin{equation*}
	\underbrace{S(D)}_{\leq S(D_2)} - \underbrace{s(D)}_{\geq S(D_1)} \leq S(D_2) - s(D_1) < \varepsilon
	\end{equation*}
\end{itemize}
\hfill $\square$

\textsc{Izrek 4:} Zvezna funkcija $f: [a, b] \to \RR$ je Darbouxovo integrabilna.

\textsc{Dokaz:}
\begin{multline*}
S(D) - s(D) = \\
= \sum_{i = 1}^{n} M_i \delta_i - \sum_{i = 1}^{n} m_i \delta_i = \sum_{i=1}^{n} \delta_i (\underbrace{M_i - m_i}_{\leq \frac{\varepsilon}{b-a}}) \leq \\
\leq \sum_{i = 1}^{n} \dfrac{\varepsilon}{b-a} \delta_i = \varepsilon 
\end{multline*}
Ker je $f$ zvezna na $[a, b]$, je na $[a, b]$ enakomerno zvezna. Torej za $\frac{\varepsilon}{b-a}$ obstaja $\delta > 0: |x - x'| < \delta \Rightarrow |f(x) - f(x')| < \frac{\varepsilon}{b-a}$ za $x, x' \in [a, b]$.

Izberemo delitev $D$, da je $\delta(D) < \delta$. Potem velja $M_i - m_i \leq \frac{\varepsilon}{b-a}$.

\hfill $\square$

\textsc{Izrek 5:} Vsaka monotona funkcija na $[a, b]$ je Darbouxovo integrabilna.

\textsc{Dokaz:} Denimo, da je $f$ nara"s"cajo"ca. Naj bo $D$ ekvidistan"cna delitev na $n$ enakih delov. Potem velja
\begin{multline*}
S(D) - s(D) = \\
= \sum_{i = 1}^{n} M_i \delta_i - \sum_{i= 1}^{n} m_i \delta_i = \sum_{i = 1}^{n}f(x_i) \delta_i - \sum_{i = 1}^{n}f(x_{i-1}) \delta_i = \\
= \dfrac{b-a}{n} \left( \sum_{i = 1}^{n} f(x_i) - \sum_{i = 1}^{n} f(x_{i-1}) \right) = \\
= (f(b) - f(a)) \dfrac{b-a}{n} < \varepsilon
\end{multline*}
za dovolj velike $n$.

\hfill $\square$

\textsc{Izrek (aditivnost domene):} Naj bo $f$ omejena funkcija na $[a, b]$ in $c \in [a, b]$. $f$ je Darbouxovo integrabilna na $[a, b]$ natanko tedaj, kadar je Darbouxovo integrabilna na $[a, c]$ in $[c, b]$.

\textsc{Dokaz:}
\begin{itemize}
	\item[($\Rightarrow$)] Ker je $f$ Darbouxovo integrabilna, za poljuben $\varepsilon > 0$ obstaja delitev $D$ intervala $[a, b]$, da velja
	\begin{equation*}
	S(D) - s(D) < \varepsilon
	\end{equation*}
	Ozna"cimo $\widetilde{D} = D \cup \{c\}$ je delitev intervala $[a, b]$. Vemo
	\begin{equation*}
	S(\widetilde{D}) - s(\widetilde{D}) \leq S(D) - s(D) < \varepsilon
	\end{equation*} 
	Ozna"cimo $D_1 = \{x_0, x_1, \ldots, x_k\}$ je delitev intervala $[a, c]$ in $D_2 = \{x_k, x_{k+1}, \ldots, x_n\}$ je delitev $[c, b]$, kjer je $x_0 = a, x_k = c, x_n = b$. Potem velja
	\begin{gather*}
	S(\widetilde{D}) - s(\widetilde{D}) = S(D_1) + S(D_2) - s(D_1) - s(D_2) = \\
	=(S(D_1) - s(D_1)) + (S(D_2) - s(D_2)) < \varepsilon
	\end{gather*}
	Torej velja $S(D_i) - s(D_i) < \varepsilon$ za $i = 1, 2$. Zato je $f$ integrabilna na $[a, c]$ in $[c, b]$.
	
	\item[($\Leftarrow$)] $f$ je integrabilna na $[a, c]$ in $[c, b]$. Za poljuben $\varepsilon > 0$ obstajata delitev $D_1$ intervala $[a, c]$ in delitev $D_2$ intervala $[c, b]$ da velja
	\begin{equation*}
	S(D_i) - s(D_i) < \varepsilon \quad i = 1, 2
	\end{equation*}
	Ozna"cimo $D = D_1 \cup D_2$ je delitev intervala $[a, b]$ in velja
	\begin{equation*}
	S(D) - s(D) = (\underbrace{S(D_1) - s(D_1)}_{< \varepsilon}) + (\underbrace{S(D_2) - s(D_2)}_{< \varepsilon}) < 2\varepsilon
	\end{equation*}
	Torej je $f$ integrabilna na $[a, b]$.
\end{itemize}
\hfill $\square$

\textbf{Opomba:} Denimo, da je $f$ Darbouxovo integrabilna na $[a, b]$. "Ce se funkcija $\widetilde{f}$ razlikuje od $f$ samo v eni to"cki, potem je tudi $\widetilde{f}$ Darbouxovo integrabilna na $[a, b]$.

\textsc{Posledica:} Naj bo $f$ omejena funkcija na $[a, b]$. "Ce obstajajo to"cke
\begin{equation*}
a = c_0 < c_1 < c_2 < \cdots < c_r = b
\end{equation*}
da je $f$ zvezna na $(c_{i-1}, c_i)$ za vse $i = 1, \ldots, r$, potem je $f$ Darbouxovo integrabilna na $[a, b]$.

\textsc{Posledica:} Vsaka odsekana zvezna funkcija je integrabilna.

\deff Pravimo, da je funkcija $f: [a, b] \to \RR$ \emph{odsekoma zvezna}, "ce je zvezna povsod razen v kon"cno mnogo to"ckan na $[a, b]$ in v to"ckah kjer ni zvezna ima skok, t.j. obstajata leva in desna limita.

\textsc{Trditev:} Naj bo $f$ Darobuxovo integrabilna na $[a, b]$. Potem za vsak $\varepsilon > 0$ obstaja $\delta > 0$, da za vsako delitev $D$ za katero je $\delta(D) < \delta$, velja
\begin{equation*}
S(D) - s(D) < \varepsilon
\end{equation*}
\textsc{Dokaz:} "Ce je $M = m$, ni kaj dokazovati, zato naj bo $M \neq m$. Izberemo poljuben $\varepsilon > 0$. Ker je $f$ integrabilna, obstaja delitev $D_0$, da velja
\begin{equation*}
S(D_0) - s(D_0) < \varepsilon
\end{equation*}
Naj bo $D_0 = \{x_0, x_1, \ldots, x_n\}$. Ozna"cimo $\delta := \frac{\varepsilon}{n(M-m)}$\footnote{na predavanjih smo ozna"cili $\delta := \frac{\varepsilon}{2n(M-m)}$, vendar bodo tako pri"sle lep"se "stevilke. Trust me.} Izberemo delitev $D$, za katero je $\delta(D) < \delta$. Zapi"semo lahko
\begin{equation*}
S(D) - s(D) = \sum_k (M_k - m_k) \delta_k = \sum' + \sum''
\end{equation*}
kjer je $\sum'$ vsota po tistih podintervalih v delitvi $D$, ki ne vsebujejo nobene to"cke $x_i \in D_0$ v svoji notranjosti, $\sum''$ pa je vsota po vseh ostalih podintervalih. 

Za $\sum''$ vemo, da je v njej najve"c $n$ "clenov. Torej lahko zapi"semo
\begin{equation*}
(M_k - m_k) \underbrace{\delta_k}_{\leq \delta(D) < \delta} \leq \underbrace{(M_k - m_k)}_{\leq M - m} \dfrac{\varepsilon}{n (M - m)} \leq \dfrac{\varepsilon}{n}
\end{equation*}
Ker je v $\sum''$ najve"c $n$ "clenov, velja
\begin{equation*}
\sum'' < \varepsilon
\end{equation*}
Za $\sum'$: vemo, da je $D \cup D_0$ finej"sa od $D_0$, zato velja
\begin{equation*}
S(D \cup D_0) - s(D \cup D_0) < \varepsilon
\end{equation*}
Po drugi strani bodo vsi "cleni iz vsote $\sum'$ vsebovani v $S(D \cup D_0) - s(D \cup D_0)$, zato velja
\begin{equation*}
\sum' < S(D \cup D_0) - s(D \cup D_0) < \varepsilon
\end{equation*}
zato je $\sum' < \varepsilon$. Torej velja
\begin{equation*}
\sum = \sum' + \sum'' < 2\varepsilon
\end{equation*}
\hfill $\square$

\textsc{Izrek:} Naj bo $f$ omejena funkcija na $[a, b]$. Tedaj je $f$ Riemannovo integrabilna natanko tedaj, kadar je $f$ Darbouxovo integrabilna. V tem primeru sta integrala enaka.

\textbf{Opomba:} Odsel bomo Riemannovo ali Darbouxovo integrailne funkcije imenovali kar \emph{integrabilne funkcije}. Obstajajo "se druge definicije integrabilnosti, ki se ne ujemajo z na"so definicijo.

\textsc{Dokaz:}
\begin{itemize}
	\item[($\Leftarrow$)] Denimo, da je $f$ Darbouxovo integrabilna. Izberemo poljuben $\varepsilon > 0$. Obstaja $\delta > 0$, da za vsako delitev $D$, kjer je $\delta(D) < \delta$ velja
	\begin{equation*}
	S(D) - s(D) < \varepsilon
	\end{equation*}
	Dokazujemo, da za poljuben $\varepsilon > 0$ obstaja $\delta > 0$, da za vsako delitev $D$, kjer je $\delta(D) < \delta$ velja, da za poljubno izbiro usklajenih testnih to"ck $T_D$, velja
	\begin{equation*}
	|\underbrace{I}_{S = s} - R(f, D, T_D)| < \varepsilon
	\end{equation*}
	Vemo, da za vsako usklajeno izbiro $T_D$ velja
	\begin{equation*}
	s(D) \leq R(f, D, T_D) \leq S(D)
	\end{equation*}
	in vemo
	\begin{equation*}
	s(D) \leq s = S \leq S(D)
	\end{equation*}
	Sledi
	\begin{equation*}
	|S - R(f, D, T_D)| < \varepsilon
	\end{equation*}
	za vsako uklajeno izbiro $T_D$. Torej ej $f$ Riemannovo integrabilna in $I = S = s$.
	
	\item[($\Rightarrow$)] Denimo, da je $f$ Riemannovo integrabilna in ozna"icmo z $I$ njen Riemannov integral. Za vsak $\varepsilon > 0$ obstaja $\delta > 0$, da za vsako delitev $D$, kjer je $\delta(D) < \delta$ velja
	\begin{equation*}
	|I - R(f, D, T_D)| < \varepsilon
	\end{equation*}
	za vse usklajene izbire $T_D$.
	
	Ideja: $T_D$ dobimo tako, da je $f(t_k) \approx M_k$ in potem bo $R(f, D, T_D) \approx S(D)$.
	
	Izberemo poljuben $\varepsilon > 0$ in dobimo $\delta$ iz definicije Riemannovega itnegrala. Naj bo $D$ delitev, da je $\delta(D) < \delta$. Za vsak $k$ obstajata
	\begin{align*}
	t_k &\in [x_{k-1}, x_k]: 0 \leq M_k - f(t_k) < \dfrac{\varepsilon}{b-a} \\
	s_k &\in [x_{k-1}, x_K]: 0 \leq f(t_k) - m_k < \dfrac{\varepsilon}{b-a}
	\end{align*}
	Torej velja
	\begin{multline*}
	|S(D) - I| = \left| \sum_k M_k \delta_k - I \right| = \\
	= \left| \sum_k f(t_k)\delta_k - I + \sum_k(M_k - f(t_k))\delta_k \right| \leq \\
	\leq \underbrace{\left| \sum_k f(t_k)\delta_k - I \right|}_{< \varepsilon} + \overbrace{\sum_k (\underbrace{M_k - f(t_k)}_{< \frac{\varepsilon}{b-a}})\delta(k)}^{< \varepsilon} < 2 \varepsilon
	\end{multline*}
	Na podoben na"cin dobimo $|s(D) - I| < 2 \varepsilon$. Sledi
	\begin{equation*}
	S(D) - s(D) < 4 \varepsilon
	\end{equation*}
\end{itemize}
\hfill $\square$

\textsc{Izrek:} Naj bo funkcija $f$ integrabilna na $[a, b]$. Ozna"cimo $m = \inf_{[a, b]} f$, $M = \sup_{[a, b]}f$ in naj bo $g$ zvezna na $[m, M]$. Potem je $g \circ f$ integrabilna na $[a, b]$.

\textsc{Posledica:} "Ce je $f$ integrabilna na $[a, b]$, potem so $|f|, f^n (n \in \NN)$ integrabilne.

\textsc{Dokaz:} Izberemo poljuben $\varepsilon > 0$. Ker je $g$ enakomerno zvezna na $[m, M]$, obstaja $\delta > 0$, da za $u, u' \in [m, M]$ velja
\begin{equation*}
|u - u'| < \delta \Rightarrow |g(u) - g(u')| < \varepsilon
\end{equation*}
Ker je $f$ integrabilna na $[a, b]$, obstaja delitev $D$ intervala $[a, b]$, da velja
\begin{align*}
S_f(D) - s_f(D) &< \varepsilon \delta \\
S_f(D) - s_f(D) &= \sum_k (M_k - m_k) \delta_k = \sum'(M_k - m_k) \delta_k + \sum''(M_k - m_k) \delta_k
\end{align*}
kjer je so v $\sum'$ "cleni za katere velja $M_k - m_k < \delta$, v $\sum''$, pa so "cleni za katere velja $M_k - m_k \geq \delta$.

Ozna"cimo
\begin{align*}
\overline{m_k} &= \inf \{g \circ f (x): x \in [x_{k-1}, x_k]\} \\
\overline{M_k} &= \sup \{g \circ f (x): x \in [x_{k-1}, x_k]\}
\end{align*}
Ker je $M_k - m_k < \delta$, zaradi enakomerne zveznosti velja, da je $\overline{M_k} - \overline{m_k} \leq \varepsilon$. Torej velja
\begin{equation*}
S_{g \circ f} (D) - s_{g \circ f}(D) = \sum_k'(\overline{M_k} - \overline{m_k}) \delta_k \leq \varepsilon \sum_k \delta_k \leq \varepsilon (b-a)
\end{equation*}
Za $\sum''$ velja
\begin{equation*}
\sum' + \sum'' < \varepsilon \delta \Rightarrow \sum'' (\underbrace{M_k - m_k}_{\geq \delta}) \delta_k < \varepsilon \delta
\end{equation*}
Iz tega sledi
\begin{gather*}
\sum'' \delta \delta_k \leq \sum'' (M_k - m_k) \delta_k < \varepsilon \delta \\
\delta \sum'' \delta_k \leq \varepsilon \delta \\
\sum '' \delta_k \leq \varepsilon
\end{gather*}
Po drugi strani velja
\begin{equation*}
\sum'' (\overline{M_k} - \overline{m_k}) \delta_k \leq \sum''(\overline{M} - \overline{m}) \delta_k = (\overline{M} - \overline{m}) \sum''\delta_k \leq (\overline{M} - \overline{m}) \varepsilon
\end{equation*}
"Ce to dvoje zdru"zimo dobimo
\begin{multline*}
S_{g \circ f}(D) - s_{g \circ f}(D) = \sum_k(\overline{M_k} - \overline{m_k}) \delta_k  = \\
= \underbrace{\sum' (\overline{M_k} - \overline{m_k}) \delta _k }_{\leq \varepsilon (b-a)} + \underbrace{\sum'' (\overline{M_k} - \overline{m_k}) \delta_k}_{\leq \varepsilon(\overline{M} - \overline{m})} \leq \\
\leq \varepsilon((b-a) + (\overline{M} - \overline{m}))
\end{multline*}
Torej je $g \circ f$ integrabilna na $[a, b]$.

\hfill $\square$
%
\subsubsection{Lastnosti dolo"cenega integrala}
\textsc{Trditev:} Naj bosta $f$ in $g$ integrabilni funkciji na $[a, b]$ in naj bo $\lambda \in \RR$. Potem velja:
\begin{enumerate}[(i)]
	\item $f+g$ in $f-g$ sta integrabilni na $[a, b]$ in velja
	\begin{align*}
	\int_{a}^{b} (f(x) + g(x)) dx &= \int_{a}^{b} f(x)dx + \int_{a}^{b} g(x)dx \\
	\int_{a}^{b} (f(x) - g(x)) dx &= \int_{a}^{b} f(x)dx - \int_{a}^{b} g(x)dx
	\end{align*}
	
	\item $\lambda f$ je integrabilna na $[a, b]$ in velja
	\begin{equation*}
	\int_{a}^{b} \lambda f(x) dx = \lambda \int_{a}^{b} f(x) dx
	\end{equation*}
	
	$\Rightarrow$ integrabilne funkcije na $[a, b]$ sestavljajo vektorski prostor.
	
	\item $f \cdot g$ je integrabilna na $[a, b]$.
	
	\item "Ce je $f(x) \leq g(x)$ za vse $x \in [a, b]$, potem velja
	\begin{equation*}
	\int_{a}^{b} f(x) \leq \int_{a}^{b} g(x)
	\end{equation*}
	Tej lastnosti pravmo \emph{monotonost integrala}. Posledica je $\forall x \in [a, b]: f(x) \geq x \Rightarrow \int_{a}^{b} f(x) dx \geq 0$
	
	\item 
	\begin{equation*}
	\left| \int_{a}^{b} f(x) dx \right| \leq \int_{a}^{b} \left| f(x) \right| dx
	\end{equation*}
	
	\item Naj bo $c \in (a, b)$. Potem velja
	\begin{equation*}
	\int_{a}^{b} f(x) dx = \int_{a}^{c} f(x) dx + \int_{c}^{b} f(x) dx
	\end{equation*}
\end{enumerate}
%
\textsc{Dogovor:} 
\begin{enumerate}
	\item Naj bo $f$ integrabilna na $[a, b]$. Velja
	\begin{equation*}
	\int_{b}^{a} f(x) dx = - \int_{a}^{b} f(x) dx
	\end{equation*}
	
	\item 
	\begin{equation*}
	\int_{a}^{a} f(x) dx = 0
	\end{equation*}
\end{enumerate}
\textsc{Posledica dogovora:} "Ce je $f$ integrabilna na $I$ in vzememo $a, b, c \in I$, potem velja
\begin{equation*}
\int_{a}^{b} f(x) dx = \int_{a}^{c} f(x) dx + \int_{c}^{b} f(x) dx
\end{equation*}
%
\textsc{Dokaz:}
\begin{enumerate}[(i)]
	\item 
	\begin{gather*}
	R(f+g, D, T_D) = R(f, T, T_D) + R(g, D, T_D) \\
	\sum(f+g) (t_k) \delta_k = \sum f(t_k) \delta_k + \sum g(t_k) \delta_k
	\end{gather*}
	Ker $R(f, D, T_D)$ limitira proti $\int_{a}^{b} f(x) dx$ in $R(g, D, T_D)$ limitira proti $\int_{a}^{b} g(x) dx$, tudi njuna vsota limitira proti $\int_{a}^{b} (f+g) (x) dx$. Da je vsota limit enaka limiti vsot je potrebno "se dokazati (bilo je za DN). Osnovna ideja je, da naredimo podobno kot pri zaporedjih ($\lim_{n \to \infty} (a_n + b_n) = \lim_{n\to \infty} a_n + \lim_{n \to \infty} b_n$).
	
	Podobno doka"zemo razliko.
	
	\item Doka"zemo na enak na"cin kot to"cko (i).
	
	\item Dokazujemo \dashuline{$f g$ je integrabilna}. Vemo, da sta $f$ in $g$ integrabilni. Torej velja:
	\begin{equation*}
	\underbrace{(\underbrace{f + g}_{\text{integrabilna po (i)}})^2}_{\text{integrabilna po izreku}} = \underbrace{f^2}_{\text{integrabilna po izreku}} + 2fg + \underbrace{g^2}_{\text{integrabilna po izreku}}
	\end{equation*}
	Zato lahko zapi"semo
	\begin{equation*}
	fg = \underbrace{\frac{1}{2} ( \underbrace{(f+g)^2 - f^2 - g^2}_\text{integrabilna po (i)})}_{\text{integrabilna po (ii)}}
	\end{equation*}
	Torej je tudi $fg$ integrabilna.
	
	\item $f(x) \leq g(x)$ za vse $x \in [a, b]$. Sledi $\underbrace{R(f, D, T_D)}_{\to I} \leq \underbrace{R(g, D, T_D)}_{\to J}$ za isto velja. Zato velja zveza v limiti.
	
	Dokazati moramo, da je $I \leq J$ ne glede na izbiro delitve. Ideja dokaza je, da doka"zemo, da za vsak $\varepsilon > 0$ velja $J - I \geq - \varepsilon$ ne glede na delitev.
	
	\item Ker je $f$ integrabilna je $|f|$ integrabilna. Velja
	\begin{gather*}
	-|f(x)| \leq f(x) \leq |f(x)| \\
	\underbrace{- \int_{a}^{b} |f(x)| dx}_{-I} \leq \int_{a}^{b} f(x) dx \leq \underbrace{\int_{a}^{b} |f(x)| dx}_I
	\end{gather*}
	Iz tega sledi
	\begin{equation*}
	\left| \int_{a}^{b} f(x) dx \right| \leq I = \int_{a}^{b} |f(x)| dx
	\end{equation*}
	
	\item Ta lastnost je posledica aditivnosti domene.
\end{enumerate}
%
\subsubsection{Povpre"cna vrednost}
Do sedaj vemo, da "ce so $x_1, x_2, \ldots, x_n \in \RR$, je povpre"cna vrednost $\frac{1}{n} (x_1 + x_2 + \cdots + x_n)$. "Ce za $[a, b]$ vzamemo $[0, 1]$ in za $D$ vzamemo ekvidistan"cno delitev, da je $\delta_k  = \frac{1}{n}$, potem velja
\begin{equation*}
R(f, D, T_D) = \frac{1}{n} (f(t_1) + f(t_2) + \cdots + f(t_n))
\end{equation*}
%
\deff Naj bo $f$ integrabilna funkcija na $[a, b]$. \emph{Povpre"cna vrednost} funkcije $f$ na $[a, b]$ je
\begin{equation*}
\dfrac{1}{b - a} \int_{a}^{b} f(x) d_x = \mu
\end{equation*}
\textbf{Opomba:} Za nenegativne integrabilne funkcije $f$ je povpre"cna vrednost $\mu$ tisto "stevilo, da je plo"s"cina pod grafom $f$ enaka plo"s"cini pravokotnika z vi"sino $\mu$.

\textsc{Izrek:} Naj bo $f$ integrabilna na $[a, b]$ in naj bo
\begin{align*}
m = \inf f && M = \sup f
\end{align*}
Potem za povpre"cno vrednost $\mu$ funkcije $f$ velja
\begin{equation*}
m \leq \mu \leq M
\end{equation*}
"Ce je $f$ zvezna na $[a, b]$, potem obstaja $c \in [a, b]$, da velja
\begin{equation*}
f(c) = \mu
\end{equation*}
\textsc{Dokaz:} Vemo $m \leq f(x) \leq M$ za vsak $x \in [a, b]$. Zaradi monotonosti dolo"cenega integrala velja
\begin{gather*}
m(b-a) = \int_a^b m dx \leq \int_a^b f(x) dx \leq \int_a^b M dx = M(b-a) \\
m \leq \mu \leq M
\end{gather*}
"Ce je $f$ zvezna na $[a, b]$, potem dose"ze vse vrednosti med $m$ in $M$, torej tudi $\mu$.
%
\subsubsection{Osnovni izrek analize}
\deff Naj bo $f: [a, b] \to \RR$ integrabilna funkcija. Funkcijo $F: [a, b] \to \RR$ definiramo s predpisom
\begin{equation*}
F(x) = \int_a^x f(t) dt
\end{equation*}
in jo imenujemo \emph{integral kot funkcija zgornje meje.}

%
\textsc{Izrek} (osnovni izrek analize):

Naj bo $f$ integrabilna funkcija na $[a, b]$. Tedaj je $F(x) = \int_a^x f(t) dt$ zvezna na $[a, b]$. "Ce je $f$ zvezna v to"cki $x \in (a, b)$, potem je $F$ v to"cki $x$ odvedljiva in $F'(x) = f(x)$.

\textsc{Posledica:} Vsaka zvezna funkcija na $[a, b]$ ima primitivno funkcija na $[a, b]$. Natan"cneje: "ce je $f$ zvezna na $[a, b]$, potem je $F(x) = \int_a^x f(t) dt$ odvedljiva na $[a, b]$ in $F' = f$.

\textsc{Dokaz:} \dashuline{"Ce je $f$ integrabilna je $f$ zvezna.}

Vemo, da je $f$ omejena. Torej obstaja $M \in \RR: |f(x)| \leq M$ za vsak $x \in [a, b]$. Torej velja
\begin{equation*}
F(x) - F(x') = \int_a^x f(t) dt - \int_a^{x'} f(t) dt = \int_{x'}^{x} f(t) dt
\end{equation*}
"Ce je $x > x'$ velja
\begin{equation*}
|F(x) - F(x')| = \left| \int_{x'}^{x} f(t) dt \right| \underbrace{\leq}_{x > x'} \int_{x'}^x |\underbrace{f(t)}_{\leq M} dt| \leq \int_{x'}^{x} M dt = M (x - x') < \varepsilon
\end{equation*}
"Ce je $|x - x'| < \delta = \frac{\varepsilon}{M}$.

"Ce je $x < x'$, potem po podobnem izra"cunu dobimo
\begin{equation*}
|F(x) - F(x')| \leq M(x' - x) < \varepsilon
\end{equation*}
Torej velja, da "ce je $|x - x'| < \frac{\varepsilon}{M}$, potem je $|F(x) - F(x')| < \varepsilon$, zaradi "cesar je $F$ enakomerno zvezna.

Denimo, da je $f$ zvezna v to"cki $x$. Velja
\begin{multline*}
\dfrac{F(x + h) - F(x)}{h} = \dfrac{1}{h} \left( \int_{a}^{x+h} f(t) dt - \int_{a}^{x} f(t) dt \right) = \\
= \dfrac{1}{h} \int_{x}^{x + h} f(t) dt = \dfrac{1}{h} \int_{x}^{x+h} f(x) dt + \dfrac{1}{h} \int_{x}^{x+h} (f(t) - f(x)) dt = \\
= f(x) + \dfrac{1}{h} \int_{x}^{x + h} (f(t) - f(x)) dt
\end{multline*}
Dokazati je potrebno, da je limita $\frac{1}{h} \int_{x}^{x + h} (f(t) - f(x)) dt$ enaka 0.

Ker je $f$ zvezna v to"cki $x$ velja
\begin{equation*}
\forall \varepsilon > 0 \exists \delta > 0: |t - x| < \delta \Rightarrow |f(t) - f(x)| < \varepsilon
\end{equation*}
"Ce je $|h| < \delta$, potem velja $|t - x| < \delta$, ker $t$ le"zi med $x$ in $x + h$. Zato velja
\begin{multline*}
\left| \dfrac{1}{h} \int_{x}^{x+h} (f(t) - f(x)) dt \right| \leq \dfrac{1}{|h|} \int_{x}^{x+h} \underbrace{|f(t) - f(x)|}_{< \varepsilon} dt \leq \\
\leq \dfrac{1}{|h|} \int_{x}^{x+h} \varepsilon dt = \dfrac{1}{h} \varepsilon h = \varepsilon
\end{multline*}
\hfill $\square$

\textsc{Posledica:} Vsaka zvezna funkcija $f$ na $[a, b]$ ima primitivno funkcijo na $[a, b]$. Natan"cneje $F(x) = \int_{a}^{x} f(t)dt$ je odvedljiva in $F'(x) = f(x)$, $F$ je primitivna funkcija od $f$.

\textbf{Opomba:} Nima vsaka integrabilna funkcija primitivne funkcije. Npr.\,funkcija, ki ima skok.

\textsc{Izrek:} (osnovni izrek interalnega ra"cuna - Leibnizova formula)

Najbo $f$ taka integrabilna funkcija na $[a, b]$, ki ima na $[a, b]$ primitivno funkcijo $F$, to je $F' = f$ na $[a, b]$. Potem velja Leibnizova formula
\begin{equation*}
\int_{a}^{b} f(x) dx = F(b) - F(a) = F(X) \rvert_a^b
\end{equation*}
\textsc{Dokaz:}
\begin{enumerate}
	\item "Ce je $f$ zvezna, potem je $F(X) = \int_{a}^{x}F(t)dt$ njena primitivna funkcija
	\begin{equation*}
	F(b) - F(a) = \int_a^b f(t) dt
	\end{equation*}
	izrek za $F$ velja. "Ce je $G$ kak"sna druga primitivna funkcija za $f$, potem velja $F' = G' = f$ na $[a, b]$ in zato obstaja $C \in \RR$, da velja
	\begin{equation*}
	G(X) = F(X) + C
	\end{equation*}
	Sledi
	\begin{equation*}
	G(b) - G(a) = F(b) - F(a)
	\end{equation*}
	torej izrek velja
	
	\item V splo"snem:
	
	Naj bo $D$ poljubna delitev intervala $[a, b]$. Potem po Lagrangeevem izreku obstaja tak $c_i \in (x_{i-1}, x_i)$, da velja
	\begin{equation*}
	F(x_i) - F(x_{i-1}) = F'(c_i) (x_i - x_{i-1}) = f(c_i) (x_i - x_{i-1})
	\end{equation*}
	Zapi"semo lahko tudi
	\begin{equation*}
	F(b) - F(a) = \sum_{i=1}^{n} (F(x_i) - F(x_{i-1}) = \sum_{i=1}^{n} f(c_i) (x_i - x_{i-1}) = R(f, D, T_D)
	\end{equation*}
	Vemo, da je $f$ integrabilna in zato obstaja limita Riemannovih vsot, to je Riemannov integral. Vemo, da si lahko za vsako delitev izberemo testne to"cke za katere bo $F(b) - F(a)$ enaka Riemannovi vsoti. Torej smo na"sli podzaporedje konvergentnega zaporedja Riemannovih vsot, ki konvergira proti $F(b) - F(a)$, torej tudi zaporedje Riemannovih vsto konvergira proti $F(b) - F(a)$, zato je integral enak $F(b) - F(a)$.
	
	\hfill $\square$
\end{enumerate}

\textsc{Primer:} Obstaja funkcija, ki ni zvezna, je integrabilna in ima primitivno funkcijo.
Naj bo $F(x) = x^2 \sin \frac{1}{x}$ primitivna funkcija, $F(0) = 0$. Vemo
\begin{align*}
F'(x) &= 2 x \sin \frac{1}{x} - \cos \frac{1}{x} \\
F'(0) &= \lim_{h \to 0} \dfrac{h^2 \sin \frac{1}{g}}{h} = 0
\end{align*}
Torej je $F$ zvezna in odvedljiva na $\RR$. Definiramo
\begin{align*}
f(x) &= F'(x) \\
f(x) &= \begin{cases}
2x \sin \dfrac{1}{x} - \cos \dfrac{1}{x} & x \neq 0 \\
0 & x = 0
\end{cases}
\end{align*}
$\lim_{x \to 0} f(x)$ ne obstaja, $f$ je zvezna na $\RR \setminus \{0\}$. Ker je omejena na $[a, b]$, je integrabilna. Sledi
\begin{equation*}
\int_{0}^{a} f(x) dx = F(a) - F(0) = F(a)
\end{equation*}
%
\subsubsection{Uvedba nove spremenljivke in integracija po delih v dolo"cenem integralu}
\textsc{Izrek:}
\begin{enumerate}[(i)]
	\item  Naj bo $\varphi: [a, b] \to \RR$ zvezno odvedljiva funkcija in $f: Z_\varphi \to \RR$ zvezna funkcija. Potem velja
	\begin{equation*}
	\int_{a}^{b} f(\varphi(t)) \varphi'(t) dt = \int_{\varphi(a)}^{\varphi(b)} f(x) dx
	\end{equation*}
	
	\item Naj bosta $f, g: [a, b] \to \RR$ zvezno odvedljivi funkciji. Potem velja
	\begin{equation*}
	\int_a^b f(x) g'(x) dx = (f(x) g(x)) \rvert_a^b - \int_a^b f'(x) g(x) dx
	\end{equation*}
\end{enumerate}
\textsc{Dokaz:}
\begin{enumerate}[(i)]
	\item Naj bo $F$ primitivna funkcija od $f$ (obstaja, ker je $f$ zvezna). Potem je
	\begin{equation*}
	G(t) = F(\varphi(t))
	\end{equation*}
	zvezno odvedljiva funkcija. Torej je
	\begin{equation*}
	G'(t) = F'(\varphi(t))\varphi'(t) = f(\varphi(t)) \varphi'(t)
	\end{equation*}
	zvezna funkcija. Zato je $G$ primitivna funkcija od $(f \circ \varphi) \varphi'$ in velja
	\begin{equation*}
	\int_a^b f(\varphi(t)) \varphi'(t) dt = G(b) - G(a) = F(\varphi(b)) - F(\varphi(a)) = \int_{\varphi(a)}^{\varphi(b)} f(x) dx
	\end{equation*}
	
	\item Velja
	\begin{equation*}
	(f(x) g(x))' = f'(x)g(x) + f(x) g'(x)
	\end{equation*}
	Zato je $fg$ primitivna funkcija od $f'g + fg'$, odtod po Leibnizovi formuli
	\begin{equation*}
	\int_a^b (f'(x)g(x) + f(x) g'(x))dx = (f(x) g(x)) \rvert_a^b
	\end{equation*}
	\hfill $\square$
\end{enumerate}
%
\textsc{Izrek:} Naj bo $\varphi$ zvezno odvedljiva, nara"s"cajo"ca funkcija na $[a, b]$ in $f$ integrabilna na $[\varphi(a), \varphi(b)]$. Tedaj je funkcija $(f \circ \varphi) \varphi'$ integrabilna na $[a, b]$ in velja
\begin{equation*}
\int_{\varphi(a)}^{\varphi(b)} f(x) dx = \int_{a}^{b} f(\varphi(t)) \varphi'(t) dt
\end{equation*}
\textsc{Dokaz:} $f$ je integrabilna, zato za vsak $\varepsilon > 0$ obstaja $\delta > 0$, da za vsako delitev $D$, $\delta(D) < \delta$, velja
\begin{equation*}
\left| \underbrace{\int_{\varphi(a)}^{\varphi(b)} f(x) dx}_I - R(f, D, T_D) \right| < \varepsilon
\end{equation*}
za vsako usklajeno izbiro $T_D$.

Delitev $D$ intervala $[\varphi(a), \varphi(b)]$ dolo"ca delitev $\{t_0, t_1, \ldots, t_n\}$ intervala $[a, b]$ (ker je $\varphi$ nara"s"cajo"ca).

Ker je $\varphi$ zvezna na $[a, b]$, je enakomerno zvezna na $[a, b]$, zato za vsak $\delta > 0$ obstaja $\delta' > 0$ da za $t, t' \in [a, b]$ velja
\begin{equation*}
|t - t'| < \delta' \Rightarrow |\varphi(t) - \varphi(t')| < \delta
\end{equation*}
Oglejmo si Riemannovo vsoto $\int_a^b f(\varphi(t)) \varphi'(t) dt$.

Naj bo $\overline{D}$ delitev intervala $[a, b]$. Velja
\begin{equation*}
R((f \circ \varphi) \varphi', \overline{D}, T_{\overline{D}}) = \sum_{j = 1}^{n} f(\varphi(c_j)) \varphi'(c_j) (t_j - t_{j-1})
\end{equation*}
Po Lagrangeevem izreku obstaja tak $\overline{c_j} \in (t_{j-1}, t_j)$, da velja
\begin{equation*}
\varphi(t_j) - \varphi(t_{j-1}) = \varphi(\overline{c_j}) (t_j - t_{j-1})
\end{equation*}
Upo"stevajmo "se enakomerno zveznost funkcije $\varphi'$. Obstaja $\delta'' > 0$, da za $t, t' \in [a, b]$ velja
\begin{equation*}
|t - t'| < \delta'' \Rightarrow |\varphi'(t) - \varphi'(t')| < \varepsilon
\end{equation*}
Ozna"cimo "se $M = \sup |f|$. Sedaj lahko zapi"semo
\begin{multline*}
R((f \circ \varphi) \varphi', \overline{D}, T_{\overline{D}}) = \sum_{j=1}^n f(\varphi(c_j)) \varphi'(c_j) (t_j - t_{j-1}) = \\
= \sum_{j=1}^n f(\varphi(c_j)) \varphi'(\overline{c_j}) (t_j - t_{j-1}) - \underbrace{\sum_{j=1}^n f(\varphi(c_j)) (t_j - t_{j-1}) (\varphi'(c_j) - \varphi'(\overline{c_j}))}_o = \\
= \sum_{j=1}^n f(\varphi(c_j))(\varphi(t_j) - \varphi(t_{j-1})) + o = \\
= R \left( f, \{\varphi(t_1), \ldots, \varphi(t_n)\}, \{\varphi(c_1), \ldots, \varphi(c_n)\} \right) + o
\end{multline*}
Poglejmo kako je z $o$
\begin{equation*}
|o| \leq \sum_{j=1}^n \underbrace{|f(\varphi(c_j))|}_{\leq M} \cdot |t_j - t_{j-1}| \cdot \underbrace{|\varphi'(c_j) - \varphi'(\overline{c_j})|}_{\text{$\leq \varepsilon$ "ce $|c_j - \overline{c_j}| < \delta''$}} \leq \varepsilon M \sum_{j=1}^n |t_j - t_{j-1}| = \varepsilon M (b-a)
\end{equation*}
Iz tega sledi
\begin{multline*}
|R((f \circ \varphi) \varphi', \overline{D}, T_{\overline{D}}) -I| = |R(f, D, T_D) +o -I| \leq \\
 \leq \underbrace{|R(f, D, T_D) -I|}_{< \varepsilon} + \underbrace{|o|}_{\leq \varepsilon M (b-a)} \leq \varepsilon (1 + M (b-a))
\end{multline*}
\hfill $\square$