\deff Naj bo $D \subseteq \RR$ in $f_n : D \to \RR$ funkcije za vsak $n \in \NN$. Potem pravimo, da je
\begin{equation*}
\{f_n\} = \{f_n: D \to \RR\}
\end{equation*}
\emph{funkcijsko zaporedje}.

"Ce za vsak $x \in D$ "stevilsko zaporedje $\{f_n(x)\}$ konvergira, pravimo da funkcijsko zaporedje $f_n$ \emph{konvergira} na $D$ (konvergira po to"ckah). V tem primeru definirmao funkcijo $f: D \to \RR$ s predpisom
\begin{equation*}
f(x) = \lim_{n \to \infty} f_n(x)
\end{equation*}
in jo imenujemo \emph{limitna funkcija}.

\deff Naj bo $D \subseteq \RR$ in $\{f_n: D \to \RR \}$ funkcijsko zaporedje. Pravimo, da $\{f_n\}$ \emph{konvergira enakomerno} proti funkciji $f: D \to \RR$ na $D$, "ce za vsak $\varepsilon > 0$ obstaja $n_0 \in \NN$, da za vse $n \geq n_0$ in vse $x \in D$ velja $|f_n (x) - f(x)| < \varepsilon$.

\textbf{Opomba:} "Ce funkcijsko zaporedje $\{ f_n \}$ konvergira enakomerno na $D$ proti $f$, potem $\{ f_n \}$ konvergira proti $f$, t.j.\,$f$ je limitna funkcija $\{ f_n \}$.

$f$ je limitna funkcija $\{ f_n \} \iff \forall x \in D: f(x) = \lim_{n \to \infty} f_n(x) \iff$
\begin{equation*}
\forall x \in D \forall \varepsilon > 0 \exists n_0 \in \NN \forall n \geq n_0 : |f_n(x) - f(x)| < \varepsilon
\end{equation*}
$f_n$ enakomerno konvergira proti $f$ na $D \iff$
\begin{equation*}
\forall \varepsilon > 0 \exists n_0 \in \NN \forall n \geq n_0 \forall x \in D: |f_n(x) - f(x)| < \varepsilon
\end{equation*}

V definiciji enakomerne konvergence ozna"cimo
\begin{equation*}
M_n := \sup_{x \in D} |f_n (x) - f(x)|
\end{equation*}
Za obravnavo enakomerne konvergence moramo obravnavati $M_n$, ki pa je "stevilsko zaporedje. $f_n$ enakomerno konvergira proti $f$ natanko takrat, kadar zaporedje $M_n$ konvergira proti 0.

\textbf{Geometrijska interpretacija:} $M_n < \varepsilon \iff$ graf funkcije $f_n$ le"zi v $\varepsilon$-pasu okrog grafa $f$. $f_n \to f$ enakomerno na $D \iff \forall \varepsilon > 0$ vsi grafi funkcije $f_n$ za dovolj velik $n$ le"zijo znotraj $\varepsilon$-pasu okrog grafa $f$.

\deff Naj bo $D \subseteq \RR$ in $\{ f_n : D \to \RR \}$ funckijsko zaporedje. Pravimo, da je $\{ f_n \}$ \emph{enakomerno Cauchyjevo} na $D$, "ce velja
\begin{equation*}
\forall \varepsilon > 0 \exists n_0 \in \NN \forall n, m \geq n_0 \forall x \in D : |f_n(x) - f_m(x)| < \varepsilon
\end{equation*}

\textsc{Izrek:} Naj bo $D \subseteq \RR, \{f_n : D \to \RR \}$ funkcijsko zaporedje. Tedaj je $\{ f_n \}$ enakomerno konvergentno na $D$ natanko takrat kadar je $\{ f_n \}$ enakomerno Cauchyjevo na $D$.

\textsc{Dokaz:} Doka"zemo podobno kot za zaporedja.

\textsc{Izrek:} Naj bo $D \subset \RR$ in $\{ f_n : D \to \RR \}$ funkcijsko zaporedje. "Ce so vse funkcije $f_n$ zvezne na $D$ in $\{ f_n \}$ enakomerno konvergira proti $f$ na $D$, potem je $f$ zvezna na $D$.

\textbf{Opomba:} Protiprimer je $f_n (x) = x^n$ na $[0, 1]$.

\textsc{Dokaz:} Dokazujemo, da je $f$ zvezna na $D$. Izberemo poljuben $a \in D$ in dokazujemo, da je $f$ zvezna v to"cki $a$. Izberemo $\varepsilon > 0$ in i"s"cemo tak $\delta > 0$, da $\forall x \in D$ velja $|x - a| < \delta \Rightarrow |f(x) - f(a)| < \varepsilon$.
\begin{multline*}
|f(x) - f(a)| = |f(x) - f_{n_0}(x) + f_{n_0}(x) - f_{n_0}(a) + f_{n_0}(a) - f(a)| \leq \\
\leq \underbrace{|f(x) - f_{n_0}(x)|}_{< \varepsilon/3} + \underbrace{|f_{n_0}(x) - f_{n_0}(a)|}_{< \varepsilon/3 \text{ za $|x - a| < \delta$}} + \underbrace{|f_{n_0}(a) - f(a)|}_{< \varepsilon/3} < \varepsilon
\end{multline*}
Ker $f_n$ enakomerno konvergira proti $f$, obstaja $n_0$, da za vsak $n \geq n_0$ velja, da za vsak $x \in D$
\begin{equation*}
|f_n(x) - f(x)| < \frac{\varepsilon}{3}
\end{equation*}
Ker je funkcija $f_{n_0}$ zvezna v to"cki $a$, obstaja $\delta > 0$, da za vsak $x \in D$ velja
\begin{equation*}
|x - a| < \delta \Rightarrow |f_{n_0}(x) - f_{n_0}(a)| < \dfrac{\varepsilon}{3}
\end{equation*}
\hfill $\square$

\deff Naj bo $D \subset \RR$ in $\{ u_n : D \to \RR \}$ funkcijsko zaporedje.
\begin{equation*}
\sum_{n=1}^{\infty} u_n
\end{equation*}
pravimo \emph{funkcijska vrsta}. Funkcijska vrsta $\sum_{n=1}^\infty$ \emph{konvergira po to"ckah}, "ce za vsak $x \in D$ "stevilska vrsta
\begin{equation*}
\sum_{n=1}^\infty u_n(x)
\end{equation*}
konvergira.

\textbf{Opomba:} Funkcijska vrsta konvergira po to"ckah natanko takrat, kadar funkcijsko zaporedje njenih delnih vsot konvergira po to"ckah. Torej za $x \in D$ vrsta $\sum_{n=1}^\infty u_n(x)$ konvergira $\iff \{ \sum_{n=1}^k u_n (x) \}_k$ konvergira $\iff s_k = \sum_{n=1}^k u_n$ konvergira v $x$.

Naj bo $s$ vsota po to"ckah konvergentne funkcijske vrste $\sum_{n=1}^\infty u_n$, $s = \lim_{k \to \infty} s_k$. Pravimo, da $\sum_{n=1}^\infty$ \emph{konvergira proti $s$ enakomerno na $D$}, "ce funkcijsko zaporedje njenih delnih vsot $s_k = \sum_{n=1}^k u_n$ enakomerno konvergira proti $s$ na $D$.

\textbf{Posledica:} "Ce je $\{ u_n: D \to \RR \}$ funkcijsko zaporedje zveznih funkcij in "ce $\sum_{n=1}^\infty u_n$ konvergira enakomerno na $D$ proti $s$, potem je $s$ zvezna funkcija na $D$.

\textbf{Posledica:} Naj bo $\{ u_n : D \to \RR \}$ funkcijsko zaporedje. Tedaj velja $\sum_{n=1}^\infty u_n$ je enakomerno konvergentna na $D \iff \sum_{n=1}^\infty u_n$ je enakomerno Cauchyjeva na $D$, t.j.:
\begin{equation*}
\forall \varepsilon > 0 \exists n_0 \in \NN \forall n > m \geq n_0 \forall x \in D: \left| \sum_{k=1}^n u_k(x) - \sum_{k=1}^m u_k(x) \right| = \left| \sum_{k = m+1}^n u_k(x) \right| < \varepsilon
\end{equation*}

\textsc{Izrek} (Weierstrassov kriterij za enakomerno konvergenco funkcijskih vrst, M-test):

Naj bo $\{ u_n : D \to \RR \}$ funkcijsko zaporedje. Denimo, da obstaja konvergentna "stevilska vrsta $\sum_{n=1}^\infty c_n$ s pozitivnimi "cleni, za katero velja
\begin{equation*}
|u_n(x)| \leq c_n \quad \forall x \in D
\end{equation*}
Potem funkcijska vrsta $\sum_{n=1}^\infty u_n$ konvergira enakomerno (in absolutno) na $D$. "Ce se $u_n$ zvezne na $D$, potem je tudi vsota vrste zvezna na $D$.

\textsc{Dokaz:} Iz $|u_n(x)| \leq c_n \forall x \in D$ z uporabo primerjalnega kriterija sledi, da $\sumalln{1} |u_n(x)|$ konvergira za vsak $x$. Torej vrsta $\sumalln{1} u_n$ po to"ckah absolutno konvergira, zato konvergira po to"ckah.

Doka"zimo, da je $\sumalln{1} u_n$ enakomerno Cauchyjeva. Naj bo $k > m$, potem za vsak $x \in D$ velja
\begin{equation*}
\left| \sum_{n=1}^ku_n(x) - \sum{n=1}^m u_n(x) \right| = \left| \sum_{n= m+1}^{k} u_n(x) \right| \leq
 \sum_{n = m+1}^k |u_n(x)| \leq \sum_{n = m+1}^k c_n < \varepsilon
\end{equation*}
Konvergentna "stevilska vrsta je Cauchyjeva, zato obstaja $n_0$, da za $k > m \geq n_0$ velja $\sum_{n = m+1}^k c_n < \varepsilon$

\subsection{Integriranje in odvajanje funkcijskih zaporedij in vrst}
\textsc{Izrek:} Naj bo $\{ f_n : [a, b] \to \RR \}$ funkcijsko zaporedje zveznih funkcij. "Ce funkcijsko zaporedje $\{ f_n \}$ konvergira proti funkciji $f: [a, b] \to \RR$ enakomernona $[a, b]$, potem velja
\begin{equation*}
\lim_{n \to \infty} \int_a^b f_n (x) dx = \int_a^b f(x) dx
\end{equation*}
\textsc{Dokaz:} Vemo, da je limitna funkcija zvezna na $[a, b]$. $f_n$ konvergira proti $f$ enakomerno na $[a, b]$:
\begin{equation*}
\forall \varepsilon > 0 \exists n_0 \in \NN \forall n \geq n_0 \forall x \in [a, b] : |f_n(x) - f(x)| < \varepsilon
\end{equation*}
\begin{multline*}
\left| \int_a^b f_n (x) dx - \int_a^b f(x) dx \right| = \left| \int_a^b (f_n(x) - f(x)) dx \right| \leq \\
\leq \int_a^b | f_n(x) - f(x) | dx < \int_a^b \varepsilon dx = \varepsilon (b-a)
\end{multline*}
Torej zaporedje konvergira za $n \geq n_0$.

\textsc{Posledica:} Naj bo $\{ u_n : [a, b] \to \RR \}$ funkcijsko zaporedje zveznih funkcij. Denimo, da $\sum_{n=1}^\infty u_n$ enakomerno konvergira na $[a, b]$. Potem velja
\begin{equation*}
\int_a^b \left( \sumalln{1} u_n (x) \right) dx = \sumalln{1} \left( \int_a^b u_n (x) dx \right)
\end{equation*}
\textbf{Opomba:} Enakomerno konvergentno funkcijsko vrsto iz zveznih funkcij lahko "clenoma integriramo.

\textsc{Dokaz:} (skica) Funkcijsko zaporedje delnih vsot enakomerno konvergira in uporabimo prej"snji izrek.

\textsc{Izrek:} Naj bo $\{ f_n : [a, b] \to \RR \}$ funkcijsko zaporedje zvezno odvedljivih funkcij. Denimo, da $\{ f_n' \}$ enakomerno konvergira na $[a, b]$ proti funkciji $g: [a, b] \to \RR$ in denimo, da obstaja $c \in [a, b]$, da $\{ f_n (c) \}$ konvergira. Potem $\{ f_n \}$ konvergira enakomerno na $[a, b]$ k neki funkciji $f$ in velja
\begin{equation*}
f' (x) = \lim_{n \to \infty} f_n'(x)
\end{equation*}
\textbf{Opomba:} Pri odvajanju izgubimo podatek o konstanti, zato potrebujemo da $\{ f_n(c) \}$ konvergira za nek $c$. Primer $f_n(x) = n$.

\textsc{Dokaz:} Ker je $f_n'$ zvena funkcija, $f_n(x) = f_n(c) + \int_c^x f_n'(t) dt$ za vse $x \in [a, b]$. Po prej"snjem izreku $\int_a^b f_n'(t) dt$ konvergira in celotna funkcija konvergira proti
\begin{equation*}
\left( \lim_{n \to \infty} f_n (c) = L + \int_c^x g(t) dt \right) = f(x)
\end{equation*}

Torej za vsak $x \in [a, b]$ obstaja $\lim_{n \to \infty} f_n (x)$. Zato je $f_n$ konvergentna po to"ckah.

Dokazati moramo "se,  da $f_n$ enakomerno konvergira. Izberemo poljuben $\varepsilon > 0$:
\begin{multline*}
|f_n(x) - f(x)| = \\
= \left| f_n(c) + \int_c^x f_n'(t) dt - \lim_{n \to \infty} f_n(c) - \int_c^x g(t) dt \right| = \\
= \left| f_n(c) - \lim_{n \to \infty} f_n (c) + \int_c^x (f_n'(t) - g(t)) dt \right| \leq \\
\leq \underbrace{|f_n - \lim_{n \to \infty} f_n(c)|}_{< \varepsilon \text{ za velik $n$}} + \int_c^x \underbrace{|f_n'(t) - g(t)|}_{< \varepsilon \text{ za velik $n$}} dt < \varepsilon + \varepsilon(b-a)
\end{multline*}
\hfill $\square$

\textsc{Posledica:} Naj bo $\{ u_n: [a, b] \to \RR \}$ funkcijsko zaporedje zvezno enakomernih funkcij. Denimo, da $\sumalln{0} u_n'$ konvergira enakomerno na $[a, b]$ in da obstaja $c \in [a, b]$, da $\sumalln{1} u_n(c)$ konvergira. Potem $\sumalln{1} u_n$ konvergira enakomerno na $[a, b]$ in velja
\begin{equation*}
\left( \sumalln{n=1} u_n (x) \right)' = \sumalln{1} u_n'(x) \quad \forall x \in [a, b]
\end{equation*}
\textsc{Dokaz:} Prej"senj izrek uporabimo na zaporedju delnih vsot.

\subsection{Poten"cne vrste}
\deff Poten"cna vrsta je vrsta oblike
\begin{equation*}
\sumalln{0} a_n (x-c)^n
\end{equation*}
kjer je $a_n$ "stevilsko zaporedje in $c \in \RR$. Re"cemo tudi poten"cna vrsta s \emph{sredi"s"cem} v $c$.

\textsc{Izrek:} Naj bo $\sumalln{0} a_n (x - c)^n$ poten"cna vrsta. Obstaja $R \in [0, \infty) \cup \{ \infty \}$ z naslednjo lastnostjo:
\begin{itemize}
    \item $\forall x, |x - c| < R$,  je vrsta $\sumalln{0} a_n (x - c)^n$ konvergentna in absolutno konvergentna
    \item $\forall x, |x - c| > R$, je vrsta $\sumalln{0} a_n (x - c)^n$ divergentna.
\end{itemize}
$R$ imanujemo \emph{konvergen"cni polmer}. "Ce je $r \in (0, R)$, potem poten"cna vrsta na $[c - r, c + r]$ enakomerno konvergira.
\textsc{Dokaz:} za $c = 0$.

Denimo, da poten"cna vrsta $\sumalln{0} a_n x^n$ konvergira pri $x = x_0 \neq 0$. Naj bo $r \in (0, |x_0|)$. Dokazujemo, da vrsta enakomerno konvergira na $[-r, r]$. Uporabimo Wirestrassov M-test.

Vemo, da $\sumalln{0}a_n x_0^n$ konvergira, zato velja $\lim_{n \to \infty} |a_n x^n| = 0$. Zato obstaja $M \in \RR: |a_n x_0^n| < M$ za vsak $n \in \NN$.
\begin{equation*}
|a_n x^n| = |a_n| |x|^n \leq |a_n| |r|^n \leq \dfrac{M}{|x_0|^n} r^n = M \left( \dfrac{r}{|x_0|} \right)^n
\end{equation*}
$\sumalln{0} M \left( \frac{r}{|x_0|} \right)^n$ je geometrijska vrsta, z $q = \frac{r}{|x_0|} < 1$, zato konvergira. Po M-testu je $\sumalln{0} a_n x^n$ konvergenta in absolutno konvergentna in enakomerno konvergeira na $[-r, r]$. Velja
\begin{equation*}
R = \sup \{ |x_0|, \text{ $x_0$ vrsta konvergira} \}
\end{equation*}
"Ce za nek $y_0$ velja, da $\sum a_n y_0^n$ divergira, potem divergira za vse $y > y_0$. "Ce to ne bi veljalo, bi obstajal $y_1$,  za katerega vrsta konvergira in po prej"snjem dokazu konvergira tudi za vse $y < y_1$, torej tudi $y_0$, kar je protislovje. "Ce ne obstaja $y_0$, za katerega vrsta divergira, potem je $R = \infty$.

\textsc{Posledica:} naj bo $\sumalln{0} a_n (x - c)^n$ poten"cna vrsta s konvergen"nim polmerom $R > 0$.
\begin{enumerate}
    \item Vsota poten"cne vrste je zvezna funkcija na $(c - R, c + R)$ (na $[c-r, c+r]$ konvergira enakomerno za vsak $r < R$.)
    \item Vsoto poten"cne vrste lahko "clenoma integriramo in "clenoma odvajamo na $(c-R, c+R)$. Konvergen"cni polmer se ohrani.
    \begin{gather*}
     \left( \sumalln{0} a_n (x - c)^n \right)' = \sumalln{1} n a_n (x - c)^{n-1} \\
     \int_c^x \left( \sumalln{0} a_n (t - c)^n \right)dt = \sumalln{0} \dfrac{a_n}{n+1} (x-c)^{n+1}
    \end{gather*}
    \item Vsota poten"cne vrste je funkcija $C^\infty ((c-R, c+R))$.
\end{enumerate}

\textsc{Izrek:} Naj bo $\sumalln{0} a_n (x - c)^n$ poten"cna vrsta in $R$ njen konvergen"cni polmer. Potem velja
\begin{enumerate}
    \item $\frac{1}{R} = \lim_{n \to \infty} \frac{|a_{n+1}|}{|a_n|}$, "ce ta limita obstaja
    \item $\frac{1}{R} = \lim_{n \to \infty} \sqrt[n]{|a_n|}$, "ce ta limita obstaja
\end{enumerate}
\textsc{Dokaz:}
\begin{enumerate}
    \item Uporabimo kvocientni kriterij. Obravnavamo absolutno konvergenco $\sumalln{0} a_n (x - c)^n$:
    \begin{gather*}
    d_n = \dfrac{|a_{n+1}| |x-c|^{n+1}}{|a_n| |x-c|^n} = \dfrac{|a_{n+1}|}{|a_n|} |x-c| \\
    \lim_{n \to \infty} d_n = \lim_{n \to \infty} \dfrac{|a_{n+1}|}{|a_n|} |x-c| = |x-c| \underbrace{\lim_{n \to \infty} \dfrac{|a_{n+1}|}{|a_n|}}_\text{obstaja po predp.} = |x-c| L
    \end{gather*}
    Po kvocientnem kriteriju velja
    \begin{itemize}
        \item $|x - c| L < 1 \iff | x - c| < \frac{1}{L}$ vrsta konvergira
        \item $|x - c| L > 1 \iff |x - c| > \frac{1}{L}$ vrsta divergira
    \end{itemize}
    Torej je $\frac{1}{L} = R$ in velja
    \begin{equation*}
    \dfrac{1}{R} = L = \lim_{n \to \infty} \dfrac{|a_{n+1}|}{|a_n|}
    \end{equation*}
    
    \item Podoben dokaz kot za to"cko (1).
\end{enumerate}
\hfill $\square$

\textsc{Izrek:} (Cauchy -- Hadamart) Za konvergen"cni polmer $R$ poten"cne vrste $\sumalln{0} a_n (x - c)^n$ velja
\begin{equation*}
\dfrac{1}{R} = \limsup_{n \to \infty} \sqrt[n]{|a_n|}
\end{equation*}