\deff Naj bo $D \subseteq \RR$ in $f_n : D \to \RR$ funkcije za vsak $n \in \NN$. Potem pravimo, da je
\begin{equation*}
\{f_n\} = \{f_n: D \to \RR\}
\end{equation*}
\emph{funkcijsko zaporedje}.

"Ce za vsak $x \in D$ "stevilsko zaporedje $\{f_n(x)\}$ konvergira, pravimo da funkcijsko zaporedje $f_n$ \emph{konvergira} na $D$ (konvergira po to"ckah). V tem primeru definirmao funkcijo $f: D \to \RR$ s predpisom
\begin{equation*}
f(x) = \lim_{n \to \infty} f_n(x)
\end{equation*}
in jo imenujemo \emph{limitna funkcija}.

\deff Naj bo $D \subseteq \RR$ in $\{f_n: D \to \RR \}$ funkcijsko zaporedje. Pravimo, da $\{f_n\}$ \emph{konvergira enakomerno} proti funkciji $f: D \to \RR$ na $D$, "ce za vsak $\varepsilon > 0$ obstaja $n_0 \in \NN$, da za vse $n \geq n_0$ in vse $x \in D$ velja $|f_n (x) - f(x)| < \varepsilon$.

\textbf{Opomba:} "Ce funkcijsko zaporedje $\{ f_n \}$ konvergira enakomerno na $D$ proti $f$, potem $\{ f_n \}$ konvergira proti $f$, t.j.\,$f$ je limitna funkcija $\{ f_n \}$.

$f$ je limitna funkcija $\{ f_n \} \iff \forall x \in D: f(x) = \lim_{n \to \infty} f_n(x) \iff$
\begin{equation*}
\forall x \in D \forall \varepsilon > 0 \exists n_0 \in \NN \forall n \geq n_0 : |f_n(x) - f(x)| < \varepsilon
\end{equation*}
$f_n$ enakomerno konvergira proti $f$ na $D \iff$
\begin{equation*}
\forall \varepsilon > 0 \exists n_0 \in \NN \forall n \geq n_0 \forall x \in D: |f_n(x) - f(x)| < \varepsilon
\end{equation*}

V definiciji enakomerne konvergence ozna"cimo
\begin{equation*}
M_n := \sup_{x \in D} |f_n (x) - f(x)|
\end{equation*}
Za obravnavo enakomerne konvergence moramo obravnavati $M_n$, ki pa je "stevilsko zaporedje. $f_n$ enakomerno konvergira proti $f$ natanko takrat, kadar zaporedje $M_n$ konvergira proti 0.

\textbf{Geometrijska interpretacija:} $M_n < \varepsilon \iff$ graf funkcije $f_n$ le"zi v $\varepsilon$-pasu okrog grafa $f$. $f_n \to f$ enakomerno na $D \iff \forall \varepsilon > 0$ vsi grafi funkcije $f_n$ za dovolj velik $n$ le"zijo znotraj $\varepsilon$-pasu okrog grafa $f$.

\deff Naj bo $D \subseteq \RR$ in $\{ f_n : D \to \RR \}$ funckijsko zaporedje. Pravimo, da je $\{ f_n \}$ \emph{enakomerno Cauchyjevo} na $D$, "ce velja
\begin{equation*}
\forall \varepsilon > 0 \exists n_0 \in \NN \forall n, m \geq n_0 \forall x \in D : |f_n(x) - f_m(x)| < \varepsilon
\end{equation*}

\textsc{Izrek:} Naj bo $D \subseteq \RR, \{f_n : D \to \RR \}$ funkcijsko zaporedje. Tedaj je $\{ f_n \}$ enakomerno konvergentno na $D$ natanko takrat kadar je $\{ f_n \}$ enakomerno Cauchyjevo na $D$.

\textsc{Dokaz:} Doka"zemo podobno kot za zaporedja.

\textsc{Izrek:} Naj bo $D \subset \RR$ in $\{ f_n : D \to \RR \}$ funkcijsko zaporedje. "Ce so vse funkcije $f_n$ zvezne na $D$ in $\{ f_n \}$ enakomerno konvergira proti $f$ na $D$, potem je $f$ zvezna na $D$.

\textbf{Opomba:} Protiprimer je $f_n (x) = x^n$ na $[0, 1]$.

\textsc{Dokaz:} Dokazujemo, da je $f$ zvezna na $D$. Izberemo poljuben $a \in D$ in dokazujemo, da je $f$ zvezna v to"cki $a$. Izberemo $\varepsilon > 0$ in i"s"cemo tak $\delta > 0$, da $\forall x \in D$ velja $|x - a| < \delta \Rightarrow |f(x) - f(a)| < \varepsilon$.
\begin{multline*}
|f(x) - f(a)| = |f(x) - f_{n_0}(x) + f_{n_0}(x) - f_{n_0}(a) + f_{n_0}(a) - f(a)| \leq \\
\leq \underbrace{|f(x) - f_{n_0}(x)|}_{< \varepsilon/3} + \underbrace{|f_{n_0}(x) - f_{n_0}(a)|}_{< \varepsilon/3 \text{ za $|x - a| < \delta$}} + \underbrace{|f_{n_0}(a) - f(a)|}_{< \varepsilon/3} < \varepsilon
\end{multline*}
Ker $f_n$ enakomerno konvergira proti $f$, obstaja $n_0$, da za vsak $n \geq n_0$ velja, da za vsak $x \in D$
\begin{equation*}
|f_n(x) - f(x)| < \frac{\varepsilon}{3}
\end{equation*}
Ker je funkcija $f_{n_0}$ zvezna v to"cki $a$, obstaja $\delta > 0$, da za vsak $x \in D$ velja
\begin{equation*}
|x - a| < \delta \Rightarrow |f_{n_0}(x) - f_{n_0}(a)| < \dfrac{\varepsilon}{3}
\end{equation*}
\hfill $\square$

\deff Naj bo $D \subset \RR$ in $\{ u_n : D \to \RR \}$ funkcijsko zaporedje.
\begin{equation*}
\sum_{n=1}^{\infty} u_n
\end{equation*}
pravimo \emph{funkcijska vrsta}. Funkcijska vrsta $\sum_{n=1}^\infty$ \emph{konvergira po to"ckah}, "ce za vsak $x \in D$ "stevilska vrsta
\begin{equation*}
\sum_{n=1}^\infty u_n(x)
\end{equation*}
konvergira.

\textbf{Opomba:} Funkcijska vrsta konvergira po to"ckah natanko takrat, kadar funkcijsko zaporedje njenih delnih vsot konvergira po to"ckah. Torej za $x \in D$ vrsta $\sum_{n=1}^\infty u_n(x)$ konvergira $\iff \{ \sum_{n=1}^k u_n (x) \}_k$ konvergira $\iff s_k = \sum_{n=1}^k u_n$ konvergira v $x$.

Naj bo $s$ vsota po to"ckah konvergentne funkcijske vrste $\sum_{n=1}^\infty u_n$, $s = \lim_{k \to \infty} s_k$. Pravimo, da $\sum_{n=1}^\infty$ \emph{konvergira proti $s$ enakomerno na $D$}, "ce funkcijsko zaporedje njenih delnih vsot $s_k = \sum_{n=1}^k u_n$ enakomerno konvergira proti $s$ na $D$.

\textbf{Posledica:} "Ce je $\{ u_n: D \to \RR \}$ funkcijsko zaporedje zveznih funkcij in "ce $\sum_{n=1}^\infty u_n$ konvergira enakomerno na $D$ proti $s$, potem je $s$ zvezna funkcija na $D$.

\textbf{Posledica:} Naj bo $\{ u_n : D \to \RR \}$ funkcijsko zaporedje. Tedaj velja $\sum_{n=1}^\infty u_n$ je enakomerno konvergentna na $D \iff \sum_{n=1}^\infty u_n$ je enakomerno Cauchyjeva na $D$, t.j.:
\begin{equation*}
\forall \varepsilon > 0 \exists n_0 \in \NN \forall n > m \geq n_0 \forall x \in D: \left| \sum_{k=1}^n u_k(x) - \sum_{k=1}^m u_k(x) \right| = \left| \sum_{k = m+1}^n u_k(x) \right| < \varepsilon
\end{equation*}

\textsc{Izrek} (Weierstrassov kriterij za enakomerno konvergenco funkcijskih vrst, M-test):

Naj bo $\{ u_n : D \to \RR \}$ funkcijsko zaporedje. Denimo, da obstaja konvergentna "stevilska vrsta $\sum_{n=1}^\infty c_n$ s pozitivnimi "cleni, za katero velja
\begin{equation*}
|u_n(x)| \leq c_n \quad \forall x \in D
\end{equation*}
Potem funkcijska vrsta $\sum_{n=1}^\infty u_n$ konvergira enakomerno (in absolutno) na $D$. "Ce se $u_n$ zvezne na $D$, potem je tudi vsota vrste zvezna na $D$.

\textsc{Dokaz:} Iz $|u_n(x)| \leq c_n \forall x \in D$ z uporabo primerjalnega kriterija sledi, da $\sumalln{1} |u_n(x)|$ konvergira za vsak $x$. Torej vrsta $\sumalln{1} u_n$ po to"ckah absolutno konvergira, zato konvergira po to"ckah.

Doka"zimo, da je $\sumalln{1} u_n$ enakomerno Cauchyjeva. Naj bo $k > m$, potem za vsak $x \in D$ velja
\begin{equation*}
\left| \sum_{n=1}^ku_n(x) - \sum{n=1}^m u_n(x) \right| = \left| \sum_{n= m+1}^{k} u_n(x) \right| \leq
 \sum_{n = m+1}^k |u_n(x)| \leq \sum_{n = m+1}^k c_n < \varepsilon
\end{equation*}
Konvergentna "stevilska vrsta je Cauchyjeva, zato obstaja $n_0$, da za $k > m \geq n_0$ velja $\sum_{n = m+1}^k c_n < \varepsilon$

\subsection{Integriranje in odvajanje funkcijskih zaporedij in vrst}
\textsc{Izrek:} Naj bo $\{ f_n : [a, b] \to \RR \}$ funkcijsko zaporedje zveznih funkcij. "Ce funkcijsko zaporedje $\{ f_n \}$ konvergira proti funkciji $f: [a, b] \to \RR$ enakomernona $[a, b]$, potem velja
\begin{equation*}
\lim_{n \to \infty} \int_a^b f_n (x) dx = \int_a^b f(x) dx
\end{equation*}
\textsc{Dokaz:} Vemo, da je limitna funkcija zvezna na $[a, b]$. $f_n$ konvergira proti $f$ enakomerno na $[a, b]$:
\begin{equation*}
\forall \varepsilon > 0 \exists n_0 \in \NN \forall n \geq n_0 \forall x \in [a, b] : |f_n(x) - f(x)| < \varepsilon
\end{equation*}
\begin{multline*}
\left| \int_a^b f_n (x) dx - \int_a^b f(x) dx \right| = \left| \int_a^b (f_n(x) - f(x)) dx \right| \leq \\
\leq \int_a^b | f_n(x) - f(x) | dx < \int_a^b \varepsilon dx = \varepsilon (b-a)
\end{multline*}
Torej zaporedje konvergira za $n \geq n_0$.

\textsc{Posledica:} Naj bo $\{ u_n : [a, b] \to \RR \}$ funkcijsko zaporedje zveznih funkcij. Denimo, da $\sum_{n=1}^\infty u_n$ enakomerno konvergira na $[a, b]$. Potem velja
\begin{equation*}
\int_a^b \left( \sumalln{1} u_n (x) \right) dx = \sumalln{1} \left( \int_a^b u_n (x) dx \right)
\end{equation*}
\textbf{Opomba:} Enakomerno konvergentno funkcijsko vrsto iz zveznih funkcij lahko "clenoma integriramo.

\textsc{Dokaz:} (skica) Funkcijsko zaporedje delnih vsot enakomerno konvergira in uporabimo prej"snji izrek.

\textsc{Izrek:} Naj bo $\{ f_n : [a, b] \to \RR \}$ funkcijsko zaporedje zvezno odvedljivih funkcij. Denimo, da $\{ f_n' \}$ enakomerno konvergira na $[a, b]$ proti funkciji $g: [a, b] \to \RR$ in denimo, da obstaja $c \in [a, b]$, da $\{ f_n (c) \}$ konvergira. Potem $\{ f_n \}$ konvergira enakomerno na $[a, b]$ k neki funkciji $f$ in velja
\begin{equation*}
f' (x) = \lim_{n \to \infty} f_n'(x)
\end{equation*}
\textbf{Opomba:} Pri odvajanju izgubimo podatek o konstanti, zato potrebujemo da $\{ f_n(c) \}$ konvergira za nek $c$. Primer $f_n(x) = n$.

\textsc{Dokaz:} Ker je $f_n'$ zvena funkcija, $f_n(x) = f_n(c) + \int_c^x f_n'(t) dt$ za vse $x \in [a, b]$. Po prej"snjem izreku $\int_a^b f_n'(t) dt$ konvergira in celotna funkcija konvergira proti
\begin{equation*}
\left( \lim_{n \to \infty} f_n (c) = L + \int_c^x g(t) dt \right) = f(x)
\end{equation*}

Torej za vsak $x \in [a, b]$ obstaja $\lim_{n \to \infty} f_n (x)$. Zato je $f_n$ konvergentna po to"ckah.

Dokazati moramo "se,  da $f_n$ enakomerno konvergira. Izberemo poljuben $\varepsilon > 0$:
\begin{multline*}
|f_n(x) - f(x)| = \\
= \left| f_n(c) + \int_c^x f_n'(t) dt - \lim_{n \to \infty} f_n(c) - \int_c^x g(t) dt \right| = \\
= \left| f_n(c) - \lim_{n \to \infty} f_n (c) + \int_c^x (f_n'(t) - g(t)) dt \right| \leq \\
\leq \underbrace{|f_n - \lim_{n \to \infty} f_n(c)|}_{< \varepsilon \text{ za velik $n$}} + \int_c^x \underbrace{|f_n'(t) - g(t)|}_{< \varepsilon \text{ za velik $n$}} dt < \varepsilon + \varepsilon(b-a)
\end{multline*}
\hfill $\square$

\textsc{Posledica:} Naj bo $\{ u_n: [a, b] \to \RR \}$ funkcijsko zaporedje zvezno enakomernih funkcij. Denimo, da $\sumalln{0} u_n'$ konvergira enakomerno na $[a, b]$ in da obstaja $c \in [a, b]$, da $\sumalln{1} u_n(c)$ konvergira. Potem $\sumalln{1} u_n$ konvergira enakomerno na $[a, b]$ in velja
\begin{equation*}
\left( \sumalln{n=1} u_n (x) \right)' = \sumalln{1} u_n'(x) \quad \forall x \in [a, b]
\end{equation*}
\textsc{Dokaz:} Prej"senj izrek uporabimo na zaporedju delnih vsot.

\subsection{Poten"cne vrste}
\deff Poten"cna vrsta je vrsta oblike
\begin{equation*}
\sumalln{0} a_n (x-c)^n
\end{equation*}
kjer je $a_n$ "stevilsko zaporedje in $c \in \RR$. Re"cemo tudi poten"cna vrsta s \emph{sredi"s"cem} v $c$.

\textsc{Izrek:} Naj bo $\sumalln{0} a_n (x - c)^n$ poten"cna vrsta. Obstaja $R \in [0, \infty) \cup \{ \infty \}$ z naslednjo lastnostjo:
\begin{itemize}
    \item $\forall x, |x - c| < R$,  je vrsta $\sumalln{0} a_n (x - c)^n$ konvergentna in absolutno konvergentna
    \item $\forall x, |x - c| > R$, je vrsta $\sumalln{0} a_n (x - c)^n$ divergentna.
\end{itemize}
$R$ imanujemo \emph{konvergen"cni polmer}. "Ce je $r \in (0, R)$, potem poten"cna vrsta na $[c - r, c + r]$ enakomerno konvergira.
\textsc{Dokaz:} za $c = 0$.

Denimo, da poten"cna vrsta $\sumalln{0} a_n x^n$ konvergira pri $x = x_0 \neq 0$. Naj bo $r \in (0, |x_0|)$. Dokazujemo, da vrsta enakomerno konvergira na $[-r, r]$. Uporabimo Wirestrassov M-test.

Vemo, da $\sumalln{0}a_n x_0^n$ konvergira, zato velja $\lim_{n \to \infty} |a_n x^n| = 0$. Zato obstaja $M \in \RR: |a_n x_0^n| < M$ za vsak $n \in \NN$.
\begin{equation*}
|a_n x^n| = |a_n| |x|^n \leq |a_n| |r|^n \leq \dfrac{M}{|x_0|^n} r^n = M \left( \dfrac{r}{|x_0|} \right)^n
\end{equation*}
$\sumalln{0} M \left( \frac{r}{|x_0|} \right)^n$ je geometrijska vrsta, z $q = \frac{r}{|x_0|} < 1$, zato konvergira. Po M-testu je $\sumalln{0} a_n x^n$ konvergenta in absolutno konvergentna in enakomerno konvergeira na $[-r, r]$. Velja
\begin{equation*}
R = \sup \{ |x_0|, \text{ $x_0$ vrsta konvergira} \}
\end{equation*}
"Ce za nek $y_0$ velja, da $\sum a_n y_0^n$ divergira, potem divergira za vse $y > y_0$. "Ce to ne bi veljalo, bi obstajal $y_1$,  za katerega vrsta konvergira in po prej"snjem dokazu konvergira tudi za vse $y < y_1$, torej tudi $y_0$, kar je protislovje. "Ce ne obstaja $y_0$, za katerega vrsta divergira, potem je $R = \infty$.

\textsc{Posledica:} naj bo $\sumalln{0} a_n (x - c)^n$ poten"cna vrsta s konvergen"nim polmerom $R > 0$.
\begin{enumerate}
    \item Vsota poten"cne vrste je zvezna funkcija na $(c - R, c + R)$ (na $[c-r, c+r]$ konvergira enakomerno za vsak $r < R$.)
    \item Vsoto poten"cne vrste lahko "clenoma integriramo in "clenoma odvajamo na $(c-R, c+R)$. Konvergen"cni polmer se ohrani.
    \begin{gather*}
     \left( \sumalln{0} a_n (x - c)^n \right)' = \sumalln{1} n a_n (x - c)^{n-1} \\
     \int_c^x \left( \sumalln{0} a_n (t - c)^n \right)dt = \sumalln{0} \dfrac{a_n}{n+1} (x-c)^{n+1}
    \end{gather*}
    \item Vsota poten"cne vrste je funkcija $C^\infty ((c-R, c+R))$.
\end{enumerate}

\textsc{Izrek:} Naj bo $\sumalln{0} a_n (x - c)^n$ poten"cna vrsta in $R$ njen konvergen"cni polmer. Potem velja
\begin{enumerate}
    \item $\frac{1}{R} = \lim_{n \to \infty} \frac{|a_{n+1}|}{|a_n|}$, "ce ta limita obstaja
    \item $\frac{1}{R} = \lim_{n \to \infty} \sqrt[n]{|a_n|}$, "ce ta limita obstaja
\end{enumerate}
\textsc{Dokaz:}
\begin{enumerate}
    \item Uporabimo kvocientni kriterij. Obravnavamo absolutno konvergenco $\sumalln{0} a_n (x - c)^n$:
    \begin{gather*}
    d_n = \dfrac{|a_{n+1}| |x-c|^{n+1}}{|a_n| |x-c|^n} = \dfrac{|a_{n+1}|}{|a_n|} |x-c| \\
    \lim_{n \to \infty} d_n = \lim_{n \to \infty} \dfrac{|a_{n+1}|}{|a_n|} |x-c| = |x-c| \underbrace{\lim_{n \to \infty} \dfrac{|a_{n+1}|}{|a_n|}}_\text{obstaja po predp.} = |x-c| L
    \end{gather*}
    Po kvocientnem kriteriju velja
    \begin{itemize}
        \item $|x - c| L < 1 \iff | x - c| < \frac{1}{L}$ vrsta konvergira
        \item $|x - c| L > 1 \iff |x - c| > \frac{1}{L}$ vrsta divergira
    \end{itemize}
    Torej je $\frac{1}{L} = R$ in velja
    \begin{equation*}
    \dfrac{1}{R} = L = \lim_{n \to \infty} \dfrac{|a_{n+1}|}{|a_n|}
    \end{equation*}
    
    \item Podoben dokaz kot za to"cko (1).
\end{enumerate}
\hfill $\square$

\textsc{Izrek:} (Cauchy -- Hadamart) Za konvergen"cni polmer $R$ poten"cne vrste $\sumalln{0} a_n (x - c)^n$ velja
\begin{equation*}
\dfrac{1}{R} = \limsup_{n \to \infty} \sqrt[n]{|a_n|}
\end{equation*}
\textsc{Dokaz:} Omejimo se na vrste s sredi"s"cem 0 $\sumalln{0} a_n x^n$. Naj bo 
\begin{equation*}
a = \limsup_{n \to \infty} \sqrt[n]{|a_n|} \in [0, \infty) \cup \{ \infty \}
\end{equation*}
\begin{enumerate}
    \item Denimo, da je $a = \infty$. Izberimo $x \neq 0$. $\infty$ je stekali"s"ce $\sqrt[n]{|a_n|}$. Torej obstajajo poljubno veliki indeksi $n$, da je $\sqrt[n]{|a_n|} > \frac{1}{|x|}$. Za te indekse $n$ velja
    \begin{equation*}
    |a_n| |x|^n > 1
    \end{equation*}
    Torej ne velja, da je $\lim_{n \to \infty} |a_n x^n| = 0$, zato vrsta $\sumalln{0} a_n x^n$ divergira. Sledi, da je $R = 0$.
    
    \item Denimo, da je $a \in [0, \infty)$. Izberimo poljuben $x, |x| < \frac{1}{a}$. Obstaja $q > 0$, da velja
    \begin{equation*}
    |x| < \frac{1}{q} < \frac{1}{a}
    \end{equation*}
    Ker je $a$ najve"cje stekali"s"ce $\sqrt[n]{|a_n|}$, obstaja $n_0 \in \NN$, da za vse $n \geq n_0$ velja
    \begin{equation*}
    \sqrt[n]{|a_n|} < q
    \end{equation*}
    Torej je $|a_n| \leq q^n$ za vse $n \geq n_0$. Zato je
    \begin{equation*}
    |a_n x^n| \leq |q x|^n
    \end{equation*}
    za vse $n \geq n_0$. Po definiciji $q$-ja vemo
    \begin{equation*}
    |qx| < 1
    \end{equation*}
    Po primerjalnem kriteriju vrsta $\sumalln{0} a_n x^n$ absolutno konvergira.
    
    Izberimo poljuben $x, |x| > \frac{1}{a}$, torej $a > \frac{1}{|x|}$. Obstajajo poljubno veliki indeksi, da je
    \begin{equation*}
    \sqrt[n]{|a_n|} > \dfrac{1}{|x|}
    \end{equation*}
    za te indekse $n$. Torej velja
    \begin{equation*}
    |a_n x|^n > 1
    \end{equation*}
    Zato ne velja, da bi $|a_n x^n|$ konvergiralo proti 0, torej vrsta $\sumalln{0} a_n x^n$ divergira.
    
    Sledi, da je
    \begin{equation*}
    a = \dfrac{1}{R}
    \end{equation*}
    \hfill $\square$.
\end{enumerate}
%
\textsc{Izrek} (odvajanje in integriranj epoten"cnih vrst): Naj bo konvergen"cni polmer $R$ dane poten"cne vrste $\sumalln{0} a_n (x-c)^n = f(x)$ pozitiven. Potem imata vrsti, ki ju dobimo s "clenskim odvajanjem $\sumalln{1} n a_n (x - c)^{n-1}$ in s "clenskim integriranjem $\sumalln{0} \frac{a_n}{n + 1} (x - c)^{n+1}$ tudi konvergen"cen polmer $R$ in velja
\begin{gather*}
f'(x) = \sumalln{1} n a_n (x - c)^{n-1} \\
\int_0^x f(t) dt = \sumalln{0} \frac{a_n}{n + 1} (x - c)^{n + 1} \quad x \in (-R, R)
\end{gather*}
\textsc{Dokaz:}
\begin{align*}
\limsup_{\toinf{n}} \sqrt[n]{|a_n|} &= \dfrac{1}{R} \\
\limsup_{\toinf{n}} \sqrt[n]{\dfrac{|a_n|}{n+1}} &= \dfrac{1}{R} \\
\limsup_{\toinf{n}} \sqrt[n]{n |a_n|} &= \dfrac{1}{R}
\end{align*}
\hfill $\square$

\textsc{Izrek} (Abelov izrek): Naj bo $R > 0$ konvergen"cni polmer poten"cne vrste $\sumalln{0} a_n (x - c)^n$. "Ce vrsta v $x = R$ konvergira, potem je njena vstoa zvezna v $x = R$. Simetri"cno velja za $x = -R$.

\subsubsection{Taylorjeva formula in Taylorjeva vrsta}
Naj bo $p$ polinom stopnje $n$
\begin{equation*}
p(x) = a_0 + \cdots + a_n x^n
\end{equation*}
Za $a \in \RR$ velja
\begin{align*}
p(a+h) &= a_0 + a_1(a+h) + a_2 (a+h)^2 + \cdots + a_n (a+h)^n = \\
&= b_0 + b_1 h + b_2 h^2 + \cdots + b_n h^n
\end{align*}
Za izra"cun koeficienta $b_0$, je o"citno da vstavimo $h=0$ in dobimo $b_0 = p(a)$. S pomo"cjo odvajanja lahko pridobimo "se ostale "clene
\begin{align*}
p'(a+h) &= b_1 + 2 b_2 h + \cdots + n b_n h^{n-1} \\
p'(a) &= b_1 \\
p''(a+h) &= 2 b_2 + 2 \cdot 3 b_3 h + \cdots + n \cdot (n-1) b_n h^{n-2} \\
p''(a) &= 2 b_2
\end{align*}
Z nadaljevanjem tega postopka dobimo nasledenj polinom
\begin{equation*}
p(a+h) = p(a) + p'(a)h + \dfrac{p''(a)}{2} h^2 + \dfrac{p'''(a)}{3!} h^3 + \cdots + \dfrac{p^{(n)}(a)}{n!} h^n
\end{equation*}
%
\deff Naj bo $f$ $n$-krat odvedljiva funkcija v okolici to"cke $a$. Polinom
\begin{equation*}
T_{n, a}(x) = f(a) + f'(a)(x-a) + \dfrac{f''(a)}{2}(x-a)^2 + \cdots + \dfrac{f^{(n)}(a)}{n!} (x-a)^n
\end{equation*}
imenujemo $n$-ti \emph{Taylorjev polinom} funkcije $f$ pri $a$.

\textbf{Opomba:} Velja, "ce je $f$ polinom stopnje $n$, potem
\begin{equation*}
f(x) = T_{n, a} (x) \quad \forall x \in \RR
\end{equation*}
%
V resnici nas zanima
\begin{equation*}
f(x) = T_{n, a}(x) + R_{n, a}(x)
\end{equation*}
kjer je $R_{n, a}$ ostanek. Za uporabo je pomembno oceniti ostanek. $T_{n, a}$ in $f$ imata v to"cki $a$ vse odvode do reda $n$ enake in aproksimacija bo ,,dobra'' blizu $a$, "ce je $n$ "cim ve"cji.

\textsc{Taylorjev izrek:} Naj bo funkcija $f$ $(n+1)$-krat odvedljiva na odprtem intervalu $I$, ki vsebuje $a$. Potem za vsak $x \in I$ obstaja $c \in I$ med $a$ in $x$, da velja
\begin{equation*}
R_{n, a}(x) = \dfrac{f^{(n+1)}(c)}{(n+1)!} (x-a)^{n+1}
\end{equation*}
\textsc{Dokaz:} Po definiciji $R$ velja
\begin{equation*}
R_{n, a}(x) = f(x) - T_{n, a}(x)
\end{equation*}
Vemo, da $R_{n, a} (a) = 0$ in $R_{n, a}^{(k)} (a) = 0$ za $k \in \{ 1, 2, \ldots, n \}$.

Fiksiramo $x$ in vemo, da obstaja $s \in \RR$, da velja
\begin{equation*}
R_{n, a} (x) = s(x - a)^{n+1}
\end{equation*}
Definiramo funkcijo
\begin{equation*}
G(y) = R_{n, a}(y)  - s(y-a)^{n+1}
\end{equation*}
Velja
\begin{align*}
G(x) = 0 && G^{(k)}(a) = 0 \quad k = 0, 1, \ldots, n
\end{align*}
Po Rollovem izrek obstaja $c_1$ med $a$ in $x$: $G'(c_1) = 0$. Ponovno uporabimo Rollov izrek in obstaja $c_2$ med $a$ in $c_1$: $G''(c_2) = 0$. Ta postopek nadaljujemo in obstaja $c$: $G^{(n+1)}(c) = 0$. Iz tu sledi
\begin{multline*}
G^{(n+1)} (y) = (R_{n, a}(y))^{(n+1)} - (s (y-a)^{n+1})^{(n+1)} = \\
= (f(y) - T_{n, a}(y))^{(n+1)} - (n+1)!s = f^{(n+1)}(y) - s (n+1)!
\end{multline*}
Iz enakosti 
\begin{equation*}
R_{n, a}(x) = s(x - a)^{n+1}
\end{equation*}
sledi
\begin{equation*}
s = \dfrac{R_{n, a}(x)}{(x-a)^{n+a}}
\end{equation*}
Torej velja
\begin{gather*}
G^{(n+1)} (c) = 0 \\
f^{(n+1)}(c)  - (n+1)! \dfrac{R_{n, a}(x)}{(x-a)^{n+1}} = 0 \\
R_{n, a}(x) = \dfrac{f^{(n+1)} (c)}{(n+1)!} (x - a)^{n+1}
\end{gather*}
\hfill $\square$

\deff "Ce je $f$ neskon"cnokrat odvedljiva v okolici to"cke $a$, ji lahko priredimo \emph{Taylorjevo vrsto}
\begin{equation*}
\sumalln{0} \dfrac{f^{(n)} (a)}{n!} (x-a)^n = \lim_{k \to \infty} T_{k, a} (x)
\end{equation*}
Za tiste $x$, za katere limita obstaja, t.j.\,za tiste $x$, za katere vrsta konvergira.

\textbf{Opomba:} Taylorjeva vrsta je poten"cna vrsta s konvergen"cnim obmo"cjem.

\textbf{Pozor:} Vsota Taylorjeve vrste, ki je prirejena $f$, ni nujno $f$.

\textsc{Izrek:} Denimo, da je funkcija $f$ vsota konvergentne poten"cne vrste
\begin{equation*}
f(x) = \sumalln{0} c_n x^n \quad |x| < R \land R > 0
\end{equation*}
Potem za vsak $|a| < R$ velja
\begin{equation*}
f(x) = \sum_{k=0}^\infty \dfrac{f^{(k)}(a)}{k!} (x - a)^k
\end{equation*}
za vse $x, |x - a| < R - |a|$. T.j.\,$f$ je vsota prirejene Taylorjeve vrste.

\textsc{Dokaz:} Za $a = 0$:
\begin{equation*}
f(x) = \sumalln{0} c_n x^n \quad f(0) = c_0
\end{equation*}
Na $(-R, R)$ poten"cno vrsto "clenoma odvajamo
\begin{align*}
f'(x) &= \sumalln{1} n c_n x^{n-1} \\
f'(0) &= c_1\\
f^{(k)}(x) &= \sumalln{k} n(n-1) \cdots (n - k + 1) c_n x^{n-k} \\
f^{(k)}(0) &= k(k-1) \cdots 1 c_k = k! c_k \Rightarrow c_k = \dfrac{f^{(k)}(0)}{k!}
\end{align*}
Torej za $a = 0$ izrek velja.

Za $a \neq 0$ definiramo $x = (x - a) + a$. Velja
\begin{equation*}
x^n = ((x-a) + a)^n = \sum_{k=0}^n \binom{n}{k} a^{n-k} (x-a)^k
\end{equation*}
To vstavimo v prvotno vrsto in dobimo
\begin{multline*}
f(x) = \sumalln{0} c_n x^n = \sumalln{n} \sum_{k=0}^n c_n \binom{n}{k} a^{n-k} (x-a)^k = \\
= \sum_{k=}^\infty \left( \underbrace{\sum_{n = k}^\infty c_n \binom{n}{k} a^{n-k}}_{d_k} \right) (x - a)^k = \sum_{k=0}^\infty d_k (x - a)^k
\end{multline*}
Na enak na"cin kot v prvem delu dokaza dobimo
\begin{equation*}
d_k = \dfrac{f^{(k)}(a)}{k!}
\end{equation*}
\hfill $\square$

\deff Naj bo $I$ odprt interval. Pravimo, da je funkcija $f: I \to \RR$ \emph{realno analiti"cna} na intervalu $I$, "ce
\begin{equation*}
\forall a \in I \exists r_a > 0: (a - r_a, a + r_a) \subset I
\end{equation*}
in je $f$ na $(a - r_a, a + r_a)$ vsota konvergentne poten"cne vrste s sredi"s"cem v $a$
\begin{equation*}
f(x) = \sumalln{0} c_n (x - a)^n
\end{equation*}
za vse $x \in (a - r_a, a + r_a)$

\textbf{Oznaka:} $C^\omega (I)$ \dots razred vseh realno analiti"cnih funkcij na $I$.

\textbf{Opombe:}
\begin{enumerate}
    \item Od prej vemo, da je $c_n = \frac{f^{(n)} (a)}{n!}$, torej je $f$ vsota prirejene Taylorjeve vrste na neki okolici to"cke $a$.
    \item $C^\omega (I) \subsetneq C^\infty (I)$
\end{enumerate}
%
\textsc{Taylorjev izrek} (splo"sna oblika ostanka): Naj bo $I$ odprt interval in $f \in C^{n+1}(I)$ Za vsak $x \in I, a \in I$ in $p \in \{ 0, \ldots, n + 1 \}$ obstaja $c$ med $a$ in $x$, da velja
\begin{equation*}
f(x) = T_{n, a}(x) + R_{n, a}(x)
\end{equation*}
kjer je
\begin{equation*}
R_{n, a}(x) = \dfrac{f^{(n+1)} (c)}{p n!} (x-a)^p (x-c)^{n+1 - p}
\end{equation*}
\textbf{Opomba:} Pri $p = n+1$
\begin{equation*}
R_{n, a}(x) = \dfrac{f^{(n+1)} (c)}{(n+1)!} (x - a)^{n+1}
\end{equation*}
\textsc{Dokaz:} Naj bodo $a, x, b \in I, p \in \{ 0, \ldots, n+1 \}$. Definiramo funkcijo
\begin{align*}
F(x) &= T_{n, x} (b) + \left( \dfrac{b - x}{b - a} \right)^p R_{n, a}(b) = \\
&= f(x) + f'(x) (b-x) + \cdots + \dfrac{f^{(n)}(x)}{n!} (b - x)^n + \left( \dfrac{b-x}{b-a} \right)^p R_{n, a}(b)
\end{align*}
Velja
\begin{align*}
F(a) &= T_{n, a}(b) + R_{n, a}(b) = f(b) \\
F(b) &= f(b)
\end{align*}
Po Rollovem izreku obstaja $c$ med $a$ in $b$, da je $F'(c) = 0$. Velja
\begin{multline*}
F'(x) = f'(x) + f'(x)(-1) + f''(x)(b-x) + f''(x)(-1) + f'''(x)\dfrac{(b-x)^2}{2} + \cdots + \\
+\dfrac{f^{(n)}(x)}{(n-1)!} (b-x)^{n-1} (-1) + \dfrac{f^{(n+1)}(x)}{n!} (b-x)^n + p \dfrac{(b-x)^{p-1}}{(b-a)^p} R_{n, a}(b) (-1)
\end{multline*}
Ve"cino "clenov se pokraj"sa in dobimo
\begin{equation*}
F'(c) = 0 = \dfrac{f^{(n+1)}(c)}{n!} (b-c)^n - p \dfrac{(b-c)^{p-1}}{(b-a)^p} R_{n, a}(b)
\end{equation*}
Torej velja
\begin{equation*}
R_{n, a}(b) = \dfrac{f^{(n+1)}(c)}{pn!} (b-c)^{n - p + 1} (b-a)^p
\end{equation*}
\hfill $\square$

\subsubsection{Taylorjeve vrste osnovnih funkcij}
\begin{enumerate}
    \item \textbf{Eksponentna funkcija} $f(x) = e^x$ okrog 0.
    
    Prirejena Taylorjeva vrsta
    \begin{equation*}
    \sumalln{0} \dfrac{f^{(n)}(0)}{n!} x^n = \sumalln{0} \dfrac{x^n}{n!}
    \end{equation*}
    Dokazati je treba $e^x = \sumalln{0} \dfrac{x^n}{n!}$.
    \begin{equation*}
    e^x = T_{n, 0}(x) + R_{n, 0} (x)
    \end{equation*}
    $T_{n, 0}(x)$ konvergira proti $\sumalln{0} \frac{x^n}{n!}$, zato je treba dokazati, da za fiksen $x$ velja
    \begin{equation*}
    \lim_{n \to \infty} R_{n, 0}(x) = 0
    \end{equation*}
    Fiksiramo $x$:
    \begin{multline*}
      |R_{n, 0}(x)| = \left| \dfrac{f^{(n+1)}(c)}{(n+1)!} x^{n+1} \right| = \left| \dfrac{e^c}{(n+1)!} x^{n+1} \right| \leq \\
      \leq \dfrac{e^{|x|}}{(n+1)!} |x|^{n+1} = e^{|x|} \dfrac{\overbrace{|x| \cdot |x| \cdots |x|}^{n+1}}{1 \cdot 2 \cdots (n+1)} \stackrel{n \to \infty}{\longrightarrow} 0
    \end{multline*}
    
    Torej velja
    \begin{equation*}
    e^x = 1 + x + \dfrac{x^2}{2} + \dfrac{x^3}{3!} + \cdots + \dfrac{x^n}{n!} + \cdots \quad \forall x \in \RR
    \end{equation*}
    
    \item \textbf{Sinus, kosinus} okoli 0
    
    Prirejena Taylorjeva vrsta:
    \begin{equation*}
    \sin x = x - \dfrac{x^3}{3!} + \dfrac{x^5}{5!} - \dfrac{x^7}{7!} + \cdots
    \end{equation*}
    Za fiksen $x$ velja
    \begin{equation*}
    |R_{n, 0}(x)| = \left| \dfrac{f^{(n+1)} (c)}{(n+1)!} x^{n+1} \right| \leq \dfrac{|x|^{n+1}}{(n+1)!} \stackrel{n \to \infty}{\longrightarrow} 0
    \end{equation*}
    
    Podobno doka"zemo za kosinus in velja
    \begin{align*}
    \sin x &= \sumalln{0} (-1)^n \dfrac{x^{2n + 1}}{(2n + 1)!} \quad \forall x \in \RR \\
    \cos x &= \sumalln{0} (-1)^n \dfrac{x^{2n}}{(2n)!} \quad \forall x \in \RR
    \end{align*}
    
    \item \textbf{Logaritemska funkcija}
    \begin{equation*}
    f(x) = \ln (x + 1)
    \end{equation*}
    razvijemo v Taylorjevo vrsto s sredi"s"cem v 0. Velja $D_f = (-1, \infty)$. Velja
    \begin{equation*}
    f'(x) = \dfrac{1}{x+1} = \dfrac{1}{1 - (-x)} = \sumalln{0} (-x)^n = \sumalln{0} (-1)^n x^n
    \end{equation*}
   za $x \in (-1, 1)$ ($R = 1$). Z integracijo po "clenih dobimo
   \begin{equation*}
   f(x) = \sumalln{0} \dfrac{(-1)^n}{n + 1} x^{n+1} + C
   \end{equation*}
    Za izra"cun konstante vstavimo $x = 0$ in dobimo
    \begin{equation*}
    f(0) = 0 = C \Rightarrow C = 0
    \end{equation*}
    Torej velja
    \begin{equation*}
    \ln(x+1) = \sumalln{0} \dfrac{(-1)^n}{n+1} x^{n+1} \quad \forall x \in (-1, 1)
    \end{equation*}
    
    \textsc{Primer:} Razvoj funkcije $\ln x$ okrog to"cke $a > 0$.
    \begin{equation*}
     \ln x = \ln(a + (x - a)) = \ln \left( a \left( 1 + \dfrac{x-a}{a} \right) \right)
    \end{equation*}
    "Ce je $\left| \frac{x - a}{a}  < 1 \right|$ velja
    \begin{multline*}
     \ln \left( a \left( 1 + \dfrac{x-a}{a} \right) \right) = \\
     =\ln a + \ln \left( 1 + \dfrac{x - a}{a} \right) = \ln a + \dfrac{x - a}{a} - \dfrac{1}{2} \left( \dfrac{x - a}{a}\right)^2 + \cdots + \dfrac{(-1)^{n+1}}{n} \left( \dfrac{x - a}{a} \right)^n + \cdots = \\
     = \ln a + \dfrac{1}{a} (x-a) - \dfrac{1}{2a^2} (x - a)^2 + \cdots + \dfrac{(-1)^{n+1}}{na^n} (x - a)^n + \cdots
    \end{multline*}
    velja za $|x - a| < |a|$.
    
    \item \textbf{Binomska vrsta} 
    \begin{equation*}
    f(x) = (1 + x)^\alpha, \quad \alpha \in \RR
    \end{equation*}
    Razvoj v Taylorjevo vrsto okrog 0. Za funkcijo velja $D_f = (-1, \infty)$. Vemo
    \begin{align*}
    f^{(k)}(x) &= \alpha (\alpha - 1) \cdots (\alpha - k + 1) (1 + x)^{\alpha - k} \\
    f^{(k)}(0) &= \alpha (\alpha - 1) \cdots (\alpha - k + 1)
    \end{align*}
    Prirejena Taylorjeva vrsta je torej
    \begin{equation*}
    \sumalln{0} \dfrac{f^{(n)}(0)}{n!} x^n = \sumalln{0} \dfrac{\alpha (\alpha -  1) \cdots (\alpha - n + 1)}{n!} x^n = \sumalln{0} \binom{\alpha}{n} x^n
    \end{equation*}
    Defnirajmo \emph{posplo"sen binomski simbol}
    \begin{equation*}
    \binom{\alpha}{k} = \dfrac{\alpha (\alpha - 1) \cdots (\alpha - k + 1)}{k!}, \quad \alpha \in \RR, k \in \NN_0
    \end{equation*}
    Doka"zimo, da velja
    \begin{equation*}
    (1 + x)^\alpha = \sumalln{0} \binom{\alpha}{n} x^n
    \end{equation*}
    za $x \in (-1, 1)$.
    \begin{itemize}
        \item[$x \in (0, 1)$:] Za $c \in (0, x)$ velja
        \begin{multline*}
        |R_n(x)| = \left| \dfrac{f^{(n+1)}(c)}{(n+1)!} x^{n+1} \right| = \left| \dfrac{\alpha (\alpha - 1) \cdots (\alpha - n)}{(n+1)!} (1+c)^{\alpha - n -1} x^{n+1}\right| \leq \\
        \leq \underbrace{\left| \binom{\alpha}{n + 1} \right| |x|^{n+1} 2^\alpha}_{a_n}
        \end{multline*}
        Za zgornjo oceno smo uporabili slede"co oceno
        \begin{equation*}
        (1+c)^{\alpha - n - 1} =\underbrace{ |(1+c)^\alpha |}_{\leq 2^\alpha} \underbrace{|(1 + c)^{-n -1}|}_{\leq 1} \leq 2^\alpha
        \end{equation*}
        Pri fiksnem $x$ dokazujemo $\lim_{n \to \infty} a_n = 0$. Dovolj je dokazati, da "stevilska vrsta $\sumalln{0} a_n$ konvergira. Uporabimo kvocientni kriterij
        \begin{multline*}
         \dfrac{a_{n+1}}{a_n} = \dfrac{|\alpha (\alpha - 1) \cdots (\alpha - n) (\alpha - n - 1)| |x|^{n+2} 2^\alpha (n+1)!}{(n+2)! |\alpha (\alpha - 1) \cdots (\alpha - n)| |x|^{n+1} 2^\alpha} = \\
         = \dfrac{|\alpha -n - 1|}{n+2} |x| \stackrel{n \to \infty}{\longrightarrow} |x|
        \end{multline*}
        "Ce je $|x| < 1$, potem $\sumalln{0} a_n$ konvergira (za na"s $x$ to velja).
        
        \item[$x \in (-1, 0)$:] Za $c \in (x, 0$ uporabimo oceno za ostanek pri $p = 1$:
        \begin{multline*}
         |R_n(x)| = \left| \dfrac{f^{(n+1)} (c)}{1 \cdot n!} x (x-c)^n \right| = \\
         = \left| \alpha (\alpha - 1) \cdots (\alpha -n) (1+c)^{\alpha -n -1} \right| \dfrac{1}{n!} |x| |x-c|^n \leq \\
         \leq \left| \dfrac{\alpha (\alpha - 1) \cdots (\alpha - n)}{n!} \right| M \left| \dfrac{x}{1+c} \right| \left| \dfrac{x-c}{1+c} \right|^n
        \end{multline*}
        Za dolo"citi $M$ smo uporabili
        \begin{align*}
        \alpha \geq 0&: 0 < 1 + c < 1 \Rightarrow (1 + c)^\alpha < 1\\
        \alpha < 0 &: 0 < 1 + x < 1 + c < 1 \Rightarrow (1+c)^\alpha \leq (1 + x)^\alpha
        \end{align*}
        Sledi, da je $M = \max \{ 1, (1+x)^\alpha \}$.
        
        $\left| \frac{x}{1+c} \right|$ ocenimo z
        \begin{equation*}
        \left| \dfrac{x}{1+c} \right| \leq \left| \dfrac{x}{1+x} \right|
        \end{equation*}
        Za oceno $\left| \frac{x - c}{1+c} \right|$ definiramo $g(c) = \frac{x - c}{1 + c}$ in i"s"cemo maksimum te funkcije. Velja
        \begin{equation*}
        g'(c) = \dfrac{-(1+c) - (x - c)}{(1+c)^2} = \dfrac{-1 - x}{(1 + c)^2} < 0
        \end{equation*}
       Za $g(c)$ vemo, da je $g(c) < 0$ in je strogo padajo"ca. Torej je najve"cja vrednost $|g(c)| \leq |g(0)| = |x|$. Tako smo dobili oceno
       \begin{equation*}
       |R_n(x)| \leq \left| \dfrac{\alpha (\alpha - 1) \cdots (\alpha - n)}{n!} \right| M \left| \dfrac{x}{1+x} \right| |x|^n
       \end{equation*}
       Na enak na"cin kot prej doka"zemo, da gre $|R_n(x)| \stackrel{n \to \infty}{\longrightarrow} 0$ pri fiksnem $x$.
    \end{itemize}
\end{enumerate}
