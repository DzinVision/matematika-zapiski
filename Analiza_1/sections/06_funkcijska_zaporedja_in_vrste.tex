\deff Naj bo $D \subseteq \RR$ in $f_n : D \to \RR$ funkcije za vsak $n \in \NN$. Potem pravimo, da je
\begin{equation*}
\{f_n\} = \{f_n: D \to \RR\}
\end{equation*}
\emph{funkcijsko zaporedje}.

"Ce za vsak $x \in D$ "stevilsko zaporedje $\{f_n(x)\}$ konvergira, pravimo da funkcijsko zaporedje $f_n$ \emph{konvergira} na $D$ (konvergira po to"ckah). V tem primeru definirmao funkcijo $f: D \to \RR$ s predpisom
\begin{equation*}
f(x) = \lim_{n \to \infty} f_n(x)
\end{equation*}
in jo imenujemo \emph{limitna funkcija}.

\deff Naj bo $D \subseteq \RR$ in $\{f_n: D \to \RR \}$ funkcijsko zaporedje. Pravimo, da $\{f_n\}$ \emph{konvergira enakomerno} proti funkciji $f: D \to \RR$ na $D$, "ce za vsak $\varepsilon > 0$ obstaja $n_0 \in \NN$, da za vse $n \geq n_0$ in vse $x \in D$ velja $|f_n (x) - f(x)| < \varepsilon$.

\textbf{Opomba:} "Ce funkcijsko zaporedje $\{ f_n \}$ konvergira enakomerno na $D$ proti $f$, potem $\{ f_n \}$ konvergira proti $f$, t.j.\,$f$ je limitna funkcija $\{ f_n \}$.

$f$ je limitna funkcija $\{ f_n \} \iff \forall x \in D: f(x) = \lim_{n \to \infty} f_n(x) \iff$
\begin{equation*}
\forall x \in D \forall \varepsilon > 0 \exists n_0 \in \NN \forall n \geq n_0 : |f_n(x) - f(x)| < \varepsilon
\end{equation*}
$f_n$ enakomerno konvergira proti $f$ na $D \iff$
\begin{equation*}
\forall \varepsilon > 0 \exists n_0 \in \NN \forall n \geq n_0 \forall x \in D: |f_n(x) - f(x)| < \varepsilon
\end{equation*}

V definiciji enakomerne konvergence ozna"cimo
\begin{equation*}
M_n := \sup_{x \in D} |f_n (x) - f(x)|
\end{equation*}
Za obravnavo enakomerne konvergence moramo obravnavati $M_n$, ki pa je "stevilsko zaporedje. $f_n$ enakomerno konvergira proti $f$ natanko takrat, kadar zaporedje $M_n$ konvergira proti 0.

\textbf{Geometrijska interpretacija:} $M_n < \varepsilon \iff$ graf funkcije $f_n$ le"zi v $\varepsilon$-pasu okrog grafa $f$. $f_n \to f$ enakomerno na $D \iff \forall \varepsilon > 0$ vsi grafi funkcije $f_n$ za dovolj velik $n$ le"zijo znotraj $\varepsilon$-pasu okrog grafa $f$.

\deff Naj bo $D \subseteq \RR$ in $\{ f_n : D \to \RR \}$ funckijsko zaporedje. Pravimo, da je $\{ f_n \}$ \emph{enakomerno Cauchyjevo} na $D$, "ce velja
\begin{equation*}
\forall \varepsilon > 0 \exists n_0 \in \NN \forall n, m \geq n_0 \forall x \in D : |f_n(x) - f_m(x)| < \varepsilon
\end{equation*}

\textsc{Izrek:} Naj bo $D \subseteq \RR, \{f_n : D \to \RR \}$ funkcijsko zaporedje. Tedaj je $\{ f_n \}$ enakomerno konvergentno na $D$ natanko takrat kadar je $\{ f_n \}$ enakomerno Cauchyjevo na $D$.

\textsc{Dokaz:} Doka"zemo podobno kot za zaporedja.

\textsc{Izrek:} Naj bo $D \subset \RR$ in $\{ f_n : D \to \RR \}$ funkcijsko zaporedje. "Ce so vse funkcije $f_n$ zvezne na $D$ in $\{ f_n \}$ enakomerno konvergira proti $f$ na $D$, potem je $f$ zvezna na $D$.

\textbf{Opomba:} Protiprimer je $f_n (x) = x^n$ na $[0, 1]$.

\textsc{Dokaz:} Dokazujemo, da je $f$ zvezna na $D$. Izberemo poljuben $a \in D$ in dokazujemo, da je $f$ zvezna v to"cki $a$. Izberemo $\varepsilon > 0$ in i"s"cemo tak $\delta > 0$, da $\forall x \in D$ velja $|x - a| < \delta \Rightarrow |f(x) - f(a)| < \varepsilon$.
\begin{multline*}
|f(x) - f(a)| = |f(x) - f_{n_0}(x) + f_{n_0}(x) - f_{n_0}(a) + f_{n_0}(a) - f(a)| \leq \\
\leq \underbrace{|f(x) - f_{n_0}(x)|}_{< \varepsilon/3} + \underbrace{|f_{n_0}(x) - f_{n_0}(a)|}_{< \varepsilon/3 \text{ za $|x - a| < \delta$}} + \underbrace{|f_{n_0}(a) - f(a)|}_{< \varepsilon/3} < \varepsilon
\end{multline*}
Ker $f_n$ enakomerno konvergira proti $f$, obstaja $n_0$, da za vsak $n \geq n_0$ velja, da za vsak $x \in D$
\begin{equation*}
|f_n(x) - f(x)| < \frac{\varepsilon}{3}
\end{equation*}
Ker je funkcija $f_{n_0}$ zvezna v to"cki $a$, obstaja $\delta > 0$, da za vsak $x \in D$ velja
\begin{equation*}
|x - a| < \delta \Rightarrow |f_{n_0}(x) - f_{n_0}(a)| < \dfrac{\varepsilon}{3}
\end{equation*}
\hfill $\square$

\deff Naj bo $D \subset \RR$ in $\{ u_n : D \to \RR \}$ funkcijsko zaporedje.
\begin{equation*}
\sum_{n=1}^{\infty} u_n
\end{equation*}
pravimo \emph{funkcijska vrsta}. Funkcijska vrsta $\sum_{n=1}^\infty$ \emph{konvergira po to"ckah}, "ce za vsak $x \in D$ "stevilska vrsta
\begin{equation*}
\sum_{n=1}^\infty u_n(x)
\end{equation*}
konvergira.

\textbf{Opomba:} Funkcijska vrsta konvergira po to"ckah natanko takrat, kadar funkcijsko zaporedje njenih delnih vsot konvergira po to"ckah. Torej za $x \in D$ vrsta $\sum_{n=1}^\infty u_n(x)$ konvergira $\iff \{ \sum_{n=1}^k u_n (x) \}_k$ konvergira $\iff s_k = \sum_{n=1}^k u_n$ konvergira v $x$.

Naj bo $s$ vsota po to"ckah konvergentne funkcijske vrste $\sum_{n=1}^\infty u_n$, $s = \lim_{k \to \infty} s_k$. Pravimo, da $\sum_{n=1}^\infty$ \emph{konvergira proti $s$ enakomerno na $D$}, "ce funkcijsko zaporedje njenih delnih vsot $s_k = \sum_{n=1}^k u_n$ enakomerno konvergira proti $s$ na $D$.

\textbf{Posledica:} "Ce je $\{ u_n: D \to \RR \}$ funkcijsko zaporedje zveznih funkcij in "ce $\sum_{n=1}^\infty u_n$ konvergira enakomerno na $D$ proti $s$, potem je $s$ zvezna funkcija na $D$.

\textbf{Posledica:} Naj bo $\{ u_n : D \to \RR \}$ funkcijsko zaporedje. Tedaj velja $\sum_{n=1}^\infty u_n$ je enakomerno konvergentna na $D \iff \sum_{n=1}^\infty u_n$ je enakomerno Cauchyjeva na $D$, t.j.:
\begin{equation*}
\forall \varepsilon > 0 \exists n_0 \in \NN \forall n > m \geq n_0 \forall x \in D: \left| \sum_{k=1}^n u_k(x) - \sum_{k=1}^m u_k(x) \right| = \left| \sum_{k = m+1}^n u_k(x) \right| < \varepsilon
\end{equation*}

\textsc{Izrek} (Weierstrassov kriterij za enakomerno konvergenco funkcijskih vrst, M-test):

Naj bo $\{ u_n : D \to \RR \}$ funkcijsko zaporedje. Denimo, da obstaja konvergentna "stevilska vrsta $\sum_{n=1}^\infty c_n$ s pozitivnimi "cleni, za katero velja
\begin{equation*}
|u_n(x)| \leq c_n \quad \forall x \in D
\end{equation*}
Potem funkcijska vrsta $\sum_{n=1}^\infty u_n$ konvergira enakomerno (in absolutno) na $D$. "Ce se $u_n$ zvezne na $D$, potem je tudi vsota vrste zvezna na $D$.

\textsc{Dokaz:} Iz $|u_n(x)| \leq c_n \forall x \in D$ z uporabo primerjalnega kriterija sledi, da $\sumalln{1} |u_n(x)|$ konvergira za vsak $x$. Torej vrsta $\sumalln{1} u_n$ po to"ckah absolutno konvergira, zato konvergira po to"ckah.

Doka"zimo, da je $\sumalln{1} u_n$ enakomerno Cauchyjeva. Naj bo $k > m$, potem za vsak $x \in D$ velja
\begin{equation*}
\left| \sum_{n=1}^ku_n(x) - \sum{n=1}^m u_n(x) \right| = \left| \sum_{n= m+1}^{k} u_n(x) \right| \leq
 \sum_{n = m+1}^k |u_n(x)| \leq \sum_{n = m+1}^k c_n < \varepsilon
\end{equation*}
Konvergentna "stevilska vrsta je Cauchyjeva, zato obstaja $n_0$, da za $k > m \geq n_0$ velja $\sum_{n = m+1}^k c_n < \varepsilon$
