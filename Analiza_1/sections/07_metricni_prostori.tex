Metri"cni prostor je mno"zica $M$, kjer je za vsak par to"ck $x, y \in M$ definirana razdalja $d(x, y)$. Lastnosti, ki jih od razdalje pri"cakujemo zdru"zimo v naslednji definiciji:

\deff \emph{Metri"cni prostor} je neprazna mno"zica $M$ skupaj s preslikavo $d: M \times M \to \RR$, ki ustreza pogojem
\begin{enumerate}
    \item $d(x, y) \geq 0 \quad \forall x, y \in M \quad \text{in} \quad d(x, y) = 0 \iff x = y$
    \item $d(x, y) = d(y, x)$
    \item \emph{trikotni"ska neenakost:} $d(x, z) \leq d(x, y) + d(y, z) \quad \forall x, y, z \in M$
\end{enumerate}
Preslikavo $d$ imenujemo \emph{metrika} ali \emph{razdalja} na $M$, par $(M, d)$ imenujemo \emph{metri"cni prostor}.

\deff Naj bo $(M, d)$ metri"cni prostor in naj bo $a \in M, r > 0$.

\emph{Odprta krogla} s sredi"s"cem v $a$ in polmerom $r$ je mno"zica
\begin{equation*}
K(a, r) = \{x \in M, d(a, x) < r \}
\end{equation*}
\emph{Zaprta krogla} s sredi"s"cem v $a$ in polmerom $r$ je mno"zica
\begin{equation*}
\overline{K}(a, r) = \{ x \in M d(a, x) \leq r \}
\end{equation*}
\emph{Okolica to"cke} $a$ je vsaka taka podmno"zica $M$, ki vsebuje neko kroglo s sredi"s"cem v $a$ in pozitivnim polmerom.

\deff Naj bo $(M, d)$ metri"cni prostor in naj bo $A \subset M$.
\begin{enumerate}[(i)]
    \item To"cka $a \in M$ je \emph{notranja to"cka} mno"zici $A$, "ce obstaja kak"sna okolica to"cke $a$, ki vsa le"zi v mno"zici $A$.
    \begin{equation*}
    \exists r > 0: K(a, r) \subset A
    \end{equation*}
    
    \item To"cka $a \in M$ je \emph{zunanja to"cka} mno"zice $A$, "ce obstaja vsaj ena okoica to"cke $a$, ki ne vsebuje nobene to"cke iz $A$.
    \begin{equation*}
    \exists r > 0: K(a, r) \cap A = \varnothing
    \end{equation*}
    
    \item To"cka $a \in M$ je \emph{robna to"cka} mno"zice $A$, "ce vsaka njena okolica seka $A$ in $A^c = M \setminus A$.
    \begin{equation*}
    \forall r > 0:(K(a, r) \cap A \neq \varnothing \land K(a, r) \cap (M \setminus A) \neq \varnothing)
    \end{equation*}
\end{enumerate}
Mno"zico vseh notranjih to"ck ozna"cimo z $\mathring{A}$ ali z $\Int A$ in jo imenujemo \emph{notranjost} mno"zice $A$.

Mno"zico vseh robnih to"ck ozna"cimo z $\partial A$ ali z $\Meja A$ in jo imenujemo \emph{rob} ali \emph{meja mno"zice} $A$.

\textbf{Opombe:}
\begin{enumerate}
    \item Zunanje to"cke mno"zice $A$ so natanko notranje mno"zice $A^c$.
    \item $M = \Int A \cup \Int(A^c) \cup \Meja A$, pri "cemer so te mno"zice paroma disjunktne.
    \item Vsaka notranja to"cka mno"zice $A$ le"zi v $A$ ($\Int A \subset A$).
    \item Nobena zunanja to"cka mno"zice $A$ ne le"zi v $A$.
    \item Robne to"cke lahko le"zijo v $A$ ali v $A^c$.
    \item Notranja to"cka mno"zice $A$ je zunanja to"cka mno"zice $A^c$.
    \item Robna to"cka mno"zice $A$ je tudi robna to"cka mno"zice $A^c$ in obratno.
\end{enumerate}
%
\deff Naj bo $(M, d)$ metri"cni prostor.

 Pravimo, da je podmno"zica $O$ \emph{odprta}, "ce je vsaka njena to"cka notranja to"cka mno"zice $O$.

Pravimo, da je podmno"zica $Z \subset M$ \emph{zaprta}, "ce vsebuje vse svoje robne to"cke.

\textbf{Opombi:}
\begin{gather*}
O \subset M \text{ je odprta} \iff \Int O = O \\
Z \subset M \text{ je zaprta} \iff \partial Z \subset Z
\end{gather*}
%
\textsc{Trditev:} Naj bo $(M, d)$ metri"cni prostor in $A \subset M$. Potem je $A$ odprta natanko takrat, kadar je $A^c$ zaprt.

\textsc{Dokaz:} Recimo, da je $O \subset M$ odprta. Torej $O = \Int O$. Zato le"zijo robne to"cke mno"zice $O$ v $O^c$. torej vse robne to"cke $O^c$ le"zijo v $O^c$. Zato je $O^c$ zaprta mno"zica.

Reicmo, da je $Z \subset M$ zaprta. Potem $Z$ vsebuje svoje robne to"cke. Ker so robne to"cke $Z$ in $Z^c$ enake, $Z^c$ vsebuje samo notranje to"cke, torej je $Z^c$ opdrt.

\textsc{Izrek:} Naj bo $\OO$ dru"zina vseh odprtih mno"zic metri"cnega prostora $(M, d)$. Potem velja
\begin{enumerate}[(1)]
    \item $M \in \OO, \quad \varnothing \in \OO$
    
    \item Unija poljubne dru"zine odprtih mno"zic je odprta, t.j.\,za poljubno mno"zico $\Lambda$ in za poljubno dru"zino $A_\lambda \in \OO$ velja
    \begin{equation*}
    \forall \lambda \in \Lambda: \bigcup_{\lambda \in \Lambda} A_\lambda \in \OO
    \end{equation*}
    
    \item Presek kon"cnega "stevila odprtih mno"zic je odprta mno"zica.
    \begin{equation*}
    \forall n \in \NN, A_1, \ldots, A_n \in \OO \Rightarrow \bigcap_{j = 1}^n A_j \in \OO
    \end{equation*}
\end{enumerate}
\textsc{Dokaz:}
\begin{enumerate}[(1)]
    \item Vemo od prej.
    \item Izberemo $\Lambda, A_\lambda \in \OO$ za vsakn $\lambda \in \Lambda$. Dokazujemo, da je $\bigcup_{\lambda \in \Lambda} A_\lambda$ odprta. Izberemo poljubno to"cko $a \in \bigcup_{\lambda \in \Lambda} A_\lambda$ in dokazujemo, da je notranja to"cka od $\bigcup_{\lambda \in \Lambda} A_\lambda$. Vemo
    \begin{equation*}
    \exists \lambda \in \Lambda: a \in A_\lambda
    \end{equation*}
    Ker je $A_\lambda$ odprta, je $a$ notranja to"cka od $A_\lambda$, zato 
    \begin{equation*}
    \exists r > 0: K(a, r) \subset A_\lambda \subset \bigcup_{\lambda \in \Lambda} A_\lambda
    \end{equation*}
    Torej je $a \in \Int (\bigcup_{\lambda \in \Lambda} A_\lambda)$.
    
    \item Dovolj je dokazati za $n=2$, naprej nadaljujemo induktivno po istem postopku.
    
    Naj bosta $A_1, A_2 \in \OO$. Dokazujemo, da je $A_1 \cap A_2 \in \OO$.
    \begin{itemize}
        \item "Ce je $A_1 \cap A_2 = \varnothing$, je presek odrpt.
        \item "Ce $A_1 \cap A_2 \neq \varnothing$, izberemo poljuben $a \in A_1 \cap A_2$. Velja $a \in A_1 \land a \in A_2$. Ker sta $A_1$ in $A_2$ odprti, je $a$ notranja to"cka obeh.
        \begin{equation*}
       (\exists r_1 > 0: K(a, r_1) \subset A_1) \land (\exists r_2 > 0: K(a, r_2) \subset A_2)
        \end{equation*}
        Izberemo $r = \min \{ r_1, r_2 \}$ in velja
        \begin{equation*}
        K(a, r) \subset A_1 \cap A_2 \Rightarrow a \in \Int (A_1 \cap A_2)
        \end{equation*}
        Zato je $A_1 \cap A_2 \in \OO$.
    \end{itemize}
\end{enumerate}
\textbf{Opomba:} Za "stevno (neskon"cne) preseke ni nujno res, da bi bil presek odprtih mno"zic odprt. Protiprimer $\RR^2$ z obi"cajno metriko in dru"zina mno"zic $A_n = K \left( 0, \frac{1}{n} \right)$. $A_n$ je odrpta za vsak $n \in \NN$, vendar
\begin{equation*}
\bigcap_{n \in \NN} A_n = \{ (0, 0) \}
\end{equation*}
ni odprta mno"zica.

\textsc{Posledica:} Naj bo $\Z$ dru"zina vseh zaprtih mno"zic metri"cnega prostora $(M, d)$. Velja
\begin{enumerate}[(1)]
    \item $M \in \Z, \quad \varnothing \in \Z$
    
    \item Unija kon"cnega "stevila zaprtih mno"zic je zaprta mno"zica.
    \begin{equation*}
    \forall n \in \NN, Z_1, \ldots Z_n \in \Z: \bigcup_{i = 1}^n Z_i \in \Z
    \end{equation*}
    
    \item Presek poljubne dru"zine zaprtih mno"zic je zaprt.
    \begin{equation*}
    \forall \Lambda \forall Z_\lambda \in \Z: \bigcap_{\lambda \in \Lambda} Z_\lambda \in \Z
    \end{equation*}
\end{enumerate}
\textsc{Dokaz:} (1) vemo, (2) in (3) se doka"ze podobno. Doka"zimo (3):
\begin{equation*}
\left( \bigcap_{\lambda \in \lambda} Z_\lambda \right)^c = \underbrace{\bigcup_{\lambda \in \Lambda} \underbrace{(Z_\lambda)^c}_\text{odprt}}_\text{odprt}
\end{equation*}
%
\textsc{Posledica:} Vsaka kon"cna podmno"zica v metri"cnem prostoru je zaprta.

\textsc{Dokaz:} Vemo, da je $\{ x \}$ zaprta.

\textsc{Trditev:} naj bo $(M, d)$ metri"cni prostor. Vsaka odrpta krogla je odprta mno"zica in vsaka zaprta krogla je zparta mno"zica.

\textsc{Dokaz:} naj bo $a \in M$ in $r > 0$. Potem je
\begin{equation*}
K(a, r) = \{ x \in M, d(a, x) < r \}
\end{equation*}
Dokazujemo, da je $K(a, r)$ odprta mno"zica, t.j.:\,$\forall x \in K(a, r)$ je $x$ notranja to"cka $K(a, r)$. I"s"cemo $r' > 0: K(x, r') \subset K(a, r)$. Naj bo
\begin{equation*}
r' := (r - d(a, x))
\end{equation*}
Izberemo $y \in K(x, r'): d(x, y) < r'$ in dokazujemo $y \in K(a, r)$, t.j.\,$d(a, y) < r$.
\begin{equation*}
d(a, y) \leq d(a, x) + d(x, y) < d(a, x) + r - d(a, x) = r
\end{equation*}
Za zaprte krogle doka"zemo, da je komplement zaprte krogle odprt, na podoben na"cin kot zgoraj.

\deff Naj bo $A$ podmno"zica v metri'cnem prostoru $(M, d)$. Z $\overline{A}$ ozna"cimo mno"zico $A \cup \partial A$ in jo imenujemo \emph{zaprtje} mno"zice $A$.

\textbf{Velja:}
\begin{enumerate}
    \item $\overline{A}$ je zaprta mno"zica (ker je $\overline{A}^c$ odprt).
    \item $\overline{\left( \overline{A} \right)} = \overline{A}$
\end{enumerate}
\textbf{Opomba:} V splo"snem ne velja
\begin{equation*}
\overline{K(a, r)} = \overline{K}(a, r)
\end{equation*}
%
\deff Naj bo $(M, d)$ metri"cni prostor. Podmno"zica $A \subset M$ je \emph{omejena}, "ce le"zi v neki krogli, t.j.
\begin{equation*}
\exists a \in M \exists r > 0: A \subset K(a, r)
\end{equation*}
Metri"cni prostor $(M, d)$ je \emph{omejen}, "ce je $M$ omejena podmno"zica.

\deff Naj bo $(M, d)$ metri"cni prostor in $A \subset M$. Pravimo, da je to"cka $a \in M$ \emph{stekali"s"ce mno"zice} $A$, "ce vsaka okolica to"cke $a$ vsebuje neskon"cno to"ck mno"zice $A$.

\textbf{Pozor:} To ni analogna definicija stekali"s"ca zaporedja.

\textbf{Opomba:} Stekali"s"ca mno"zice imajo lahko samo neskon"cne mno"zice.

\textsc{Izrek:} Naj bo $(M, d)$ metri"cni prostor in $A \subset M$. To"cka $a \in M$ je stekali"s"ce mno"zice $A$ natanko tedaj, kadar vsaka okolica to"cke $a$ vsebuje vsaj eno od $a$ razli"cno to"cko mno"zice $A$.

\textsc{Dokaz:}
\begin{itemize}
    \item[$(\Rightarrow)$] O"citno, direktno iz definicije.
    \item[$(\Leftarrow)$] Naj bo $U$ poljubna okolica to"cke $a$. Dokazujemo, da v $U$ le"zi neskon"cno to"ck iz $A$. Po predpostavki obstaja $a_1 \in U, a_1 \in A, a_1 \neq a$. Naj bo
    \begin{equation*}
    U_2 = K(a, d(a, a_1))
    \end{equation*}
    $U_2$ je okolica to"cke $a$ in $a_1 \notin U_2$. Po predpostavki $\exists a_2 \in U_2, a_2 \in A, a_2 \neq a$. Postopek nadaljujemo in dobimo $a_1, a_2, \ldots$, ki so med seboj razli"cne in vse le'zijo v $U$ (in v $A$).
\end{itemize}
\hfill $\square$

\textsc{Posledica:} Naj bo $(M, d)$ metri"cni prostor in $A \subset M$. Mno"zica $A$ je zaprta natanko tedaj, kadar vsebuje vsa svoje stekali"s"ca.

\textsc{Dokaz:} Stekali"s"ce mno"zice $A$ ni zunanja to"cka $A$. Zato je bodisi notranja, bodisi robna to"cka mno"zice $A$.
\begin{itemize}
    \item[$(\Rightarrow)$] "Ce je $A$ zaprta, potem $A$ vsebuje vse svoje robne to"cke, zato vsebuje vsa svoja stekali"s"ca.
    \item[$(\Leftarrow)$] Denimo, da $A$ ni zaprta. Potem obstaja neka robna to"cka $b$ mno"zice $A$, ki ne le"zi v $A$. Po definiciji robne to"cke vsaka njena okolica seka $A$ in $A^c$. Ker $b \notin A$ je torej v vsaki okolici $b$ vsaj ena to"cka $a \in A, a \neq b$. Torej je po izreku to"cka $b$ stekali"s"ce mno"zice $A$, zato $A$ ne vsebuje vseh svojih stekali"s"c.  $\rightarrow\leftarrow$
\end{itemize}
\hfill $\square$

\textsc{Posledica:} Naj bo $(M, d)$ metri"cni prostor in $A \subset M$. Potem je mno"zica vseh stekali"s"c mno"zice $A$ zaprta.

\textsc{Dokaz:} $S$ naj bo mno"zica vseh stekali"s"c mno"zice $A$. Dokazujemo, da je $M \setminus S$ odprta mno"zica. Izberemo poljuben $x \in M \setminus S$ in dokazujemo, da je notranja to"cka $M \setminus S$. Obstaja $r > 0: K(x, r) \cap A \subset \{ x \}$.
\begin{equation*}
x' \in K(x, r): K(x', \min \{ r - d(x, x'), d(x, x') \} ) \cap A = \varnothing
\end{equation*}
Zato $x' \neq S$.

\hfill $\square$

\subsection{Zaporedja v metri"cnih prostorih}
\deff Zaporedje v metri"cnem prostoru $(M, d)$ je preslikava $\varphi: \NN \to M$. To"cko $\varphi(n)$ imenujemo $n$-ti "clen zaporedja in pi"semo $a_n = \varphi(n)$.

\deff Pravimo, da je to"cka $a \in M$ \emph{stekali"s"ce} zaporedja $\{ a_n \}$, "ce vsaka njena okolica vsebuje neskon"cno mnogo "clenov zaporedja, t.j.
\begin{equation*}
\forall \varepsilon > 0 \exists \text{ neskon"cno indeksov $n$}: d(a, a_n) < \varepsilon
\end{equation*}

\deff Pravimo, da zaporedje $\{ a_n \}$ v $M$ \emph{konvergira} proti $a \in M$, "ce vsaka okolica to"cke $a$ vsebuje vse "clene zaporedja od nekega indeksa naprej, t.j.
\begin{equation*}
\forall \varepsilon > 0 \exists n_0 \in \NN \forall n \geq n_0: d(a, a_n) < \varepsilon
\end{equation*}
V tem primeru $a$ imenujemo \emph{limita zaporedja} $\{ a_n \}$ in pi"semo $a = \lim_{n \to \infty} a_n$.

\textbf{Opombe:}
\begin{itemize}
    \item "Ce je zaporedje $\{ a_n \}$ v $M$ konvergentno z limito $a \in M$, potem je $a$ stekali"s"ce zaporedja. 
    \item Ni vsako stekali"s"ce limita. 
    \item Limita je edino stekali"s"ce konvergentnega zaporedja.
    
    \textsc{Dokaz:} Denimo, da je $\{ a_n \}$ v $M$ konvergentno z limito $a \in M$. Recimo, da je $b \in M, b \neq a$. Dokazujemo, da $b$ ni stekali"s"ce. Naj bo $\varepsilon = \frac{1}{2} d(a, b) > 0$. Velja
    \begin{equation*}
    \exists n_0 \in \NN \forall n \geq n_0: d(a, a_n) < \varepsilon
    \end{equation*}
    Poglejmo si razdaljo
    \begin{equation*}
    d(b, a_n) \geq \underbrace{d(b,a)}_{2 \varepsilon} - \underbrace{d(a, a_n)}_{< \varepsilon} \geq \varepsilon
    \end{equation*}
    za vse $n \geq n_0$. Torej $b$ ni stekali"s"ce, ker v $K(b, \varepsilon)$ le"zi kve"cjemo prvih $n_0$ "clenov zaporedja
\end{itemize}

\deff Pravimo, da zaporedje $\{ a_n \}$ v metri"cnem prostoru $(M, d)$ zado"s"ca \emph{Cauchyjevem pogoju} "ce
\begin{equation*}
\forall \varepsilon > 0 \exists n_0 \in \NN \forall n, m \geq n_0 : d(a_n, a_m) < \varepsilon
\end{equation*}
%
\textsc{Izrek:} Vsako konvergentno zaporedje v $(M, d)$ je Cauchyjevo.

\textsc{Dokaz:} Recimo, da $\{ a_n \}$ v $M$ konvergira z limito $a \in M$. Torej velja
\begin{equation*}
\forall \varepsilon > 0 \exists n_0 \in \NN \forall n \geq n_0: d(a, a_n) < \varepsilon
\end{equation*}
Za $n, m \geq n_0$ velja
\begin{equation*}
d(a_n, a_m) \leq \underbrace{d(a_n, a)}_{< \varepsilon} + \underbrace{d(a, a_m)}_{< \varepsilon} < 2 \varepsilon
\end{equation*}
Zato je zaporedje $\{ a_n \}$ Cauchyjevo.

\textsc{Primer:} Naj bo $M = \{ 1, \frac{1}{2}, \ldots \} = \{ \frac{1}{n}, n \in \NN \}$ z inducirano razdaljo v $\RR$. Obstaja zaporedje v $M$, ki je Cauchyjevo, ni pa konvergentno. Primer takega zaporedja je $a_n = \frac{1}{n}$. Zaporedje $\{ a_n \}$ je konvergenton zaporedje v $\RR$ z limito 0. Vemo, da je v vsakem metri"cnem prostoru konvergentno zaporedje Cauchyjejvo. Na"se zaporedje je torej Cauchyjevo v $\RR$ in zato tudi v $M$, ni pa konvergentno, ker to"cke 0 ni v $M$.

\deff Naj bo $(M, d)$ metri"cni prostor. Pravimo, da je $M$ \emph{poln}, "ce je vsako Cauchyjevo zaporedje v $M$ konvergentno v $M$.

\textsc{Izrek:} Naj bo $C[a, b]$ metri"cni prostor zveznih funkcij s standardno metriko (max. metrika). Zaporedje $\{ f_n \}$ v $C[a, b]$ konvergira proti $f \in C[a, b]$ natanko tedaj, kadar funkcijsko zaporedje proti $f$ konvergira enakomerno na $[a, b]$.

\textsc{Dokaz:}
\begin{multline*}
f = \lim_{n \to \infty} f_n \in C[a, b] \iff \forall \varepsilon> 0 \exists n_0 \in \NN \forall n \geq n_0: d(f, f_n) < \varepsilon \iff \\
\iff \forall \varepsilon > 0 \exists n_0 \in \NN \forall n \geq n_0: \sup_{x \in [a, b]} |f(x) - f_n(x)| < \varepsilon \iff \\
\forall x \in [a, b]: \iff |f(x) - f_n(x)| < \varepsilon \iff f_n \stackrel{n \to \infty}{\longrightarrow} f \text{ enakomerno na $[a, b]$}
\end{multline*}
\textbf{Opomba:} Zvene funkcije z obi"cajno metriko: konvergentna v $C[a, b]$ je natanko enakomerna konvergenca na $[a, b]$.

\textsc{Izrek:} Metri"cni prostor $C[a, b]$ z obi"cajno metriko je poln.

\textsc{Dokaz:} Naj bo $\{ f_n \}$ Cauchyjevo zaporedje v $C[a, b]$. Dokazujemo, da je konvergentno. Vemo
\begin{equation*}
\forall \varepsilon > 0 \exists n_0 \in \NN \forall n, m \geq n_0: \underbrace{d(f_n, f_m)}_{\max_{x \in [a, b]} |f_n(x) - f_m(x)|} < \varepsilon
\end{equation*}
Naj bo $x \in [a, b]: |f_n(x) - f_m(x)| < \varepsilon$ za vse $m, n \geq n_0$. $\{ f_n(x) \}$ je Cauchyjevo (v $\RR$) zato je konvergentno in njegovo limito ozna"cimo z $f(x)$. S tem smo dobili limitno funkcijo $f$ po to"ckah. Dokazujemo, da $f_n$ proti $f$ konvergira v $C[a, b]$. Vemo
\begin{equation*}
\forall \varepsilon > 0 \exists n_0 \in \NN \forall n, m \geq n_0 \forall x \in [a, b]: |f_n(x) - f_m(x)| < \varepsilon
\end{equation*}
Ko po"sljemo $m$ proti neskon"cno za dani $\varepsilon$ dobimo, da za dani $\varepsilon$ in vsak $n \geq n_0$ velja
\begin{equation*}
|f_n(x) - f(x)| < \varepsilon \quad \forall x \in [a, b]
\end{equation*}
Torej $f_n$ konvergira proti $f$ enakomerno na $[a, b]$. Po izreku torej $f_n$ konvergira proti $f$ v $C[a, b]$.

\subsection{Kompaktnost}
\deff Naj bo $(M, d)$ metri"cni prostor in naj bo $K \subset M$. Pravimo, da je dru"zina mno"zic $(A_\gamma)_{\gamma \in \Gamma}$ \emph{pokritje} mno"zice $K$, "ce velja
\begin{equation*}
A_\gamma \subset M \forall \gamma \in \Gamma \quad \text{in} \quad K \subset \bigcup_{\gamma \in \Gamma} A_\gamma
\end{equation*}
"Ce so vse mno"zice $A_\gamma$ odprte, pravimo da je $(A_\gamma)_{\gamma \in \Gamma}$ \emph{odprto pokritje} $K$.

"Ce so vse $A_\gamma$ zaprte, potem je $(A_\gamma)_{\gamma \in \Gamma}$ \emph{zaprto pokritje}.

"Ce je v dru"zini le kon"cno mno"zic, je to \emph{kon"cno pokritje}.

\emph{Podpokritje} pokritja $(A_\gamma)_{\gamma \in \Gamma}$ je vsaka poddru"zina $(A_\gamma)_{\gamma \in \Gamma}$, ki je pokritje od $K$.

\deff Naj bo $(M, d)$ metri"cni prostor. Pravimo, da je podmno"zica $K \subset M$ \emph{kompaktna}, "ce vsako odprto pokritje $(O_\gamma)_{\gamma \in \Gamma}$ vsebuje kon"cno podpokritje.

Pravimo, da je $M$ \emph{kompakten} metri"cni prostor, "ce je mno"zica $M$ kompaktna mno"zica.

\textsc{Izrek:} Vsaka kompaktna podmno"zica $K$ v metri"cnem prostoru $(M, d)$ je zaprta in omejena.

\textsc{Dokaz:} \dashuline{$K$ je omejena}

$K$ je kompaktna $\iff$ za vsako odprto pokritje $K$ obstaja kon"cno podpokritje

$K$ je omejena $\iff \exists a \in M, \exists r > 0: K \subset K(a, r)$

Izberemo $a \in M$. $\{ K(a, n), n \in \NN \}$ je odprto pokritje $K$. Ker je $K$ kompaktno, obstaja kon"cno podpokritje
\begin{equation*}
K \subset K(a, n_1) \cup K(a, n_2), \cup \cdots \cup K(a, n_m)
\end{equation*}
Naj bo $N := \max \{ n_1, \ldots, n_m \}$. Potem je $K \subset K(a, N)$, zato je omejena.

\dashuline{$K$ je zaprta $\iff K^c$ je odprt}

Izberemo poljuben $c \in M \setminus K$ in dokazujemo, da je notranja to"cka $K^c$.
\begin{equation*}
O_r = \{ x \in M: d(x, c) > r \} = \left( \overline{K}(c, r) \right)^c
\end{equation*}
Velja
\begin{equation*}
K \subset \bigcup_{r > 0} O_r = M \setminus \{ c \}
\end{equation*}
Zaradi kompaktnosti $K$ obstaja kon"cno podpokritje:
\begin{equation*}
K \subset O_{r_1} \cup O_{r_2} \cup \cdots \cup O_{r_k} = O_r
\end{equation*}
kjer je $r := \min \{ r_1, \ldots, r_k \}$. Torej je $K \subset \left( \overline{K}(c, r) \right)^c$. Od tod sledi $\overline{K}(c, r) \subset K^c$. Zato je $c$ notranja to"cka v $K^c$.

\hfill $\square$

\textsc{Izrek:} Vsaka zaprta podmno"zica $Z$ kompaktne mno"zice $K$ v metri"cnem prostoru $(M, d)$ je kompaktna.

\textsc{Dokaz:} Naj bo $Z$ zaprta podmno"zica kompaktne mno"zice $K$, t.j. $Z \subset K$. Vemo, da je $Z^c$ odprt v $M$. Naj bo $\{ O_\gamma \}_{\gamma \in \Gamma}$ odprto pokritje mno"zice $Z$.
\begin{equation*}
\{ O_\gamma: \gamma \in \Gamma \} \cup Z^c
\end{equation*}
je odprto pokritje $M$ in zato tudi $K$. Ker je $K$ kompaktna, obstaja kon"cno podpokritje
\begin{equation*}
Z \subset K \subset O_{\gamma_1} \cup O_{\gamma_2} \cup \cdots \cup O_{\gamma_k} \cup Z^c
\end{equation*}
Torej je
\begin{equation*}
Z \subset O_{\gamma_1} \cup O_{\gamma_2} \cup \cdots \cup O_{\gamma_k}
\end{equation*}
zato je $Z$ kompaktna.

\hfill $\square$

\textsc{Izrek} (Heine -- Borel): Naj bo $A \subset \RR$ z obi"cajno metriko. $A$ je kompaktna natanko takrat, kadar je zaprta in omejena.

\textbf{Opomba:} Izrek velja tudi za podmno"ice v $\RR^n$.

\textsc{Dokaz:}
\begin{itemize}
    \item[($\Rightarrow$)] Vemo po izreku.
    \item[$(\Leftarrow)$] Naj bo $A$ zaprta in omejena. Ker je $A$ omejena, obstajata $a < b: A \subset [a, b]$. Ker je $A$ zaprta podmno"zica kompaktne mno"zice, je kompaktna.
\end{itemize}
\hfill $\square$

\textsc{Lema 1:} Naj bo $\{ I_n \}$ zaporedje vlo"zenih zaprtih intervalov v $\RR$, t.j. $I_{n+1} \subset I_n$ za vsak $n \in \NN$. Potem velja
\begin{equation*}
\bigcap_{n \in \NN} \neq \varnothing
\end{equation*}
\textsc{Dokaz:} $I_n = [a_n, b_n]$. $\{ a_n \}$ je nara"s"cajo"ce in navzgor omejeno (z $b_1$), zato je konvergentno in ozna"cimo $a = \lim_{n \to \infty} a_n$. Velja
\begin{equation*}
a_n \leq a \leq b_n \quad \forall n \in \NN
\end{equation*}
Od tod sledi $a \subset \bigcap_{n \in \NN} I_n$.

\textsc{Lema 2:} naj bo $\{ P_n \}$ padajo"ce zaporejde kvadrov v $\RR^k$, t.j.
\begin{equation*}
P_n = [a_1^n, b_1^n] \times [a_2^n, b_2^n] \times \cdots \times [a_k^n, b_k^n]
\end{equation*}
Potem je $\bigcap_{n \in \NN} P_n \neq \varnothing$.

\textsc{Dokaz:} Dokaz analogen dokazu Leme 1.

\textsc{Lema 3:} Kvader $P = [a_1, b_1] \times \cdots \times [a_k, b_k]$ je kompakten v $\RR^k$ z evklidsko metriko.

\textsc{Dokaz:} Dol"zina diagonale v $P$ je
\begin{equation*}
\delta := \sqrt{\sum_{i = 1}^k (b_i - a_i)^2}
\end{equation*}
in velja $\forall x, y \in P: d(x, y) \leq \delta$.

Denimo, da $P$ ni kompakten. Potem obstaja odprto pokritje $\{ O_\gamma \}_{\gamma \in \Gamma}$ kvadra $P$, ki nima kon"cnega podpokritja. Naj bo
\begin{equation*}
c_j := \dfrac{a_j + b_j}{2}
\end{equation*}
Intervali $[a_j, c_j], [c_j, b_j]$ dolo"cajo $2^k$ kvadrov $Q_j$, katerih unija je $P$. Vsaj enega od kvadrov $Q_j$ ne moremo pokriti s kon"cno mnogo "clanicami pokritja $\{ O_\gamma \}_{\gamma \in \Gamma}$. Postopek nadaljujemo in na ta na"cin dobimo zaporedje kvadrov
\begin{equation*}
P_j: P \supset P_1 \supset P_2 \supset \cdots
\end{equation*}
za katerega velja $P_{j+1} \subset P_j$ in nobenega kvadra ne moremo pokriti s kon"cno mnogo $O_\gamma$. Po lemi 2 obstaja $x \in \bigcap_{n \in \NN} P_n$ in obstaja $\gamma \in \Gamma: x \in O_\gamma$, kjer je $O_\gamma$ odprta mno"zica. Vemo, da je dol"zina diagonale $P_j = 2^{-j} \delta$, zato obstaja $P_j$, da velja $P_j \subset O_\gamma$. Torej smo na"sli kvader, ki ga lahko pokrijemo s kon"cno mnogo $O_\gamma$. $\rightarrow \leftarrow$

Torej je $P$ kompaktna.

\hfill $\square$

Zato velja:

\textsc{Izrek} (Heine -- Borel: Naj bo $E \subset \RR^k$ z evklidsko metriko. Mno"zica $E$ je kompaktna natanko takrat, kadar je $E$ zaprta in omejena.

\textsc{Izrek:} Vsaka neskon"cna mno"zica to"ck $A$, ki le"zi v kompaktni podmno"zici $K$ metri"cnega prostora $(M, d)$ ima stekali"s"ce v $K$.

\textsc{Dokaz:} Denimo, da $A$ nima stekali"s"ca v $K$. Torej za poljuben $x \in K$ velja, da obstaja $r_x > 0: K(x, r_x) \cap A$ je kve"cjemu kon"cna.
\begin{equation*}
\{ K(x, r_x): x \in K \}
\end{equation*}
je odprto pokritje $K$. Obstaja kon"cno podpokritje
\begin{equation*}
K \subset K(x_1, r_{x_1}) \cup \cdots \cup K(x_k, r_{x_k})
\end{equation*}
Od tod sledi
\begin{multline*}
A \cap K \subset A \cap (K(x_1, r_{x_1}) \cup K(x_2, r_{x_2}) \cup \cdots \cup K(x_k, r_{x_k})) = \\
= \underbrace{(A \cap K(x_1, r_{x_1})) \cup \cdots \cup (A \cap K(x_k, r_{x_k}))}_\text{kve"cjemu kon"cna}
\end{multline*}
$A \cap K$ je neskon"cna $\rightarrow \leftarrow$

\textsc{Izrek:} Naj bo $K$ kompaktna podmno"zica v metri"cnem prostoru $(M, d)$. Tedaj ima poljubno zaporedje $\{ a_n \}$ v $K$ vsaj eno stekali"s"ce.

\textsc{Dokaz:} Denimo, da nobena to"cka iz $K$ ni stekali"s'ce $\{ a_n \}$. Potem za vsak $x \in K$ obstaja $r_x > 0$, da v $K(x, r_x)$ le"zi kve"cjem kon"cno "clenov zaporedja.
\begin{equation*}
\{ K(x, r_x): x \in K \}
\end{equation*}
je odprto pokritje kompaktne mno"zice $K$. Obstaja kon"cno podpokritje
\begin{equation*}
K \subset \underbrace{K(x_1, r_{x_1}) \cup \cdots \cup K(x_k, r_{x_k})}_\text{v tej uniji le"zi kve"cjemu kon"cno "clenov zaporedja}
\end{equation*}
V $K$ le"zi neskon"cno mnogo "clenov zaporedja. $\rightarrow \leftarrow$.

\textsc{Posledica:} Naj bo $\{ a_n \}$ zaporedje v kompaktni mno"zici $K$ v metri"cnem prostoru $(M, d)$. Tedaj ima $\{ a_n \}$ konvergentno podzaporedje (z limito v $K$).

\textsc{Dokaz:} Kot pri realnih zaporedjih doka"zemo: za vsako stekali"s"ce $s$ zaporedja $\{ a_n \}$ obstaja konvergentno podzaporedje z limito $s$.

\textsc{Izrek:} Vsak kmopakten metri"cni prostor je poln.

\textsc{Lema:} "Ce je $a$ stekali"s"ce Cauchyjevega zaporedja $\{ a_n \}$ v metri"cnem prostoru $(M, d)$, potem je $\{ a_n \}$ konvergentno zaporedje z limito $a$.

\textsc{Dokaz} (izrek): $M$ je kompakten. Izberemo poljubno Cauchyjevo zaporedje $\{ a_n \}$ v $M$. Po izreku ima $\{ a_n \}$ stekali"s"ce $a$ v $M$, ki je po lemi limita zaporedja $\{ a_n \}$.

\hfill $\square$

\textsc{Dokaz} (Leme): $a$ je stekali"s"ce $\{ a_n \} \iff \forall \varepsilon > 0$ obstaja neskon"cno mnogo indeksov $n \in \NN$, za katere velja
\begin{equation*}
a_n \in K(a, \varepsilon) \iff d(a, a_n) < \varepsilon
\end{equation*}
$\{a_n\}$ je Cauchyjevo:
\begin{equation*}
\forall \varepsilon > 0 \exists n_0 \in \NN \forall n, m \geq n_0: d(a_n, a_m) < \varepsilon
\end{equation*}
Ker je $a$ stekali"s"ce, obstaja $n_1 \geq n_0: d(a, a_{n_1}) < \varepsilon$ in velja
\begin{equation*}
\forall m \geq n_0: d(a, a_m) \leq \underbrace{d(a, a_{n_1})}_{< \varepsilon} + \underbrace{d(a_{n_1}, a_m)}_{< \varepsilon} < 2 \varepsilon
\end{equation*}
Od tod sledi, da je $a = \lim_{n \to \infty} a_n$.

\deff Metri"cni prostor $(M, d)$ je \emph{lokalno kompakten}, "ce ima vsaka to"cka $x \in M$ kompaktno okolico.

\textsc{Izrek:} Naj bodo $K_j$ zaprte podmno"zice v kompaktnem metri"cnem prostoru $(M, d)$, za katere velja
\begin{equation*}
K_1 \supset K_2 \supset K_3 \supset \cdots \quad K_j \neq \varnothing
\end{equation*}
Potem je $\bigcap_{j=1}^\infty k_j \neq \varnothing$.

\textbf{Opombi:}
\begin{itemize}
    \item "Ze vemo za primer zaprtih vlo"zenih intervalov.
    \item $\bigcap_{j=1}^\infty K_j$ je kompakten (ker je zaprt v kompaktnem metri"cnem prostoru).
\end{itemize}
\textsc{Dokaz:} $V_j = M \setminus K_j$ je odprta mno"zica za vsak $j$. "Ce bi veljalo
\begin{equation*}
\bigcap_{j=1}^\infty K_j  = \varnothing
\end{equation*}
lahko naredimo komplement nad to ena"cbo in dobimo
\begin{equation*}
\left( \bigcap_{j=1}^\infty K_j \right)^c = \bigcup_{j=1}^\infty V_j = M
\end{equation*}
Velja, da je $M \subset \bigcup_{j=1}^\infty V_j = M$ in ker je $M$ kompaktna, obstaja kon"cno podpokritje, t.j.
\begin{equation*}
\exists k \in \NN: M \subset V_{j_1} \cup V_{j_1} \cup \cdots \cup V_{j_k} = V_{j_k}
\end{equation*}
kjer smo brez "skode za splo"snost predpostavili $j_1 < j_2 < \cdots < j_k$. Torej velja
\begin{equation*}
V_{j_k} = M \setminus K_{j_k}
\end{equation*}
Vemo, da $K_{j_k} \neq \varnothing$, torej smo pri"sli v protislovje z $M \subset V_{j_k}$. Od tod sledi 
\begin{equation*}
\bigcap_{j=1}^\infty K_j \neq \varnothing
\end{equation*}
\hfill $\square$

\subsection{Podprostori v metri"cnem prostoru}
Vemo: "Ce je $(M, d)$ metri"cni prostor in je $A \subset M$, potem je $(A, d_{|A \times A})$ metri"cni prostor z inducirano razdaljo. Vemo tudi
\begin{equation*}
\forall a \in A: K_A(a, r) = K_M(a, r) \cap A
\end{equation*}

\textsc{Izrek:} Naj bo $(M, d)$ metri"cni prostor in $A \subset M$. Mno"zica $O_A \subset A$ je odprta v $(A, d_{|A \times A})$ natanko tedaj, kadar je $O_A = O \cap A$ za neko odprto mno"zico $O \subset M$.

\textsc{Dokaz:} "Ce je $O$ odprta mno"zica v $M$, potem
\begin{equation*}
\forall x \in O \exists r_x > 0: K(x, r_x) \subset O
\end{equation*}
Zato velja
\begin{equation*}
O = \bigcup_{x \in O} K(x, r_x)
\end{equation*}
\begin{itemize}
    \item[($\Rightarrow$)] Naj bo $O_A$ odprta podmno"zica v $(A, d_{|A \times A})$. Velja
    \begin{equation*}
    O_A = \bigcup_{x \in O_A} K_A(x, r_x) = \bigcup_{x \in O_A} \left( K_M(x, r_x) \cap A \right) = \underbrace{\left( \bigcup_{x \in O_A} K_M(x, r_x) \right)}_\text{unija odprtih je odprta} \cap A = O \cap A
    \end{equation*}
    
    \item[($\Leftarrow$)] Denimo, da velja $O_A = O \cap A$, kjer je $O$ odprta v $M$. Dokazujemo, da je $O_A$ odprta v $A$. Vemo
    \begin{equation*}
    O = \bigcup_{x \in O} K_M(x, r_x)
    \end{equation*}
    Torej velja
    \begin{equation*}
    O_A = \left( \bigcup_{x \in O} K_M(x, r_x) \right) \cap A = \bigcup_{x \in O} \left( K_M(x, r_x) \cap A \right) = \bigcup_{x \in O \cap A} K_A(x, r_x)
    \end{equation*}
    Unija odprtih v $A$ je odrpta, torej je $O_A$ odprta.
\end{itemize}
\hfill $\square$

\textsc{Izrek:} Naj bo $(M, d)$ metri"cni prostor in $A \subset M$. Za mno"zico $Z_A \subset A$ velja, da je $Z_A$ zaprta natnko tedaj, kadar je $Z_A = Z \cap A$ za neko zaprto mno"zico $Z$ v $M$.

\textsc{Dokaz:} S prehodom na komplemente in uporabo prej"snjega izreka.

\textsc{Izrek:} Naj bo $(M, d)$ metri"cni prostor in $A \subset M$. Mno"zica $K \subset A$ je kompaktna v $(A, d_{|A \times A})$ natanko tedja, kadar je $K$ kompaktna v $M$.

\textsc{Dokaz:}
\begin{itemize}
    \item[($\Rightarrow$)] Denimo, da je $K$ kompaktna v $A$. Dokazujemo, da je $K$ kompatkna v $M$. Izberemo poljubno odprto pokritje $(O_\gamma)_{\gamma \in \Gamma}$ za $K$ v $M$. Potem je
    \begin{equation*}
    \{ O_\gamma \cap A \}_{\gamma \in \Gamma}
    \end{equation*}
    odprto pokritje $K$ v $A$, ker
    \begin{equation*}
    K \subset \bigcup_{\gamma \in \Gamma} O_\gamma \Rightarrow K \cap A = K \subset \bigcup_{\gamma \in \Gamma} \underbrace{\left( O_\gamma \cap A \right)}_\text{odprte v $A$}
    \end{equation*}
    Ker je $K$ v $A$ kompaktna, obstaja kon"cno podpokritje
    \begin{equation*}
    K \subset (O_{\gamma_1} \cap A) \cup \cdots \cup (O_{\gamma_k} \cap A)
    \end{equation*}
    Sledi
    \begin{equation*}
    K \subset O_{\gamma_1} \cup \cdots \cup O_{\gamma_k}
    \end{equation*}
    Torej je $K$ kompaktna v $M$.
    
    \item[($\Leftarrow$)] Denimo, da je $K$ kompaktna v $M$. Dokazujemo, da je $K$ kompaktna v $A$. Izberemo poljubno odprto pokritje $(O_\gamma^A)_{\gamma \in \Gamma}$ v $A$ od $K$. Po izreku vemo, da za vsak $\gamma \in \Gamma$ obstaja $O_\gamma$, ki je odprta v $M$, da velja
    \begin{equation*}
    O_\gamma^A = O_\gamma \cap A
    \end{equation*}
    Torej je $K \subset \bigcup_{\gamma \in \Gamma} O_\gamma$. Ker je $K$ kompaktna v $M$, obstaja kon"cno podpokritje
    \begin{equation*}
    K \subset O_{\gamma_1} \cup \cdots \cup O_{\gamma_k}
    \end{equation*}
    Sledi
    \begin{equation*}
    K \cap A = K \subset (O_{\gamma_1} \cap A) \cup \cdots \cup (O_{\gamma_k} \cap A) = \bigcup_{i=1}^k O_{\gamma_i}^A
    \end{equation*}
    Torej je $K$ kompaktna v $M$.
\end{itemize}
\hfill $\square$

\subsection{Preslikave med metri"cnimi prostori}
Naj bosta $(M, d)$ in $(M', d')$ metri"cna prostora in naj bo $D \subset M$ neprazna podmno"zica. Ukvarjali se bomo s preslikavami $f: D \to M'$.

\deff Oznake kot zgoraj. Naj bo $x_0 \in D$. Pravimo, da je preslikava $f$ \emph{zvezna v to"cki} $x_0$, "ce velja
\begin{equation*}
\forall \varepsilon > 0 \exists \delta > 0 \forall x \in D: d(x, x_0) < \delta \Rightarrow d'(f(x), f(x_0)) < \varepsilon
\end{equation*}
\textbf{Opombe:}
\begin{itemize}
    \item $f: D \to M'$ je zvezna v to"cki $x_0 \in D \iff$ za vsako okolico $V$ to"cke $f(x_0)$ obstaja okolica $U$ to"cke $x_0$ v $M$, da je $f(U \cap D) \subset V$.
    \begin{itemize}
        \item[($\Leftarrow$)] Naj bo $V = K'(f(x_0), \varepsilon)$. Po predpostavki obstaja okolica $U$ to"cke $x_0: f(U \cap D) \subset V$. Po definiciji okolice $\exists \delta > 0: K(x_0, \delta) \subset U$. Torej "ce je $x \in D$ in $x \in K(x_0, \delta)$, potem je $f(x) \in K'(f(x_0), \varepsilon)$.
        
        \item[($\Rightarrow$)] Izberemo poljubno okolico $V$ to"cke $f(x_0)$. Po definiciji okolice
        \begin{equation*}
        \exists \varepsilon > 0: K'(f(x_0), \varepsilon) \subset V
        \end{equation*}
        Po definiciji zveznosti v to"cki $x_0$
        \begin{equation*}
        \exists \delta > 0: f(\underbrace{K(x_0, \delta)}_U \cap D) \subset K'(f(x_0), \varepsilon)
        \end{equation*}
    \end{itemize}
    \hfill $\square$
    
    \item $f: D \to M'$ je zvezna v to"cki $x_0 \in D \iff$ za vsako okolico $V$ to"cke $f(x_0)$ v $M'$ obstaja okolica $U$ od $x_0$ v $D$ , da velja $f(U) \subset V$.
\end{itemize}
%
\textsc{Izrek:} Preslikava $f: D \to M'$ je zvezna v to"cki $x_0 \in D$ natanko takrat, kadar za vsako zaporedje $\{ x_n \}$ v $D$, ki konvergira proti $x_0$, zaporedje $\{ f(x_n) \}$ konvergira proti $f(x_0)$.

\textsc{Dokaz:} Kot pri realnih funkcijah.

\deff Naj bo $A \subset D$. Pravimo, da je preslikava $f: D \to M'$ \emph{zvezna na $A$}, "ce je $f$ zvezna v vsaki to"cki iz $A$. Pravimo, da je $f$ \emph{zvezna preslikava}, "ce je zvezna na $D$.

\textsc{Izrek:} Preslikava $f: D \to M'$ je zvezna natanko tedaj, kadar je za vsako odprto mno"zico $O'$ v $M'$ praslika $f^{-1} (O')$ odprta.

\textsc{Dokaz:}
\begin{itemize}
    \item[($\Rightarrow$)] Denimo, da je $f: D \to M'$ zvezna. Izberemo poljubno odprto mno"zico $O' \in M'$. Ozna"cimo $O = f^{-1}(O')$. Dokazujemo, da je $O$ odprta.
    \begin{itemize}
        \item $O = \varnothing$ je odprta.
        \item $O \neq \varnothing$. Izberemo poljubno to"cko $x_0 \in O$ in dokazujemo, da je notranja to"cka od $O$. Vemo, da je $f(x_0) \in O'$ in ker je $O'$ odprta, je $o'$ okolica $f(x_0)$. Ker je $f$ zvezna v to"cki $x_0$, velja
        \begin{equation*}
        \exists \delta > 0: f(\underbrace{K(x_0, \delta) \cap D}_{\subset O}) \subset O'
        \end{equation*}
        Torej je $x_0$ notranja to"cka od $O$.
    \end{itemize}

    \item[($\Leftarrow$)] Izberemo poljuben $x_0 \in D$ in dokazujemo, da je \dashuline{$f$ zvezna v $x_0$}. 
    
    Izberemo poljubno okolico $V$ to"cke $f(x_0)$ v $M'$. Obstaja $K'(f(x_0), \varepsilon) \subset V$. Po predpostavki je 
    \begin{equation*}
    f^{-1}(\underbrace{K'(f(x_0), \varepsilon)}_{\ni x_0})
    \end{equation*}
    odprta. Ker je $x_0$ element odprte mno"zice, je notranjost te mno"zice, velja
    \begin{equation*}
    \exists \delta > 0: K(x_0, \delta) \subset f^{-1}(K'(f(x_0), \varepsilon))
    \end{equation*}
    Torej velja
    \begin{equation*}
    f(K(x_0, \delta)) \subset K'(f(x_0), \varepsilon)
    \end{equation*}
    \hfill $\square$
\end{itemize}
%
\textsc{Izrek:} Naj bosta $(M, d)$ in $(M', d')$ metri"cna prostora in naj bo $D \subset M$. Preslikava $f: D \to M'$ je zvezna natanko tedaj, kadar je za vsako zaprto mno"zico $Z' \subset M'$ njena praslika $f^{-1}(Z')$ zaprta v $M$.

\textsc{Dokaz:} S prehodom na komplemente in uporabo prej"snjega izreka.

\textbf{Opomba:} "Ce je $f: D \to M'$ zvezna ni nujno res, da bi bila slika odprte mno"zice odprta. Primer: $f: \RR \to \RR$ z obi"cajno metriko definiramo s predpisom $f(x) = 0$ za vsak $x \in \RR$.

\textsc{Izrek:} Naj bodo $(M, d), (M', d'), (M'', d'')$ metri"cni prostori in $D \subset M$ ter preslikavi $f: D \to M'$, $g: M' \to M''$. "Ce sta $f$ in $g$ zvezni, potem je $g \circ f: D \to M''$ zvezna.

\textsc{Dokaz:} "Ce je $Z'' \subset M''$ zaprta, potem je zaradi zveznosti $g$, $g^{-1}(Z'')$ zaprta v $M'$. Zaradi zveznosti $f$, je $f^{-1}(g^{-1}(Z''))$ v $M$ zaprta. Po prej"snjem izreku je $g \circ f$ zvezna.

\hfill $\square$

\textsc{Izrek:} Naj bo $K \subset M$ kompaktna podmno"zica v metri"cnem prostoru $(M, d)$ in $f: K \to M'$ zvezna preslikava. Potem je $f(K)$ kompaktna.

\textsc{Dokaz:} Denimo, da je $K$ kompaktna in $f: K \to M'$ zvezna. Dokazujemo \dashuline{$f(K)$ je kompaktna}.

Izberemo poljubno odprto pokritje 
\begin{equation*}
\{ O'_\gamma \}_{\gamma \in \Gamma}: f(K) \subset \bigcup_{\gamma \in \Gamma} O'_\gamma
\end{equation*}
$f^{-1}(O'_\gamma)$ je odprta v $M$, ker je $f$ zvezna. Velja
\begin{equation*}
K \subset \bigcup_{\gamma \in \Gamma} \underbrace{f^{-1}(O'_\gamma)}_\text{odprta}
\end{equation*}
Ker je $K$ kompaktna, obstaja kon"cno podpokritje
\begin{align*}
K &\subset f^{-1}(O'_{\gamma_1}) \cup \cdots \cup f^{-1}(O'_{\gamma_k}) \\
f(K) &\subset O'_{\gamma_1} \cup \cdots \cup O'_{\gamma_k}
\end{align*}
Torej je $f(K)$ kompaktna.

\hfill $\square$

\textsc{Izrek:} Naj bo $K \subset M$ kompaktna podmno"zica v metri"cnem prostoru $(M, d)$ in $f: K \to \RR$ zvezna funkcija\footnote{Pod besedo funkcija se razume obi"cajna metrika na $\RR$.}. Tedaj je $f$ omejena in dose"ze minimum in maksimum.

\textsc{Skica dokaza:} $f(K) \subset \RR$ je kopaktna (prej"snji izrek), zato je zaprta in omejena. Sledi, da je $f$  omejena. ker je $f$ omejena, potem $\inf f$ in $\sup f$ obstajata in sta robni to"cki. Ker je $f(K)$ zaprta, sta $\inf f$ in $\sup f$ dose"zena.

\deff Naj bosta $(M, d)$ in $(M', d')$ metri"cna prostora in $D \subset M$ ter $C \subset D$. Pravimo, da je preslikava $f: D \to M'$ \emph{enakomerno zvezna na $C$}, "ce velja
\begin{equation*}
\forall \varepsilon > 0 \exists \delta > 0 \forall x, x' \in C: d(x, x') < \delta \Rightarrow d'(f(x), f(x')) < \varepsilon
\end{equation*}
\textbf{Opomba:} "Ce je $f$ enakomerno zvezna na $C$, je $f$ zvezna na $C$, obratno v splo"snem ni res.

\textsc{Izrek:} Naj bo $K$ kompaktna podmno"zica v metri"cnem prostoru $(M, d)$ in $f: K \to M'$ zvezna preslikava. Potem je $f$ enakomerno zvezna na $K$.

\textsc{Dokaz:} Izberemo poljuben $\varepsilon > 0$. Za vsak $x \in K$ obstaja $\delta_x > 0$, da za vsak $x' \in K, d(x, x') < \delta_x$ velja $d'(f(x), f(x')) < \frac{\varepsilon}{2}$.

Dru"zina $\{ K(x, \frac{1}{2} \delta_x) \}_{x \in K}$ je odprto pokritje kompaktne mno"zice $K$, zato obstaja kon"cno podpokritje
\begin{equation*}
K \subset K(x_1, \delta_{x_1} / 2) \cup \cdots \cup K(x_k, \delta_{x_k} / 2)
\end{equation*}
Naj bo
\begin{equation*}
\delta := \min \left\{ \dfrac{\delta_{x_1}}{2}, \ldots, \dfrac{\delta_{x_k}}{2} \right\}
\end{equation*}
Izberemo poljubena $x, x': d(x, x') < \delta$. Obstaja $j: x' \in K(x_j, \delta_{x_j} / 2)$.
\begin{equation*}
d(x, x_j) \leq \underbrace{d(x, x')}_{< \delta \leq \delta_{x_j} / 2} + \underbrace{d(x', x_j)}_{< \delta_{x_j} / 2} < \delta_{x_j}
\end{equation*}
Ker $d(x', x_j) < \frac{\delta_{x_j}}{2}$ zaradi zveznosti velja
\begin{equation*}
d'(f(x'), f(x_j)) < \dfrac{\varepsilon}{2}
\end{equation*}
Ker $d(x, x_j) < \frac{\delta_{x_j}}{2}$ zaradi zveznosti velja
\begin{equation*}
d'(f(x), f(x_j) < \dfrac{\varepsilon}{2}
\end{equation*}
Torej velja
\begin{equation*}
d'(f(x), f(x')) \leq d'(f(x), f(x_j)) + d'(f(x_j), f(x') < \varepsilon
\end{equation*}
\hfill $\square$

\subsection{Banachovo skr"citveno na"celo}
\deff Naj bo $(M, d)$ metri"cni prostor. Preslikava $f: M \to M$ je \emph{skr"citev}, "ce obstaja $q \in \RR, q < 1$, da velja
\begin{equation*}
d(f(x), f(y)) \leq q d(x, y) \quad \forall x, y \in M
\end{equation*}
\textbf{Opombe:}
\begin{itemize}
    \item $q \geq 0$
    \item Vsaka skr"citev je enakomerno zvezna na $M$.
\end{itemize}

\textsc{Izrek} (Banachovo skr"citveno na"celo): Naj bo $(M, d)$ poln metri"cni prostor in $f: M \to M$ skr"citev. Potem obstaja natanko ena \emph{negibna to"cka} preslikave $f$, t.j.
\begin{equation*}
a \in M: f(a) = a
\end{equation*}
Za poljubno to"cko $x \in M$ zaporedje $x, f(x), f(f(x)), \ldots$ konvergira proti $a$.

\textsc{Dokaz:}
\begin{itemize}
    \item \textbf{enoli"cnost stacionarne to"cke}
    
    Recimo, da sta $a, b \in M$ in da je $f(a) = a$, ter $f(b) = b$. Ker je $f$ skr"citev, obstaja $q \in [0, 1)$, da je
    \begin{equation*}
    d(f(x), f(y)) \leq q d(x, y) \quad \forall x,y \in M
    \end{equation*}
    Torej 
    \begin{gather*}
    d(f(a), f(b)) \leq q d(a, b) \\
    d(a, b) \leq q d(a, b) \\
    \Rightarrow d(a, b) = 0 \Rightarrow a = b
    \end{gather*}
    
    \item \textbf{obstoj}
    
    Naj bo $x \in M$:
    \begin{equation*}
    \underbrace{f(x)}_{x_1}, \underbrace{f(f(x))}_{x_2}, \ldots
    \end{equation*}
    Definiramo
    \begin{equation*}
    x_m = \underbrace{f(f( \cdots f}_{m \times f}(x) \cdots ))
    \end{equation*}
    Dokazujemo, da je $\{ x_m \}$ Cauchyjevo. Brez "skode za splo"snost lahko predpostavimo $n > m$: in velja
    \begin{multline*}
    d(x_n, x_m) \leq d(x_n, x_{n-1}) + d(x_{n-1}, x_{n-2}) + \cdots + d(x_{m+2}, x_{m+1}) + d(x_{m+1}, x_m) \leq \\
    \leq d(x_m, x_{m+1}) (1 + q + q^2 + \cdots) = \dfrac{1}{1-q} d(x_m, x_{m+1})
    \end{multline*}
    Za vse $m$ velja
    \begin{equation*}
    d(x_m, x_{m+1}) \leq q d(x_{m-1}, x_m) \leq q^2 d(x_{m-2}, x_{m-1}) \leq \cdots \leq q^m d(x_1, x_0)
    \end{equation*}
    Torej za $n > m$ velja
    \begin{equation*}
    d(x_n, x_m) \leq \dfrac{q^m}{1 - q} d(x_1, x_0)
    \end{equation*}
    Ker $q^m \stackrel{m \to \infty}{\longrightarrow} 0$, je $\{ x_n \}$ Cauchyjevo. Ker je $M$ poln, je zaporedje $\{ x_m \}$ konvergentno in njegovo limito ozna"cimo z $a$.
    
    Doka"zimo, da je $a$ negibna, torej $f(a) = a$.
    \begin{equation*}
    f(a) = f(\lim_{n \to \infty} x_n) = \lim_{n \to \infty} f(x_n) = \lim_{n \to \infty} x_{n+1} = a
    \end{equation*}
\end{itemize}
\hfill $\square$

\textbf{Opomba:} $d(x_n, x_m) \leq \frac{q^m}{1-q} d(x_1, x_0)$. "Ce po"sljemo $n \to \infty$ dobimo
\begin{equation*}
d(a, x_m) \leq \dfrac{q^m}{1-q} d(x_1, x_0)
\end{equation*}
Tu smo uporabili, da je funkcija $x \mapsto d(x, x_0)$ zvezna. Dokaz je bil za DN.
