\documentclass[a4paper, 12pt]{article}
\usepackage[slovene]{babel}
\usepackage[utf8]{inputenc}
\usepackage[T1]{fontenc}
\usepackage{hyperref}
\usepackage{graphicx}
\usepackage{amssymb}
\usepackage{amsmath}
\usepackage{mathtools}
\usepackage{float}
\usepackage{multicol}
\usepackage{wasysym}
\usepackage{multicol}
\usepackage{enumerate}

\setlength{\parindent}{0px}
\setlength{\parskip}{10px}

\newcommand{\RR}{\mathbb{R}}
\newcommand{\NN}{\mathbb{N}}

\newcommand{\set}{\ensuremath{\text{Set}}}

\title{Logika in Množice}
\author{Vid Drobnič}
\date{}

\begin{document}
	\maketitle
	\thispagestyle{empty}
	\pagebreak
	\setcounter{page}{1}
	
	\tableofcontents
	\pagebreak
	
	\section{Mno"zice}
	\(A\) - mno"zica\\
\(x \in A\) - x je element \(A\)

\subsubsection*{Na"celo ekstenzionalnosti:} 
"Ce imata mno"zici iste elemte, sta enaki.

\underline{Kon"cna mno"zica:} \(\{a, b, c, ... z\}\), primer:
\[A=\{1, 2, 5\}\]
\[B = \{2, 1, 1, 5\}\]
\[A = B\]

\underline{Prazna mno"zica:} \(\{\}\) oznaka \(\varnothing \)

\underline{Enojec:} \(\{a\}\)

\underline{Dvojec ali neurejeni par:} \(\{a, b\}\) za katerikoli \(a\) in \(b \Rightarrow\) lahko sta enaka \(\Rightarrow\) enojec je posebni primer dvojca.
\[\{c, c\} = \{c\}\]

\underline{Standardni enojec:} \(1 = \{()\}\)

	
	\section{Preslikava ali Funkcija}
	\begin{itemize}
	\item[(1)] \textbf{domena}: mno"zica \(A\)
	\item[(2)] \textbf{kodomena}: mno"zica \(B\)
	\item[(3)] \textbf{prirejanje}: pove kako elementom iz \(A\) priredimo elemnte iz \(B\)
	\begin{itemize}
		\item \textbf{Celovitost:} vsakemu elementu iz \(A\) priredi vsaj 1 element iz \(B\)
		\item \textbf{Enoli"cnost:} "ce sta elementu \(x\) prirejena \(y_1\) in \(y_2\), potem velja \(y_1 = y_2\)
	\end{itemize}
\end{itemize}

\(A \rightarrow B\) (brezimna )preslikava iz \(A\) v \(B\) \\
\(A\) - domena\\
\(B\) - kodomena

\(f: A \rightarrow B\) funkcija (preslikava) poimenovana \(f\)\\
\(A \stackrel{f}{\rightarrow} B\)

\subsubsection*{Funkcijski predpis}
\[x \mapsto 1 + x^2\]
\(x\) se slika v \(1 + x^2\)

\[f: x \mapsto 1 + x^2\]
\[f(x) = 1 + x^2\]
\underline{Opomba:} funkciji manjka "se domena in kodomena.

\[\{1, 2, 5\} \rightarrow \{1, 2, 3, 4, ... 10\}\]
\[x \mapsto 1+ x^2\]

\(g(2)\): \(g\) uporabimo ali apliciramo na argumentu 2

\(g: \mathbb{R} \rightarrow \mathbb{R}\): predpis\\
\(g\): preslikava\\
\(g(3)\): "stevilo\\
\(g(x)\): "stevilo

\begin{itemize}
	\item[(1)] \(x \mapsto ax + b\) (\(x\) je vezana spremenljivka, \(a\) in \(b\) sta parametra)
	\item[(2)] \(a \mapsto ax + b\)
	\item[(3)] \(y \mapsto ay + b\)
\end{itemize}

(1) in (2) sta isti preslikavi.

\[g: \mathbb{R} \rightarrow \mathbb{R}\]
\[g(x) = 1 + x^3\]
\[g(7) = 1 + 7^3\]
\underline{Opomba:} ni treba izra"cunati.

\[\mathbb{R} \rightarrow \mathbb{R}\]
\[x \mapsto 1 + x^3\]
\[(x \mapsto 1 + x^3)(7) = 1 + 7^3\]
\[(x \mapsto ax + b)(7) = 7x + b\]
Uporaba funkcije - \textbf{aplikacija.}

\underline{Preslikave \(\varnothing \rightarrow A\)?}
\[\varnothing \rightarrow \{1, 2, 3\}\]
Prirejanje ``vsi elemtni domene se presliakjo v 1''.

\[x \mapsto 1\]
\[x \mapsto 2\]
Preslikavi sta enaki.

\underline{Sklep:} iz \(\varnothing \rightarrow A\) imamo natanko eno preslikavo.

\underline{Opomba:} Za vse elemte prazne mno"zice velja karkoli.
\[\mathbb{R} \rightarrow \mathbb{R}\]
\[x \mapsto x \cdot x\]
\[x \mapsto x \cdot x + x - x\]
Preslikavi sta enaki.

\subsubsection*{Na"celo ekstenzionalnosti preslikav:} "Ce imata preslkavi enaki domeni in enaki kodomeni, ter prirejata elementom domene enake vrednosti, potem sta enaki.
\[f: A \rightarrow B\]
\[g: C \rightarrow D\]

"Ce \(A = C\) in \(B = D\) in za vsak \(x \in A\) velja \(f(x) = g(x)\), potem \(f = g\).

Druga"ce povedano (se izpelje):\\
"Ce \(A = C\) in \(B = D\) in za vsak \(x_1, x_2 \in A\) velja, da iz \(x_1 = x_2\) sledi: \(f(x_1) = g(x_2)\), potem \(f = g\).

	
	\section{Aritmetika Mno"zic}
	\subsection{Kartezi"cni produkt ali zmno"zek}
\(A\) in \(B\) mno"zici\\
\(A \times B\) zmno"zek

Elementi \(A \times B\) so urejeni pari \((a, b)\), kjer sta \(a \in A\) in \(b \in B\).

\underline{Projekciji:}
\[\pi_1: A \times B \rightarrow A\]
\[\pi_2: A\times B \rightarrow B \]

\underline{Ena"cbe:}

Za vse \(a \in A\) in \(b \in B\) velja:
\[\pi_1(a, b) = a\]
\[\pi_2(a, b) = b\]

\subsubsection*{Ekstenzionalnost za zmno"zke:} Za vse \(p, q \in A \times B\), "ce \(\pi_1(p) = \pi_1(q)\) in \(\pi_2(p) = \pi_2(q)\),  potem \(p = q\)

\[f: A \times B \rightarrow C\]
\[f: p \mapsto ...\]
\[f: (x, y) \mapsto ... x ... y ...\]

\[g: A \rightarrow B \times C\]
\[g: a \mapsto (...a..., ...a...)\]

Kaj je \(\varnothing \times A\)? \(\varnothing \times A = \varnothing\)


\subsection{Eksponentna mno"zica}
"Ce sta \(A\) in \(B\) mno"zici, je \(B^A\) mno"zica vseh preslikav z domeno \(A\) in kodomeno \(B\).

\subsection{Vsota mno"zic}
"Ce sta \(A\) in \(B\) mno"zici je vsota \(A + B\) mno"zica.

Za vsak \(a \in A\) je \(\iota_1(a) \in A + B\)

Za vsak \(b \in B \) je \(\iota_2(b) \in A + B\)

Elementa \(u\) in \(v\) iz \(A + B\) sta enaka, "ce bodisi obstaja \(a \in A\) da je \(u = \iota_1(a)\) in \(v = \iota_1(a)\), bodisi obstaja \(b \in B\) da je \(u = \iota_2(b)\) in \(v = \iota_2(b)\).

\[\{1, 2\} + \{1, 2\} = \{\iota_1(1), \iota_1(2), \iota_2(1), \iota_2(2)\}\]

\subsection{Izomorfni mno"zici}
\underline{Def.:} Izomorfizem je preslikava \(f: A \rightarrow B\), za katero obstaja preslikava \(g: B \rightarrow A\), da je:
\begin{itemize}
	\item za vsak \(x \in A\) je \(g(f(x)) = x\) in
	\item za vsak \(y \in B\) je \(f(g(y)) = y\)
\end{itemize}

Pravimo da je \(g\) inverz \(f\).

"Ce obstaja izomorfizem \(X \rightarrow Y\), pravimo, da sta \(X\) in \(Y\) \textbf{izomorfni}, pi"semo \(X \cong Y\)

\subsection{Kompozitum}
\(B^A\) je mno"zica preslikav iz \(A\) v \(B\).

Kompozicija preslikav \(g \circ f\).
\[A \stackrel{f}{\rightarrow} B \stackrel{g}{\rightarrow} C\]
\begin{align*}
	\circ&: C^B \times B^A \rightarrow C^A\\
	\circ&: (g, f) \mapsto (x \mapsto g(f(x))) \text{ (ugnezden funkcijski prepis)}
\end{align*}

Pi"semo \(g \circ f\)

\underline{Zakaj ne raje \(f\) \textbullet \(\ g\)?} 

Npr., da imamo:
\begin{align*}
	\text{\textbullet}&: B^A \times C^B \rightarrow C^A\\
	\text{\textbullet}&: (f, g) \mapsto (x \mapsto g(f(x)))
\end{align*}

Ra"cunsko pravilo za \(\circ\):

\((g \circ f)(a) = g(f(a)) \checkmark\) izberemo, ker se ohrani vrstni red.\\
\((f \text{ \textbullet }\ g)(a) = g(f(a))\)

Imamo dve preslikavi:
\begin{multicols}{2}
	\[\mathbb{R} \rightarrow \mathbb{R}\]
	\[x \mapsto 4 - x^2\]
	
	\columnbreak
	\[\mathbb{R} \rightarrow \mathbb{R}\]
	\[x \mapsto 2 - x\]
\end{multicols}

\[(x \mapsto 4 - x^2)\circ(x \mapsto 2 - x)  = (x \mapsto (x\mapsto4-x^2)((x \mapsto 2-x)x))\]
Zaradi dvoumnosti preimenujemo vezane spremenljivke:
\begin{multicols}{2}
	\[x \mapsto 4 - x^2 \Rightarrow y \mapsto 4-y^2\]
	
	\columnbreak
	\[x \mapsto 2 - x \Rightarrow z \mapsto 2 - z\]
\end{multicols}
\[(y \mapsto 4 - y^2)\circ(z \mapsto 2 - z)  = (x \mapsto (y\mapsto4-y^2)((z \mapsto 2-z)x))\]

\textbf{Identiteta} na mno"zici \(A\) je preslikava:
\begin{align*}
	id_A&: A \rightarrow A\\
	id_A&: x \mapsto x
\end{align*}

\underline{Def:} \(f: A \rightarrow B, g: B \rightarrow A\) re"cemo, da je \(g\) \textbf{inverz} \(f\), ko velja:
\[f \circ g = id_B \land g \circ f = id_A\]

"Ce ime \(f\) inverz, pravimo, da je \(izomorfizem.\)

"Ce obstaja izomorfizem \(A \rightarrow B\), pravimo, da sta \(A\) in \(B\) \textbf{izomorfni} mno"zici. Pi"semo \(A \cong B\)       

\underline{Primeri:}
\begin{itemize}
	\item[(a)] 
	\(A \times \varnothing \cong \varnothing\)\\
	\(f: A \times \varnothing \rightarrow \varnothing\)\\
	\hspace*{24pt}Predpis ni potreben, ker ni nobenih elementov.\\
	\(g: \varnothing \rightarrow A \times \varnothing\)\\
	\hspace*{24pt}Iz prazne mno"zice obstaja ena sama preslikava.
	
	\item[(b)]
	\(1 = \{()\}\)\\
	\(A \times 1 \cong A\)
	\begin{multicols}{2}
		\begin{align*}
			f:& A \times 1 \rightarrow A\\
			&(x, y) \mapsto x
		\end{align*}
		
		\columnbreak
		\begin{align*}
			g:& A \rightarrow A \times 1\\
			& x \mapsto (x, ())
		\end{align*}
	\end{multicols}
	\[A \times 1 \rightarrow A \rightarrow A \times 1\]
	\[(x, y) \stackrel{f}{\mapsto} x \stackrel{g}{\mapsto} (x, ())\]
	
	\item[(c)] \(A^{B \times C} \cong (A^B)^C\)
	\begin{align*}
		\theta:& A^{B \times C} \rightarrow (A^B)^C\\
		\theta:& \sun \mapsto (c \mapsto (b \mapsto \sun(b, c)))
	\end{align*}
	\begin{align*}
		\phi:& (A^B)^C \rightarrow A^{B \times C}\\
		\phi:& \leftmoon \mapsto ((\beta, \gamma) \mapsto (\leftmoon(\gamma))(\beta))
	\end{align*}
\end{itemize}

	
	\section{Simbolni zapis}
	\subsection{Izjavni ra"cun:}
\begin{itemize}
	\item konstanti
	\begin{itemize}
		\item \(\bot\) - neresnica
		\item \(\top\) - resnica
	\end{itemize}

	\item  logi"cni vezniki:
	\begin{itemize}
		\item \(p \land q \) - \(p\) in \(q\) (\(p, q\) sta izjavi)
		\item \(p \lor q\) - \(p\) ali \(q\)
		\begin{multicols}{2}
			\item \(p \Rightarrow q\) 
			
			\columnbreak
			"ce \(p\) potem \(q\)\\
			iz \(p\) sledi \(q\)\\
			\(p\) je zadosten (pogoj) za \(q\)\\
			\(q\) je potreben (pogoj) za \(p\)
		\end{multicols}
	\begin{multicols}{2}
		\item \(p \Leftrightarrow q\) 
		
		\columnbreak
		\(p\) "ce in samo "ce \(q\)\\
		\(p\) "cee \(q\)\\
		\(p\) iff \(q\) (if and only if)
		\(p\) in \(q\) sta enakovredna ali ekvivalentna
	\end{multicols}
	\item \(\neg p\) - ne p
	\end{itemize}
\end{itemize}

\subsection{Predikatni ra"cun:}

Izjavni + \textbf{kvantifikatorja}
\begin{itemize}
	\item univerzalni kvantifikator:
	\begin{multicols}{2}
		\(\forall x \in B . p\)\\
		\((\forall x \in B)p\)\\
		\(\forall x \in B: p\)\\
		\(\forall x \in B (p)\)
		
		\columnbreak
		``za vsak \(x\) iz \(B\) velja \(p\)''
		
		``vsi \(x\)-i iz \(B\) zado"s"cajo \(p\)''
	\end{multicols}

	\item eksisten"cni kvalifikator
	\begin{multicols}{2}
		\(\exists x \in B . p\)
		
		\columnbreak
		``obstaja \(x\) iz \(B\), da velja \(p\)''
		
		``obstaja \(x\) iz \(B\), za katerega \(p\)''
		
		``za neki \(x\) iz \(B\) velja \(p\)''
	\end{multicols}
\end{itemize}

\subsection{Prednosti veznikov:}
Vezniki si po prednosti sledijo od tistega z najve"cjo, do tistega z najmanj"so v naslednjem vrstnem redu:
\[\neg, \land, \lor, (\Rightarrow, \Leftrightarrow), (\forall, \exists) \]

	
	\section{Dokazovanje}
	Dokaz ima drevesno strukturo in more biti kon"cen.

Vedeti moramo:
\begin{enumerate}
	\item Kaj trenutno dokazujemo
	\item Katere \emph{spremenljivke} in \emph{predpostavke} imamo na voljo (kontekst).
\end{enumerate}

\subsection{Oblika dokaza}
Za obliko glej zvezek. "Zal se mi ne da prepisovati vseh razli"cnih dokazov in skic kako naj izgledajo.

\subsection{Pravila sklepanja}
\subsubsection{Pravila upeljave}
\begin{enumerate}
	\item \emph{Resnica $\top$:} je res
	\item \emph{Neresnica $\bot$:} ni pravila
	\item \emph{Konjunkcija:} da doka"zemo $p \land q$ moramo dokazati $p$, nato pa "se $q$.
	\item \emph{Disjunkcija:} da doka"zemo $p \lor q$ lahko doka"zemo $p$, ali pa $q$.
	\item \emph{Implikacija:} da doka"zemo $p \Rightarrow q$, predpostavimo $p$ in nato doka"zemo $q$.
	\item \emph{Ekvivalenca:} ker je $p \Leftrightarrow q$ okraj"sava za $(p \Rightarrow q) \land (q \Rightarrow p)$, to doka"zemo tako, da po pravilu 5. najprej doka"zemo $p \Rightarrow q$, nato pa "se $q \Rightarrow p$.
	\item \emph{Negacija:} za dokaz $\lnot p$ predpostavimo $p$ in nato doka"zemo $\bot$. Druga"ce povedano: ``i"s"cemo protislovje''.
	\item \emph{Zakon o izklju"ceni tretji mo"znosti:}\footnote{posebno, osnovno pravilo} vemo da je $q$ ali pa $\lnot q$. Ne more biti oboje.
	\item \emph{Univerzalni kvalifikator:} za dokaz $\forall x \in A: p(x)$, najprej izberemo poljubni $x$ s trditvijo: ``Naj bo $x \in A$''\footnote{$x$ mora bit ``sve"z'', t.j: trenutno "se ne uporabljen.}, nato pa doka"zemo $p(x)$.
	\item \emph{Eksisten"cni kvalifikator:} da doka"zemo $\exists x \in A: p(x)$, si izberemo $x$ s trditvijo: ``Vzemimo $x := a$''. Nato najprej doka"zemo $a \in A$ in potem "se $p(a)$.
\end{enumerate}

\subsubsection{Pravila uporabe}
\begin{enumerate}
	\item \emph{Resnica $\top$:} ni uporabno.
	\item \emph{Neresnica $\bot$:} "ce vemo neresnico, lahko doka"zemo katerokli izjavo tako, da uporabimo neresnico.
	\item \emph{Konjunkcija:} "ce vemo $p \land q$, lahko re"cemo da vemo $p$, ali pa da vemo $q$.
	\item \emph{Disjunkcija:} "ce vemo $p \lor q$, lahko doka"zemo izjavo tako da ``Obravnavamo primera $p, q$ zaradi $p \lor q$''. Nato imamo dva primera. V enem predpostavimo $p$, v drugem pa $q$.
	\item \emph{Implikacija:} "ce vemo $p \Rightarrow q$ in vemo $p$, potem vemo $q$.
	\item \emph{Ekvivalenca:} "ce vemo $p \Leftarrow q$ vemo $p \Rightarrow q$ in $q \Rightarrow p$. Prav tako imamo tudi \emph{pravilo zamenjave}, ki pravi, da lahko $p$ nadomestimo s $q$ in obratno.
	\item \emph{Negacija:} "ce vemo $q$ in vemo $\lnot q$, velja $\bot$.
	\item \emph{Univerzalni kvantifikator:} "ce vemo $\forall a \in A: p(a)$ in vemo $a \in A$, potem vemo $p(a)$.
	\item \emph{Eksisten"cni kvantifikator:} "ce vemo $\exists x \in A: p(x)$. lahko re"cemo da imamo $x \in A$. Potem vemo $p(x)$.
\end{enumerate}

	
	\section{Boolova algebra}
	Izjava $p$ ima \emph{pomen} in \emph{resni"cnostno vrednost} ($\bot$ ali $\top$).

V izjavi $\lnot p \lor q$ sta $p$ in $q$ \emph{izjavna simbola}.

Mno"zica $2 = \{\bot, \top\}$ je \emph{mno"zica resni"cnostnih vrednosti}.

\emph{$n$-"clena Boolova preslikava} je
\begin{equation*}
\underbrace{2\times 2 \times \cdots \times 2}_{n} \rightarrow 2
\end{equation*}
Primer:
\begin{align*}
2\times 2 &\rightarrow 2\\
(p, q) &\mapsto \lnot p \lor q
\end{align*}

\emph{Tavtologija} je izjava, ki je resni"cna ne glede na vrednosti parametrov.

\subsubsection*{Zakon o zamenjavi ekvivalentnih izjav}
"Ce $p \iff q$ potem lahko $p$ nadomestimo s $q$, "ce gledamo le na resni"cnostno vrednost izjav.

\subsection{Zakoni Boolove algebre}
Operacije:
\begin{itemize}
	\item Konstanti: $\top, \bot$
	\item Negacija: $\lnot$
	\item Konjunkcija: $\land$
	\item Disjunkcija: $\lor$
\end{itemize}

Konjunkcija:
\begin{itemize}
	\item $p \land q = q \land p$
	\item $p \land (q \land r) = (p \land q) \land r$
	\item $p \land \top = p$
	\item $p \land p = p$
\end{itemize}

Disjunkcija:
\begin{itemize}
	\item $p \lor q = q \lor p$
	\item $p \lor (q \lor r) = (p \lor q) \lor r$
	\item $p \lor \bot = p$
	\item $p \lor p = p$
\end{itemize}

Distributivnost:
\begin{itemize}
	\item $(p \land q) \lor r = (p \lor r) \land (q \lor r)$
	\item $(p \lor q) \land r = (p \land r) \lor (q \land r)$
\end{itemize}

Absorpcija:
\begin{itemize}
	\item $(q \land p) \lor p = p$
	\item $(q \lor q) \land p = p$
\end{itemize}

Negacija:
\begin{itemize}
	\item $p \land \neg p = \bot$
	\item $p \lor \lnot p = \top$
\end{itemize}

\emph{Izrek:} (za izjavo $p$ v kateri nastopajo samo izjavni simboli $q_1 \ldots q_n$)
\begin{enumerate}
	\item "Ce ima izjava dokaz je tavtologija.
	\item "Ce je izjava tavtologija ima dokaz.
\end{enumerate}
Izrek ne velja za izjave, ki vsebujejo parametre iz mno"zic.

\subsection{Polni nabori}
Nabor operacij je \emph{poln}, "ce lahko z njim dobimo poljubno resni"cnostno tabelo.

Primeri:
\begin{itemize}
	\item $\top, \bot, \land, \lor, \lnot$ je poln
	\item $\top, \lnot, \land$ je poln
	\item $\bot, \uparrow \text{(nand)}$ je poln
\end{itemize}

\subsection{Ra"cunska pravila}
Pravila za $\top$:
\begin{itemize}
	\item $p \lor \top = \top$
	\item $p \land \top = p$
	\item $\lnot \top = \bot$
\end{itemize}

Pravila za $\bot$:
\begin{itemize}
	\item $p \lor \bot = p$
	\item $p \land \bot = \bot$
	\item $\lnot \bot = \top$
\end{itemize}

Pravila za negacijo:
\begin{itemize}
	\item $\lnot \lnot p = p$
	\item de Morganova pravila:
	\begin{itemize}
		\item $\lnot (p \land q) = \lnot p \lor \lnot q$
		\item $\lnot (p \lor q) = \lnot p \land \lnot q$
	\end{itemize}
\end{itemize}

Ostalo (\emph{kontrapozitivna oblika}):
\begin{itemize}
	\item $(p \Rightarrow q) = (\lnot q \Rightarrow \lnot p)$
	\item $(p \lor q) = (\lnot p \Rightarrow q)$
	\item $(p \Rightarrow q) = (\lnot p \lor q)$
\end{itemize}

Izjava ima lahko dve obliki:
\begin{itemize}
	\item \emph{konjunktivna} oblika: $(\lnot p \lor q) \land r \land (r \lor \lnot p)$
	\item \emph{disjunktivna} oblika: $(u \land \lnot v) \lor (u \land w \land \lnot u)$
\end{itemize}

\subsection{Pravila za kvantifikatorje}
\begin{itemize}
	\item $(\lnot \exists x \in A. p(x)) \iff (\forall x \in A . \lnot p(x))$
	\item $(\lnot \forall x \in A . p(x)) \iff (\exists x \in A . \lnot p(x))$
	\item $(\forall x \in \varnothing . p(x)) \iff \top$
	\item $(\exists x \in \varnothing . p(x)) \iff \bot$
	\item $(p \Rightarrow \forall x \in A . q(x)) \iff (\forall x \in A . p \Rightarrow q(x))$
	\item $(\forall u \in A \times B . p(u)) \iff (\forall x \in A \forall y \in B . p(x, y))$
	\item $(\exists u \in A \times B . p(u)) \iff (\exists x \in A \exists y \in B . p(x, y))$
	\item $(\forall u \in A + B . p(u)) \iff (\forall x \in A . p(\iota_1(x))) \land (\forall y \in B . p (\iota_2(y)))$
	\item $(\forall u \in A \cup B . p(u)) \iff (\forall a \in A. p(a)) \land (\forall b \in B . p(b))$
	\item $(\forall x \in \{a\} . p(x)) \iff p(a)$
	\item $(\exists x \in \{a\} . p(x)) \iff p(a)$
\end{itemize}
	
	\section{Definicije in enoli"cen opis}
	\begin{enumerate}[1)]
	\item Okraj"sava, uvedemo nov simbol
	\begin{gather*}
		c := \cdots\\
		c \stackrel{\triangle}{=} \cdots\\
		c \stackrel{\text{def}}{=} \cdots\\
		c = \cdots\\
		f(x): = \cdots
	\end{gather*}
	
	\item Enoli"cen opis
	\begin{gather*}
		\exists! x \in A . p(x)\\
		\exists^1 x \in A .p(x)
	\end{gather*}
	``obstaja natanko en $x \in A$, da velja $p(x)$''
	
	To je okraj"sva za:
	\begin{equation*}
	(\exists x \in A .p(x)) \land (\forall y, z \in A . p(y) \land p(z) \Rightarrow y = z)
	\end{equation*}
	
	"Ce doka"zemo
	\begin{equation*}
	\exists! x \in A . p(x)
	\end{equation*}
	potem lahko uvedemo novo oznako $c$ in pravilo
	\begin{equation*}
	c \in a \text{ in } p(c)
	\end{equation*}
	
	Lahko pi"semo tudi:
	\begin{equation*}
	\iota x \in A . p(x)
	\end{equation*}
	kar pomeni ``tisti $x \in A$, za katerega velja $p(x)$'', podobno kot anonimna funkcija. Primer uporabe:
	\begin{equation*}
	(\iota y \in \RR . y^3 = 2)^6 + 7 = 11
	\end{equation*}
\end{enumerate}

	
	\section{Podmno"zice}
	\emph{Definicija:} Za mno"zici $A$ in $B$:
\begin{align*}
A \subseteq B &:= \forall x \in A . x \in B\\
\subseteq &:= (A, B) \mapsto \forall x \in A . x \in B
\end{align*}
Namesto $\subseteq(A, B)$ pi"semo $A \subseteq B$.

Konstrukcija podmno"zice:
\begin{itemize}
	\item mno"zica $A$
	\item izjava $p(x)$, kjer $x \in A$
\end{itemize}
Tvorimo mno"zico:
\begin{equation*}
\{x \in A | p(x)\}
\end{equation*}
Elementi te mno"zico so natanko tisti $a \in A$, za katere velja $p(a)$.

Ostali zapsi so:
\begin{gather*}
	\{x \in A: p(x)\}\\
	\{x \in A ; p(x)\}
\end{gather*}

Ra"cunski pravili:
\begin{enumerate}[1)]
	\item
	\begin{equation*}
	(\forall x \in \{y \in A | p(y)\} . q(x)) \iff (\forall z \in A. p(z) \Rightarrow q(z))
	\end{equation*}
	
	\item 
	\begin{equation*}
	(\exists x \in \{y \in A | p(y)\} . q(x)) \iff (\exists z \in A. p(z) \land q(z))
	\end{equation*}
\end{enumerate}

	
	\section{Poten"cne mno"zice}
	$\mathcal{P}(A)$ je poten"cna mno"zica $A$. Njeni elementi so natanko vse podmno"zice $A$.

Primeri:
\begin{gather*}
\mathcal{P}(\{1, 7\}) = \{\varnothing, \{1\}, \{7\}, \{1, 7\}\}\\
\mathcal{P}(\varnothing) = \{\varnothing\}
\end{gather*}

Spomnimo: $2 = \{\bot, \top\}$

Podmno"zice $A$ so preslikave $A \rightarrow 2$.

Izrek: $\mathcal{P}(A) \cong 2^A$
\begin{align*}
\mathcal{P}(A) &\rightarrow 2^A\\
\chi : S &\mapsto \left(x \mapsto \begin{cases}
\bot & x \notin S\\
\top & x \in S
\end{cases}\right)
\end{align*}
%
\begin{align*}
2^A &\rightarrow \mathcal{P}(A)\\
f &\mapsto \{x \in A | f(x)\}
\end{align*}

Nato te funkcije se preverimo, kot smo delali "ze mnogokrat na vajah.

\subsection{Boolova algebra na $\mathcal{P}(A)$}
Imamo operacije $\cup, \cap$, komplement
\begin{align*}
S \cap T &:= \{x \in A | x \in S \land x \in T\}\\
S \cup T &:= \{x \in A | x \in S \lor x \in T\}\\
\varnothing &:= \{x \in A | \bot\}\\
A &:= \{x \in A| \top\}\\
S^C &:= \{x \in A| \lnot (x \in S)\}
\end{align*}

	
	\section{Razredi}
	Vzemimo mno"zico vseh mno"zic
\begin{equation*}
V = \{x | \text{$x$ je mno"zica}\}
\end{equation*}
Definirajmo podmno"zico:
\begin{equation*}
R  = \{x \in V | x \notin x\}
\end{equation*}
Dokazali bomo $R \notin R$ in $R \in R$:
\begin{enumerate}[1)]
	\item $R \notin R$
	
	Predpostavimo $R \in R$ in i"s"cemo protislovje. Po predpostavki vemo $R \in R$. To pomeni, da po definiciji $R$ velja $R \notin R$, s "cimer smo pri"sli do protilsovja, torej velja $R \notin R$.
	
	\item $R \in R$
	
	To bomo dokazali s protislovjem (pozor: prej"sen dokaz je bil dokaz negacije!). Predpostavimo $R \notin R$ in i"s"cemo protislovje. Po predpostavki vemo, da $R \notin R$, kar pomeni da po definiciji $R$ velja $R \in R$. Pri"sli smo do protislovja, kar pomeni da velja $R \in R$.
\end{enumerate}
Dokazali smo $\bot$, torej velja vse. Tudi tak"sne nesmiselnosti kot $0 = 1$.

Da se znebimo tega problema uvedemo razred, ki ga tvorimo\footnote{Tvorba je razli"cna od tvorbe mno"zic. Za mno"zice imamo to"cno dolo"cene na"cine tvorbe (kartezi"cni produkt, podmno"zica, presek, unija, \dots)}:
\begin{equation*}
\{x | p(x)\}
\end{equation*}
Velja:
\begin{equation*}
a \in \{x | p(x)\} \iff p(a)
\end{equation*}
Pri tem je $a$ bodisi osnovni matemati"cni objekt ("stevilo, urejeni par) ali mno"zica, ne sme pa biti razred. Druga"ce povedano: razredi niso elementi.

Razred $C$ je mno"zica, "ce lahko tvorimo mno"zico, ki ima iste elemente kot $C$
\begin{equation*}
a \in C \iff a \in S
\end{equation*}
kjer je $S$ mno"zica.

Vsaka mno"zica $S$ je razred:
\begin{equation*}
\{x | x \in S\}
\end{equation*}
Razred, ki ni mno"zica se imenuje \emph{pravi razred}.

Primeri pravih razredov:
\begin{itemize}
	\item Razred vseh mno"zic:
	\begin{equation*}
	V = \{x | \text{$x$ je mno"zica}\} = \{x | \top\}
	\end{equation*}
	oznaka za tak razred je Set.
	
	\item $R = \{x | x \notin x\}$
	
	\item  $\{A | \exists!x \in A:\top\}$ razred vseh enojcev
	
	\item $\{X | \text{$X$ je vektorski prostor}\}$
	
	$\{X | \text{$X$ je grupa}\}$
\end{itemize}
	
	\section{Dru"zine mno"zic}
	Imamo naslednje mno"zice:
\begin{align*}
A_0 &= \cdots \\
A_1 &= \cdots \\
A_2 &= \cdots
\end{align*}
Dru"zina mno"zic je preslikava:
\begin{equation*}
A : I \rightarrow \set
\end{equation*}
kjer $I$ je indeksna mno"zica in $i \in I$ so indeksi.

Namesto $A(i)$ pi"semo $A_i$.

\textsc{Primeri:}
\begin{enumerate}[1)]
	\item "Ce imamo mno"zice $A, B, C, D, E$, lahko tvorimo dru"zino:
	\begin{gather*}
		I = \{1, 2, 3, 4, 5\}\\
		Q : I \rightarrow \set \\
		Q_1 = A, Q_2 = B, Q_3 = C, Q_4 = D, Q_5 = E
	\end{gather*}
	
	\item Dru"zina vseh zaprtih intervalov:
	\begin{gather*}
		K = \{(a, b) \in \RR \times \RR | a \leq b\}\\
		I: K \rightarrow \set\\
		I(a, b) := [a, b] = \{x \in \RR | a \leq x \leq b\}
	\end{gather*}
	
	\item Nekateri elementi dru"zine so lahko enaki:
	\begin{gather*}
		I = \{1, 2, 3, 4, 5\} \\
		A : I \rightarrow \set
	\end{gather*}
	lahko velja $A_1 = A_3$.
	
	\item Konstanta dru"zina $A: I \rightarrow \set$.
	\begin{equation*}
	\forall i, j \in I: A_i = A_j
	\end{equation*}
	
	\item \emph{Prazna dru"zina} $\varnothing \rightarrow \set$
	
	\item \emph{Dru"zina praznih mno"zic}
	\begin{gather*}
		A : I \rightarrow \set \\
		\forall i \in I: A_i = \varnothing
	\end{gather*}
	
	\item \emph{Neprazna dru"zina}
	\begin{gather*}
		A: I \rightarrow \set \\
		I \neq \varnothing
	\end{gather*}
	
	\item \emph{Dru"zina nepraznih}
	\begin{gather*}
		A : I \rightarrow \set \\
		\forall i \in I: A_i  \neq \varnothing
	\end{gather*}
\end{enumerate}

\subsection{Konstrukcija z dru"zinami mno"zic}
Naj bo $A: I \rightarrow \set$ dru"zina.

\emph{Funkcija izbire} $f$ za dano dru"zino $A$ je prirejanje, ki vsakemu $i \in I$ priredi natanko en element $f(i) \in A_i$.

\textsc{Primer:} dru"zina vseh zaprtih intervalov
\begin{gather*}
I = \{(a, b) \in \RR \times \RR | a \leq b\} \\
K(a,b) = [a, b] \\
f(a, b) = \dfrac{a + b}{2}\\
g(a, b) = b
\end{gather*}
$f$ in $g$ sta primera funkcije izbire.

"Ce imamo $A: I \rightarrow \set$ in $A_j = \varnothing$ za neki $j \in I$, potem za $A$ ni nobene funkcije izbire.
\subsubsection{Kartezi"cni produkt}
\begin{equation*}
\prod_{i \in I} A_i
\end{equation*}
Elementi so funkcije izbire za $A$.

Za vsak $i \in I$ imamo $i$-to projekcijo:
\begin{gather*}
\pi_i : \prod_{j \in I} A_j \rightarrow A_i \\
f \mapsto f(i)
\end{gather*}

$B \times C$ je poseben primer:
\begin{equation*}
B \times C \cong \prod_{i \in I}A_i
\end{equation*}
kjer $I = \{1, 2\}$ in $A_1 = B, A_2 = C$.

Tudi $C^B$ je poseben primer
\begin{equation*}
C^B \cong \prod_{j \in J} D_j
\end{equation*}
kjer $J = B$ in $D_j = C$.

\subsubsection{Unija in presek}
\begin{gather*}
\bigcup_{i \in I} A_i = \{x; \exists i \in I : x \in A_i\}\\
\bigcap_{i \in I} A_i = \{x; \forall i \in I : x \in A_i\}
\end{gather*}

Presek prazne dru"zine:
\begin{equation*}
\bigcap_{i \in \varnothing} A_i \ \{x; \forall i \in \varnothing: x \in A_i\} = \{x; \top\} = V
\end{equation*}
je pravi razred.

Presek neprazne dru"zine je mno"zica, "ce imamo $j \in I$
\begin{equation*}
\bigcap_{i \in I} A_i = \{x; \forall i \in I: x \in A_i\} = \{x \in A_j; \forall i \in I: x \in A_i\}
\end{equation*}

\textsc{Aksiom o uniji: } Unija dru"zine mno"zic je mno"zica.

\textsc{Primer:}
\begin{gather*}
A : \NN \rightarrow \set\\
A_0 = \NN\\
A_1 = P(\NN)\\
A_2 = P(P(\NN))\\
A_{n+1} = P(A_n)
\end{gather*}

$\bigcup_{n \in \NN} A_n$ je unija po aksiomu.

Ra"cunska pravila z $\in$:
\begin{itemize}
	\item $x \in \varnothing \iff \varnothing$
	\item $x \in A \times B \iff \pi_1(x) \in A \land \pi_2(x) \in B$
	\item $x \in \{y \in A| P(y)\} \iff x \in A \land P(x)$
	\item $x \in A \cup B \iff x \in A \lor x \in B$
	\item $x \in \bigcup_{i \in I} A_i \iff \exists i \in I: x \in A_i$
	\item $x \in \bigcap_{i \in I} A_i \iff \forall i \in I: x \in A_i$
\end{itemize}

\subsubsection{Vsota ali koprodukt dru"zine mno"zic}
Dru"zina $A: I \rightarrow \set$

$\coprod_{i \in I} A_i$ je koprodukt dru"zine $A$. Elementi takega koprodukta so $\iota_k(x)$, kjer je $k \in I$ in $x \in A_k$.
\begin{equation*}
\coprod_{i \in I} A_i = \{\iota_k(x) | k \in I \land x \in A_k\}
\end{equation*}
\textbf{Opomba:} Na tak na"cin ponavadi zapi"semo razred, ki pa v tem primeru ni pravi razred in ga zato lahko obravnavamo kot mno"zico.

$\sum_{i \in I} A_i$ je vsota dru"zine $A$. Elementi so tako kot pri koproduktu $\iota_k(x)$ za $k \in I$ in $x \in A_k$. Elemente lahko zapi"semo tudi kot \emph{odvisne pare} $(k, x)$ za $k \in I$ in $x \in A_k$, kar je samo drug zapis za $\iota_k(x)$.

\textsc{Velja:}
\begin{align*}
B + C \cong \sum_{i \in \{1, 2\}} A_i && A: \{1, 2\} \rightarrow \set\\
&& A_1 = B \\
&& A_2 = C \\
B \times C \cong \sum_{b \in B} A_b && A: B \rightarrow \set \\
&& A_b = C
\end{align*}
\end{document}
