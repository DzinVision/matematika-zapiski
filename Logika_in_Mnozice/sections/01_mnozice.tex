\(A\) - mno"zica\\
\(x \in A\) - x je element \(A\)

\subsubsection*{Na"celo ekstenzionalnosti:} 
"Ce imata mno"zici iste elemte, sta enaki.

\underline{Kon"cna mno"zica:} \(\{a, b, c, ... z\}\), primer:
\[A=\{1, 2, 5\}\]
\[B = \{2, 1, 1, 5\}\]
\[A = B\]

\underline{Prazna mno"zica:} \(\{\}\) oznaka \(\varnothing \)

\underline{Enojec:} \(\{a\}\)

\underline{Dvojec ali neurejeni par:} \(\{a, b\}\) za katerikoli \(a\) in \(b \Rightarrow\) lahko sta enaka \(\Rightarrow\) enojec je posebni primer dvojca.
\[\{c, c\} = \{c\}\]

\underline{Standardni enojec:} \(1 = \{()\}\)
