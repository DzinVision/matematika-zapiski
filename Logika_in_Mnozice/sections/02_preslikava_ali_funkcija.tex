\begin{itemize}
	\item[(1)] \textbf{domena}: mno"zica \(A\)
	\item[(2)] \textbf{kodomena}: mno"zica \(B\)
	\item[(3)] \textbf{prirejanje}: pove kako elementom iz \(A\) priredimo elemnte iz \(B\)
	\begin{itemize}
		\item \textbf{Celovitost:} vsakemu elementu iz \(A\) priredi vsaj 1 element iz \(B\)
		\item \textbf{Enoli"cnost:} "ce sta elementu \(x\) prirejena \(y_1\) in \(y_2\), potem velja \(y_1 = y_2\)
	\end{itemize}
\end{itemize}

\(A \rightarrow B\) (brezimna )preslikava iz \(A\) v \(B\) \\
\(A\) - domena\\
\(B\) - kodomena

\(f: A \rightarrow B\) funkcija (preslikava) poimenovana \(f\)\\
\(A \stackrel{f}{\rightarrow} B\)

\subsubsection*{Funkcijski predpis}
\[x \mapsto 1 + x^2\]
\(x\) se slika v \(1 + x^2\)

\[f: x \mapsto 1 + x^2\]
\[f(x) = 1 + x^2\]
\underline{Opomba:} funkciji manjka "se domena in kodomena.

\[\{1, 2, 5\} \rightarrow \{1, 2, 3, 4, ... 10\}\]
\[x \mapsto 1+ x^2\]

\(g(2)\): \(g\) uporabimo ali apliciramo na argumentu 2

\(g: \mathbb{R} \rightarrow \mathbb{R}\): predpis\\
\(g\): preslikava\\
\(g(3)\): "stevilo\\
\(g(x)\): "stevilo

\begin{itemize}
	\item[(1)] \(x \mapsto ax + b\) (\(x\) je vezana spremenljivka, \(a\) in \(b\) sta parametra)
	\item[(2)] \(a \mapsto ax + b\)
	\item[(3)] \(y \mapsto ay + b\)
\end{itemize}

(1) in (2) sta isti preslikavi.

\[g: \mathbb{R} \rightarrow \mathbb{R}\]
\[g(x) = 1 + x^3\]
\[g(7) = 1 + 7^3\]
\underline{Opomba:} ni treba izra"cunati.

\[\mathbb{R} \rightarrow \mathbb{R}\]
\[x \mapsto 1 + x^3\]
\[(x \mapsto 1 + x^3)(7) = 1 + 7^3\]
\[(x \mapsto ax + b)(7) = 7x + b\]
Uporaba funkcije - \textbf{aplikacija.}

\underline{Preslikave \(\varnothing \rightarrow A\)?}
\[\varnothing \rightarrow \{1, 2, 3\}\]
Prirejanje ``vsi elemtni domene se presliakjo v 1''.

\[x \mapsto 1\]
\[x \mapsto 2\]
Preslikavi sta enaki.

\underline{Sklep:} iz \(\varnothing \rightarrow A\) imamo natanko eno preslikavo.

\underline{Opomba:} Za vse elemte prazne mno"zice velja karkoli.
\[\mathbb{R} \rightarrow \mathbb{R}\]
\[x \mapsto x \cdot x\]
\[x \mapsto x \cdot x + x - x\]
Preslikavi sta enaki.

\subsubsection*{Na"celo ekstenzionalnosti preslikav:} "Ce imata preslkavi enaki domeni in enaki kodomeni, ter prirejata elementom domene enake vrednosti, potem sta enaki.
\[f: A \rightarrow B\]
\[g: C \rightarrow D\]

"Ce \(A = C\) in \(B = D\) in za vsak \(x \in A\) velja \(f(x) = g(x)\), potem \(f = g\).

Druga"ce povedano (se izpelje):\\
"Ce \(A = C\) in \(B = D\) in za vsak \(x_1, x_2 \in A\) velja, da iz \(x_1 = x_2\) sledi: \(f(x_1) = g(x_2)\), potem \(f = g\).
