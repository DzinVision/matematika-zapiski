\subsection{Kartezi"cni produkt ali zmno"zek}
\(A\) in \(B\) mno"zici\\
\(A \times B\) zmno"zek

Elementi \(A \times B\) so urejeni pari \((a, b)\), kjer sta \(a \in A\) in \(b \in B\).

\underline{Projekciji:}
\[\pi_1: A \times B \rightarrow A\]
\[\pi_2: A\times B \rightarrow B \]

\underline{Ena"cbe:}

Za vse \(a \in A\) in \(b \in B\) velja:
\[\pi_1(a, b) = a\]
\[\pi_2(a, b) = b\]

\subsubsection*{Ekstanzionalnost za zmno"zke:} Za vse \(p, q \in A \times B\), "ce \(\pi_1(p) = \pi_1(q)\) in \(\pi_2(p) = \pi_2(q)\),  potem \(p = q\)

\[f: A \times B \rightarrow C\]
\[f: p \mapsto ...\]
\[f: (x, y) \mapsto ... x ... y ...\]

\[g: A \rightarrow B \times C\]
\[g: a \mapsto (...a..., ...a...)\]

Kaj je \(\varnothing \times A\)? \(\varnothing \times A = \varnothing\)


\subsection{Eksponentna mno"zica}
"Ce sta \(A\) in \(B\) mno"zici, je \(B^A\) mno"zica vseh preslikav z domeno \(A\) in kodomeno \(B\).

\subsection{Vsota mno"zic}
"Ce sta \(A\) in \(B\) mno"zici je vsota \(A + B\) mno"zica.

Za vsak \(a \in A\) je \(\iota_1(a) \in A + B\)

Za vsak \(b \in B \) je \(\iota_2(b) \in A + B\)

Elementa \(u\) in \(v\) iz \(A + B\) sta enaka, "ce bodisi obstaja \(a \in A\) da je \(u = \iota_1(a)\) in \(v = \iota_1(a)\), bodisi obstaja \(b \in B\) da je \(u = \iota_2(b)\) in \(v = \iota_2(b)\).

\[\{1, 2\} + \{1, 2\} = \{\iota_1(1), \iota_1(2), \iota_2(1), \iota_2(2)\}\]

\subsection{Izomorfni mno"zici}
\underline{Def.:} Izomorfizem je preslikava \(f: A \rightarrow B\), za katero obstaja preslikava \(g: B \rightarrow A\), da je:
\begin{itemize}
	\item za vsak \(x \in A\) je \(g(f(x)) = x\) in
	\item za vsak \(y \in B\) je \(f(g(y)) = y\)
\end{itemize}

Pravimo da je \(g\) inverz \(f\).

"Ce obstaja izomorfizem \(X \rightarrow Y\), pravimo, da sta \(X\) in \(Y\) \textbf{izomorfni}, pi"semo \(X \cong Y\)

\subsection{Kompozitum}
\(B^A\) je mno"zica preslikav iz \(A\) v \(B\).

Kompozicija preslikav \(g \circ f\).
\[A \stackrel{f}{\rightarrow} B \stackrel{g}{\rightarrow} C\]
\begin{align*}
	\circ&: C^B \times B^A \rightarrow C^A\\
	\circ&: (g, f) \mapsto (x \mapsto g(f(x))) \text{ (ugnezden funkcijski prepis)}
\end{align*}

Pi"semo \(g \circ f\)

\underline{Zakaj ne raje \(f\) \textbullet \(\ g\)?}

Npr, da imamo:
\begin{align*}
	\text{\textbullet}&: B^A \times C^B \rightarrow C^A\\
	\text{\textbullet}&: (f, g) \mapsto (x \mapsto g(f(x)))
\end{align*}

Ra"cunsko pravilo za \(\circ\):

\((g \circ f)(a) = g(f(a))\) \checkmark izberemo, ker se ohrani vrstni red.\\
\((f \text{ \textbullet }\ g)(a) = g(f(a))\)

Imamo dve preslikavi:
\begin{multicols}{2}
	\[\mathbb{R} \rightarrow \mathbb{R}\]
	\[x \mapsto 4 - x^2\]
	
	\columnbreak
	\[\mathbb{R} \rightarrow \mathbb{R}\]
	\[x \mapsto 2 - x\]
\end{multicols}

\[(x \mapsto 4 - x^2)\circ(x \mapsto 2 - x)  = (x \mapsto (x\mapsto4-x^2)((x \mapsto 2-x)x))\]
Zaradi dvoumnosti preimenujemo vezane spremenljivke:
\begin{multicols}{2}
	\[x \mapsto 4 - x^2 \Rightarrow y \mapsto 4-y^2\]
	
	\columnbreak
	\[x \mapsto 2 - x \Rightarrow z \mapsto 2 - z\]
\end{multicols}
\[(y \mapsto 4 - y^2)\circ(z \mapsto 2 - z)  = (x \mapsto (y\mapsto4-y^2)((z \mapsto 2-z)x))\]

\textbf{Identiteta} na mno"zici \(A\) je preslikava:
\begin{align*}
	id_A&: A \rightarrow A\\
	id_A&: x \mapsto x
\end{align*}

\underline{Def:} \(f: A \rightarrow B, g: B \rightarrow A\) re"cemo, da je \(g\) \textbf{inverz} \(f\), ko velja:
\[f \circ g = id_B \land g \circ f = id_A\]

"Ce ime \(f\) inverz, pravimo, da je \(izomorfizem.\)

"Ce obstaja izomorfizem \(A \rightarrow B\), pravimo, da sta \(A\) in \(B\) \textbf{izomorfni} mno"zici. Pi"semo \(A \cong B\)       

\underline{Primeri:}
\begin{itemize}
	\item[(a)] 
	\(A \times \varnothing \cong \varnothing\)\\
	\(f: A \times \varnothing \rightarrow \varnothing\)\\
	\hspace*{24pt}Predpis ni potreben, ker ni nobenih elementov.\\
	\(g: \varnothing \rightarrow A \times \varnothing\)\\
	\hspace*{24pt}Iz prazne mno"zice obstaja ena sama preslikava.
	
	\item[(b)]
	\(1 = \{()\}\)\\
	\(A \times 1 \cong A\)
	\begin{multicols}{2}
		\begin{align*}
			f:& A \times 1 \rightarrow A\\
			&(x, y) \mapsto x
		\end{align*}
		
		\columnbreak
		\begin{align*}
			g:& A \rightarrow A \times 1\\
			& x \mapsto (x, ())
		\end{align*}
	\end{multicols}
	\[A \times 1 \rightarrow A \rightarrow A \times 1\]
	\[(x, y) \stackrel{f}{\mapsto} x \stackrel{g}{\mapsto} (x, ())\]
	
	\item[(c)] \(A^{B \times C} \cong (A^B)^C\)
	\begin{align*}
		\theta:& A^{B \times C} \rightarrow (A^B)^C\\
		\theta:& \sun \mapsto (c \mapsto (b \mapsto \sun(b, c)))
	\end{align*}
	\begin{align*}
		\phi:& (A^B)^C \rightarrow A^{B \times C}\\
		\phi:& \leftmoon \mapsto ((\beta, \gamma) \mapsto (\leftmoon(\gamma))(\beta))
	\end{align*}
\end{itemize}
