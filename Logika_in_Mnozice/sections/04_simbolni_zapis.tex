\subsection{Izjavni ra"cun:}
\begin{itemize}
	\item konstanti
	\begin{itemize}
		\item \(\bot\) - neresnica
		\item \(\top\) - resnica
	\end{itemize}

	\item  logi"cni vezniki:
	\begin{itemize}
		\item \(p \land q \) - \(p\) in \(q\) (\(p, q\) sta izjavi)
		\item \(p \lor q\) - \(p\) ali \(q\)
		\begin{multicols}{2}
			\item \(p \Rightarrow q\) 
			
			\columnbreak
			"ce \(p\) potem \(q\)\\
			iz \(p\) sledi \(q\)\\
			\(p\) je zadosten (pogoj) za \(q\)\\
			\(q\) je potreben (pogoj) za \(p\)
		\end{multicols}
	\begin{multicols}{2}
		\item \(p \Leftrightarrow q\) 
		
		\columnbreak
		\(p\) "ce in samo "ce \(q\)\\
		\(p\) "cee \(q\)\\
		\(p\) iff \(q\) (if and only if)
		\(p\) in \(q\) sta enakovredna ali ekvivalentna
	\end{multicols}
	\item \(\neg p\) - ne p
	\end{itemize}
\end{itemize}

\subsection{Predikatni ra"cun:}

Izjavni + \textbf{kvantifikatorja}
\begin{itemize}
	\item univerzalni kvantifikator:
	\begin{multicols}{2}
		\(\forall x \in B . p\)\\
		\((\forall x \in B)p\)\\
		\(\forall x \in B: p\)\\
		\(\forall x \in B (p)\)
		
		\columnbreak
		``za vsak \(x\) iz \(B\) velja \(p\)''
		
		``vsi \(x\)-i iz \(B\) zado"s"cajo \(p\)''
	\end{multicols}

	\item eksisten"cni kvalifikator
	\begin{multicols}{2}
		\(\exists x \in B . p\)
		
		\columnbreak
		``obstaja \(x\) iz \(B\), da velja \(p\)''
		
		``obstaja \(x\) iz \(B\), za katerega \(p\)''
		
		``za neki \(x\) iz \(B\) velja \(p\)''
	\end{multicols}
\end{itemize}

\subsection{Prednosti veznikov:}
Vezniki si po prednosti sledijo od tistega z najve"cjo, do tistega z najmanj"so v naslednjem vrstnem redu:
\[\neg, \land, \lor, (\Rightarrow, \Leftrightarrow), (\forall, \exists) \]
