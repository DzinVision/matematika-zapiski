Izjava $p$ ima \emph{pomen} in \emph{resni"cnostno vrednost} ($\bot$ ali $\top$).

V izjavi $\lnot p \lor q$ sta $p$ in $q$ \emph{izjavna simbola}.

Mno"zica $2 = \{\bot, \top\}$ je \emph{mno"zica resni"cnostnih vrednosti}.

\emph{$n$-"clena Boolova preslikava} je
\begin{equation*}
\underbrace{2\times 2 \times \cdots \times 2}_{n} \rightarrow 2
\end{equation*}
Primer:
\begin{align*}
2\times 2 &\rightarrow 2\\
(p, q) &\mapsto \lnot p \lor q
\end{align*}

\emph{Tavtologija} je izjava, ki je resni"cna ne glede na vrednosti parametrov.

\subsubsection*{Zakon o zamenjavi ekvivalentnih izjav}
"Ce $p \iff q$ potem lahko $p$ nadomestimo s $q$, "ce gledamo le na resni"cnostno vrednost izjav.

\subsection{Zakoni Boolove algebre}
Operacije:
\begin{itemize}
	\item Konstanti: $\top, \bot$
	\item Negacija: $\lnot$
	\item Konjunkcija: $\land$
	\item Disjunkcija: $\lor$
\end{itemize}

Konjunkcija:
\begin{itemize}
	\item $p \land q = q \land p$
	\item $p \land (q \land r) = (p \land q) \land r$
	\item $p \land \top = p$
	\item $p \land p = p$
\end{itemize}

Disjunkcija:
\begin{itemize}
	\item $p \lor q = q \lor p$
	\item $p \lor (q \lor r) = (p \lor q) \lor r$
	\item $p \lor \bot = p$
	\item $p \lor p = p$
\end{itemize}

Distributivnost:
\begin{itemize}
	\item $(p \land q) \lor r = (p \lor r) \land (q \lor r)$
	\item $(p \lor q) \land r = (p \land r) \lor (q \land r)$
\end{itemize}

Absorpcija:
\begin{itemize}
	\item $(q \land p) \lor p = p$
	\item $(q \lor q) \land p = p$
\end{itemize}

Negacija:
\begin{itemize}
	\item $p \land \neg p = \bot$
	\item $p \lor \lnot p = \top$
\end{itemize}

\emph{Izrek:} (za izjavo $p$ v kateri nastopajo samo izjavni simboli $q_1 \ldots q_n$)
\begin{enumerate}
	\item "Ce ima izjava dokaz je tavtologija.
	\item "Ce je izjava tavtologija ima dokaz.
\end{enumerate}
Izrek ne velja za izjave, ki vsebujejo parametre iz mno"zic.

\subsection{Polni nabori}
Nabor operacij je \emph{poln}, "ce lahko z njim dobimo poljubno resni"cnostno tabelo.

Primeri:
\begin{itemize}
	\item $\top, \bot, \land, \lor, \lnot$ je poln
	\item $\top, \lnot, \land$ je poln
	\item $\bot, \uparrow \text{(nand)}$ je poln
\end{itemize}

\subsection{Ra"cunska pravila}
Pravila za $\top$:
\begin{itemize}
	\item $p \lor \top = \top$
	\item $p \land \top = p$
	\item $\lnot \top = \bot$
\end{itemize}

Pravila za $\bot$:
\begin{itemize}
	\item $p \lor \bot = p$
	\item $p \land \bot = \bot$
	\item $\lnot \bot = \top$
\end{itemize}

Pravila za negacijo:
\begin{itemize}
	\item $\lnot \lnot p = p$
	\item de Morganova pravila:
	\begin{itemize}
		\item $\lnot (p \land q) = \lnot p \lor \lnot q$
		\item $\lnot (p \lor q) = \lnot p \land \lnot q$
	\end{itemize}
\end{itemize}

Ostalo (\emph{kontrapozitivna oblika}):
\begin{itemize}
	\item $(p \Rightarrow q) = (\lnot q \Rightarrow \lnot p)$
	\item $(p \lor q) = (\lnot p \Rightarrow q)$
	\item $(p \Rightarrow q) = (\lnot p \lor q)$
\end{itemize}

Izjava ima lahko dve obliki:
\begin{itemize}
	\item \emph{konjunktivna} oblika: $(\lnot p \lor q) \land r \land (r \lor \lnot p)$
	\item \emph{disjunktivna} oblika: $(u \land \lnot v) \lor (u \land w \land \lnot u)$
\end{itemize}

\subsection{Pravila za kvantifikatorje}
\begin{itemize}
	\item $(\lnot \exists x \in A. p(x)) \iff (\forall x \in A . \lnot p(x))$
	\item $(\lnot \forall x \in A . p(x)) \iff (\exists x \in A . \lnot p(x))$
	\item $(\forall x \in \varnothing . p(x)) \iff \top$
	\item $(\exists x \in \varnothing . p(x)) \iff \bot$
	\item $(p \Rightarrow \forall x \in A . q(x)) \iff (\forall x \in A . p \Rightarrow q(x))$
	\item $(\forall u \in A \times B . p(u)) \iff (\forall x \in A \forall y \in B . p(x, y))$
	\item $(\exists u \in A \times B . p(u)) \iff (\exists x \in A \exists y \in B . p(x, y))$
	\item $(\forall u \in A + B . p(u)) \iff (\forall x \in A . p(\iota_1(x))) \land (\forall y \in B . p (\iota_2(y)))$
	\item $(\forall u \in A \cup B . p(u)) \iff (\forall a \in A. p(a)) \land (\forall b \in B . p(b))$
	\item $(\forall x \in \{a\} . p(x)) \iff p(a)$
	\item $(\exists x \in \{a\} . p(x)) \iff p(a)$
\end{itemize}