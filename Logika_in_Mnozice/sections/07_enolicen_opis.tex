\begin{enumerate}[1)]
	\item Okraj"sava, uvedemo nov simbol
	\begin{gather*}
		c := \cdots\\
		c \stackrel{\triangle}{=} \cdots\\
		c \stackrel{\text{def}}{=} \cdots\\
		c = \cdots\\
		f(x): = \cdots
	\end{gather*}
	
	\item Enoli"cen opis
	\begin{gather*}
		\exists! x \in A . p(x)\\
		\exists^1 x \in A .p(x)
	\end{gather*}
	``obstaja natanko en $x \in A$, da velja $p(x)$''
	
	To je okraj"sva za:
	\begin{equation*}
	(\exists x \in A .p(x)) \land (\forall y, z \in A . p(y) \land p(z) \Rightarrow y = z)
	\end{equation*}
	
	"Ce doka"zemo
	\begin{equation*}
	\exists! x \in A . p(x)
	\end{equation*}
	potem lahko uvedemo novo oznako $c$ in pravilo
	\begin{equation*}
	c \in a \text{ in } p(c)
	\end{equation*}
	
	Lahko pi"semo tudi:
	\begin{equation*}
	\iota x \in A . p(x)
	\end{equation*}
	kar pomeni ``tisti $x \in A$, za katerega velja $p(x)$'', podobno kot anonimna funkcija. Primer uporabe:
	\begin{equation*}
	(\iota y \in \RR . y^3 = 2)^6 + 7 = 11
	\end{equation*}
\end{enumerate}
