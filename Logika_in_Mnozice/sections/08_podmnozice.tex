\emph{Definicija:} Za mno"zici $A$ in $B$:
\begin{align*}
A \subseteq B &:= \forall x \in A . x \in B\\
\subseteq &:= (A, B) \mapsto \forall x \in A . x \in B
\end{align*}
Namesto $\subseteq(A, B)$ pi"semo $A \subseteq B$.

Konstrukcija podmno"zice:
\begin{itemize}
	\item mno"zica $A$
	\item izjava $p(x)$, kjer $x \in A$
\end{itemize}
Tvorimo mno"zico:
\begin{equation*}
\{x \in A | p(x)\}
\end{equation*}
Elementi te mno"zico so natanko tisti $a \in A$, za katere velja $p(a)$.

Ostali zapsi so:
\begin{gather*}
	\{x \in A: p(x)\}\\
	\{x \in A ; p(x)\}
\end{gather*}

Ra"cunski pravili:
\begin{enumerate}[1)]
	\item
	\begin{equation*}
	(\forall x \in \{y \in A | p(y)\} . q(x)) \iff (\forall z \in A. p(z) \Rightarrow q(z))
	\end{equation*}
	
	\item 
	\begin{equation*}
	(\exists x \in \{y \in A | p(y)\} . q(x)) \iff (\exists z \in A. p(z) \land q(z))
	\end{equation*}
\end{enumerate}
