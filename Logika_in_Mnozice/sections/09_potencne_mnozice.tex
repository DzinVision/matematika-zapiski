$\mathcal{P}(A)$ je poten"cna mno"zica $A$. Njeni elementi so natanko vse podmno"zice $A$.

Primeri:
\begin{gather*}
\mathcal{P}(\{1, 7\}) = \{\varnothing, \{1\}, \{7\}, \{1, 7\}\}\\
\mathcal{P}(\varnothing) = \{\varnothing\}
\end{gather*}

Spomnimo: $2 = \{\bot, \top\}$

Podmno"zice $A$ so preslikave $A \rightarrow 2$.

Izrek: $\mathcal{P}(A) \cong 2^A$
\begin{align*}
\mathcal{P}(A) &\rightarrow 2^A\\
\chi : S &\mapsto \left(x \mapsto \begin{cases}
\bot & x \notin S\\
\top & x \in S
\end{cases}\right)
\end{align*}
%
\begin{align*}
2^A &\rightarrow \mathcal{P}(A)\\
f &\mapsto \{x \in A | f(x)\}
\end{align*}

Nato te funkcije se preverimo, kot smo delali "ze mnogokrat na vajah.

\subsection{Boolova algebra na $\mathcal{P}(A)$}
Imamo operacije $\cup, \cap$, komplement
\begin{align*}
S \cap T &:= \{x \in A | x \in S \land x \in T\}\\
S \cup T &:= \{x \in A | x \in S \lor x \in T\}\\
\varnothing &:= \{x \in A | \bot\}\\
A &:= \{x \in A| \top\}\\
S^C &:= \{x \in A| \lnot (x \in S)\}
\end{align*}
