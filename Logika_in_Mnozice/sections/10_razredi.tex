Vzemimo mno"zico vseh mno"zic
\begin{equation*}
V = \{x | \text{$x$ je mno"zica}\}
\end{equation*}
Definirajmo podmno"zico:
\begin{equation*}
R  = \{x \in V | x \notin x\}
\end{equation*}
Dokazali bomo $R \notin R$ in $R \in R$:
\begin{enumerate}[1)]
	\item $R \notin R$
	
	Predpostavimo $R \in R$ in i"s"cemo protislovje. Po predpostavki vemo $R \in R$. To pomeni, da po definiciji $R$ velja $R \notin R$, s "cimer smo pri"sli do protilsovja, torej velja $R \notin R$.
	
	\item $R \in R$
	
	To bomo dokazali s protislovjem (pozor: prej"sen dokaz je bil dokaz negacije!). Predpostavimo $R \notin R$ in i"s"cemo protislovje. Po predpostavki vemo, da $R \notin R$, kar pomeni da po definiciji $R$ velja $R \in R$. Pri"sli smo do protislovja, kar pomeni da velja $R \in R$.
\end{enumerate}
Dokazali smo $\bot$, torej velja vse. Tudi tak"sne nesmiselnosti kot $0 = 1$.

Da se znebimo tega problema uvedemo razred, ki ga tvorimo\footnote{Tvorba je razli"cna od tvorbe mno"zic. Za mno"zice imamo to"cno dolo"cene na"cine tvorbe (kartezi"cni produkt, podmno"zica, presek, unija, \dots)}:
\begin{equation*}
\{x | p(x)\}
\end{equation*}
Velja:
\begin{equation*}
a \in \{x | p(x)\} \iff p(a)
\end{equation*}
Pri tem je $a$ bodisi osnovni matemati"cni objekt ("stevilo, urejeni par) ali mno"zica, ne sme pa biti razred. Druga"ce povedano: razredi niso elementi.

Razred $C$ je mno"zica, "ce lahko tvorimo mno"zico, ki ima iste elemente kot $C$
\begin{equation*}
a \in C \iff a \in S
\end{equation*}
kjer je $S$ mno"zica.

Vsaka mno"zica $S$ je razred:
\begin{equation*}
\{x | x \in S\}
\end{equation*}
Razred, ki ni mno"zica se imenuje \emph{pravi razred}.

Primeri pravih razredov:
\begin{itemize}
	\item Razred vseh mno"zic:
	\begin{equation*}
	V = \{x | \text{$x$ je mno"zica}\} = \{x | \top\}
	\end{equation*}
	oznaka za tak razred je Set.
	
	\item $R = \{x | x \notin x\}$
	
	\item  $\{A | \exists!x \in A:\top\}$ razred vseh enojcev
	
	\item $\{X | \text{$X$ je vektorski prostor}\}$
	
	$\{X | \text{$X$ je grupa}\}$
\end{itemize}