Imamo naslednje mno"zice:
\begin{align*}
A_0 &= \cdots \\
A_1 &= \cdots \\
A_2 &= \cdots
\end{align*}
Dru"zina mno"zic je preslikava:
\begin{equation*}
A : I \rightarrow \set
\end{equation*}
kjer $I$ je indeksna mno"zica in $i \in I$ so indeksi.

Namesto $A(i)$ pi"semo $A_i$.

\textsc{Primeri:}
\begin{enumerate}[1)]
	\item "Ce imamo mno"zice $A, B, C, D, E$, lahko tvorimo dru"zino:
	\begin{gather*}
		I = \{1, 2, 3, 4, 5\}\\
		Q : I \rightarrow \set \\
		Q_1 = A, Q_2 = B, Q_3 = C, Q_4 = D, Q_5 = E
	\end{gather*}
	
	\item Dru"zina vseh zaprtih intervalov:
	\begin{gather*}
		K = \{(a, b) \in \RR \times \RR | a \leq b\}\\
		I: K \rightarrow \set\\
		I(a, b) := [a, b] = \{x \in \RR | a \leq x \leq b\}
	\end{gather*}
	
	\item Nekateri elementi dru"zine so lahko enaki:
	\begin{gather*}
		I = \{1, 2, 3, 4, 5\} \\
		A : I \rightarrow \set
	\end{gather*}
	lahko velja $A_1 = A_3$.
	
	\item Konstanta dru"zina $A: I \rightarrow \set$.
	\begin{equation*}
	\forall i, j \in I: A_i = A_j
	\end{equation*}
	
	\item \emph{Prazna dru"zina} $\varnothing \rightarrow \set$
	
	\item \emph{Dru"zina praznih mno"zic}
	\begin{gather*}
		A : I \rightarrow \set \\
		\forall i \in I: A_i = \varnothing
	\end{gather*}
	
	\item \emph{Neprazna dru"zina}
	\begin{gather*}
		A: I \rightarrow \set \\
		I \neq \varnothing
	\end{gather*}
	
	\item \emph{Dru"zina nepraznih}
	\begin{gather*}
		A : I \rightarrow \set \\
		\forall i \in I: A_i  \neq \varnothing
	\end{gather*}
\end{enumerate}

\subsection{Konstrukcija z dru"zinami mno"zic}
Naj bo $A: I \rightarrow \set$ dru"zina.

\emph{Funkcija izbire} $f$ za dano dru"zino $A$ je prirejanje, ki vsakemu $i \in I$ priredi natanko en element $f(i) \in A_i$.

\textsc{Primer:} dru"zina vseh zaprtih intervalov
\begin{gather*}
I = \{(a, b) \in \RR \times \RR | a \leq b\} \\
K(a,b) = [a, b] \\
f(a, b) = \dfrac{a + b}{2}\\
g(a, b) = b
\end{gather*}
$f$ in $g$ sta primera funkcije izbire.

"Ce imamo $A: I \rightarrow \set$ in $A_j = \varnothing$ za neki $j \in I$, potem za $A$ ni nobene funkcije izbire.
\subsubsection{Kartezi"cni produkt}
\begin{equation*}
\prod_{i \in I} A_i
\end{equation*}
Elementi so funkcije izbire za $A$.

Za vsak $i \in I$ imamo $i$-to projekcijo:
\begin{gather*}
\pi_i : \prod_{j \in I} A_j \rightarrow A_i \\
f \mapsto f(i)
\end{gather*}

$B \times C$ je poseben primer:
\begin{equation*}
B \times C \cong \prod_{i \in I}A_i
\end{equation*}
kjer $I = \{1, 2\}$ in $A_1 = B, A_2 = C$.

Tudi $C^B$ je poseben primer
\begin{equation*}
C^B \cong \prod_{j \in J} D_j
\end{equation*}
kjer $J = B$ in $D_j = C$.

\subsubsection{Unija in presek}
\begin{gather*}
\bigcup_{i \in I} A_i = \{x; \exists i \in I : x \in A_i\}\\
\bigcap_{i \in I} A_i = \{x; \forall i \in I : x \in A_i\}
\end{gather*}

Presek prazne dru"zine:
\begin{equation*}
\bigcap_{i \in \varnothing} A_i \ \{x; \forall i \in \varnothing: x \in A_i\} = \{x; \top\} = V
\end{equation*}
je pravi razred.

Presek neprazne dru"zine je mno"zica, "ce imamo $j \in I$
\begin{equation*}
\bigcap_{i \in I} A_i = \{x; \forall i \in I: x \in A_i\} = \{x \in A_j; \forall i \in I: x \in A_i\}
\end{equation*}

\textsc{Aksiom o uniji: } Unija dru"zine mno"zic je mno"zica.

\textsc{Primer:}
\begin{gather*}
A : \NN \rightarrow \set\\
A_0 = \NN\\
A_1 = P(\NN)\\
A_2 = P(P(\NN))\\
A_{n+1} = P(A_n)
\end{gather*}

$\bigcup_{n \in \NN} A_n$ je unija po aksiomu.
