Naj bodo:
\begin{align*}
f: A \rightarrow B && \\ 
S \subseteq A && S \in \mathcal{P}(A) \\
T \subseteq B && B \in \mathcal{P}(B)
\end{align*}
\textsc{Definicije:}
\begin{itemize}
	\item \emph{Slika} je mno"zica:
	\begin{equation*}
	f_*(S) = \{y \in B | \exists x \in S: f(x) = y\}
	\end{equation*}
	
	\item \emph{Praslika} je mno"zica:
	\begin{equation*}
	f^*(T) = \{x \in A | f(x) \in T\}
	\end{equation*}
\end{itemize}
Poznamo tudi ostale zapise, ki pa so slab"si:
\begin{itemize}
	\item $f_*(S)$ se pi"se tudi kot $f(S)$ ali $f[S]$.
	\item $f^*(S)$ se pi"se tudi kot $f^{-1}(S)$ ali $f^{-1}[S]$.
\end{itemize}
\begin{align*}
&f: A \rightarrow B \\
&f_*: \mathcal{P}(A) \rightarrow \mathcal{P}(B) \\
&f^*: \mathcal{P}(B) \rightarrow \mathcal{P}(A)
\end{align*}
Pravimo, da je $f_*$ \emph{kovariantna} (ne obrne smeri $f$) in da je $f^*$ \emph{kontravariantna} (obrne smer $f$).

\textsc{Velja:}
\begin{align*}
f^*(\varnothing) = \varnothing && f_*(\varnothing) = \varnothing \\
f^*(B) = A && \underbrace{f_*(A)}_{Z_f} \subseteq B
\end{align*}

\subsection{Ra"cunska pravila}
\begin{align*}
f: A \rightarrow B && S: I \rightarrow \mathcal{P}(A)
\end{align*}
\begin{gather*}
	f^*\left(\bigcup_{i \in I}S_i\right) = \bigcup_{i \in I}f^*(S_i) \\
	f^*\left(\bigcap_{i \in I}S_i\right) = \bigcap_{i \in I}f^*(S_i) \\
	f^*(S_1 \cup S_2) = f^*(S_1) \cup f^*(S_2) \\
	f^*(S_1 \cap S_2) = f^*(S_1) \cap f^*(S_2) \\
	f_*\left(\bigcup_{i \in I}S_i\right) = \bigcup_{i \in I}f_*(S_i) \\
	f_*\left(\bigcap_{i \in I}S_i\right) \subseteq \bigcap_{i \in I}f_*(S_i) \\
	f^*(S^\complement) = (f^*(S))^\complement
\end{gather*}
\textsc{Definicije} injektivne, surjektivne, bijektivne, epi in mono

Naj bo $f: A \rightarrow B$ preslikava
\begin{itemize}
	\item $f$ je \emph{injektivna} "ce velja:
	\begin{gather*}
		\forall x, y \in A: f(x) = f(y) \Rightarrow x = y 
		\intertext{v"casi uporabimo tudi:} 
		\forall x, y \in A: x \neq y \Rightarrow f(x) \neq f(y)
	\end{gather*}
	
	\item $f$ je \emph{surjektivna}, "ce velja:
	\begin{equation*}
	\forall y \in B \exists x \in A: f(x) = y
	\end{equation*}
	lahko re"cemo tudi, da je zaloga vrednosti za $f$ celoten $B$, kar zapi"semo s pomo"cjo slike:
	\begin{equation*}
	f_*(A) = B
	\end{equation*}

	\item $f$ je \emph{bijektivna} kadar je surjektivna in injektivna. Simbolno to zapi"semo kot:
	\begin{equation*}
	\forall y \in B \exists! x \in A: f(x) = y
	\end{equation*}
	
	\item $f$ je \emph{monomorfizem} (pravimo, da je $f$ \emph{mono}).
	
	"Ce za preslikavi $g, h: C \to A$ velja:
	\begin{equation*}
	f \circ g = f \circ h \Rightarrow g = h
	\end{equation*}
	pravimo, da lahko $f$ \emph{kraj"samo} na levi.
	
	\textsc{Definicija:} $f: A \rightarrow B$ je \emph{mono}, kadar za vse preslikave $g, h: C \to A$ velja:
	\begin{equation*}
	f \circ g = f \circ h \Rightarrow g = h
	\end{equation*}
	
	\item $f$ je \emph{epimorfizem} (pravimo, da je $f$ \emph{epi}), kadar velja:
	\begin{equation*}
	\forall C \in \set \forall g, h: B \to C: g \circ f = h \circ f \Rightarrow g = h
	\end{equation*}
\end{itemize}
Doka"zimo nekatere izjeve, ki so na voljo na \url{https://github.com/andrejbauer/ucbenik-logika-in-mnozice/blob/master/predavanja-2017/07-funkcije.md}.
\begin{enumerate}
	\item[1)] \dashuline{$f$ mono in $g$ mono $\Rightarrow g \circ f$ mono.}
	
	Naj bo $f: A \to B$ in $g: B \to C$ in $k, l: D \to A$. Dokazujemo:
	\begin{equation*}
	(g \circ f) \circ k = (g \circ f) \circ l \Rightarrow k = l
	\end{equation*}
	Predpostavimo
	\begin{equation*}
	(g \circ f) \circ k = (g \circ f) \circ l
	\end{equation*}
	po definiciji je kompozitum asociativen, torej lahko zapi"semo:
	\begin{equation*}
	g \circ (f \circ k) =g \circ (f \circ l)
	\end{equation*}
	Ker je $g$ mono, lahko kraj"samo $g$:
	\begin{equation*}
	f \circ k = f \circ l
	\end{equation*}
	Ker je $f$ mono, lahko kraj"samo $f$:
	\begin{equation*}
	k = l
	\end{equation*}
	\hfill $\square$
	
	\item[3)] \dashuline{$g \circ f$ mono $\Rightarrow f$ mono}
	
	Dokazujemo:
	\begin{equation*}
		f \circ k = f \circ l \Rightarrow k = l
	\end{equation*}
	Predpostavimo:
	\begin{equation*}
	f \circ k = f \circ l
	\end{equation*}
	Na vsaki strani lahko ena"cbo ``raz"sirimo'' z $g$:
	\begin{equation*}
	g \circ f \circ k = g \circ f \circ l
	\end{equation*}
	Ker je kompozitum asociativen velja:
	\begin{equation*}
	(g \circ f) \circ k = (g \circ f) \circ l
	\end{equation*}
	Lahko kraj"samo $g \circ f$ po predpostavki:
	\begin{equation*}
	k = l
	\end{equation*}
	\hfill $\square$
\end{enumerate}

Naj bo $f: A \to B$
\begin{enumerate}
	\item[1)] \dashuline{$f$ je mono $\iff f$ je injektivna}
	\begin{itemize}
		\item[($\Rightarrow$)] Prepostavimo: $f$ je mono in dokazujemo:
		\begin{equation*}
		\forall x, y \in A: f(x) = f(y) \Rightarrow x = y
		\end{equation*}
		Naj bosta $x, y \in A$. Predpostavimo $f(x) = f(y)$ in dokazujemo $x = y$.
		
		Definirajmo:
		\begin{align*}
		k: 1 \to A && l: 1 \to A \\
		* \mapsto x && * \mapsto y
		\end{align*}
		Spomnimo se: $1$ je enojec, $1 = \{*\}$
		
		Trdimo: $f \circ k = f \circ l$ ker:
		\begin{align*}
		(f \circ k)(*) &= f(k(*)) = f(x) \\
		(f \circ l)(*) &= f(l(*)) = f(y)
		\end{align*}
		Po predpostavki $f(x) = f(y)$ zgornja trditev velja.
		
		Ker je $f \circ k = f \circ l$ sledi, $k = l$, ker je $f$ mono.
		
		Funkcij $k$ in $l$ slikata iz enojca, torej lahko zapi"semo:
		\begin{equation*}
		k(*) = l(*)
		\end{equation*}
		Torej po definiciji $k$ in $l$ velja:
		\begin{equation*}
		x = y
		\end{equation*}
		
		\item[($\Leftarrow$)] Predpostavimo, da je $f$ injektivna in dokazujemo, da je mono.
		
		Naj bosta $g, h: C \to A$. Predpostavimo $f\circ g = f\circ h$. Dokazujemo:
		\begin{equation*}
		g =h \iff \forall c \in C: g(c) = h(c)
		\end{equation*}
		Naj bo $c \in C$. Dokazujemo $g(c) = h(c)$. Vemo $f \circ g = f \circ h$. Sledi:
		\begin{equation*}
			\Rightarrow (f\circ g)(c) = (f\circ h)(c) \iff f(g(c)) = f(h(c))
		\end{equation*}
		Ker je $f$ injektivna sledi:
		\begin{equation*}
		g(c) = h(c)
		\end{equation*}
	\end{itemize}
\end{enumerate}

\textsc{Trditvi:}
\begin{itemize}
	\item $f$ je \emph{epi} $\iff f$ surjektivna
	\item $f$ je izomorfizem $\iff f$ bijekcija
\end{itemize}
Doka"zimo drugo trditev:
\begin{itemize}
	\item[($\Rightarrow$)] Dokazujemo $f$ izo $\Rightarrow f$ bijekcija. Predpostavimo, da je $f$ izomorfizem in dokazujemo, da je bijekcija. Po definiciji bijekcije to pomeni, da je injektivna in surjektivna. Po prej"snjih trditvah velja, da mora biti \dashuline{$f$ mono in epi}.
	\begin{enumerate}
		\item $f$ je mono
		
		Vemo $id_A = f^{-1} \circ f$ in $id_A$ je mono. Torej je $f^{-1} \circ f$ mono. Spomnimo se trditve od zadnji"c:
		\begin{equation*}
		g \circ h \text{ mono } \Rightarrow h \text{ mono}
		\end{equation*}
		Torej je $f$ mono.
		
		\item $f$ je epi: podoben dokaz kot za 1. to"cko.
		
		Vemo $id_B = f \circ f^{-1}$ in $id_B$ je epi. Torej je $f \circ f^{-1}$ epi. Ponovno se spomnimo trditve od zadnji"c:
		\begin{equation*}
		g \circ h \text{ epi } \Rightarrow g \text{ epi}
		\end{equation*}
		Sledi $f$ je epi.
	\end{enumerate}

	\item[($\Leftarrow$)] $f$ je bijekcija $\Rightarrow f$ je izomofizem.
	
	Predpostavimo, da je $f$ bijekcija in dokazujemo, da je izomorfizem. Po definiciji izomorfizma:
	\begin{equation*}
	\exists g : B \to A: f\circ g = id_B \land g \circ f = id_A
	\end{equation*}
	Definirajmo $g: B \to A$ s predpisom:
	\begin{equation*}
	g(y) = \text{ ``tisti $x \in A$ za katerega je $f(x) = y$''}
	\end{equation*}
	Utemeliti moramo:
	\begin{equation*}
	\forall y \in B \exists!x \in A: f(x) = y
	\end{equation*}
	Z drugimi besedami:
	\begin{enumerate}
		\item $g$ je celovit predpis:
		\begin{equation*}
		\forall y \in B \exists x \in A: f(x) = y
		\end{equation*}
		Opazimo, da je to definicija surjektivnosti in velja, ker je $f$ bijektivna.
		
		\item $g$ je enoli"cen predpis:
		\begin{equation*}
		\forall y \in B \forall x_{1,2} \in A: f(x_1) = y \land f(x_2) = y \Rightarrow x_1 = x_2
		\end{equation*}
		Vemo injektivnost $f$: 
		\begin{equation*}
		\forall z_1, z_2 \in A: f(z_1) = f(z_2) \Rightarrow z_1 = z_2
		\end{equation*}
		"Ce velja $f(x_1) = y$ in $f(x_2) = y$, potem velja $f(x_1) = f(x_2)$, torej tudi $x_1 = x_2$ ker $f$ injektivna.
	\end{enumerate}
	Sedaj vemo, da je $g$ dobro definirana funkcija. Preverimo:
	\begin{enumerate}
		\item $f \circ g = id_B$
		
		Naj bo $y \in B$ Preverimo $f(g(y)) = y$. Velja po definiciji $g$.
		
		\item $g \circ f = id_A$
		
		Naj bo $x \in A$ Preverimo $g(f(x)) = x$. Po definiciji $g$ je to tisti element, ki ga $f$ slika v $f(x)$. Torej velja.
	\end{enumerate}
\hfill $\square$
\end{itemize}
