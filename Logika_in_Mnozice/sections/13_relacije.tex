\textsc{Definicija:} Relacija na mno"zicah $A_1, A_2, \ldots, A_n$ je podmno"zica $A_1 \times A_2 \times \ldots \times A_n$.

\textsc{Primeri:}
\begin{itemize}
	\item ``to"cka $A$ je med to"ckama $B$ in $C$''. (troji"ska relacija)
	\item $R \subseteq A_1 \times A_2$ dvoji"ska relacija na $A_1, A_2$
	\item $R \subseteq A \times A$ relacija na $A$.
\end{itemize}
\textsc{Primeri:}
\begin{itemize}
	\item  $\leq$ je relacija na $\RR$ in lahko zapi"semo:
	\begin{equation*}
	\leq \subseteq \RR \times \RR
	\end{equation*}
	
	\item $R \subseteq A \times B, a \in A, b \in B$
	$(a, b) \in \RR$ preberemo kot: ``$a$ in $b$ sta v relaciji $R$''. Zapi"semo tudi
	\begin{equation*}
	a R b
	\end{equation*}
\end{itemize}
\textsc{Primer:} $a \leq b$ lahko zapi"semo kot $(a, b) \in \leq$

\subsection{Osnovne lastnosti}
Naj bo $R \subseteq A \times A$.
\begin{itemize}
	\item \textbf{refleksivnost} $\forall x \in A: x R x$ \hfill $=, \leq$
	\item \textbf{irefleksivnost} $\forall x \in A: \lnot (x R x)$ \hfill $<, \perp$
	\item \textbf{simetri"cnost} $\forall x, y \in A: x R y \Rightarrow y R x$ \hfill $\perp, \parallel, =$
	\item \textbf{asimetri"cnost} $\forall x, y \in A: x R y \Rightarrow \lnot(y R x)$ \hfill $<$
	\item \textbf{antisimetri"cnost} $\forall x, y \in A: x R y \land y R a \Rightarrow x = y$ \hfill $\leq$
	\item \textbf{tranzitivnost} $\forall x, y, z \in A: x R y \land y R z \Rightarrow x R z$ \hfill $<, \leq, \parallel$
	\item \textbf{sovisnost} $\forall x, y \in A: x \neq y \Rightarrow x R y \lor y R x$ \hfill $<, \leq$
	\item \textbf{stroga sovisnost} $\forall x, y \in A: x R y \lor y R x$ \hfill $\leq$
\end{itemize}
\textsc{Definicije:}
\begin{itemize}
	\item \emph{Prazna relacija} na $A$ je $\varnothing$.
	\item \emph{Polna relacija} na $A$ je $A \times A$.
	\item \emph{Enakost} na $A$ je relacija $\{(x, y) \in A \times A | x = y\} \subseteq A \times A$.
\end{itemize}
Relacije lahko predstavimo kot grafe (tiste iz teorije grafov, ne kot grafe funkcij).

\subsection{Operacije na relacijah}
\subsubsection*{Transponirana relacija}
Naj bo $R \subseteq A \times B$.
\begin{equation*}
R^\intercal \subseteq B \times A
\end{equation*}
Definiramo kot
\begin{equation*}
R^\intercal := \{(b, a) \in B \times A| (a,b) \in R\}
\end{equation*}
\textsc{Primera:} $\leq^\intercal = \geq$ in $\subseteq^\intercal = \supseteq$

Velja:
\begin{equation*}
(R^\intercal)^\intercal = R
\end{equation*}
Relacijo in njeno transpozicijo lahko predstavimo kot tabelo:
\begin{table}[!htbp]
	\begin{minipage}{.5\linewidth}
		\centering
		\caption{Relacija $R$}
		\begin{tabular}{c|ccc}
			$R$ & 1 & 2 & 3 \\ \hline
			$a$ & $\bot$ & $\top$ & $\top$ \\
			$b$ & $\bot$ & $\bot$ & $\top$
		\end{tabular}
	\end{minipage}%
	\begin{minipage}{.5\linewidth}
		\centering
		\caption{Transponirana relacija $R$}
		\begin{tabular}{c|cc}
			$R^\intercal$ & $a$ & $b$ \\ \hline
			1 & $\bot$ & $\bot$ \\
			2 & $\top$ & $\bot$ \\ 
			3 & $\top$ & $\top$ 
		\end{tabular}
	\end{minipage}
\end{table}

\subsubsection*{Kompozicija relacij}
Naj bosta $R \subseteq A \times B$ in $S \subseteq B \times C$ relaciji.

Kompozicijo $S \circ R \subseteq A \times C$ definiramo kot\footnote{Vrstni red je nekoliko zmeden in je potebno biti nanj pozoren. Jaz si zapomnim na slede"c na"cin: gremo iz $A$ v $C$, torej gremo najprej "cez relacijo $R$, in nato "cez relacijo $S$. Tako kot kompozitum funkcij, pa se ta zapi"se iz desne proti levi. Torej $S \circ R$ preberemo: ``gremo "cez $R$ in nato "cez $S$.'' "Cedalje bolj verjamem, da si je kompozitum izmislil fizik, ker gre vse v rikverc in je zmedeno.}:
\begin{equation*}
S \circ R = \{(a, c) \in A \times C | \exists b \in B: a R b \land b S c\}
\end{equation*}
\textsc{Trditev:} Kompozicija relacij je asociativna:
\begin{equation*}
(S \circ R) \circ T = S \circ (R \circ T)
\end{equation*}
\textsc{Definiramo:}
\begin{equation*}
\underbrace{R \circ R \circ R \circ \ldots \circ R}_n =: R^n
\end{equation*}

\subsection{Graf preslikave}
Naj bo $f: A \to B$. \emph{Graf} je $\Gamma_f \subseteq A \times B$ definiran z:
\begin{equation*}
\Gamma_f := \{(x, y)| f(x) = y\}
\end{equation*}
$\Gamma_f$ ima lastnost:
\begin{enumerate}
	\item Celovita relacija
	\item Enoli"cna relacija
\end{enumerate}
\textsc{Definicija:} $R \subseteq A \times B$ je:
\begin{enumerate}
	\item \emph{celovita}, "ce velja:
	\begin{equation*}
	\forall x \in A \exists y \in B: x R y
	\end{equation*}
	
	\item \emph{enoli"cna}, "ce velja:
	\begin{equation*}
	\forall x \in A \forall y, z \in B: x R y \land x R z \Rightarrow y = z
	\end{equation*}
\end{enumerate}
$R$ je \emph{funkcijska relacija}, "ce je celovita in enoli"cna.

\textsc{Trditev:}
\begin{enumerate}
	\item Za vsako $f: A \to B$ je $\Gamma_f$ funkcijska relacija.
	\item Vsaka funkcijska relacija je graf neke funkcije.
\end{enumerate}
\textsc{Dokaz:}
\begin{enumerate}
	\item Opazimo, da je $R$ funkcijska $\iff \forall x \in A \exists! y \in B: x R y$. Ali je $\Gamma_f$ funkcijska?
	\begin{equation*}
	\forall x \in A \exists! y \in B: (x, y) \in \Gamma_f \iff \forall x \in A \exists! y \in B: f(x) = y
	\end{equation*}
	Veja, ker je $f$ preslikava.
	
	\item Denimo, da je $R \subseteq A \times B$ funkcijska. Dokazujemo:
	\begin{equation*}
	\exists f : A \to B: R = \Gamma_f
	\end{equation*}
	Vzemimo $f: A \to B$ s predpisom:
	\begin{equation*}
	f(x) = \text{ tisti $y \in B$, da velja $x R y$} = \iota y \in B: x R y
	\end{equation*}
	Preverimo $R = \Gamma_f$
\end{enumerate}

\textbf{Poanta}\footnote{za ljubitelje sloven"s"cine: izraz je uporabil profesor, jaz pa se ne morem spomniti bolj"sega}
\begin{equation*}
B^A \cong \{R \subseteq A \times B | R \text{ funkcijska}\}
\end{equation*}
Torej lahko funkcije definiramo kot funkcijske relacije.

\subsection{Ekvivalen"cne relacije in kvocientne mno"zice}
\textsc{Definicija}: $R \subseteq A \times A$ je \emph{ekvivalen"cna}, ko je refleksivna, simetri"cna in tranzitivna. Uporabljamo simbole $=\ \equiv\ \equiv\ \sim\ \approx\ \cong$.

\textsc{Primeri:} enakost $=$, polna relacija na $A$.

Naj bo $f: A \to B$ in definiramo $R \subseteq A \times A$ s 
\begin{equation*}
x R y \iff f(x) = f(y)
\end{equation*}
Tedaj je $R$ ekvivalen"cna. Pravimo, da je $R$ \emph{inducirana} s $f$.

\subsubsection{Ekvivalen"cni razredi}
Naj bo $R \subseteq A \times A$ ekvivalen"cna. Za $x \in A$ definiramo ekvivalen"cni razred $x$
\begin{equation*}
[x]_R := \{y \in A| x R y\}
\end{equation*}
Za $x, y$ velja:
\begin{align*}
x R y &\iff [x]_R = [y]_R \\
&\iff x \in [y]_R \\
&\iff y \in [x]_R
\end{align*}
\textsc{Primer:} Relacija $\sim$ na $\ZZ$.
\begin{align*}
m \sim n &\iff m | n \land n | m \\
[12]_\sim &= \{12, -12\} \\
[-2]_\sim &= \{2, -2\} \\
[0]_\sim &= \{0\}
\end{align*}
"Ce je $R \subseteq A \times A$ ekvivalen"cna, velja:
\begin{itemize}
	\item ekvivalen"cni razredi so neprazni: $x \in [x]_R$ ("Ce $A = \varnothing \Rightarrow$ ni ekvivalen"cnih razredov)
	\item $[x]_R \cap [y]_R \neq \varnothing \Rightarrow [x]_R = [y]_R$
\end{itemize}
Pravimo, da ekvivalen"cni razredi tvorijo \emph{particijo} ali \emph{razdelitev} $A$.

Ekvivalen"cno relacijo lahko podamo z ekvivalen"cnimi razredi tako, da podamo dru"zino $\{E_i\}, i \in I$, da velja:
\begin{itemize}
	\item $E_i \neq \varnothing \forall i \in I$ neprazni
	\item paroma diskunktni
	\item $\bigcup E_i = A$ tvorijo pokritje $A$
\end{itemize}
Pripadajo"ca $R \subseteq A \times A$ je:
\begin{equation*}
x R y \iff \exists i \in I: x \in E_i \land y \in E_i
\end{equation*}
%
\subsubsection{Univerzalne lastnosti kvocientnih mno"zic}
\textsc{Definicija:} Naj bo $R \subseteq A \times A$. Kvocientna mno"zica je:
\begin{align*}
A/_R &:= \{[x]_R : x \in A\} \\
&:= \{S: \exists x \in A: S = [x]_R\} \\
&:= \{S \in \mathcal{P}(A):\exists x \in A: S = [x]_R\} \\
&:= \{S \in \mathcal{P}(A):\exists x \in A \forall y \in A: y \in S \iff x R y \}
\end{align*}
\subsubsection*{Kvocientna preslikava}
\begin{align*}
q_R : A &\to A/_R \\
x &\mapsto  [x]_R
\end{align*}
\dashuline{$q_R$ je surjektivna}: $\forall \xi \in A/_R \exists x \in A: q_R(x) = \xi$

Naj bo $\xi \in A/_R$. Tedaj obstaja $y \in A$, da je $\xi = [y]_R$. Vzamemo $x := y$. Preverimo $q_R(x) = \xi$.
\begin{equation*}
q_R(x) = q_R(y) = [y]_R = \xi
\end{equation*}
\textsc{Izrek:} Naj bo $R \subseteq A \times A$ ekvivalen"cna relacija in $f: A \to B$ preslikava, ki je \emph{skladna} z $R$, kar pomeni:
\begin{equation*}
x R y \Rightarrow f(x) = f(y)
\end{equation*}
Pravimo tudi, da je $f$ \emph{kongluenca} za $R$.

Tedaj obstaja natanko ena preslikava $\overline{f}: A/_R \to B$, da velja
\begin{equation*}
f = \overline{f} \circ q_R
\end{equation*}
\begin{figure}[!htbp]
	\centering
	\begin{tikzpicture}
		\node (A) at (-2, 0) {$A/_R$};
		\node (B) at (2, 0) {$B$};
		\node (C) at (-2,3) {$A$};
		
		\draw[->] (A) -- node[midway, below] {$\overline{f}$} ++ (B);
		\draw[->] (C) -- node[midway, above] {$f$} ++ (B);
		\draw[->] (C) -- node[midway, left] {$g_R$} ++ (A);
	\end{tikzpicture}
\end{figure}
\textsc{Dokaz:} Pokazati moramo, enoli"cnost in obstoj $\overline{f}$.
\begin{itemize}
	\item[\textbf{Enoli"cnost:}] denimo, da imamo $\overline{f_1}, \overline{f_2}: A/_R \to B$ in
	\begin{equation*}
	f = \overline{f_1} \circ g_R \quad \text{in} \quad f = \overline{f_2} \circ g_R
	\end{equation*}
	Dokazujemo $\overline{f_1} = \overline{f_2}$. Vemo 
	\begin{equation*}
	\overline{f_1} \circ g_R = f = \overline{f_2} \circ g_R
	\end{equation*}
	Ker je $g_R$ surjektivna, je epi
	\begin{equation*}
	\overline{f_1} = \overline{f_2}
	\end{equation*}
	
	\item[\textbf{Obstoj:}] Vzemimo $\overline{f}: A/_R \to B$ definirano z:
	\begin{equation*}
	\overline{f}(S) := \iota b \in B \exists x \in A: b = f(x) \land [x]_R = b
	\end{equation*}
	Preverimo, da je $\overline{f}$ dobro definirana:
	\begin{enumerate}
		\item \textbf{Celovitost:} naj bo $S \in A/_R$. Ker je $S \in A/_R$, obstaja $x \in A$, da je $[x]_R = S$. Za $b$ vzemimo $b := f(x)$. Tedaj velja $b = f(x)$.
		
		\item \textbf{Enoli"cnost:} "Ce imamo:
		\begin{align*}
		b_1 &\in B \exists x_1 \in A: b_1 = f(x_1) \land [x_1] = S \\
		b_2 &\in B \exists x_2 \in A: b_2 = f(x_2) \land [x_2] = S
		\end{align*}
		Dokazujemo $b_1 = b_2$. Ker $[x_1] = S = [x_2]$, velja $x_1 R x_2$. Ker je $f$ skladna z $R$, velja $f(x_1) = f(x_2)$, torej:
		\begin{equation*}
		b_1 = f(x_1) = f(x_2) = b_2
		\end{equation*}
	\end{enumerate}
	\hfill $\square$
\end{itemize}
%
\subsection{Delne ureditve}
\textsc{Definicija:} Relacija $R \subseteq A \times A$ je \emph{"sibka ureditev}, "ce je refleksivna in tranzitivna. $R$ je \emph{delna ureditev}, "ce je refleksivna, tranzitivna in antisimetri"cna. Za delno ureditev uporabljamo simbole $\leq\ \sqsubseteq\ \preccurlyeq\ \subseteq$.

\textsc{Primeri:}
\begin{itemize}
	\item obi"cajna relacija  ,,manj"si ali enak'' na $\RR$
	\item tudi ,,ve"cji ali enak''
	\item ,,deli'' na $\NN$
	\item $=$ na $A$
\end{itemize}
\textsc{Protiprimeri:}
\begin{itemize}
	\item $<$ na $\RR$
	\item ,,deli'' na $\ZZ$ ($2 | -2 \land -2 | 2$ ampak $2 \neq -2$)
\end{itemize}
%
\textsc{Definicija:} $R \subseteq A \times A$ delna ureditev, je \emph{linearna}, "ce velja
\begin{equation*}
\forall x,y \in A: x R y \lor y R x
\end{equation*}
\textsc{Primeri:}
\begin{itemize}
	\item $\leq$ na $\QQ$ linearna
	\item $=$ na $\QQ$ ni linearna
	\item $\subset$ na $\mathcal{P}(\NN)$ ni linearna
\end{itemize}
Nari"semo lahko \emph{Hassejev diagram} (glej zvezek kako izgleda).

\textsc{Definicija:} Naj bo $(A, \leq)$ delna ureditev in naj bo $S \subseteq A$.
\begin{itemize}
	\item $x \in A$ je \emph{spodnja meja} za $S$, "ce velja $\forall y \in S: x \leq y$
	\item $x \in A$ je \emph{zgornja meja} za $S$, "ce velja $\forall y \in S: x \geq y$
	\item $x \in A$ je \emph{natan"cna zgornja meja} za $S$, "ce velja
	\begin{enumerate}
		\item $x$ je zgornja meja
		\item $\forall y \in A: y \text{ zgornja meja za $S$} \Rightarrow x \leq y$
	\end{enumerate}
	Pravimo, da je $x$ \emph{supremum} $S$.
	\item \emph{infimum} ali \emph{natan"cna spodnja meja} podobno.
\end{itemize}