\subsection{Peanovi aksiomi:}
\begin{enumerate}
	\item $\forall n \in \NN: n^+ \neq 0$
	\item $\forall n, m \in \NN: n^+ = m^+ \Rightarrow n = m$
	\item $\forall n \in \NN: n + 0 = n$
	\item $\forall n, m \in \NN: n + m^+ = (n + m)^+$
	\item $\forall n \in \NN: n \cdot 0 = 0$
	\item $\forall n, m \in \NN: n \cdot m^+ = n + n \cdot m$
	\item \emph{Princip indukcije:} Za vsako izjavo $\varphi(n)$, kjer $n \in \NN$ velja:
	\begin{gather*}
		\varphi(0) \land (\forall k \in \NN: (\varphi(k) \Rightarrow \varphi(k^+)) \Rightarrow \forall n \in \NN: \varphi(n) \\
		\forall S \subseteq \NN: 0 \in S \land (\forall k \in \NN: k \in S \Rightarrow k^+ \in S) \Rightarrow S = \NN
	\end{gather*}
\end{enumerate}
\textsc{Uporaba indukcije}: Za vsak $n \in \NN$ doka"zi $\varphi(n)$.

Dokaz z indukcijo:
\begin{itemize}
	\item baza (osnova) indukcije: preverimo $\varphi(0)$
	\item  indukcijski korak: predpostavimo $\varphi(k)$ in dokazujemo $\varphi (k^+)$
\end{itemize}
%
\textsc{Izrek:} $\forall n \in \NN: 0+n = n$

\textsc{Dokaz:} z indukcijo
\begin{itemize}
	\item baza: $0 + 0 = 0 $ zaradi (3)
	\item korak: predpostavimo $0 + n = n$ (IH)
	
	Dokazujemo $0 + n^+ = n^+$
	\begin{equation*}
	0 + n^+ = (0 + n)^+ = n^+
	\end{equation*}
\end{itemize}
%
\subsection{Indukcija na dvoji"skih drevesih}
Imamo prazno drevo in sestavljeno drevo.

\textbf{Aksiomi za drevesa:} ($\mathbb{D}$, Empty, Tree)
\begin{itemize}
	\item  Empty $\in \mathbb{D}$
	\item Tree(Empty, Empty)
	\item Tree(Empty, Tree(Empty, Tree(Empty, Empty)))
\end{itemize}

\subsection{Razli"cica indukcije za $\NN$}
\begin{equation*}
\forall S \subseteq \NN: (\forall m \in \NN: (\forall k \in \NN: k < m \Rightarrow k \in S) \Rightarrow m \in S) \Rightarrow S = \NN
\end{equation*}
Z besedami: Denimo, da ima $S$ lastnost:\\
"Ce so vsi predhodniki $m$ v $S$ je tudi $M \in S$.

Potem je $S = \NN$.

Iz tega sledi, da je $0 \in S$ na prazno izpolnjen.

\textsc{Definicija:} \emph{Stroga} delna ureditev je $R \subseteq A \times A$, ki je 
\begin{enumerate}
	\item irefleksivna
	\item tranzitivna
\end{enumerate}
Stroga delna ureditev je \emph{linearna}, "ce je
\begin{enumerate}[3.]
	\item sovisna $\forall x, y \in A: x \neq y \Rightarrow x R y \lor y R x$
\end{enumerate}
Za stroge ureditve uporabljamo: $<, \sqsubset, \subset, \prec$

\textsc{Definicija:} Relacija $R \subseteq A \times A$ je \emph{dobro osnovana}, "ce
\begin{equation*}
\forall S \subseteq A: (\forall y \in A: (\forall x \in A: x R y \Rightarrow x \in S) \Rightarrow y \in S) \Rightarrow S = A
\end{equation*}
$R$ je \emph{dobra ureditev}, "ce je strogo linearna in je dobro osnovana.

\textsc{Izrek:} Naj bo $\sqsubset$ stroga linearna ureditev na $A$. Ekvivalentne so izjave:
\begin{enumerate}
	\item $\sqsubset$ je dobra ureditev\item vsaka neprazna $S \subseteq A$ ime prvi element
	\begin{equation*}
	\exists x \in S \forall y \in S: x \neq y \Rightarrow x \sqsubset y
	\end{equation*}
	\item $A$ nima padajo"ce verige:
	
	Padajo"ca veriga je zaporedje $a: \NN \to A$, da velja $a_{n+1} \sqsubset a_n$ za vse $n \in \NN$. To je:
	\begin{equation*}
	\cdots \sqsubset a_3 \sqsubset a_2 \sqsubset a_1 \sqsubset a_0
	\end{equation*}
\end{enumerate}
\textsc{Primeri}
\begin{enumerate}
	\item Relacija $<$ na $\RR$: (2) ne velja za $(0, 1) \Rightarrow <$ na $\RR$ ni dobra ureditev
	
	\item $A = \NN \cup \{\omega\}$ uredimo:
	\begin{equation*}
	0 < 1 < 2 < \cdots < \omega
	\end{equation*}
	\begin{equation*}
	x < y \iff (y = \omega \land x \in \NN) \lor (y, x \in \NN \land x < y \text{ obi"cajno za $\NN$})
	\end{equation*}
	Velja (3): ni neskon"cnih padajo"cih verig $\Rightarrow$ je dobra ureditev
	
	\item 
	\begin{equation*}
	0 < 1 \cdots < \omega < \omega + 1 < cdots < \omega + \omega < \omega + \omega + 1 \cdots < \omega + \omega + \omega
	\end{equation*}
\end{enumerate}
%
Denimo, da je $<$ stroga urejenost na $A$. Pravimo, da je $S \subseteq A$ \emph{progresivna} (glede na $<$), ko velja
\begin{equation*}
\forall x \in A: (\forall y \in A: y < x \Rightarrow y \in S) \Rightarrow x \in S
\end{equation*}
Relacija $<$ je \emph{dobro osnovona}, "ce velja
\begin{equation*}
\forall S \subseteq A: \text{$S$ progresivna} \Rightarrow S = A
\end{equation*}
Relacija $<$ je \emph{dobro urejena}, "ce je linearna in dobro osnovana.

\textsc{Lema:} Naj bo $<$ stroga urejenost na $A, A\neq \varnothing$. "Ce $A$ nima $\leq$-minimalnega elementa, potem v $A$ obstaja padajo"ca veriga. Ponovimo: $A$ ima $\leq$-minimalni element:
\begin{equation*}
\exists x \in A \forall y \in A: y \leq x \Rightarrow y = x
\end{equation*}
$A$ nima minimalnega elementa:
\begin{equation*}
\forall x \in a \exists y \in A: y \leq x \land y \neq x \iff \forall x \in A \exists y \in A: y < x
\end{equation*}
\textsc{Dokaz:} Dokazujemo, da v $A$ obstaja $a: \NN \to A$, da velja $a(n+1) < a(n)$ za vse $n \in \NN$. Ker $A \neq \varnothing$, obstaja $a(0) \in A$.

Denimo, da smo "ze skonstruirali $a(n) < a(n-1) < \cdots < a(2) < a(1) < a(0)$. Ker $a(n)$ ni minimalni, obstaja $y \in A$, da je $y < a(n)$. Za $a(n+1)$ \emph{izberemo} enega od $y < a(n)$.

Postopek nadaljujemo in dobimo $a(n+2), a(n+3), \ldots$

\hfill $\square$

\textsc{Izrek:} Naj bo $\sqsubset$ stroga urejenost na $A$. Ekvivalentne so izjave:
\begin{enumerate}
	\item $\sqsubset$ je dobro osnovana
	\item Vsaka neprazna $S \subseteq A$ ima $\sqsubseteq$-minimalni element
	\item $A$ nima padajo"ce $\sqsubset$-verige
\end{enumerate}
\textsc{Dokaz}
\begin{itemize}
	\item[$1 \Rightarrow 2$] Denimo, da je $\sqsubset$ dobro osnovana.
	
	Nj bo $S \subseteq A$ neprazna in naj bo 
	\begin{equation*}
	M := \{x \in S: x \text{ je minimalni v $S$}\}
	\end{equation*}
	Dokazujemo $M \neq \varnothing$. V ta namen definiramo:
	\begin{equation*}
	T := \{x \in A: (\exists y \in S: y \sqsubset x) \Rightarrow \exists m \in M: m \sqsubset x\}
	\end{equation*}
	Trdimo, da je $T$ progresivna.
	
	Naj bo $v \in A$ in denimo, da velja
	\begin{equation*}
	\forall u \in A: u \sqsubset v \Rightarrow u \in T \tag{\hexstar}
	\end{equation*}
	Dokazujemo $v \in T$.
	
	Predpostavimo, da obstaja $y \in S$, da je $y \sqsubset v$. Dokazujemo
	\begin{equation*}
	\exists m \in M: m \sqsubset v
	\end{equation*}
	Iz (\hexstar) sledi, $y \in T$. Obravnavamo dva primera:
	\begin{enumerate}[(a)]
		\item "Ce $\exists z \in S: z \sqsubset y$:
		
		Ker $y \in T$, obstaja $m' \in M$, da je $m' \sqsubset y$.
		
		Imamo $m' \sqsubset y \sqsubset v$
		
		Torej $\exists m \in M: m \sqsubset v$, namre"c $m := m'$
		
		\item "Ce $\lnot \exists z \in S: z \sqsubset y$
		
		Tedaj je $y \in M$.
		
		Torej $\exists m \in M: m \sqsubset v$, namre"c $m := y$.
	\end{enumerate}
	Ker je $T$ progresivna in velja 1, sledi $T = A$.
	
	Ker je $S$ neprazna, obstaja $t \in S$. Dva primera:
	\begin{enumerate}[(a)]
		\item "Ce $\exists z \in S: z \sqsubset t$
		
		Velja $t \in T$. Po definicija $T$, torej $\exists m \in M: m \sqsubset t$.
		
		Torej $M \neq \varnothing$.
		
		\item "Ce $\lnot \exists z \in S: z \sqsubset t$:
		
		Potem je $t \in M$. Torej $M \neq \varnothing$.
	\end{enumerate}

	\item[$2 \Rightarrow 3$] Predpostavimo: vsaka neprazna $S \subseteq A$ ima minimalni element.
	
	Dokazujemo: $A$ nima padajo"ce verige.
	\begin{equation*}
	\lnot \exists a: \NN \to A: \text{$a$ padajo"ca veriga}
	\end{equation*}
	Predpostavimo, da je $a: \NN \to A$ padajo"ca veriga. I"s"cemo protislovje.
	
	Mno"zica $C = \{a(n): n \in \NN\} \subseteq A$ je neprazna ($a(0)$ vsebuje).
	
	Po predpostavki ima minimalni element $a(j)$, vendar $C$ nima minimalnega elementa, ker za $\forall i \in \NN: a(i+1) \sqsubset a(i)$
	
	$\rightarrow \leftarrow$
	
	\item[$3 \Rightarrow 1$] Predpostavimo $A$ nima padajo"ce verige. 
	
	Dokazujemo: $\sqsubset$ je dobro osnovana.
	
	Naj bo $S \subseteq A$ progresivna. Dokazujemo $S = A$. Trdimo, da $C := A \setminus S$ nima minimalnega elementa. "Ce bi bil $c \in S$ minimalni, bi to pomenilo:
	\begin{gather*}
	\forall x \in A: x \sqsubset c \Rightarrow x \notin C \iff \\
	\forall x \in A: x \sqsubset c \Rightarrow x \in S \text{ ker je } A \setminus C = S
	\end{gather*}
	Ker je $S$ progresivna, sledi $c \in S$, kar je v nasprotju z $c \in A \setminus S$.
	
	Torej $C$ nima minimalnega elementa.
	
	Doka"zimo $S = A$ s protislovjem.
	
	Denimo $S \neq A$. Potem obstaja element v $A \setminus S$. Torej $C = S \setminus A$ ni prazna in nima minimalnega elementa. Po lemi v $C$ obstaja padajo"ca veriga, ki je tudi padajo"ca veriga v $A$. Protislovje s predpostavko (3). $\rightarrow \leftarrow$
	
	\hfill $\square$
\end{itemize}
%
\subsection{Aksiom Izbire}
V lemi smo uporabili \textbf{aksiom odvisne izbire:}

Naj bo $A$ neprazna in $R \subseteq A \times S$ celovita: $\forall x \in A \exists y \in A: x R y$. Tedaj obstaja $f: \NN \to A$, da velja $\forall n \in \NN: f(n) R f(n+1)$.

V lemi: $R$ je bila relacija $x R y \iff y < x$ in $f$ je bila padjo"ca veriga.

Bolj splo"sen je \textbf{aksiom izbire:}

Vsaka dru"zina nepraznih mno"zic ima funkcijo izbire.

"Ce je $A : I \to \set$ dru"zina mno"zic in $\forall i \in I: A_i \neq \varnothing$, potem 
\begin{equation*}
\exists f \in \prod_{i \in I} A_i: \top
\end{equation*}
To pomeni $f: I \to \bigcup_{i \in I} A_i$ in velja
\begin{equation*}
\forall i \in I: f(i) \in A_i
\end{equation*}
\textbf{Posledica aksioma izbire:} Vsaka surjekcija ima prerez. To pomeni $f: A \to B$ surjektivna, prerez $f$ je $g: B \to A$, da vleja $f \circ g = id_B$

Uporabimo izbiro na dru"zini $D: I \to \set$
\begin{gather*}
I := B \\
D_y := f^*(\{y\}) = \{x \in A: f(x) = y\}
\end{gather*}
$D_y \neq \varnothing$, ker je $f$ surjektivna.

Torej obstaja funkcija izbire $g: B \to \bigcup_{y \in B} D_y = A$, da je
\begin{gather*}
\forall y \in B: g (y) \in D_y \\
\forall y \in B: f(g(y)) = y \\
f \circ g = id_B
\end{gather*}
\textbf{Premislek:} "Ce ima vsaka surjekcija prere, potem velja aksiom izbire.