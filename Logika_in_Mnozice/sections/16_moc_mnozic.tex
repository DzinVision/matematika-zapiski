Za vsako $n \in \NN$ definiramo \emph{standardno mno"zico} z $n$ elementi:
\begin{equation*}
n := \{k \in \NN: k < n\} = \{0, 1, \ldots, n-1\}
\end{equation*}
\textbf{Primer:}
\begin{align*}
[0] &= \{\}\\
[1] &= \{0\} \\
[4] &= \{0, 1, 2, 3\}
\end{align*}
\textsc{Definicija:} Mno"zica je \emph{kon"cna}, "ce je izomorfn kaki standardni mno"zici
\begin{equation*}
\text{$A$ kon"cna} \iff \exists n \in \NN: A \cong [n]
\end{equation*}
\textsc{Izrek:} $A \cong [m] \land A \cong [n] \Rightarrow m = n$

\textsc{Dokaz:} Opustimo, ker je o"citno.

\textsc{Definicija:} \emph{Mo"c kon"cne mno"zice} $A$ je tisti $n \in \NN$, za katerega velja
\begin{equation*}
A \cong [n]
\end{equation*}
Mo"c $A$ ozna"cimo z $|A|$
\subsection{Ra"cunanje mo"ci}
\begin{align*}
|A \times B| &= |A| \cdot |B| \\
|A + B|& = |A| + |B| \\
\left|B^A\right| &= |B|^{|A|} \\
|A \cup B| &= |A| + |B| - |A \cap B|
\end{align*}
\textbf{Princip vklju"citve in izklju"citve}
\begin{equation*}
|A \cup B \cup C| = |A| + |B| + |C| - |A \cap B| - |A \cap C| - |B \cap C| + |A \cap B \cap C|
\end{equation*}

\textsc{Definicija:} $A$ je \emph{neskon"cna}, "ce ni kon"cna.

\textsc{Izrek:} $A$ je nekson"cna $\iff$ obstaja injektivna $\NN \to A$.

\textsc{Dokaz:}
\begin{itemize}
	\item[($\Rightarrow$)] Denimo, da je $A$ neskon"cna. Ideja dokaza: injektivno $e: \NN \to A$ definiramo po korakih.
	
	Zaporedje $e(0), e(1), e(2), \ldots$ definiramo rekurzivno: "ce smo "ze definirali $e(0), \ldots, e(n-1)$, potem lahko dodamo "se $e(n)$.
	
	Definiramo $e(0)$: ker $A \neq [0], A \neq \varnothing$, torej lahko izberemo $e(0) \in A$.
	
	Denimo, da "ze imamo $e(0), \ldots, e(n-1)$ in da so $e(0), \ldots, e(n-1)$ paroma razli"cni ($e$ je injektivna). I"s"cemo element $A \setminus \{e(0), \ldots, e(n-1)\}$. "Ce bi veljajo $A \setminus \{e(0), \ldots, e(n-1)\} = \varnothing$, bi imeli $A \cong [n]$, kar ni res. Torej lahko izberemo $e(n) \in A \setminus \{e(0), \ldots, e(n-1)\}$.
	
	\item[($\Leftarrow$)] Denimo $e: \NN \to A$ injektivna. Dokazujemo, da je $A$ neskon"cna, t.j.: $\neq \exists n \in \NN: A \cong [n]$.
	
	Predpostavimo, da je $n \in \NN$ in $A \cong [n]$, ter i"s"cemo protislovje.
	
	Potem bi imeli $\NN \stackrel{e}{\to} A \stackrel{\text{bijekcija}}{\longrightarrow}[n]$ injekcijo, kar ni mo"zno. Na predavanjih nismo dokazali, vendar v smo dobili za premislek: $\NN$ je neskon"cna, t.j.: $\neq \exists m \in \NN : \NN \cong [m]$).
\end{itemize}
%
"Ce je $A$ kon"cna, je mo"c $|A|$ naravno "stevilo. V splo"snem je mo"c $|A|$ \emph{kardinalno "stevilo}.

\textsc{Definicija:} Naj bosta $A$ in $B$ mno"zici.
\begin{enumerate}
	\item $A$ ima enako mo"c kot $B$ ($A$ in $B$ sta \emph{ekvipolentni}), ko velja $A \cong B$. Pi"semo $|A| = |B|$.
	
	\item $A$ ima manj"so ali enako mo"c kot $B$, ko obstaja injektivna preslikava $A \to B$. Pi"semo $|A| \leq |B|$.
	
	\item $A$ ima manj"so mo"c kot $B$, ko velja
	\begin{equation*}
	|A| \leq |B| \land |A| \neq |B|
	\end{equation*}
	Pi"semo $|A| < |B|$.
\end{enumerate}
\textsc{Primer:}
\begin{gather*}
S = \{n \in \NN: \text{$n$ je sodo}\} \\
|S| \leq |\NN| \text{ ker $n \mapsto n$ injektivna $S \to |NN$} \\
|S| = |NN| \text{ ker $n \mapsto 2n$ bijektivna $\NN \to S$}
\end{gather*}
%
\textsc{Izrek:} $|A| \leq |B| \iff A \neq \varnothing$ ali obstaja surjektivna $B \to A$.

\textsc{Dokaz:}
\begin{itemize}
	\item[($\Rightarrow$)] Denimo $f: A \to B$ injektivna in $A \neq \varnothing$. Potem obstaja $x_0 \in A$. Surjekcijo $g: B \to A$ definiramo s predpisom
	\begin{equation*}
	g(y) = \begin{cases}
	x, & f(x) = y \\
	x_0, & \forall z \in A: f(z) \neq y
	\end{cases}
	\end{equation*}
	
	\item[($\Leftarrow$)] Denimo $A$ prazna ali obstaja surjektivna $f: B \to A$.
	
	"Ce $A = \varnothing$, je $\varnothing \to B$ injektivna.
	
	"Ce $f: B \to A$ surjektivna, potem ima prerez $g: A \to B$. $f \circ g = id_A$, torej je $g$ injektivna.
\end{itemize}
\hfill $\square$

\textsc{Izrek:} (Cantor) $|A| < |\mathcal{P}(A)|$.

\textsc{Dokaz:} Najprej doka"zimo $|A| \leq |\mathcal{P}(A)|$.

I"s"cemo injektivno preslikavo $f: A \to \mathcal{P}(A)$.
\begin{equation*}
f (x) = \{x\}
\end{equation*}
Ta je injektivna, ker iz $\{x\} = \{y\}$ sledi $x \in \{y\}$ sledi $x = y$.

Doka"zimo $|A| \neq |\mathcal{P}(A)|$. Dokazujemo
\begin{align*}
\lnot \exists g : A \to \mathcal{P}(A)&: g \text{ bijekcija} \\
\forall g: A \to \mathcal{P}(A)&: g \text{ ni bijekcija}
\end{align*}
Naj bo $g: A \to \mathcal{P}(A)$. Dokazujemo, da $g$ ni surjektivna ali $g$ ni injektivna. Doka"zimo, da $g$ ni surjektivna.

Trdimo, da mno"zica
\begin{equation*}
S := \{x \in A: x \notin g(x)\} \in \mathcal{P}(A)
\end{equation*}
ni v sliki $g$ (in torej $g$ ni surjektivna).
\begin{align*}
\lnot \exists y \in A&: g(y) = S \\
\forall y \in A&: g(y) \neq S
\end{align*}
Naj bo $y \in A$. Dokazujemo $g(y) \neq S$. Predpostavimo $g(y) = S$ in i"s"cemo protislovje
\begin{enumerate}
	\item velja $y \notin S$
	
	"Ce $y \in S$, bi sledilo $y \notin g(y) = S$. $\rightarrow \leftarrow$
	
	\item velja $\lnot (y \notin S)$
	
	"Ce $y \notin S$, bi sledilo $y \notin g(y) = S$. Po definiciji $S$ sledi, $y \in S$. $\rightarrow \leftarrow$
\end{enumerate}
Torej $(1) \rightarrow \leftarrow (2)$
%
\subsection{"Stevne in ne"stevne mno"zice}
Moc mno"zice $\NN$ ozna"cimo z $\aleph_0$.
\begin{equation*}
|\NN| = \aleph_0
\end{equation*}
%
\textsc{Definiicija:} Mno"zica je \emph{"stevna}, "ce je njea mo"c $\leq \aleph_0$. Mno"zica je \emph{ne"stevna} "ce ni "stevna.

\textsc{Izrek:} Za mno"zico $A$ je ekvivalentno:
\begin{enumerate}
	\item $A$ je "stevna
	\item obstaja injektivna $A \to \NN$
	\item $A$ je prazna ali obstaja surjektvina $\NN \to A$
	\item obstaja preslikava $\NN \to 1 + A$, katere slika vsebuje $A$
	\item $A$ je kon"cna ali izomorfna $\NN$
\end{enumerate}
\textsc{Dokaz:} (1) in (2) sta ekvivalentni po definiciji. (2) in (3) sta ekvivalentni po izreku. (1) in (5) sta ekvivalentni po definiciji.

\textsc{Izjava:} $\NN \cong \NN \times \NN$

\textsc{Dokaz:} Da je $\NN \times \NN$ "stevna lahko doka"zemo na podoben na"cin, kot dokazujemo, da je $\QQ$ "stevna. Urejene pare $(x, y), x, y \in \NN$ uredimo v kvadrat, in jih navedemo po diagonalah. Bolj korekten dokaz smo pokazali na vajah.

\textsc{Izrek:} "Stevna unija "stevnih mno"zic je "stevna.

\textsc{Dokaz:} Imamo $I$ "stevna, $A: I \to \set$ in $A_i$ je "stevna za vsak $i \in I$. Dokazujemo, da je $\bigcup_{i \in I} A_i$ "stevna. Iz dokaza za posebene primer lahko izpeljemo splo"sen dokaz, zato smo na vajah dokazali samo za poseben primer.

\textbf{Poseben primer:} $I = \NN$ in $A_i$ "stevna neprazna.

Imamo $A: \NN \to Set$, $A_n \neq \varnothing$ "stevna.

Vemo $\forall n \in \NN: A_n \neq \varnothing$ "stevna. Po to"cki (3) v prej"snjem izreku, velja
\begin{equation*}
\forall n \in \NN \exists e : \NN\to A_n \text{ surjektivna}
\end{equation*}
Po aksiomu izbire obstaja $f$ preslikava
\begin{equation*}
f \in \prod_{n \in \NN} \{g: \NN \to A_n: g \text{ surjekcija}\}
\end{equation*}
Se pravi: za $\forall n \in \NN$ smo izbrali surjekcijo
\begin{equation*}
f_n : \NN \to A_n
\end{equation*}
Definiramo $h: \NN \times \NN \to \bigcup_{n \in \NN} A_n, \quad h(i, j) = f_i (j)$

Trdimo, da je $h$ surjekcija.

Naj bo $x \in \bigcup_{n \in \NN} A_n$. Torej obstaja $m \in \NN$, da je $x \in A_m$.

Ker je $f_m: \NN \to A_m$ surjektivna, obstaja $l \in \NN$, da je $f_m(l) = x$, torej
\begin{equation*}
h(m, l) = f_m(l) = x
\end{equation*}
Imamo Surjekcijo $\NN \stackrel{\text{bijektivna}}{\longrightarrow} \NN \times \NN \stackrel{h}{\longrightarrow} \bigcup_{n \in \NN} A_n$

\hfill $\square$

\textsc{Izrek:} (Cantor-Schr\"oder-Bernstein) "Ce obstaja injektivni preslikavi $A \to B$ in $B \to A$, potem obstaja bijektivna preslikava $A \to B$.
\begin{equation*}
|A| \leq |B| \land |B| \leq |A| \Rightarrow |A| = |B|
\end{equation*}
\textsc{Dokaz:} Denimo, da sta $f: A \ to B$ in $g: B \to A$ injektivni. Dokazujemo, da obstaja bijekcija $A \to B$.

\emph{Orbita} za $x \in A$ je zaporedje
\begin{equation*}
\ldots, f^{-1}\left(g^{-1}(x)\right), x, f(x), g(f(x)), f(g(f(x))), \ldots
\end{equation*}
Na levi se orbita kon"ca, "ce dobimo element, ki ni v sliki $f$ oziroma ni v sliki $g$ (ker potem ne moremo uporabiti $f^{-1}$ oziroma $g^{-1}$).

Na desni se orbita ne kon"ca.

Imamo tri mo"znosti:
\begin{enumerate}
	\item Orbita se na levi ne kon"ca
	\item Na levi se orbita kon"ca z elementom iz $A$
	\item Na levi se orbita kon"ca z elementom iz $B$
\end{enumerate}
Trdimo, da orbite tvorijo razbitje $A$ in $B$:
\begin{itemize}
	\item element orbite "ze dolo"ca celotno orbito
	\item "ce je element v dveh orbitah sta enaki
	\item vsak element je v natanko eni orbiti
\end{itemize}
\begin{enumerate}
	\item "Ce se orbita ne kon"ca na levi:
	
	Vsak $x \in A$ se preslika s $f$ v svojega soseda ne desni (za skico glej zvezek).
	\item "Ce se orbita kon"ca na levi z $A$:
	
	$x \in A$ v tej orbiti preslikamo z $f$ v sosede na desni (za skico ponovno glej zvezek).
	\item  "Ce se orbita kon"ca z $B$:
	
	$\forall x \in A$ v orbiti slikamo z $g^{-1}$ v levega soseda (v tretje gre rado: skica je v zvezku).
\end{enumerate}
\hfill $\square$