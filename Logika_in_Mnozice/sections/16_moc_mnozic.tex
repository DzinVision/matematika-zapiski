Za vsako $n \in \NN$ definiramo \emph{standardno mno"zico} z $n$ elementi:
\begin{equation*}
n := \{k \in \NN: k < n\} = \{0, 1, \ldots, n-1\}
\end{equation*}
\textbf{Primer:}
\begin{align*}
[0] &= \{\}\\
[1] &= \{0\} \\
[4] &= \{0, 1, 2, 3\}
\end{align*}
\textsc{Definicija:} Mno"zica je \emph{kon"cna}, "ce je izomorfn kaki standardni mno"zici
\begin{equation*}
\text{$A$ kon"cna} \iff \exists n \in \NN: A \cong [n]
\end{equation*}
\textsc{Izrek:} $A \cong [m] \land A \cong [n] \Rightarrow m = n$

\textsc{Dokaz:} Opustimo, ker je o"citno.

\textsc{Definicija:} \emph{Mo"c kon"cne mno"zice} $A$ je tisti $n \in \NN$, za katerega velja
\begin{equation*}
A \cong [n]
\end{equation*}
Mo"c $A$ ozna"cimo z $|A|$
\subsection{Ra"cunanje mo"ci}
\begin{align*}
|A \times B| &= |A| \cdot |B| \\
|A + B|& = |A| + |B| \\
\left|B^A\right| &= |B|^{|A|} \\
|A \cup B| &= |A| + |B| - |A \cap B|
\end{align*}
\textbf{Princip vklju"citve in izklju"citve}
\begin{equation*}
|A \cup B \cup C| = |A| + |B| + |C| - |A \cap B| - |A \cap C| - |B \cap C| + |A \cap B \cap C|
\end{equation*}